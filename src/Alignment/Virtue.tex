\section{Tumour deformation modelling}
\label{chapterlabel3}

The tract orientation atlas represents the expected orientation and location of the tract for a typical healthy subject.
In order to correct for displacement of white matter tracts due to space occupying lesions, the atlas can be deformed before comparing with the native FOD map.

The idea is to obtain a deformation model which is simple enough to compute that little temporal and computational resources are required.
This unfortunately rules out any advanced tumour growth modelling using finite element methods which explicitly model the biomechanical and cytoarchitectural \note{? word for cell proliferation etc} properties of the tumoural and peritumoural environment.
While such models are valuable for understanding tumour growth, progression at effects on surrounding tissue from an oncological perspective, we are interested solely in the gross spatial effects of tumour growth: what effect does the presence of a tumour have on the physical location a given volume unit of tissue?
Of course, thinking this question can be fully answered without considering the complicated factors described above, such as whether a tumour is encapsulated or infiltrating, is folly.
But focussing only on the spatial deformation and beginning with the simplest possible model to capture its effects can keep the complexity as low as feasible while allowing for more components to be introduced if necessary \note{??}.

We begin with a radial deformation model described by Nowinski and Belov \citep{Nowinski2005}.
Their motivation was remarkably similar: the rapid deformation of a morphological brain atlas to aid the interpretation of brain anatomies affected by tumour mass effect.
The required model inputs are the segmentations of the tumour and brain volumes.
We define the direction $\mathbf{\hat{e}}$, which is the unit vector along the line connecting a point anywhere within the brain $P(x,y,z)$ to the tumour centre of mass, $S$.
This is the direction along which we assume the tissue at that point to be shifted by the tumour: radially outward from the tumour centre.
Along $\mathbf{\hat{e}}$ we also define $D_p$ as the distance  $\|\overrightarrow{SP}\|$, $D_b$ as the distance from $S$ to the brain surface and $D_t$ as the distance from $S$ to the tumour surface (Fig. \ref{fig:virtue}).

\begin{figure}[htp]
  \centering
  \includegraphics{virtue_vars.pdf}
  \caption{Graphical schema of the variables defined in the radial deformation model}
  \label{fig:virtue}
\end{figure}

Then for a point in the original image $P = (x,y,z)$ the transformed location in the deformed image $P' = (x',y',z')$ is

\begin{align}\label{eq:forwardP}
  P' = f(P) = P + \mathbf{\hat{e}}kD_ts.
\end{align}

The amount of displacement $\Delta P = kD_ts$ is thus determined by $D_t$, a scale factor $0<s \leq 1$ and a displacement factor $k$.
In \textcite{Nowinski2005}, $k$ is a linear function of $D_p$: $k = 1-\frac{D_p}{D_b}$. \footnote[2]{In the original \textcite{Nowinski2005} article, the deformation is described in reverse, as a shrinking model, and the variables there look a little different. They are consistent with the formulation used here, which has been chosen for ease of conceptualisation. Note that both forward and reverse models are required for different types of image transformation, and will be derived later.}
This can be conceptualised as a displacement force radiating from the centre of the tumour and decaying linearly with distance, reaching 0 only at the brain boundary.
However, initial experimentations with this model revealed that such a linearly decaying force doesn't do well at capturing the displacement fields observed in real tumour cases.
The elasticity and compressibility of brain tissue means that the radial force is absorbed by surrounding tissue more rapidly than the linear function suggests.
Even with very large tumours, it is common for parts of the brain some distance from the tumour surface to experience no displacement at all, suggesting a  more rapidly decaying function would be a more appropriate choice for $k$.

An exponentially decaying function captures this well, while remaining easily computable, close-form and invertible.
We begin with a function in the form $k \propto e^{-\lambda \frac{D_p}{D_t}}$.
There are two boundary conditions: points on the brain surface should not be displaced ($k(D_p = D_b) = 0$) and points at the centre of the tumour should be displaced by exactly $D_t$ ($k(D_p = 0) = 1$).
Note that the latter boundary condition is an assumption reflecting a fully encapsulated tumour, where no normal tumour remains inside the final tumour boundary after displacement.
Solving for these boundary conditions gives us a normalisation constant:

\begin{align*}
  k &= a e^{-\lambda x} + c &\text{ where } x = D_p / D_b \\
  k(x=0)=1 \longrightarrow 1 &= a e^{-0} + c = a + c \\
  k(x=1)=0 \longrightarrow 0 &= a e{^-\lambda} + c \\
  -c &= (1-c) e^{-\lambda} \\
  c &= \frac{e^{-\lambda}}{e^{-\lambda} - 1}
\end{align*}

giving

\begin{align}\label{eq:forwardk}
  k(P) = (1-c)e^{-\lambda \frac{D_p}{D_b}} +c.
\end{align}

Equations (\ref{eq:forwardP}) and (\ref{eq:forwardk}) describe the deformation field in forward warp convention, where a point in the original image is mapped to a position in the transformed image.
Forward warping works well for continuous valued data, for example streamlines.
However, for discreet data such as pixels (or voxels), forward warping brings the problem of ``holes", when a mapped location in the destination image falls between coordinates on the discreet voxel grid, and the original sample value has to be distributed among the neighbouring grid points.
Thus it is usually preferable, if the warping function is invertible, to determine for each grid point in the transformed image, the corresponding continuous-valued point in the source image from which the appropriate value can be interpolated from the surrounding grid points. \note{does this make sense??}
Reverse warp convention is used by most \note{??} medical image manipulation packages to deform (and resample) a gridded image.
Thus we need to obtain the inverse mapping $P = f^{-1}(P')$ by solving equation (\ref{eq:forwardP}) for $P$:

\begin{align}
  P = P' - \mathbf{\hat{e}}(D_t c - \frac{D_b}{\lambda}\mathcal{W}_0(\frac{-\lambda D_t (1-c) e^{-\lambda(D_{p'}-D_tc)/D_b}}{D_b}))
\end{align}

where $\mathcal{W}_0(y)$ is the principal branch of the lambert $\mathcal{W}$ function, defined as the inverse function of $ y(x) = xe^x $ for $x,y \in \mathbb{R}$. \note{derive this ?in appendix??}

The inverse function for the linear model is (as formulated in \textcite{Nowinski2005}):

\begin{align}
  P = P' - \mathbf{\hat{e}}(1-\frac{D_{P'}-D_t}{D_b-D_t})D_ts
\end{align}

The appropriate value for the decay parameter $\lambda$ will depend on the specific lesion being modelled. For example, smaller lesions (20-30mm diameter) typically displace tissue only in their immediate surroundings, with distant tissue remaining virtually unmoved. In such cases, a higher value of $\lambda$ ($\geq 3$), indicating stronger decay of deformation, would be appropriate (Figure \ref{fig:virtue}).
In any case, in order to keep the transforms well behaved, we need to maintain the condition that every point $P$ in the source image that is within the tumour boundary ends up strictly outside the tumour in the eventual deformed image.
In other words,

\begin{equation}\label{eq:lambdabound}
  k(P) \geq 1 - \frac{D_P}{D_t}
\end{equation}

must hold for all $P$.

Given that the gradient of $k$ is strictly decreasing and $g(D_P) = 1 - \frac{D_P}{D_t}$ is linear, it is sufficient to set
\begin{align*}
  \frac{d}{dP}\bigg\rvert_{D_P=0}k(D_P) &= \frac{d}{dP}\bigg\rvert_{D_P=0}g(D_P) &\text{ where } \frac{dk}{dP} &= -\frac{\lambda}{D_b}(1-c)e^{-D_p/D_b} &\text{ and } \frac{dg}{dP} &= -\frac{1}{D_t}
\end{align*}

We solve for $\lambda$:

\begin{align*}
  -\frac{\lambda_{max}}{D_b}(1-c) &= -\frac{1}{D_t} \\
  \lambda_{max} &= \frac{D_b}{D_t (1-c)}
\end{align*}

This value can be determined iteratively, or analytically with the expression

\begin{equation}
  \lambda_{max} = \mathcal{W}_0(-\frac{D_b}{D_t}e^{-D_b/D_t})+\frac{D_b}{D_t}
\end{equation}

Thus for strictly non-infiltrating lesions, we set $\lambda \leq \lambda_{max}$ to satisfy equation (\ref{eq:lambdabound}), where $\lambda_{max}$ is used as the default value if none is specified. Note that $\lambda_{max}$ varies throughout the brain, as it depends on the relative distances to brain and tumour surfaces for each specific $P$.

The tumour deformation model is implemented in Python, and full execution takes on average 1 min for a 208 x 256 x 256 voxel image.
If lookup tables for $ D_t$ and $D_b$ are precomputed and saved, then subsequent executions of the model (e.g. with different values for $\lambda$ and $s$, as appropriate for a given tumour) take less than 10 seconds, as long as the tumour and brain segmentations remain unchanged.

\note{Maybe also some speculative stuff about infiltration}
