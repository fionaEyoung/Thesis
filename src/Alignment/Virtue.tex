\section{Tumour deformation modelling}
\label{chapterlabel3}

\note{this has been dumped from IJCARS paper}

The tract orientation atlas represents the expected orientation and location of the tract for a typical healthy subject.
In order to correct for displacement of white matter tracts due to space occupying lesions, a simple radial tumour expansion model is employed.

The model has been adapted from the one described by Nowinski and Belov \citep{Nowinski2005}.
The model inputs are the segmentations of the tumour and brain volumes.
Tumour segmentations were drawn manually for this study, while brain masks are readily computed by standard MRI analysis software.
We define the direction $\mathbf{\hat{e}}$, which is the unit vector along the line connecting a point $P(x,y,z)$ to the tumour centre of mass, $S$.
Along $\mathbf{\hat{e}}$ we also define $D_p$ as the distance  $\|\overrightarrow{SP}\|$, $D_b$ as the distance from $S$ to the brain surface and $D_t$ as the distance from $S$ to the tumour surface (Fig. \note{fig:virtue}).

Then for a point in the original image $P = (x,y,z)$ the transformed location in the deformed image $P' = (x',y',z')$ is

\begin{align}\label{eq:forwardP}
  P' = f(P) = P + \mathbf{\hat{e}}kD_ts.
\end{align}

An exponentially decaying function is used to model the displacement of each voxel (Fig. \ref{fig:virtue-demo}).
This choice was made in contrast to the linear relationship used in \citep{Nowinski2005} as it provides a better approximation to typically observed tumour displacement patterns, while remaining an easily computable, closed-form and invertible function.
The amount of displacement depends exponentially on the relative distance to the tumour and brain surfaces via the following relationship:

\begin{align}\label{eq:forwardk}
  k(P) = (1-c)e^{-\lambda \frac{D_p}{D_b}} +c
\end{align}

where the normalisation constant $ c = \frac{e^{-\lambda}}{e^{-\lambda}-1} $ ensures that $k = 1$ when $D_P = 0$ and $k = 0$ when $D_p = D_b$. The appropriate value for the decay parameter $\lambda$ will depend on the specific lesion being modelled. For example, smaller lesions (20-30mm diameter) typically displace tissue only in their immediate surroundings, with distant tissue remaining virtually unmoved. In such cases, a higher value of $\lambda$ ($\geq 3$), indicating stronger decay of deformation, would be appropriate (Figure \ref{fig:virtue}).

Equations (\ref{eq:forwardP}) and (\ref{eq:forwardk}) describe the deformation field in forward warp convention. To deform an image using reverse warp (``pull-back") convention, the inverse mapping $P' = f^{-1}(P)$ is needed, which is obtained by solving equation (\ref{eq:forwardP}) for $P$:

\begin{align}
  P = P' - \mathbf{\hat{e}}(D_t c - \frac{D_b}{\lambda}\mathcal{W}_0(\frac{-\lambda D_t (1-c) e^{-\lambda(D_p'-D_tc)/D_b}}{D_b}))
\end{align}

where $\mathcal{W}_0(y)$ is the principal branch of the lambert $\mathcal{W}$ function, defined as the inverse function of $ y(x) = xe^x $ for $x,y \in \mathbb{R}$.

If the lesion is not invading the surrounding tissue but instead fully displacing it (non-infiltrative), then under the simplified assumption that no original, healthy tissue is destroyed, $\lambda$ should be set to a value that ensures that every point $P$ within the lesion boundary is displaced to a new position $P'$ that is strictly outside the boundary. In other words,

\begin{equation}\label{eq:lambdabound}
  k(P) = (1-c)e^{-\lambda \frac{D_P}{D_b}} +c \geq 1 - \frac{D_P}{D_t}
\end{equation}

must hold for all $P$.

Given that the gradient of $k$ is strictly decreasing and $g(D_P) = 1 - \frac{D_P}{D_t}$ is linear, it is sufficient to set
\begin{align}
  \frac{d}{dP}\bigg\rvert_{D_P=0}k(D_P) = \frac{d}{dP}\bigg\rvert_{D_P=0}g(D_P)
\end{align}.

Differentiating both functions at $D_P=0$ and solving for $\lambda$, we have $\lambda_{max} = \frac{D_b}{D_t (1-c)}$.
Thus for strictly non-infiltrating lesions, we set $\lambda \leq \lambda_{max}$ to satisfy equation (\ref{eq:lambdabound}), where $\lambda_{max}$ is used as the default value if none is specified. Note that $\lambda_{max}$ varies throughout the brain, as it depends on the relative distances to brain and tumour surfaces for each specific $P$.

The tumour deformation model is implemented in Python, and full execution takes on average 1 min for a 208 x 256 x 256 voxel image.
If lookup tables for $ D_t$ and $D_b$ are precomputed and saved, then subsequent executions of the model (e.g. with different values for $\lambda$ and $s$, as appropriate for a given tumour) take less than 10 seconds, as long as the tumour and brain segmentations remain unchanged.

\note{Maybe also some speculative stuff about infiltration}
