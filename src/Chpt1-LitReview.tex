\chapter{Review of Literature}
\label{chapterlabel1}

\begin{markdown}

  # Chapter contents

  ## Basic background

  - really first principles stuff like diffusion, MRI and neuroanatomy

  ## Diffusion MRI

  - image formation basics
  - fiber orientations
  - FOD modelling
    - DTI
    - CSD
    - etc.

  ## Tractography and tract segmentation

  - streamline tracking
    - deterministic vs probabilistic
    - targeted, ROIs etc.
  - non-tractography tract segmentation
    - ...
  - Applications

  ## Neurosurgery

  - Basics: resection, tumours, epilepsy etc.
  - Imaging in neurosurgery
    - presurgical planning etc.
    - intraoperative

  ## Tumour deformation modelling


\end{markdown}

\section{Neuroanatomy: Micro to macro}

The building blocks of living organisms begin at the level of molecules and atoms, through macromolecules such as proteins and lipids, cell organelles, cells, tissues, organs and finally whole organisms.
For the purposes of this report, though, we can start at the cellular level, and focus on a single organ: the brain.
Brain tissues consist of numerous cell types.
The principle functional cells are neurons, which perform the computations underpinning all aspects of neural function. They are supported, and outnumbered ( by a \note{factor of?}), by a network of glial cells, each with specialised functions.

\section{Diffusion magnetic resonance imaging}

\subsection{Diffusion in tissue}

In living tissue, water molecules are not free to diffusion for long distances in all directions. Some tissue environments are highly constrained, with diffusion occurring principally along a single direction \note{too simplistic}, while in others, fewer barriers allow free diffusion.

\subsection{Image acquisition}

MRI images are obtained by detecting the spin relaxation resonance stuff of principally hydrogen atoms (which are abundant in the body in H2O molecules).

The application of a strong linear magnetic field causes all spins to align themselves with the direction of the field. This is the bulk magnetisation, in its "relaxed" state (aligned with B0).






