
%DIF LATEXDIFF DIFFERENCE FILE
%DIF DEL .drafts/v1.1.tex   Wed Mar  6 00:08:34 2024
%DIF ADD .drafts/v2.0.tex   Wed Mar  6 00:08:36 2024






\RequirePackage[l2tabu, orthodox]{nag}


\documentclass[12pt,phd,a4paper,twoside]{ucl_thesis}



\usepackage{emptypage}

\usepackage{graphicx}
\usepackage{graphbox}

\usepackage{float}

\usepackage{amsmath, bm}
\usepackage{amssymb}

\usepackage{gensymb}
\usepackage{textcomp}



\usepackage{setspace}
\setstretch{1.2}











\usepackage{ifdraft}
\ifdraft{
\usepackage[margin=1in]{geometry}
  \setstretch{1}
  \setlength \topmargin{10mm}
  \setlength \oddsidemargin {20mm} \setlength \evensidemargin {20mm}
\usepackage[modulo]{lineno}
  \linenumbers
}


\usepackage{multirow}






\usepackage{bibentry}



\usepackage[format=hang,font=small,labelfont={bf,sf},labelsep=quad]{caption}
\usepackage[labelformat=simple]{subcaption}
\usepackage[innercaption,raggedright]{sidecap}
\sidecaptionvpos{figure}{t}
\AtBeginEnvironment{SCfigure}{\captionsetup{format=plain}}

\usepackage{etoolbox}

\usepackage{microtype}



\usepackage{pdflscape}
\usepackage{rotating}




\usepackage[parfill,indent,skip=0pt]{parskip}


\usepackage[sf,bf,raggedright]{titlesec}
\titleformat{\paragraph}[hang]{\large\sffamily\raggedright}{}{0pt}{}[]
\titleformat{\subparagraph}[hang]{\normalsize\sffamily\bfseries\raggedright}{}{0pt}{}[]

\titlespacing*{\paragraph}{0pt}{0.5\baselineskip}{0.5\baselineskip}
\titlespacing*{\section}{0pt}{3\baselineskip}{\baselineskip}
\titlespacing*{\subsection}{0pt}{2\baselineskip}{\baselineskip}
\titlespacing*{\subsubsection}{0pt}{\baselineskip}{\baselineskip}

\usepackage{verbatim}
\usepackage{spverbatim} 



\usepackage{courier}
\usepackage[semibold,oldstyle]{sourcesanspro}
\usepackage{libertinust1math}
\usepackage[T1]{fontenc}

\usepackage[final]{listings} \lstset{basicstyle=\ttfamily\small, basewidth=0.6em, keywordstyle=\bfseries, morekeywords={while,do,not}}

\usepackage{enumerate, enumitem}
\usepackage{framed, color}

\usepackage{tabularx, bigstrut, multirow, booktabs, array}
\newcolumntype{+}{>{\global\let\currentrowstyle\relax}}
\newcolumntype{^}{>{\currentrowstyle}}
\newcommand{\rowstyle}[1]{\gdef\currentrowstyle{#1}#1\ignorespaces
}\usepackage[table]{xcolor}

\renewcommand{\thefootnote}{\fnsymbol{footnote}}
 
\raggedbottom
\widowpenalty10000
\clubpenalty10000
\interfootnotelinepenalty10000



\usepackage{bibentry}
\makeatletter\let\saved@bibitem\@bibitem\makeatother
\usepackage[hidelinks]{hyperref}
\makeatletter\let\@bibitem\saved@bibitem\makeatother
\makeatletter
\AtBeginDocument{
    \hypersetup{
        pdfsubject={Medical image computing},
        pdfkeywords={diffusion MRI, neurosurgery, white matter tracts, tractography},
        pdfauthor={Fiona Young},
        pdftitle={Fibre tract imaging with intraoperative diffusion-weighted MRI for neurosurgical navigation}
    }
}
\makeatother
 






\renewcommand{\topfraction}{0.9}	\renewcommand{\bottomfraction}{0.8}	

\setcounter{topnumber}{2}
\setcounter{bottomnumber}{2}
\setcounter{totalnumber}{4}     \setcounter{dbltopnumber}{2}    \renewcommand{\dbltopfraction}{0.9}	\renewcommand{\textfraction}{0.2}	

\renewcommand{\floatpagefraction}{0.7}	\renewcommand{\dblfloatpagefraction}{0.7}	



\usepackage[inkscapearea=page,
            inkscapeexe=/Applications/Inkscape.app/Contents/MacOS/inkscape,
            draft=false,
            inkscapelatex=true]{svg}

\ifdraft{\graphicspath{{draft_figs/}{figs/}}
  \svgpath{{draft_figs/}{figs/}}
  \setkeys{Gin}{draft=false}
}{\graphicspath{{figs/}{draft_figs/}}
  \svgpath{{figs/}{draft_figs/}}
}

\newcommand{\x}{\nobreak\hspace{.1em minus .045em}$\times$\nobreak\hspace{.1em minus .045em}}
 



\usepackage[doi=true,
            isbn=true,
            url=false,
            natbib=true,
            maxcitenames=1,
            maxbibnames=50,
            style=nature,
%DIF 180a180
            defernumbers=true, %DIF > 
%DIF -------
            backend=bibtex]{biblatex} \IfFileExists{bibliographies/Neuroimaging.bib}
          {\addbibresource{bibliographies/Neuroimaging.bib}}
          {\addbibresource{bibs/Neuroimaging.bib}}
\IfFileExists{bibliographies/Neurosurgery.bib}
          {\addbibresource{bibliographies/Neurosurgery.bib}}
          {\addbibresource{bibs/Neurosurgery.bib}}
\IfFileExists{bibliographies/Figures.bib}
          {\addbibresource{bibliographies/Figures.bib}}
          {\addbibresource{bibs/Figures.bib}}
\IfFileExists{bibliographies/Oncology.bib}
          {\addbibresource{bibliographies/Oncology.bib}}
          {\addbibresource{bibs/Oncology.bib}}
\addbibresource{bibliographies/Maths.bib}
\addbibresource{bibliographies/Neuroscience.bib}
\addbibresource{bibliographies/Thesis.bib}


\urlstyle{same}



\usepackage{ifthen}
\renewcommand{\textcite}[2][]{
\ifthenelse { \equal {#1} {} }  {\citeauthor{#2}\autocite{#2}}   {\citeauthor{#1}\autocite{#2}}}

\DeclareCiteCommand{\fullcite}
  {\usebibmacro{prenote}}
  {\usedriver
     {\defcounter{maxnames}{50}}
     {\thefield{entrytype}}.}
  {\multicitedelim}
  {\usebibmacro{postnote}}
%DIF 212a213-220
 %DIF > 
 %DIF > 
\newrobustcmd*{\nextcitefullname}{\AtNextCite{\DeclareNameAlias{labelname}{given-family}\togglefalse{abx@bool@giveninits}}} %DIF > 
 %DIF > 
\newcommand\citeauthorfullname{\nextcitefullname\citeauthor} %DIF > 
 %DIF > 
\newcommand\Citeauthorfullname{\nextcitefullname %DIF > 
  \Citeauthor} %DIF > 
%DIF -------
 \usepackage[acronym,
            toc=false,
            style=super,
            seeautonumberlist,
            nogroupskip=true]
            {glossaries-extra}
\setabbreviationstyle[acronym]{long-short}
\GlsXtrSetDefaultGlsOpts{noindex}
\renewcommand*\glspostdescription{\quad}
\preto\chapter{\glsresetall}
\makeglossaries
\newacronym{mri}{MRI}{magnetic resonance imaging}
\newacronym{nmr}{NMR}{nuclear magnetic resonance}
\newacronym{dmri}{dMRI}{diffusion-weighted \protect\ifglsused{mri}{MRI}{magnetic resonance imaging}}
\newacronym{imri}{iMRI}{intraoperative \protect\ifglsused{mri}{MRI}{magnetic resonance imaging}}
\newacronym{fmri}{fMRI}{functional \protect\ifglsused{mri}{MRI}{magnetic resonance imaging}}
\newacronym{hardi}{HARDI}{high angular resolution diffusion imaging}
\newacronym{epi}{EPI}{echo planar imaging}
\newacronym{rsepi}{RS-EPI}{readout-segmented \protect\ifglsused{epi}{EPI}{echo planar imaging}}
\newacronym{ssepi}{SS-EPI}{single-shot \protect\ifglsused{epi}{EPI}{echo planar imaging}}
\newacronym{rf}{RF}{radio frequency}
\newacronym{snr}{SNR}{signal to noise ratio}
\newacronym{fid}{FID}{free induction decay}
\newacronym{2dft}{2DFT}{two dimensional \protect\ifglsused{ft}{FT}{Fourier transform}}
\newacronym{ft}{FT}{Fourier transform}
\newacronym{fe}{FE}{frequency encoding}
\newacronym{pe}{PE}{phase encoding}
\newacronym[longplural={proton densities}]{pd}{PD}{proton density}
\newacronym{ir}{IR}{inversion recovery}
\newacronym{mprage}{MPRAGE}{magnetization prepared rapid gradient echo}
\newacronym{dce}{DCE}{dynamic contrast enhanced}
\newacronym{csd}{CSD}{constrained spherical deconvolution}
\newacronym{odf}{ODF}{orientation distribution function}
\newacronym{dodf}{dODF}{diffusion orientation distribution function}
\newacronym{fodf}{fODF}{fibre \protect\ifglsused{odf}{ODF}{orientation distribution function}}
\newacronym{fod}{FOD}{fibre orientation distribution}
\newacronym{tod}{TOD}{track orientation distribution}
\newacronym{dti}{DTI}{diffusion tensor imaging}
\newacronym{dt}{DT}{diffusion tensor}
\newacronym{fa}{FA}{fractional anisotropy}
\newacronym{dec}{DEC}{directionally encoded colour}
\newacronym{ssst}{SSST}{single-shell, single-tissue}
\newacronym{msmt}{MSMT}{multi-shell, multi-tissue}
\newacronym[longplural={regions of interest}]{roi}{ROI}{region of interest}
\newacronym{afd}{AFD}{apparent fibre density}
\newacronym{adc}{ADC}{apparent diffusion coefficient}
\newacronym{fact}{FACT}{fibre assignment by continuous tracking}
\newacronym{tdi}{TDI}{track density imaging}
\newacronym{mppca}{MPPCA}{Marchenko-Pastur principal component analysis}
\newacronym{cns}{CNS}{central nervous system}
\newacronym{pns}{PNS}{peripheral nervous system}
\newacronym{bbb}{BBB}{blood-brain barrier}
\newacronym{cst}{CST}{corticospinal tract}
\newacronym{or}{OR}{optic radiation}
\newacronym[longplural={lateral geniculate nuclei}]{lgn}{LGN}{lateral geniculate nucleus}
\newacronym{ml}{ML}{Meyer's loop}
\newacronym[longplural={arcuate fasciculi}]{af}{AF}{arcuate fasciculus}
\newacronym[longplural={inferior fronto-occipital fasciculi}]{ifof}{IFOF}{inferior fronto-occipital fasciculus}
\newacronym[longplural={uncinate fasciculi}]{uf}{UF}{uncinate fasciculus}
\newacronym{wm}{WM}{white matter}
\newacronym{gm}{GM}{grey matter}
\newacronym{csf}{CSF}{cerebrospinal fluid}
\newacronym{crp}{CrP}{cerebellar peduncle}
\newacronym{cp}{CP}{cerebral peduncle}
\newacronym{cc}{CC}{corpus callosum}
\newacronym{ec}{EC}{external capsule}
\newacronym[longplural={superior longitudinal fasciculi}]{slf}{SLF}{superior longitudinal fasciculus}
\newacronym[longplural={superior fronto-occipital fasciculi}]{sfof}{SFOF}{superior fronto-occipital fasciculus}
\newacronym[longplural={vertical occipital fasciculi}]{vof}{VOF}{vertical occipital fasciculus}
\newacronym{ss}{SS}{sagittal stratum}
\newacronym{dbs}{DBS}{deep brain stimulation}
\newacronym[longplural={extents of resection}]{eor}{EOR}{extent of resection}
\newacronym{gtr}{GTR}{gross total resection}
\newacronym{str}{STR}{subtotal resection}
\newacronym{hgg}{HGG}{high-grade glioma}
\newacronym{lgg}{LGG}{low-grade glioma}
\newacronym{eeg}{EEG}{electroencephalography}
\newacronym{ct}{CT}{computed tomography}
\newacronym{des}{DES}{direct electrical stimulation}
\newacronym{who}{WHO}{World Health Organisation}
\newacronym{tg}{TG}{tractography}
\newacronym{tgr}{TGR}{reference tractography}
\newacronym{tf}{TF}{tractfinder}
\newacronym{tsd}{TSD}{TractSeg DKFZ}
\newacronym{tsx}{TSX}{TractSeg XTRACT}
\newacronym{at}{AT}{atlas registration}
\newacronym{hcp}{HCP}{human connectome project}
\newacronym{btc}{BTC}{Brain Tumour Connectomics}
\newacronym{dice}{DSC}{Dice-Soerensen similarity coefficient}
\newacronym{sh}{SH}{spherical harmonic}
\newacronym{fsl}{FSL}{FMRIB software library}
\newacronym{gosh}{GOSH}{Great Ormond Street Hospital}
\newacronym{nhnn}{NHNN}{National Hospital for Neurology and Neurosurgery}



\def\acrodefsep{ = }
\def\acrolistdelim{; }

\NewDocumentCommand{\acrolist}{>{\SplitList{,}}m}
    {\ProcessList{#1}{\acroformat}\firstitemtrue}
\newif\iffirstitem
\firstitemtrue
\newcommand\acroformat[1]{\iffirstitem
    \firstitemfalse
  \else
    \acrolistdelim \fi
  \glsname{#1}\acrodefsep \glsdesc{#1}}


 \newcommand\epigraph[2]{\begin{flushright}
    \parbox{0.75\textwidth}{\raggedleft #1}
    \vskip 1.5\baselineskip
    --- #2
  \end{flushright}}

\newcommand\epipage[2]{\clearpage\thispagestyle{empty}
  \vspace*{\fill}
  {\sffamily\epigraph{#1}{#2}}
  \vspace*{\fill}\pagebreak
}

\newcommand\epart[2][]{\cleardoublepage          \thispagestyle{empty}
  \vspace*{.3\vsize}        \refstepcounter{part}
  \addcontentsline{toc}{part}{#2}{\centering \textbf{\sffamily\Huge#2}\par}\vspace*{\fill}
  {\sffamily#1}
  \vspace*{\fill}\pagebreak
}
 
\setcounter{secnumdepth}{3}
\setcounter{tocdepth}{2}

\newcommand\note[1]{\ifdraft{{\large\textcolor{red}{#1}}}{}}
%DIF PREAMBLE EXTENSION ADDED BY LATEXDIFF
%DIF CFONT PREAMBLE %DIF PREAMBLE
\RequirePackage{color}\definecolor{RED}{rgb}{1,0,0}\definecolor{BLUE}{rgb}{0,0,1} %DIF PREAMBLE
\DeclareOldFontCommand{\sf}{\normalfont\sffamily}{\mathsf} %DIF PREAMBLE
\providecommand{\DIFaddtex}[1]{{\protect\color{blue} \sf #1}} %DIF PREAMBLE
\providecommand{\DIFdeltex}[1]{{\protect\color{red} \scriptsize #1}} %DIF PREAMBLE
%DIF SAFE PREAMBLE %DIF PREAMBLE
\providecommand{\DIFaddbegin}{} %DIF PREAMBLE
\providecommand{\DIFaddend}{} %DIF PREAMBLE
\providecommand{\DIFdelbegin}{} %DIF PREAMBLE
\providecommand{\DIFdelend}{} %DIF PREAMBLE
\providecommand{\DIFmodbegin}{} %DIF PREAMBLE
\providecommand{\DIFmodend}{} %DIF PREAMBLE
%DIF FLOATSAFE PREAMBLE %DIF PREAMBLE
\providecommand{\DIFaddFL}[1]{\DIFadd{#1}} %DIF PREAMBLE
\providecommand{\DIFdelFL}[1]{\DIFdel{#1}} %DIF PREAMBLE
\providecommand{\DIFaddbeginFL}{} %DIF PREAMBLE
\providecommand{\DIFaddendFL}{} %DIF PREAMBLE
\providecommand{\DIFdelbeginFL}{} %DIF PREAMBLE
\providecommand{\DIFdelendFL}{} %DIF PREAMBLE
%DIF HYPERREF PREAMBLE %DIF PREAMBLE
\providecommand{\DIFadd}[1]{\texorpdfstring{\DIFaddtex{#1}}{#1}} %DIF PREAMBLE
\providecommand{\DIFdel}[1]{\texorpdfstring{\DIFdeltex{#1}}{}} %DIF PREAMBLE
\newcommand{\DIFscaledelfig}{0.5}
%DIF HIGHLIGHTGRAPHICS PREAMBLE %DIF PREAMBLE
\RequirePackage{settobox} %DIF PREAMBLE
\RequirePackage{letltxmacro} %DIF PREAMBLE
\newsavebox{\DIFdelgraphicsbox} %DIF PREAMBLE
\newlength{\DIFdelgraphicswidth} %DIF PREAMBLE
\newlength{\DIFdelgraphicsheight} %DIF PREAMBLE
% store original definition of \includegraphics %DIF PREAMBLE
\LetLtxMacro{\DIFOincludegraphics}{\includegraphics} %DIF PREAMBLE
\newcommand{\DIFaddincludegraphics}[2][]{{\color{blue}\fbox{\DIFOincludegraphics[#1]{#2}}}} %DIF PREAMBLE
\newcommand{\DIFdelincludegraphics}[2][]{% %DIF PREAMBLE
\sbox{\DIFdelgraphicsbox}{\DIFOincludegraphics[#1]{#2}}% %DIF PREAMBLE
\settoboxwidth{\DIFdelgraphicswidth}{\DIFdelgraphicsbox} %DIF PREAMBLE
\settoboxtotalheight{\DIFdelgraphicsheight}{\DIFdelgraphicsbox} %DIF PREAMBLE
\scalebox{\DIFscaledelfig}{% %DIF PREAMBLE
\parbox[b]{\DIFdelgraphicswidth}{\usebox{\DIFdelgraphicsbox}\\[-\baselineskip] \rule{\DIFdelgraphicswidth}{0em}}\llap{\resizebox{\DIFdelgraphicswidth}{\DIFdelgraphicsheight}{% %DIF PREAMBLE
\setlength{\unitlength}{\DIFdelgraphicswidth}% %DIF PREAMBLE
\begin{picture}(1,1)% %DIF PREAMBLE
\thicklines\linethickness{2pt} %DIF PREAMBLE
{\color[rgb]{1,0,0}\put(0,0){\framebox(1,1){}}}% %DIF PREAMBLE
{\color[rgb]{1,0,0}\put(0,0){\line( 1,1){1}}}% %DIF PREAMBLE
{\color[rgb]{1,0,0}\put(0,1){\line(1,-1){1}}}% %DIF PREAMBLE
\end{picture}% %DIF PREAMBLE
}\hspace*{3pt}}} %DIF PREAMBLE
} %DIF PREAMBLE
\LetLtxMacro{\DIFOaddbegin}{\DIFaddbegin} %DIF PREAMBLE
\LetLtxMacro{\DIFOaddend}{\DIFaddend} %DIF PREAMBLE
\LetLtxMacro{\DIFOdelbegin}{\DIFdelbegin} %DIF PREAMBLE
\LetLtxMacro{\DIFOdelend}{\DIFdelend} %DIF PREAMBLE
\DeclareRobustCommand{\DIFaddbegin}{\DIFOaddbegin \let\includegraphics\DIFaddincludegraphics} %DIF PREAMBLE
\DeclareRobustCommand{\DIFaddend}{\DIFOaddend \let\includegraphics\DIFOincludegraphics} %DIF PREAMBLE
\DeclareRobustCommand{\DIFdelbegin}{\DIFOdelbegin \let\includegraphics\DIFdelincludegraphics} %DIF PREAMBLE
\DeclareRobustCommand{\DIFdelend}{\DIFOaddend \let\includegraphics\DIFOincludegraphics} %DIF PREAMBLE
\LetLtxMacro{\DIFOaddbeginFL}{\DIFaddbeginFL} %DIF PREAMBLE
\LetLtxMacro{\DIFOaddendFL}{\DIFaddendFL} %DIF PREAMBLE
\LetLtxMacro{\DIFOdelbeginFL}{\DIFdelbeginFL} %DIF PREAMBLE
\LetLtxMacro{\DIFOdelendFL}{\DIFdelendFL} %DIF PREAMBLE
\DeclareRobustCommand{\DIFaddbeginFL}{\DIFOaddbeginFL \let\includegraphics\DIFaddincludegraphics} %DIF PREAMBLE
\DeclareRobustCommand{\DIFaddendFL}{\DIFOaddendFL \let\includegraphics\DIFOincludegraphics} %DIF PREAMBLE
\DeclareRobustCommand{\DIFdelbeginFL}{\DIFOdelbeginFL \let\includegraphics\DIFdelincludegraphics} %DIF PREAMBLE
\DeclareRobustCommand{\DIFdelendFL}{\DIFOaddendFL \let\includegraphics\DIFOincludegraphics} %DIF PREAMBLE
%DIF LISTINGS PREAMBLE %DIF PREAMBLE
\RequirePackage{listings} %DIF PREAMBLE
\RequirePackage{color} %DIF PREAMBLE
\lstdefinelanguage{DIFcode}{ %DIF PREAMBLE
%DIF DIFCODE_CFONT %DIF PREAMBLE
  moredelim=[il][\color{red}\scriptsize]{\%DIF\ <\ }, %DIF PREAMBLE
  moredelim=[il][\color{blue}\sffamily]{\%DIF\ >\ } %DIF PREAMBLE
} %DIF PREAMBLE
\lstdefinestyle{DIFverbatimstyle}{ %DIF PREAMBLE
	language=DIFcode, %DIF PREAMBLE
	basicstyle=\ttfamily, %DIF PREAMBLE
	columns=fullflexible, %DIF PREAMBLE
	keepspaces=true %DIF PREAMBLE
} %DIF PREAMBLE
\lstnewenvironment{DIFverbatim}{\lstset{style=DIFverbatimstyle}}{} %DIF PREAMBLE
\lstnewenvironment{DIFverbatim*}{\lstset{style=DIFverbatimstyle,showspaces=true}}{} %DIF PREAMBLE
%DIF END PREAMBLE EXTENSION ADDED BY LATEXDIFF

\begin{document}



\clearpage{}

\title{ Fibre tract imaging with intraoperative diffusion MRI for neurosurgical navigation }
\author{ Fiona Young }
\department{ Department of Medical Physics and Biomedical Engineering \\ and \\ UCL Great Ormond Street Institute of Child Health }

\maketitle
\makedeclaration

\begin{abstract} Mapping and understanding the brain's structure and function is never more critical than when it suffers injury or illness.
Lifesaving neurosurgical procedures may put essential neural communication pathways called \glsentrylong{wm} tracts at risk, with grave consequences for the patient, so accurately depicting their location using \gls{dmri} is becoming a key component of modern neurosurgical practice.
More recently, obtaining new \gls{imri} partway through surgery has demonstrated potential to further improve outcomes by providing updated anatomical information after the dynamic effects of intraoperative brain shift have diminished the accuracy of preoperative imaging.
With the ability to sample directional water diffusivity in tissue, \gls{dmri} produces millimetre-scale maps of \glsentrylong{wm} fibre orientations which are key to reconstructing individual tracts.
However, established image computational methods suffer from limitations in accuracy and practicality which restrict the wider clinical uptake of \gls{dmri} \glsentrylong{wm} imaging generally, and particularly for \gls{imri}.
After an in depth review of the state of the art in \glsentrylong{wm} imaging and image-guided neurosurgery, this thesis explores the development of a novel \glsentrylong{wm} tract mapping tool, named \textit{tractfinder}, which applies \textit{a priori} anatomical knowledge encoded within a statistical tract orientation and location atlas to achieve rapid tract segmentation in a patient \gls{dmri} scan.
The proposed pipeline includes explicit patient-specific modelling of tumour deformation effects, an element missing from many research-oriented tract reconstruction approaches.
Tractfinder's effectiveness in a range of applications is detailed through thorough quantitative evaluation, while clinical case studies demonstrate its key advantages over existing approaches.
In addition, the technical and practical challenges of intraoperative imaging are explored together with their implications for effective clinical translation of advanced \gls{dmri}-based \glsentrylong{wm} imaging.
\end{abstract}

\begin{impactstatement}
This thesis explores the use of \gls{dmri}'s ability to map brain fibre pathways (``tracts'') in neurosurgical planning and guidance, and in particular the potential for more advanced \gls{dmri} analysis methods than those currently under routine clinical use to improve outcomes for neurology patients.

Though \gls{imri} has gradually been recognised as a beneficial technology for improving neurosurgery, the use of high angular resolution \gls{dmri} specifically has not been widely adopted, partly because the associated potential benefits have yet to be determined.
The methods described in this thesis may make \gls{dmri} analysis more accessible to \gls{imri} procedures, which would allow their advantages to be systematically studied.
Over the longer term, intraoperative \gls{dmri}, if found to be beneficial, may be incorporated into national healthcare guidelines, as preoperative \gls{dti} and general \gls{imri} currently are.
In addition, the proposal of an alternative to streamline tractography, the current standard for \gls{wm} tract imaging, which relies less on the availability of imaging experts, time, and computational resources, could widen the availability of tract imaging for preoperative planning to more health centres, who may have the requisite equipment (\gls{mri} capabilities) but lack expertise in and/or staff availability for applying modern research methods.

The tract orientation atlas framework put forth here could inform the study of \gls{wm} architecture and connectivity in individuals and across populations, an area in which we are still lacking a lot of understanding, and which will be critical in the future study and combatting of psychological and neurological disorders.
It also contributes to current discussions on how to adopt modern data science methods and innovation in a field, medical imaging, where appropriate data is often scarce and difficult to annotate.
Brain shift is an intraoperative phenomenon which presents a key challenge to image guided surgery techniques.
The tumour deformation modelling described in this thesis represents a novel solution to achieving acceptable \gls{wm} tract segmentations on both pre- and intraoperative imaging, in patients with deforming tumours, which may be applicable to other imaging modalities as well as to tumours outside of the brain.

Findings presented in this thesis have been shared with the wider research community at several international meetings, and have led to funding being granted for a two year follow-on prospective study to apply these methods in intraoperative cases at \gls{gosh}, London.
All image processing tools developed for this thesis, including data and code for tumour deformation modelling and tract segmentation, have been made available for researchers to apply in their own data.
In addition, a patent application covering much of the methodology has been submitted and is currently under examination.
If granted, the intellectual property could be licensed to national or international commercial producers of neuronavigational equipment.
\end{impactstatement}

\begin{acknowledgements}
  Instead of living the sad doctoral student cliché, I've had a thoroughly good time working on this project, and I owe much of that enjoyment to having incredibly supportive and insightful supervisors: Professor Jon Clayden, who has kept me on course with dedicated kindness, immeasurable patience, and good humour, Mr Kristian Aquilina, who has given me the most valuable feedback and encouragement, and Professor Chris Clark, who has provided a wealth of knowledge and experience.
  \DIFaddbegin \DIFadd{I also sincerely thank my examiners, Dr Eleftheria (Laura) Panagiotaki and Professor Stephen Price, for their time, constructive and engaging discussions, and for ensuring my doctoral experience wrapped up in the most fulfilling way possible.
}\DIFaddend 

  My thanks go to the present and former staff and researchers at Great Ormond Street Hospital and the Institute for Child Health who have provided invaluable support and assistance, shared their data, and allowed me in small ways to participate in the ongoing development of the iMRI clinical service at GOSH, including Dr Kiran Seunarine, Dr Patrick Hales, Dr Enrico de Vita, and Dr Jan Sedlacik. Thanks as well to Dr Laura Mancini of the National Hospital for Neurology and Neurosurgery, for sharing their data and experiences with intraoperative diffusion imaging, and for improving this project with feedback and insight throughout.

  I'm grateful to the endless community of open source developers, maintainers, and forum members who give their time and talents to create the tools that make conducting and documenting research manageable and fun. Also, writing isn't my favourite activity, and I could hardly have produced this manuscript without the companionable and beautiful works of Hans Zimmer and Nils Frahm.

  I have amazing friends and family who have lifted my spirits from near and far, over walks, shared TV nights, pub quizzes, and voice notes; I love you all. And thank you Barley, for never (as far as I know) asking me when my thesis is due, and only ever asking me for my yoghurt.
\end{acknowledgements}





\DIFdelbegin %DIFDELCMD < \newcommand{\signdate}{16 Jan 2024}
%DIFDELCMD < %%%
\DIFdelend \DIFaddbegin \newcommand{\signdate}{05 Mar 2024}
\DIFaddend {\sffamily\footnotesize\setstretch{1.0}
\markboth{Research paper declaration}{Research paper declaration}
\chapter*{UCL Research Paper Declaration Form: referencing the doctoral candidate’s own published work(s)}
\textit{
	Please use this form to declare if parts of your thesis are already available in another format,
	e.g. if data, text, or figures:
	\begin{itemize}[itemsep=0pt,parsep=0pt]
	\item[--]	have been uploaded to a preprint server
	\item[--]	are in submission to a peer-reviewed publication
	\item[--]	have been published in a peer-reviewed publication, e.g. journal, textbook.
	\end{itemize}
	This form should be completed as many times as necessary. For instance, if a student had seven
	thesis chapters, two of which having material which had been published, they would complete this form twice.
}

\section*{Young, F. \textit{et al.}, International Journal of Computer Assisted Radiology and Surgery (2022)}
	\begin{enumerate}[leftmargin=*,label={\bfseries\arabic*.}]\itemsep0em
		\item \textbf{For a research manuscript that has already been published} (if not yet published, please skip to section 2)\textbf{:}
\begin{enumerate}[label={\alph*)}]\itemsep0em
\item \textbf{What is the title of the manuscript?}

		\citetitle{Young2022}

		\item \textbf{Please include a link to or doi for the work:}

		\url{https://doi.org/10.1007/s11548-022-02617-z}

		\item \textbf{Where was the work published?}

		\citefield{Young2022}{journaltitle}
		\item \textbf{Who published the work?}

		\citelist{Young2022}{publisher}
		\item \textbf{When was the work published?}

		\citefield{Young2022}{year}
		\item \textbf{List the manuscript's authors in the order they appear on the publication:}
		\DIFdelbegin %DIFDELCMD < \citeauthor*{Young2022}
%DIFDELCMD < 		%%%
\DIFdelend \DIFaddbegin \citeauthorfullname*{Young2022}
		\DIFaddend \item \textbf{Was the work peer reviewd?}

		Yes

		\item \textbf{Have you retained the copyright?}

		Yes, from the Licence to Publish agreement:
		\begin{quote}
			Ownership of copyright in the Article will be vested in the name of the Author.
		\end{quote}
		\item \textbf{Was an earlier form of the manuscript uploaded to a preprint server (e.g. medRxiv)? If `Yes’, please give a link or doi}

		No\\
		If ‘No’, please seek permission from the relevant publisher and check the box next to the below statement:
\begin{itemize}\itemsep0em
\item[$\boxtimes$] {\itshape I acknowledge permission of the publisher named under 1d to include in this thesis portions of the publication named as included in 1c.}
			\end{itemize}
\end{enumerate}
\item \textbf{For a research manuscript prepared for publication but that has not yet been published} (if already published, please skip to section 3)\textbf{:}
\begin{enumerate}[label={\alph*)}]\itemsep0em
\item \textbf{What is the current title of the manuscript?}
\item \textbf{Has the manuscript been uploaded to a preprint server `e.g. medRxiv'?
		\\
		If `Yes', please please give a link or doi:}
\item \textbf{Where is the work intended to be published?}
\item \textbf{List the manuscript's authors in the intended authorship order:}
\item \textbf{Stage of publication:}
\end{enumerate}

	\item \textbf{For multi-authored work, please give a statement of contribution covering all authors} (if single-author, please skip to section 4)\textbf{:}
	\begin{description}[font=\sffamily]
		\item[Fiona Young] Methodology (conceptualisation and implementation), analysis, manuscript original draft and graphics
		\item[Kristian Aquilina] Supervision, manuscript review and editing
		\item[Chris A. Clark] Supervision, manuscript review and editing
		\item[Jonathan D. Clayden] Conceptualisation, supervision, manuscript review and editing
	\end{description}
\item \textbf{In which chapter(s) of your thesis can this material be found?}

	Chapters \ref{chap:reg}, \ref{chap:applications}

	\end{enumerate}

	\textbf{e-Signatures confirming that the information above is accurate}
	(this form should be co-signed by the supervisor/ senior author unless this is not appropriate, e.g. if the paper was a single-author work)\textbf{:}\\
	\\[\baselineskip]
	\textit{Candidate:}
	\\[\baselineskip]
	\textit{Date:}\\\signdate
\\[\baselineskip]
	\textit{Supervisor/Senior Author signature} (where appropriate)\textbf{:}
	\\[\baselineskip]
	\textit{Date:}\\\signdate


\section*{Young, F. \textit{et al.}, Human Brain Mapping (\DIFdelbegin \DIFdel{in press}\DIFdelend \DIFaddbegin \DIFadd{2024}\DIFaddend )}
	\begin{enumerate}[leftmargin=*,label={\bfseries\arabic*.}]\itemsep0em
\item \textbf{For a research manuscript that has already been published} (if not yet published, please skip to section 2)\textbf{:}
\begin{enumerate}[label={\alph*)}]\itemsep0em
\item \textbf{What is the title of the manuscript?}

		\citetitle{Young2024}

		\item \textbf{Please include a link to or doi for the work:}

		\citefield{Young2024}{doi}

		\item \textbf{Where was the work published?}

		\citefield{Young2024}{journaltitle}
		\item \textbf{Who published the work?}

		\citelist{Young2024}{publisher}
		\item \textbf{When was the work published?}

		\citefield{Young2024}{year}
		\item \textbf{List the manuscript's authors in the order they appear on the publication:}

		\DIFdelbegin %DIFDELCMD < \citeauthor*{Young2024}
%DIFDELCMD < 		%%%
\DIFdelend \DIFaddbegin \citeauthorfullname*{Young2024}
		\DIFaddend \item \textbf{Was the work peer reviewd?}

		Yes

		\item \textbf{Have you retained the copyright?}

		Yes, from license agreement:
		\begin{quote}
			The Author and each Co-author or, if applicable, the Author’s or Co-author’s employer, retains all proprietary rights, such as copyright
		\end{quote}

		\item \textbf{Was an earlier form of the manuscript uploaded to a preprint server (e.g. medRxiv)? If `Yes’, please give a link or doi}

		No\\
		If ‘No’, please seek permission from the relevant publisher and check the box next to the below statement:
\begin{itemize}\itemsep0em
\item[$\boxtimes$] {\itshape I acknowledge permission of the publisher named under 1d to include in this thesis portions of the publication named as included in 1c.}
		\end{itemize}
\end{enumerate}
\item \textbf{For a research manuscript prepared for publication but that has not yet been published} (if already published, please skip to section 3)\textbf{:}
\begin{enumerate}[label={\alph*)}]\itemsep0em
\item \textbf{What is the current title of the manuscript?}
\item \textbf{Has the manuscript been uploaded to a preprint server `e.g. medRxiv'?
		\\
		If `Yes', please please give a link or doi:}
\item \textbf{Where is the work intended to be published?}
\item \textbf{List the manuscript's authors in the intended authorship order:}
\item \textbf{Stage of publication:}
\end{enumerate}

	\item \textbf{For multi-authored work, please give a statement of contribution covering all authors} (if single-author, please skip to section 4)\textbf{:}
	\begin{description}[font=\sffamily]
		\item[Fiona Young] Methodology (conceptualisation and implementation), analysis, manuscript original draft and graphics
		\item[Kristian Aquilina] Supervision, conceptualisation, manuscript review and editing
		\item[Laura Mancini] Data contribution, manuscript review and editing
		\item[Kiran K. Seunarine] Data contribution and curation
		\item[Chris A. Clark] Supervision, manuscript review and editing
		\item[Jonathan D. Clayden] Conceptualisation, supervision, manuscript review and editing
	\end{description}
\item \textbf{In which chapter(s) of your thesis can this material be found?}

	Chapters \ref{chap:neuroimaging}, \ref{chap:atlas}, \ref{chap:reg} (Section \ref{sec:reg1}), \ref{chap:eval}, Appendix
	\end{enumerate}

	\textbf{e-Signatures confirming that the information above is accurate}
	(this form should be co-signed by the supervisor/ senior author unless this is not appropriate, e.g. if the paper was a single-author work)\textbf{:}\\
	\\[\baselineskip]
	\textit{Candidate:}
	\\[\baselineskip]
	\textit{Date:}\\\signdate
\\[\baselineskip]
	\textit{Supervisor/Senior Author signature} (where appropriate)\textbf{:}
	\\[\baselineskip]
	\textit{Date:}\\\signdate

\section*{Young, F. \textit{et al.}, IEEE EMBS Data Science (conference abstract, 2023)}
\begin{enumerate}[leftmargin=*,label={\bfseries\arabic*.}]\itemsep0em
\item \textbf{For a research manuscript that has already been published} (if not yet published, please skip to section 2)\textbf{:}
\begin{enumerate}[label={\alph*)}]\itemsep0em
\item \textbf{What is the title of the manuscript?}

	\item \textbf{Please include a link to or doi for the work:}

	\item \textbf{Where was the work published?}

	\item \textbf{Who published the work?}

	\item \textbf{When was the work published?}

	\item \textbf{List the manuscript's authors in the order they appear on the publication:}

	\item \textbf{Was the work peer reviewd?}

	\item \textbf{Have you retained the copyright?}

	\item \textbf{Was an earlier form of the manuscript uploaded to a preprint server (e.g. medRxiv)? If ‘Yes’, please give a link or doi}
	\\
	If ‘No’, please seek permission from the relevant publisher and check the box next to the below statement:
\begin{itemize}\itemsep0em
\item[$\Box$] {\itshape I acknowledge permission of the publisher named under 1d to include in this thesis portions of the publication named as included in 1c.}
\end{itemize}
\end{enumerate}
\item \textbf{For a research manuscript prepared for publication but that has not yet been published} (if already published, please skip to section 3)\textbf{:}
\begin{enumerate}[label={\alph*)}]\itemsep0em
\item \textbf{What is the current title of the manuscript?}

	\citetitle{Young2023}
	\item \textbf{Has the manuscript been uploaded to a preprint server `e.g. medRxiv'?
	\\
	If `Yes', please please give a link or doi:}

	Yes, \url{https://www.researchgate.net/publication/375601134_Training_Data_Requirements_for_Atlas-Based_Brain_Fibre_Tract_Identification}
\item \textbf{Where is the work intended to be published?}

	Proceedings of the \citefield{Young2023}{eventtitle}
	\item \textbf{List the manuscript's authors in the intended authorship order:}

	\DIFdelbegin %DIFDELCMD < \citeauthor*{Young2023}
%DIFDELCMD < 	%%%
\DIFdelend \DIFaddbegin \citeauthorfullname*{Young2023}
	\DIFaddend \item \textbf{Stage of publication:}

	Work presented at conference, copyright transferred to IEEE, yet now indication of if/when proceedings may be published (copyright on submitted version retained).
\end{enumerate}

\item \textbf{For multi-authored work, please give a statement of contribution covering all authors} (if single-author, please skip to section 4)\textbf{:}
\begin{description}[font=\sffamily]
	\item[Fiona Young] Methodology (conceptualisation and implementation), analysis, manuscript original draft and graphics
	\item[Kristian Aquilina] Supervision
	\item[Chris A. Clark] Supervision
	\item[Jonathan D. Clayden] Conceptualisation, supervision, manuscript review and editing
\end{description}
\item \textbf{In which chapter(s) of your thesis can this material be found?}

Chapter \ref{chap:atlas}
\end{enumerate}

\textbf{e-Signatures confirming that the information above is accurate}
(this form should be co-signed by the supervisor/ senior author unless this is not appropriate, e.g. if the paper was a single-author work)\textbf{:}\\
\\[\baselineskip]
\textit{Candidate:}
\\[\baselineskip]
\textit{Date:}\\\signdate
\\[\baselineskip]
\textit{Supervisor/Senior Author signature} (where appropriate)\textbf{:}
\\[\baselineskip]
\textit{Date:}\\\signdate

\section*{Young, F. \textit{et al.}, ISMRM Wokshop on Diffusion MRI (conference abstract, 2022)}
\begin{enumerate}[leftmargin=*,label={\bfseries\arabic*.}]\itemsep0em
\item \textbf{For a research manuscript that has already been published} (if not yet published, please skip to section 2)\textbf{:}
\begin{enumerate}[label={\alph*)}]\itemsep0em
\item \textbf{What is the title of the manuscript?}

	\item \textbf{Please include a link to or doi for the work:}

	\item \textbf{Where was the work published?}

	\item \textbf{Who published the work?}

	\item \textbf{When was the work published?}

	\item \textbf{List the manuscript's authors in the order they appear on the publication:}

	\item \textbf{Was the work peer reviewd?}

	\item \textbf{Have you retained the copyright?}

	\item \textbf{Was an earlier form of the manuscript uploaded to a preprint server (e.g. medRxiv)? If ‘Yes’, please give a link or doi}
	\\
	If ‘No’, please seek permission from the relevant publisher and check the box next to the below statement:
\begin{itemize}\itemsep0em
\item[$\Box$] {\itshape I acknowledge permission of the publisher named under 1d to include in this thesis portions of the publication named as included in 1c.}
\end{itemize}
\end{enumerate}
\item \textbf{For a research manuscript prepared for publication but that has not yet been published} (if already published, please skip to section 3)\textbf{:}
\begin{enumerate}[label={\alph*)}]\itemsep0em
\item \textbf{What is the current title of the manuscript?}

	\citetitle{Young2022a}
	\item \textbf{Has the manuscript been uploaded to a preprint server `e.g. medRxiv'?
	\\
	If `Yes', please please give a link or doi:}

	Yes, \url{https://www.researchgate.net/publication/367116849_Stability_of_white_matter_tract_segmentation_methods_with_decreasing_data_quality}
\item \textbf{Where is the work intended to be published?}

	N/A
	\item \textbf{List the manuscript's authors in the intended authorship order:}

	\DIFdelbegin %DIFDELCMD < \citeauthor*{Young2022a}
%DIFDELCMD < 	%%%
\DIFdelend \DIFaddbegin \citeauthorfullname*{Young2022a}
	\DIFaddend \item \textbf{Stage of publication:}

	Work presented at conference, no proceedings published.
\end{enumerate}

\item \textbf{For multi-authored work, please give a statement of contribution covering all authors} (if single-author, please skip to section 4)\textbf{:}
\begin{description}[font=\sffamily]
	\item[Fiona Young] Methodology (conceptualisation and implementation), analysis, manuscript original draft and graphics
	\item[Jonathan D. Clayden] Conceptualisation, supervision, manuscript review and editing
\end{description}
\item \textbf{In which chapter(s) of your thesis can this material be found?}

Chapter \ref{chap:applications}
\end{enumerate}

\textbf{e-Signatures confirming that the information above is accurate}
(this form should be co-signed by the supervisor/ senior author unless this is not appropriate, e.g. if the paper was a single-author work)\textbf{:}\\
\\[\baselineskip]
\textit{Candidate:}
\\[\baselineskip]
\textit{Date:}\\\signdate
\\[\baselineskip]
\textit{Supervisor/Senior Author signature} (where appropriate)\textbf{:}
\\[\baselineskip]
\textit{Date:}\\\signdate
} 
\tableofcontents
\setcounter{tocdepth}{1}
\listoffigures
\listoftables
{\setstretch{1.0}
\setlength{\glsdescwidth}{0.8\linewidth}
\printglossary[type=\acronymtype,title={List of Abbreviations}]
}
\clearpage{}

\epipage{I'll tell you where the real road lies\\
Between your ears, behind your eyes}{Anais Mitchell, \textit{Wait For Me (Reprise)}}
\clearpage{}\DIFaddbegin \markboth{Introduction}{Introduction}
\DIFaddend \chapter*{Introduction}\addcontentsline{toc}{chapter}{Introduction}
\label{chapterlabel0}

Few objects of our scientific pursuits rival the human brain in mystery and complexity.
Understanding its structure and function has been a centuries-long undertaking, yet we are still far from forming a complete picture, all while the scientific and medical needs for such knowledge grow more pressing.
Much of the world is caught in a mental health crisis\autocite{Patel2018,Liu2020,Yang2021a}, in ageing populations the burden of neurodegenerative diseases is accelerating\autocite{Deuschl2020,Li2022}, after leukaemia, children's cancers arise most frequently in the brain and spinal cord\autocite{Ostrom2015}.

For many years, study of the brain was limited to postmortem dissections, or observing patients who had suffered life-altering brain injuries.
An entirely new medium for peering into the brain came with the development of \gls{mri} in the latter 20th century, which has since become a cornerstone technology of modern \textit{in vivo} neuroscience.
\Gls{mri} has also gained indispensable importance in clinical neuroimaging, wherever it is available\autocite{Geethanath2019}, with applications including the ability to map brain function and connectivity in pathology.

In \textit{A Manual of Diseases of the Nervous System}, published in 1888, pioneering neurologist Sir William Gowers wrote of brain tumours:
``The treatment of new growths [...] is always a sufficiently gloomy subject, and not least so when they are seated in an organ like the brain, in which they cause peculiar and varied suffering, and in which their development, even to a moderate degree, is rarely compatible with life.''\autocite{Gowers1888}
He goes on to note the sobering risks and almost unavoidable neural sequelae accompanying any attempt at surgical removal.
Today, surgical treatment of brain tumours is routine, brings great benefits to quality of life and prognosis, and can even be curative.
Even so, a brain tumour diagnosis remains a devastating prospect, and the outlook and survival rates for malignant brain tumours remain somber\autocite{Aldape2019}.
We wield countless weapons in the campaign against brain cancer, from temozolomide chemotherapy, gamma knife surgery, and 5-aminolevulinic acid (5-ALA) fluorescence guidance, to proton beam therapy, genomics, and focussed ultrasound, all while pre-clinical research continues searching for a better understanding of how brain cancers grow and how they can be stopped.
Among this ensemble cast, \gls{mri}-guided treatment may play a small but nonetheless significant role in improving brain cancer care globally.
Its clinical impact may be greater still, if technological limitations in the acquisition and processing of advanced high-dimensional \gls{mri} data can be overcome.

An \gls{mri} modality called \gls{dmri} is able to give insights into the structural organisation and connectivity of neural fibres within the brain.
The pioneering technique tractography, which uses \gls{dmri} data to trace fibre pathways, has opened up the study of \textit{in vivo} whole-brain network neuroscience and found use as a clinical tool for planning delicate neurosurgical procedures involving critical brain structures.
It does, however, suffer from multiple limitations and requires significant resources and expertise, which aren't as widely available as conventional \gls{mri} capabilities\autocite{GeorgeZakiGhali2020}.
In 2020, \gls[noindex=false]{gosh} in London unveiled a new \gls{mri} system for intraoperative imaging, heralding new possibilities for more precise neurosurgery and better patient outcomes.
\gls{dmri} has the potential to play a part in advanced intraoperative imaging, if image processing techniques can be developed that are compatible with the stringent efficiency and technical requirements.

This research project was motivated by the need for a brain fibre mapping technique that could be applied to intraoperative \gls{dmri} images, or in any other environment where resources are limited.
Starting from the premise that we already have, in most cases, a lot of prior information about the shape and location of brain pathways, a technique will be described in the following pages for directly estimating the location of a fibre bundle in healthy and clinical \gls{dmri} scans.
The proposed technique's benefits and limitations are evaluated in the context of the current state-of-the art in neuroimaging and neurosurgery, providing a comprehensive overview and future outlook for the role of advanced \gls{dmri} in improving the outcomes of neurosurgical interventions.
\clearpage{}


\epart[\epigraph{Too much knowledge never makes for simple decisions.}{Frank Herbert, \textit{Children of Dune}}]{Background}
\clearpage{}\chapter{Neuroanatomy}
\label{chap:neuroanatomy}

\section{Cells of the brain}

The building blocks of living complex organisms begin at the level of molecules and atoms, through macromolecules such as proteins and lipids, organelles, cells, tissues, organs and finally whole organisms.
For the purposes of this thesis, we can start at the cellular level and focus on a single organ:
The brain, which, together with the spinal cord, forms the \gls[noindex=false]{cns}.
Brain tissues comprise numerous cell types.
The principal functional cell of the human nervous system is the neuron, which performs the integration and transmission of electrical signals underpinning all aspects of neural function.
It wasn't until the work of Santiago Ramón y Cajal (1852--1934), utilising the silver staining technique developed by Camillo Golgi (1843--1926) to meticulously draw and label neural cells, that the nervous system was recognised as comprising discrete signalling units which interface with one another only at specific points, a principle called the \textit{neuron doctrine}\autocite{Kandel2021a}.
The brain's approximately 80 billion neurons are supported by a network of glial cells of equal number and importance\autocite{Herculano-Houzel2014,vonBartheld2016}, and of which there are several specialised subtypes.

Neurons fulfil a wide spectrum of specialised functions across the nervous system, but all have the following basic components.
The cell body, or soma, houses the cell nucleus, metabolic activity and organelles responsible for the day-to-day upkeep and maintenance of the entire cell (Fig. \ref{fig:neuron}).
Neurons receive signal inputs from other neurons via one or multiple protrusions called dendrites.
Most neurons of the \gls{cns} possess a large number of dendritic processes, forming a beautiful dendritic tree collecting countless inputs.
Output signals are sent via a single, specialised protrusion called an axon, which can extend across great distances to deliver impulses to their targets.
Axon terminals interface with the dendrites of other neurons or, in the \gls[noindex=false]{pns}, with muscles and other innervated tissues, at points of contact called synapses.
The speed of signal conduction in some axons, including those that travel great distances, is enhanced through sheaths of a fatty substance called myelin which provide electrical insulation.

Glial cells (from Greek \textit{glia}, ``glue'') are the other major class of nervous system residents, and though long thought to outnumber neurons up to ten fold, newer evidence indicates that they are more or less equally numerous as neurons throughout the brain as a whole, although local ratios vary greatly\autocite{Herculano-Houzel2014}.
Astrocytes are supportive glial cells providing a range of services including phagocytosis, scar tissue formation and \gls[noindex=false]{bbb} formation, the latter being a protective layer around neural vasculature selectively regulating molecular exchange between the bloodstream and brain parenchyma.
Oligodendrocytes are the \gls{cns}'s myelinating glia.
They form processes which wrap around axons as myelin sheaths, with each oligodendrocyte able to myelinate multiple axons simultaneously.
In the \gls{pns}, myelination is performed by Schwann cells, which each form a single myelin sheath.
Alongside these three classes of macroglia are the microglia, the resident immunological cells of the \gls{cns}, which patrol the brain for pathogens and clean up waste.

\begin{figure}[hbt!]
  \centering
  \includesvg[width=0.7\textwidth,pretex=\ttfamily\small]{chapter_1/neuron.svg}
  \caption[Structure of a neuron]{A neuron receives inputs across its dendritic tree (d) which converge on the cell body (c). Action potentials propagate down the axon (a) to the axon terminals (t) where they trigger chemical transmission of the signal at the synapses (s). Electrical conduction is accelerated by insulating myelin sheaths (m) separated by the nodes of Ranvier (n).}
  \label{fig:neuron}
\end{figure}

\section{Neural signalling}

Transmission of information in the brain is both electrical and chemical.
Neural impulses begin their life at the axon hillock, the transition between the soma and the axon.
At resting state, an electrical potential exists across the axonal membrane, resulting from separation of charges between the extra- and intra-axonal spaces.
This membrane potential arises from non-equal concentrations of different ionic species on opposite sides of the membrane, and is maintained by a combination of active ion pumps and passive ion channels embedded within the membrane, through which potassium (K$^+$), sodium (Na$^+$), and chloride (Cl$^-$) ions continuously diffuse or are actively transported across the membrane.
The interacting chemical and electrical potentials of primarily these ions, together with the concentration of ion channels selective for each species, give rise to a cell's resting potential, a steady net separation in charge maintained by a balanced flux of ions.
All neural signalling is both initiated and propagated by disturbances to this resting potential, when large changes to the membrane's selective permeability cause significant fluxes of ions to change the net charge separation.
When the membrane potential at the axon hillock is depolarised enough to reach a threshold, voltage-gated ion channels open to allow ions to flood across the membrane.

Each ion species experiences an interplay of the overall charge potential and its own chemical gradient:
At rest, Na$^+$ concentration is much higher outside the cell, while K$^+$ concentration is higher inside the cell, with those chemical gradients maintained by the relative membrane permeability to each species.
At the beginning of an axonal impulse the opening of voltage-sensitive sodium ion channels suddenly increases the membrane permeability to Na$^+$, allowing it to flow into the cell down its concentration gradient and further depolarise the membrane.
As nearby voltage-gated channels are triggered in turn, a wave of depolarisation propagates down the axon as an action potential.
After a brief delay, voltage-gated K$^+$ also activate, allowing potassium to repolarise the membrane by flowing down its own chemical gradient, out of the cell.
This rapid sequence of de- and repolarisation lasts only milliseconds, after which the membrane is again susceptible to excitation.
In myelinated axons, propagation of the action potential is greatly accelerated by the insulating effects of myelin sheaths, separated only by the narrow myelin-free nodes of Ranvier.
Myelin acts to decrease the membrane capacitance and thus increase conduction velocity down the axon, such that, in myelinated axons, action potentials travel at speeds of up to 120~m~s$^{-1}$ depending on axon diameter, compared with up to 2~m~s$^{-1}$ in unmyelinated axons.

Upon reaching the axon terminal and synapse, the signal is converted from electrical to chemical.
The arrival of the action potential triggers the release of signalling molecules called neurotransmitters, which diffuse across the synaptic cleft and ligase with receptors in the postsynaptic cell to influence its membrane potential.
The effect on the target neuron depends on the neurotransmitter:
It may induce hyperpolarisation, in effect an inhibitory signal which decreases the likelihood of the postsynaptic cell firing, or depolarisation, which brings the target cell closer to firing threshold.
Axons may branch to synapse with many targets, and each neuron may itself receive thousands of synaptic inputs across its dendritic tree.
Each impulse causes a slight de- or hyperpolarisation of the membrane, which diffuses towards the soma and axon hillock.
It's the cumulative effects of all those impulses that determine whether a neuron fires, transmitting a new action potential down its axon.

\section{Macroscopic neuroanatomy}

The human brain is divided macroscopically into the hind-, mid- and forebrain, which are densely interconnected via a system of axon fibre bundles (``tracts'').
With reference to a particular brain structure or functional hub, incoming connections from other parts of the brain or spinal cord are called \textit{afferent}, compared to outgoing \textit{efferent} fibres.

The hind brain comprises the caudal brainstem (medulla and pons) and cerebellum (Figs. \ref{fig:anat1}--\ref{fig:ardnoldVII1}).
The former forms a continuum between the forebrain to the spinal cord and houses control centres for vital, involuntary physiological functions and the cranial nerves.
Posterior to the brainstem, and seated within the posterior fossa, sits the cerebellum (``little brain''),
a highly specialised and complex structure whose full significance to cerebral function is yet to be fully described, but which is known to play an integral role in automated motor and speech functions.
The cerebellum connects with the pons and brainstem via three fibre tracts, the inferior, middle and superior cerebellar peduncles, of which the middle is the largest.
The pons also houses a dense arrangement of nuclei and tracts, connecting and integrating functions of the forebrain and cerebellum.

\begin{SCfigure}[][h!]
  \includesvg[width=0.6\textwidth,pretex=\sffamily\footnotesize]{chapter_1/anat_gross.svg}
  \caption[Macroscale subdivisions of the human brain]{Macroscale subdivisions of the human brain. Adapted from original by \citeauthor{Carter1918} (\citeyear{Carter1918}), public domain, via Wikimedia Commons.}
  \label{fig:anat1}
\end{SCfigure}

The forebrain consists of the diencephalon (thalamus and hypothalamus) and the telencephalon (cerebral hemispheres).
Central both in importance and in location, the thalamus forms the gateway through which almost all afferent inputs to the forebrain are channeled.
Its cluster of nuclei relay sensory inputs to their relevant cortical regions, and project diffusely to all regions of the cortex to regulate, it is thought, widespread arousal and attention, and form a feedback loop for sensory and motor integration.
Ventral to the thalamus, in the suprasellar region, sits the hypothalamus, a centre for \DIFdelbegin \DIFdel{endocrine }\DIFdelend \DIFaddbegin \DIFadd{neuroendocrine }\DIFaddend and metabolic regulation.

Evolutionarily speaking, the cerebral hemispheres are the youngest developments, responsible for much of the complex neural functionality we associate with the general intelligence of modern humans, as well as memory, sensorimotor control, and personality.
The surface of the cerebral hemispheres, called the cortex, is the site of integration of neural signals and complex neural computation, and is divided into the frontal, parietal, occipital and temporal lobes (Fig. \ref{fig:anat1}).
Cortical tissue consists of dense layers of neural somata, dendrites, interneurons, all locally connected by unmyelinated axons, forming a tissue called ``grey matter" (\glsunset{gm}\gls[noindex=false]{gm}).
So extensive is the cortical surface in the human brain that, in order to fit inside the skull, it develops a folded topography of ridges (gyri) and valleys (sulci) which form distinctive landmarks on its surface which are recognisable between individuals.

At the heart of the brain flows a network of ventricles filled with the protective and supportive \gls[noindex=false]{csf}, which surrounds the \gls{cns} to provide cushioning and maintain homeostasis\autocite{Wichmann2022}.
Two lateral ventricles curve through each hemisphere, connecting with the third ventricle separating the thalami at the midline, which in turn adjoins the fourth ventricle, a triangular cavity between the midbrain and cerebellum.
From there, \gls{csf} flows into the spaces surrounding the brain and spinal cord.

\begin{figure}[h!]
  \begin{subfigure}[b]{0.5\textwidth}
    \includesvg[width=\textwidth,pretex=\ttfamily\small]{chapter_1/arnold1838bd1_0040_TABVII.svg}
    \caption{\citeauthor{Arnold1838} (\citeyear{Arnold1838})\autocite{Arnold1838}, Tabula VII, Fig. 1}\label{fig:ardnoldVII1}
  \end{subfigure}\begin{subfigure}[b]{0.5\textwidth}
    \includesvg[width=\textwidth,pretex=\ttfamily\small]{chapter_1/arnold1838bd1_0046_TABX.svg}
    \caption{\citeauthor{Arnold1838} (\citeyear{Arnold1838})\autocite{Arnold1838}, Tabula X, Fig. 4}\label{fig:ardnoldX4}
  \end{subfigure}
  \caption[Anatomical landmarks and subcortical white matter of the human brain]{Anatomical landmarks and subcortical white matter of the human brain, labelled on illustrations by Friedrich Arnold\autocite{Arnold1838}. a. spinal cord, b. medulla, c. pons, d. midbrain, e. cerebellum, f. hypothalamus, g. thalamus, h. lateral ventricle, j. fourth ventricle, k. subcortical white matter, l. cortical grey matter, m. corpus callosum, F. frontal lobe, P. parietal lobe, O. occipital lobe, T. temporal lobe, S. central sulcus}
  \label{fig:anat2}
\end{figure}


\section{Functional organisation: From topology to hodology}\label{sec:hodology}

Since antiquity scientists have suspected that different cognitive functions, rather than being equally distributed throughout the brain, arise from distinct locations.
While early incarnations of this idea may have featured wildly unfounded and mistaken conclusions regarding the functions that can be ascribed to specific localities, the underlying principle of functional localisation has been subsequently borne out and continuously refined.

The earliest theories focussed on the ventricles as the centres of cognition, before attention shifted to the cortex\autocite{Folzenlogen2019}.
Physicians of the 18th and 19th centuries popularised the study of bumps and dips on the skull as a means for assessing myriad character traits and capabilities, a pseudoscience known as phrenology.
Significant advances in the theory of functional localisation came with the seminal studies of aphasia, or disorders of language function, undertaken by Pierre Paul Broca (1824--1880) and Carl Wernicke (1848--1905).
Each gave their names to regions of the frontal and temporal cortex respectively, after consistently linking specific patterns of aphasia with lesions caused by stroke or injury to those regions.
Historically, much of the understanding of functions seated in a given cortical area was in this way predicated on case studies involving injury (e.g. through trauma or stroke) to that region.
This resulted in \DIFaddbegin \DIFadd{an }\DIFaddend inevitably imperfect understanding of the affected areas and functions, given that such random injuries are unlikely to precisely hit one isolated functional centre.
Modern non-invasive imaging and measuring techniques have accelerated and refined our study of the brain, however clinical case studies remain significant contributors to the field.

Some cortical regions have been studied and mapped extensively, particularly those associated with easily measured quantities such as external stimuli and motor outputs.
Consequently, the organisation of the visual, motor, auditory and somatosensory cortices have been comprehensively described.
The precise localisation of higher-level functions, on the other hand, such as advanced language, memory, planning, and personality remains elusive, not least because such complex multifaceted phenomena are unlikely to be associated with just a single structural-functional unit\autocite{Catani2007}.
Indeed, modern neuroscience understands that even seemingly primitive functions, such as visual object identification, involve the integration of information and processing across multiple, often distant cortical loci\autocite{ffytche2005}, and thus there can be no complete understanding of any cognitive function by considering only demarcated cortical areas in isolation:
The brain is a network, and the unit of connection between nodes is the neural axon.
\DIFaddbegin 

\DIFaddend Medium and long range connecting axons project from the cortex into a dense subcortical network, in aggregate forming a substance named ``white matter'' (\glsunset{wm}\gls[noindex=false]{wm}) after the pale appearance given by the abundance of insulating myelin.
\Gls{wm} axons, rather than existing in an unordered tangle, form coherent bundles, or tracts, of fibres connecting similar brain regions (Fig. \ref{fig:ardnoldX4}).
There are three categories of \gls{wm} tracts:
Projection tracts connect the cortex with other, non-cortical brain structures and the spinal cord.
They include both efferent (motor) and afferent (sensory) connections.
Association tracts connect different cortical regions within the same cerebral hemisphere, while commissural tracts form interhemispheric connections.
The major commissural bundle is the corpus callosum, forming a thick band bridging the largely separated cerebral hemispheres (Fig. \ref{fig:anat2}).

While the division of \gls{wm} into individually described tracts introduces a certain level of clarity to the intricate brain network and a common ground from which to study it, many of their anatomical and functional definitions are far from clear-cut.
As we saw in the context of cortical functional localisation, the projection tracts, which have clear physiological correlates of sensory input and motor outputs, are relatively easy to study and have been correspondingly well defined.
Examples include the corticospinal tract, responsible for much of voluntary motor control, and the optic radiations, which bring visual inputs to the primary visual cortex in the occipital lobes, although as we will see in Chapter \ref{chap:atlas} even these structures' exact courses can be called into question.

Meanwhile, the task of precisely delineating and ascribing functional correlates to the many association tracts is monumental and woefully incomplete.
Fibres from different tracts cross, mingle, and converge throughout the brain, and while fibres of the same tract may form a coherent bundle in deep \gls{wm}, they proceed to diverge and fan out, terminating in broad swathes of cortex.
In this way, fibre tracts form fractal echos of their individual neuron constituents, drawing on many inputs to disseminate the output signals across countless targets.
There is no perfect method for observing the full extent of a tract from one end to another.
Postmortem dissection remains the gold standard for anatomical identification, but it is delicate work and disentangling the full course of one bundle of fibres from all other surrounding ones is impossible.
So it is that anatomical dissections have at times identified and defined tracts which turned out to be entirely artifactual.
In Chapter \ref{chap:atlas} we will further explore difficulties of defining \gls{wm} tracts.

Neuroscientists, anatomists, and neurologists have increasingly come to recognise the importance of looking beyond the purely localisationist models that focus on cortico-functional organisation, towards the integration of \gls{wm} connectivity into our understanding of cognitive function\autocite{ffytche2005,Catani2007}.
After all, the axons in \gls{wm} are extensions of the cells that make up \gls{gm}, so there can be no activity of the latter that does not involve the former.
\DIFdelbegin \DIFdel{Indeed, }\DIFdelend \DIFaddbegin \DIFadd{The study of white matter connectivity, termed ``hodology" (from Greek \textit{hodos} ``path'') and more recently ``connectomics", }\DIFaddend far from being a mere philosophical question about under what framework we study and write about the brain, \DIFdelbegin \DIFdel{the so-called hodological view (from Greek \textit{hodos} ``path'') }\DIFdelend has profound translational implications.
Traditional localisationist approaches to neurosurgery hold, crudely put, that tumours seated within or in close association to ``eloquent'' areas of cortex subserving well studied functions of language, low-level sensory processing and movement, are fundamentally inoperable.
Meanwhile tumours in areas of ``non-eloquent'' cortex with no known overt functional correlates can be radically resected with few substantial neurological consequences.
Much of the study into and tools for functional preservation of so-called eloquent areas that will be discussed in Chapter \ref{chap:neurosurgery} adopts this framing.
But the postoperative deterioration in neurocognitive function and quality of life often experienced by patients undergoing surgery for epilepsy or tumours in ``non-eloquent'' areas flies in the face of such thinking
\autocite{Satoer2017,Mandonnet2017a,Rijnen2019,Herbet2019,Vigren2020a,Dadario2021}.
Furthermore, emerging evidence suggests adopting a hodological approach to neurosurgery can inform successful resection of tumours from areas typically considered inoperable\autocite{Dadario2021,DeBenedictis2011b,Suzuki2023}.

As frameworks of neural function continue to evolve alongside our technological means to study it, we find ourselves reaching across the disciplinary divides for tools and ideas with which to advance our understanding of the brain.
Modern neuroscience is a blend of cognitive, systems, computational, psychological, anatomical and molecular domains, each influencing and adding context to the others.
As such, while the methodologies and experiments presented in the thesis are rooted in \textit{in vivo} structural imaging of individual \gls{wm} tracts, questions around functional tract definitions and surgical relevance will recur throughout.

\DIFaddbegin \vfill
\DIFaddend \begin{figure}[hb!]
  \centering
  \includegraphics[width=0.8\textwidth]{chapter_1/xkcd_1163_debugger.png}
  \DIFdelbeginFL %DIFDELCMD < \caption*{\href{xkcd \#1163: ``Debugger''}{https://xkcd.com/1163/}}
%DIFDELCMD <   %%%
\DIFdelendFL \DIFaddbeginFL \caption*{\href{https://xkcd.com/1163/}{xkcd \#1163: ``Debugger''}}
  \DIFaddendFL \label{fig:xkcd}
\end{figure}
\DIFaddbegin \vfill
\DIFaddend 

\chapter{MRI physics}
\label{chap:NMR}

The study of living human neuroanatomy at the millimetre and macro scale in unprecedented detail was unlocked in the latter 20th century by the advent of \gls[noindex=false]{mri} techniques.
To understand the workings of the ``Wonder Machine'' (in the words of podcasters Josh Clark and Charles W. ``Chuck'' Bryant), we have to leave behind our microscopic neurons and enter the realm of atomic and subatomic particles to cover the mechanisms of \gls[noindex=false]{nmr}.
Throughout this section, vector quantities and their associated scalar magnitudes will be formatted as $\mathbf{a}$ and $a$ respectively.

\section{Nuclei and spin}

\Gls{nmr} is a phenomenon and a technique, in which the interactions between atomic nuclei and external magnetic fields produce characteristic electromagnetic signals.
Atomic nuclei are made up of protons and neutrons (collectively termed nucleons) held together, despite the electrostatic forces repelling the positively charged protons, by the nuclear force\DIFdelbegin \DIFdel{, or residual strong force}\DIFdelend .
The number of protons, or atomic number, in a nucleus determines the element: Every hydrogen atom has exactly one proton, every oxygen atom exactly eight, and so on.
A given element can have a variable number of neutrons, with the total number of nucleons determining the \textit{isotope} of that atom.
While \gls{nmr} applies to any atomic species with the appropriate quantum nuclear properties, the vast majority of \gls{mri} applied to human imaging concerns itself only with one:
The nucleus of a hydrogen atom which, in its most abundant isotope (protium, $^1$H), is a single, positively charged proton.

The particular combination of nucleons determines that nucleus' spin and quantum magnetic moment, properties which in turn determine the effects wrought by an external magnetic field on that nucleus.
Spin (or spin angular momentum) $\mathbf{S}$ is an intrinsic property of elemental particles, and of their composites, including nuclei\autocite{Eisberg1961}.\footnote[2]{The unique spin quantity of nuclei \DIFdelbegin \DIFdel{$I$ }\DIFdelend \DIFaddbegin \DIFadd{$\mathbf{I}$ }\DIFaddend arises non-trivially from the interactions of its constituent nucleons' individual spins, however we are concerning ourselves here only with individual protons.}
Each particle has a fixed, dimensionless spin quantum number $s$ ($\frac{1}{2}$ for single electrons, protons and neutrons) which determines the spin vector's magnitude given by $S = \hslash \sqrt{s(s+1)}$, where $\hslash$ is the reduced Planck constant.
However, according to the uncertainty principle of quantum mechanics, which posits that the location and momentum of a quantum system cannot be known simultaneously, the total spin angular momentum vector (magnitude \textit{and} direction) cannot be measured.
Instead, the associated observable quantity is the component of $\mathbf{S}$ along the observed direction, typically arbitrarily designated the $z$-axis, which is determined by another quantum number $S_z = \hslash m_s$.
The spin magnetic quantum number $m_s$, like $s$, is a quantised property, meaning it can inhabit a discrete number $2s+1$ of values ranging from $-s$ to $+s$.
As a spin-$\frac{1}{2}$ particle, the proton thus has two possible spin states given by $m_s = \pm \frac{1}{2}$, commonly called ``spin-up'' and ``spin-down''.

The existence of an intrinsic angular momentum of a particle with electrical charge (i.e. a single electron, proton or nucleus) gives rise directly to another vector quantity, a magnetic moment $\bm{\mu}$.
Following the spin picture, the observable component $\mu_z$ is related to the spin angular momentum: $\mu_z = \gamma S_z = \gamma \hslash m_s $.
where $\gamma$ is the \textit{gyromagnetic ratio}, a species-specific quantity derived from its charge, mass and intrinsic $g$-factor.
Under quantum mechanics, a spin system can only inhabit a quantised set of possible states, called \textit{eigenstates} which have associated total energy \textit{eigenvalues}.
In the absence of an external magnetic field, the spin eigenstates $m_s = \pm \frac{1}{2}$ are degenerate, meaning their energies are equal and neither is more likely than the other, but they split into different energy levels, in a phenomenon called the Zeemann effect, under the influence of an external magnetic field $\mathbf{B}_0$.
The ratio of spins in spin-up vs spin-down state is given by the Boltzmann distribution, which describes the expected value of the energy states over all possible quantum superpositions:
\begin{align}
  \frac{N_{\uparrow}}{N_{\downarrow}} = e^{\Delta E / k T} = e\DIFdelbegin \DIFdel{^{\gamma \hslash B / k T}}\DIFdelend \DIFaddbegin \DIFadd{^{\gamma \hslash B_0 / k T}}\DIFaddend , \label{eq:boltzmann}
\end{align}
where $\Delta E$ is the difference in energy between the two states and is proportional to the external field strength \DIFdelbegin \DIFdel{$B$}\DIFdelend \DIFaddbegin \DIFadd{$B_0$}\DIFaddend .
The external magnetic field also exerts a torque on the spin magnetic moments, causing them to precess about $\mathbf{B}_0$, in a phenomenon known as \textit{Larmor precession} (Fig. \ref{fig:larmor}), at an angular frequency of
\begin{equation}
  \omega = \gamma B\DIFaddbegin \DIFadd{_0
}\DIFaddend \end{equation}

\begin{figure}
    \centering
    \includesvg[height=0.4\textwidth,pretex=\small\sffamily]{chapter_1/nmr.svg}\qquad \includesvg[height=0.4\textwidth,pretex=\small\sffamily]{chapter_1/relaxation.svg}
  \caption[Larmor precession and magnetisation relaxation]{\Glsentrylong{mri} depends on quantum interactions between nuclei and magnetic fields.
  \textbf{\sffamily a.} Larmor precession of \DIFaddbeginFL \DIFaddFL{a }\DIFaddendFL nucleus with spin magnetic moment $\bm{\mu}$ \DIFdelbeginFL \DIFdelFL{around }\DIFdelendFL \DIFaddbeginFL \DIFaddFL{about an }\DIFaddendFL external magnetic field $\mathbf{B}_0$.
  \textbf{\sffamily b.} Immediately after a 90\textdegree\ radio frequency excitation into the $xy$-plane, the bulk magnetisation $\mathbf{M}$ relaxes back to equilibrium where $\mathbf{M}_z=\mathbf{M}_0$ and $\mathbf{M}_{xy}=0$ in a spiral pattern, inducing an oscillating electrical signal in the \glsentryshort{mri} receiver equipment.}\label{fig:larmor}
\end{figure}

\section{Resonance and relaxation}



The application of a strong linear magnetic field $\mathbf{B}_0$ along $z$ causes spins in a sample to, on average, align themselves with the direction of that field, with the ratio of spins in each of the possible energy states determined by (\ref{eq:boltzmann}).
At equilibrium, more spin magnetic moments will occupy the lower energy state aligned with $\mathbf{B}_0$, resulting in a net bulk magnetisation $\mathbf{M}$ with magnitude $M_z=M_0$ at equilibrium.
As the difference in energy between the two states is determined by $\Delta E = \gamma h B_0$, the stronger the external field, the larger the ratio of up to down spins and thus the larger $M_0$.
This in turn underpins the strength of any subsequently sampled signals, so that stronger $\mathbf{B}_0$ fields, as a general rule, produce images with better \gls[noindex=false]{snr}.
Magnetic field strengths are measured in units of tesla (T), with Earth's magnetic field ranging from 25 to 65~\textmu T, while typical modern high-field MR imaging systems use superconducting magnets to produce main field strengths of 1.5 to 3 T.
Just as the distribution of spin states among two possible opposing states produces a net \textit{longitudinal} magnetisation along $z$, the lack of any directional coherence perpendicular to $z$ precludes any net \textit{transverse} magnetisation $\mathbf{M}_{xy}$.
In other words, at thermal equilibrium the magnetic moments within the sample are precessing at the Larmor frequency with random phases (Fig. \ref{fig:larmor}).

Manipulation of the net magnetisation away from, and subsequent relaxation back to this equilibrium state, is the basis of all signal production in \gls{mri}.
Spins in the lower energy state can be excited (resonated) to occupy the higher energy state by absorbing electromagnetic radiation tuned precisely to the Larmor frequency.
This energy is transmitted in the form of an oscillating magnetic field $\mathbf{B}_1$, called a \glsreset{rf}\gls{rf} pulse, as the frequency falls within the \gls[noindex=false]{rf} range of the electromagnetic spectrum, and has two effects on the bulk magnetisation.
The first is a reduction in $M_z$, as the absorbed energy reduces the ratio of spins in the relaxed state, with the total change in $M_z$ dependent on the duration $\tau$ of the \gls{rf} pulse.
Secondly, spins will align themselves with an effective field combining $\mathbf{B}_0$ and the rotating $\mathbf{B}_1$, resulting in a phase synchronisation of spins' precessions and a net, rotating transverse magnetisation with magnitude $M_{xy}$.
It is the sweep of this rotating bulk magnetisation vector which induces an oscillating current in the receiver equipment of the \gls{mri} scanner, which is registered as the signal called \gls[noindex=false]{fid}.
The angle between the total magnetisation $\mathbf{M} = \mathbf{M}_{xy} + \mathbf{M}_{z}$ after the \gls{rf} pulse and $\mathbf{B}_0$ is the flip angle $\alpha$ determined by $\alpha = \gamma B_1 \tau$.

As soon as the excitation field is switched off, the bulk magnetisation gradually relaxes to equilibrium as the induced effects on the individual moments revert.
Interactions between spins and their molecular environment cause de-phasing of their precession and consequent reduction in $M_{xy}$ (spin-spin or transverse relaxation), while the energy gained from the \gls{rf} pulse is gradually thermally dissipated among the lattice of neighbouring molecules as the spins revert to the statistically favourable lower energy state in alignment with $\mathbf{B}_0$ causing gradual recovery of $M_z$ (spin-lattice or longitudinal relaxation):
\begin{align}
  M_z(t)  &= M_0 (1-e^{-t/T_1}) \\
  M_{xy}(t) &= M_0 e^{-t/T_2} \label{eq:recovery}
\end{align}
where the recovery rate of $M_z$ and the rate of decay of $M_{xy}$ are described by the time constants $T_1$ and $T_2$ respectively, each dependent on the specific molecular environment and are tissue-specific quantities.
It is the varying values of $T_1$ and $T_2$, as well as the overall differences in \glspl{pd} between tissues which lend contrast to MR images.

\section{Signal echos}\label{sec:echos}

Immediately after application of the \gls{rf} excitation pulse, magnetisation will return to equilibrium in a spiral pattern (Fig. \ref{fig:larmor}) inducing the rapidly decaying \gls{fid} signal.
Due in part to its short lifetime, the simple \gls{fid} is not particularly useful on its own for biological imaging.
In fact, the phase coherence of the system is not entirely lost even as the \gls{fid} signal reaches zero, but only masked by competing effects.
This hidden signal can be partially resurrected though the spin echo technique, but to appreciate how we must first look more closely at the transverse relaxation process, which is the result of a range of different effects through which spins lose their phase coherence.
One is an intrinsic tissue property, in which through interactions with the molecular environment spins experience minute perturbations in $\omega$ and accumulate phase differences as a result.
This dephasing is associated with the ``true'' $T_2$.
In addition, inhomogeneities in the applied $\mathbf{B}_0$ field also produce slightly varying $\omega$ values across the sample, with the same dephasing effect.
The combination of intrinsic $T_2$ and the effects of field inhomogeneities produce an overall dephasing effect characterised by $T_2$*, where  $T_2$* $\leqslant$ $T_2$.

If, at a time TE/2 after a 90\textdegree\ \gls{rf} pulse is applied to the sample, an additional ``refocussing'' 180\textdegree\ pulse follows, $\textbf{M}$ will be flipped in the $xy$ plane.
Recall that variations in precession speeds will mean that some spins will have accumulated a certain amount of phase ``ahead'' of the ensemble average, while others will lag behind.
After being flipped, while the frequencies stay the same, the relative phase differences will be reversed, allowing the slower spins to catch up.
Thus after an additional symmetrical time span of TE/2, the uneven effects of field inhomogeneities will have been entirely cancelled out and the spins will find themselves briefly rephased, producing a signal peak at echo time TE called a \textit{spin echo}.
The signal at echo peak is then only dependent on the intrinsic $T_2$, not on $T_2$*.

The need for an additional \gls{rf} pulse to produce a spin echo introduces instrumentation delay times and corresponding loss of signal magnitude.
An alternative form of signal echo is the \textit{gradient echo}, which can work with shorted TEs and produce different types of image contrast.
To produce a gradient echo, a static, spatially varying magnetic field gradient $G_x$ is applied across the sample causing spins to precess at different rates according to their local magnetic field $B_0 + G_xx$ and dephase, spoiling the \gls{fid} more rapidly than it would decay on its own.
After a time TE/2 the gradient polarity is reversed, and with it the rates of dephasing across the sample.
Spins previously precessing fastest are now the slowest and vice versa, such that after an equal amount of time TE/2 under the reversed gradients, the gradient-induced dephasing effects are reversed, producing a signal echo at TE.
Since the reversal of gradient polarity effects only the gradient coil generated fields, the dephasing caused by local inhomogeneities in the $\mathbf{B}_0$ field cannot be reversed, and the resulting signal is $T_2$*-weighted.
As a consequence, gradient echo sequences are more prone to $T_2$*-related imaging artefacts, including those caused by local variations in magnetic susceptibility.

\section{Spatial encoding}

The signal induced in an \gls{mri} receiver coil is a complex superposition of contributions from spins located throughout the excited sample.
To generate a three dimensional image from this time-varying electrical current, the individual signal components need to be separated and attributed to positions in space.
Most commonly in medical imaging spatial encoding is achieved with the \gls[noindex=false]{2dft} method using slice selection, phase encoding and frequency encoding\autocite{Hendee2002}.
In theory, these three encoding dimensions can apply to any permutation of the three spatial axes, but in practice, and for the purposes of this simplified description, it is easiest and conventional to select slices along the $z$-axis, parallel with the main magnetic field, and we will designate the $x$ and $y$ axes as the frequency and phase encoding dimensions respectively.

Slice selection ensures that the \gls{rf} pulse only excites a specific slab within the sample, and is achieved by overlaying a magnetic field gradient $G_s$ along the slice selecting dimension $z$, altering the Larmor frequencies of the spins along $z$ according to $\omega_0 + \gamma G_sz$, where $\omega_0 = \gamma B_0$ is the Larmor precession frequency under $\mathbf{B}_0$.
An \gls{rf} pulse with (central) frequency $\nu_c = \frac{1}{2\pi} (\omega_0 + \gamma G_s z_c)$ will induce resonance in spins located only at position $z_c$, producing a slice of excitation. All other spins in the sample will be unaffected by subsequent manipulation of the magnetisation vector throughout the imaging sequence and will not generate signal.
The slice width $\Delta z$ is determined by the \gls{rf} pulse bandwidth $\Delta\nu$ as $\Delta z = \frac{2 \pi \Delta \nu}{\gamma G_s}$.

Within the selected plane, manipulation of two spin quantities determines localisation in the remaining two dimensions.
The first is spin precession frequency, which can be modulated along an axis in the same manner as is used in slice selection---a \gls[noindex=false]{fe} magnetic field gradient $G_{f}$ applied along $x$ imparts spatially varying frequencies according to $\omega_0 + \gamma G_{f}x$.
Thus each spin precesses at a frequency that uniquely identifies its position along the \gls{fe} direction.
In practice, frequencies are grouped into pixels of finite width, with the range of frequencies per pixel termed \textit{pixel bandwidth}, while the full range of frequencies (total receiver bandwidth) across the sample determines the image field of view (FOV) in the \gls{fe} direction.
Each pixel's width is determined by the analog-to-digital sampling rate of the received signal, such that if the signal is sampled at dwell time intervals of $t_d$ for a total amount of time $t_s$, the total \gls{fe} bandwidth is $t_d^{-1}$ and the pixel bandwidth $t_s^{-1}$.
Although the term \textit{pixel} is used in this context to refer to a discrete image element in a two-dimensional plane, as in reality those elements have finite thickness in the slice dimension, we will more often be referring to \textit{voxels} (``volume pixels") to denote a three-dimensional imaging volume element.
The \gls{fe} gradient is applied throughout the signal readout period to maintain the spatially dependent frequencies and is correspondingly also called the readout gradient.
During readout, the receiver coil digitally samples the echo signal at discrete points, with the number of samples corresponding to the image matrix size along the $x$ axis (Fig. \ref{fig:kspace}a).

\begin{figure}
  \includesvg[width=\textwidth,pretex=\sffamily\small]{chapter_1/kspace_axial.svg}
  \caption[MRI spatial encoding and $k$-space]{Sampled MR signals encode 2D images.
  \textbf{\sffamily a.} A single echo train is sampled at dwell time intervals $t_d$, where each sample corresponds to a point in reciprocal $k$-space, and $t_d$ determines the frequency resolution along $k_x$ and the image matrix size along $x$.
  \textbf{\sffamily b.} Repeated echos with varying \gls{pe} gradient strengths sample the image along $k_y$.
  \textbf{\sffamily c--d.} The image in spatial dimensions is reconstructed by taking the \gls{2dft} of the $k$-space image.}
  \label{fig:kspace}
\end{figure}

The second encoded property is spin phase.
If differences in spin frequency can be likened to clocks running at different speeds, then two spins with different phases (but the same frequency) are like two clocks which both keep perfect time, except that one has not been set correctly and is perpetually five minutes behind the other (these two clocks would have a phase difference of five minutes).
For the two clocks running at different speeds, their phase difference would constantly change as one fell further and further behind the other.
Before signal readout, a \gls[noindex=false]{pe} gradient $G_p$ is applied along $y$ imparting different precession frequencies as a function of $y$ as we saw for frequency encoding, with spins slowing acquiring growing phase differences according to their $y$ positions.
This can be likened to taking our two clocks and moving one closer to a very large object, whose gravitational field will cause it to run slower and gradually fall behind the second clock.
After a time $\Delta t_p$, $G_p$ is switched off, and in its absence spins return to precessing at the same rates, except that they have accumulated $y$-dependent phase differences $\phi_p$ to the tune of $\phi_p=\gamma G_p y \Delta t_p$.
In other words, the clocks have been brought back together such that they keep the same time, but without correcting the five minute lag one clock accumulated while it was experiencing gravitational time dilation.
Since the phase encoding gradient is switched off before signal readout, phase differences become locked in and serve to identify spins along the \gls{pe} direction.
A single echo provides insufficient information to fully separate out the mixed phase contributions and uniquely determine the $y$ coordinates of received signal components.
Instead, the excitation--phase-encoding--echo-readout sequence is repeated with varying \gls{pe} strengths resulting in a series of echo train signals, with the number of \gls{pe} steps $N_{p}$ corresponding to the image matrix size along the $y$ axis (Fig. \ref{fig:kspace}b).

After acquiring all \gls{pe} steps, the full data from a single slice has been recorded as a series of time-varying electrical signals.
Thanks to our spatial encoding gradients, we know that those signals are complex mixtures of frequencies originating from different positions in the sample.
The spatial frequencies and phases in this raw signal can be represented in the 2-dimensional frame known as $k$-space, a reciprocal space defined as the inverse spatial \gls{2dft} of the measured image.
To reconstruct an image in the spatial domain, two Fourier transforms are computed, first for each signal train, and secondly for each time sample across all phase encoding steps (Fig. \ref{fig:kspace}).

\section{Pulse sequences and contrast}

As we have seen, MR images are acquired by measuring signals within the spatial frequency dimension, $k$-space, from which we can retrieve the spatial image via a \gls{2dft} operation.
Pulse sequences, meaning specific arrangements, durations and magnitudes of the \gls{rf} pulses, field gradients, and signal readout, encode the trajectory and resolution in which data-points in $k$-space are measured (Fig. \ref{fig:mprage}).
MR imaging is a stunningly diverse modality.
Beyond the basics of spatial encoding and echo signal generation, there lies a whole host of techniques for generating specific contrasts, neutralising unwanted signal contributions and artefacts, and accelerating acquisition times, which are outside of the scope of this introduction.

\begin{figure}[htb!]
  \includesvg[width=\textwidth]{chapter_1/mprage.svg}
  \caption[MPRAGE pulse sequence diagram]{A simplified pulse sequence diagram representing a \gls[noindex=false]{mprage} $T_1$-weighted acquisition. This popular acquisition uses a 180\textdegree\ \gls[noindex=false]{ir} \glsentryshort{rf} pulse, followed after inversion time TI by an excitation--acquisition sequence, to enhance the contrast between tissues based on their different $T_1$ relaxation times. (In reality, multiple image echos are sampled using fast imaging techniques in-between repeated \gls{ir} pulses.) $N_p$ signal echos are sampled with decreasing \glsentryshort{pe} gradient strengths $G_p$ to build up an entire 2D image.}
  \label{fig:mprage}
\end{figure}

Modification of pulse sequence parameters enables the production of images in which different physiological properties dominate the image contrast, the principal ones being the tissue-specific relaxation time constants $T_1$ and $T_2$, as well as \gls{pd}.
The development of novel pulse sequences and contrasts is a perennial and active field of research which will continue to expand and improve the capabilities of \gls{mri} for years to come.
Here we will focus on the few fundamental contrasts directly relevant to this thesis.
From (\ref{eq:recovery}), the signal at $t=TE$ in a spin echo sequence is described approximately by
\begin{equation}
  S(\text{TE}) \propto M_0 (1-e^{-\text{TR}/T_1})e^{-\text{TE}/T_2}
\end{equation}
where TR is the repetition time between successive \gls{rf} excitations and $M_0$ is proportional to \gls[noindex=false]{pd}.
With a very long repetition time ($TR\gg T_1$) the $M_z$ term tends to 1 for all tissues, and the contrast between them based on their different $T_1$ properties disappears.
Paired with an echo time of $TE\approx T_2$, the resulting image will be characterised by the differences in $T_2$ between tissues and is therefore said to be \textit{$T_2$-weighted}.
Similarly, very short echo times ($TE\ll T_2$) produce images with no $T_2$-based tissue contrast, and when $TR\approx T_1$ the image will be \textit{$T_1$-weighted}.
Finally if both $TR\gg T_1$ and $TE\ll T_2$ then contrast derived from different relaxation properties disappears altogether, producing a \gls{pd}-weighted image.

Sequences with $T_1$- or $T_2$-weighted contrast are some of the most common sequences employed in clinical and research \gls{mri} as they can produce images with high resolution and anatomical detail (Fig. \ref{fig:ortho}).
Alongside these so-called ``conventional'' contrasts exist a large collection of advanced imaging techniques that are sensitive to particular tissue properties or activity.\footnote[2]{The term ``contrast'' is also associated in clinical contexts with the use of external agents (typically injected) with paramagnetic properties that shorten local $T_1$ times and increase image brightness particularly in blood vessels and in areas of \gls{bbb} disruption, making it an important tool in oncological imaging. This will be referred to specifically as \gls[noindex=false]{dce} \gls{mri}, or simply contrast enhancement, where applicable, while here we will use ``contrast'' more generally to refer to manipulation of pulse sequence elements to highlight different tissue properties.}
\DIFdelbegin %DIFDELCMD < \gls{fmri}%%%
\DIFdelend \DIFaddbegin \Gls{fmri}\DIFaddend , for example, involves the detection of subtle contrast changes produced by changes in the relative concentrations of oxygenated and deoxygenated blood, which have different magnetic properties and influence the local magnetic field accordingly.
A spike in neural activity within a particular region causes increased flow of oxygen-rich blood to that area, and the resulting local magnetic field disturbances have a measurable effect on proton relaxation, producing a \textit{blood oxygentation level dependent} (BOLD) signal.
\Gls{fmri} is an important clinical and research tool for measuring association between activity in particular brain areas and observable behaviour or stimuli, such as motor activity, speech, or visual inputs.
Additionally, by analysing temporal correlations between activation in different brain regions, we can deduce functional connectivity between them and establish a model of a brain's functional network.

Establishing that two brain regions' neural activity is temporally correlated isn't enough to confirm the existence of a direct structural connection between them, as their correlation may be incidental, or mediated by a third intermediate node.
And while \gls{fmri} is a key tool for identifying eloquent cortex for neurosurgical planning, the paucity of blood vessels within \gls{wm} and \gls{fmri}'s low spatial resolution make it unsuitable for mapping \gls{wm} fibre activity.
The only tool available for non-invasively imaging the microscopic structure of \gls{wm} \textit{in vivo} is \glsentrylong{dmri}.

\begin{figure}[hbt!]
  \includesvg[width=\textwidth,pretex=\ttfamily\small]{chapter_1/anat_ortho.svg}
  \caption[Neuroanatomical landmarks on $T_1$-weighted MRI]{Orthogonal slices (from left to right: Sagittal, coronal, axial views) of an \glsentryshort{mprage} acquisition of the author's brain, with key anatomical features and landmarks labelled. a. spinal cord, b. medulla, c. pons, d. midbrain, e. cerebellum, f. hypothalamus, g. thalamus, h. lateral ventricle, i. third ventricle, j. fourth ventricle, k. subcortical white matter, l. cortical grey matter, m. corpus callosum, F. frontal lobe, P. parietal lobe, O. occipital lobe, T. temporal lobe, S. central sulcus.}
  \label{fig:ortho}
\end{figure}

\section{Diffusion-weighted MRI}\label{sec:dmri}



Molecules in a fluid travel through their environment in a stochastic pattern of thermal movement, and this incoherent motion is the basis for the contrast formed in \gls[noindex=false]{dmri}.
In a pure, unlimited volume, the mean squared displacement $x$ travelled in time $t$ is described in three dimensions by a material specific property, the diffusion coefficient $D$:
\begin{equation}
  \langle |\mathbf{x}|^2 \rangle = 6Dt
\end{equation}
Water molecules in living tissue are not as free to diffuse for long distances in all directions as they would be in a pure, endless medium.
The presence of cell membranes, cytoskeletons, organelles and other microstructural barriers reduces the theoretical distance a diffusing molecule can cover within a given timeframe as it has to navigate around obstacles, resulting in a \textit{hindered diffusion} pattern with a lower \gls[noindex=false]{adc}.
If the environment poses such unsurmountable barriers to diffusion that, after a sufficiently long time, the mean displacement will appear to plateau, then a \textit{restricted diffusion} pattern is observed where $D$ becomes dependent on the diffusion time and geometry of the limiting environment \autocite{LeBihan1986,LeBihan1995}.
In \gls{gm}, the cell membranes of somata, glial cells, dendrites and axons as well as subcellular structures and organelles all present obstacles to diffusing molecules, resulting in a hindered diffusion pattern.
However, since these barriers are arranged in a random fashion, and are to be encountered in any direction with equal likelihood, there is no preferential direction in which diffusion is freer than in others, thus the diffusivity is \textit{isotropic}.
In \gls{wm}, conversely, the highly coherent arrangement of axons poses very strong restrictions on diffusion occurring perpendicular to those axons, while diffusivity is relatively unhindered along the direction parallel to the bundled axon fibres.
In such an environment, diffusion is \textit{anisotropic}, biased towards certain directions over others.

That molecular displacement, both passive diffusion effects and active perfusion (such as the flow of \gls{csf} and blood\autocite{LeBihan1988}), has an effect on measured MR signals was known some time before those effects were harnessed as a form of contrast weighting.
\textcite{Hahn1950} described in 1950 how diffusion attenuates the measured spin echo signal, as spins diffusing through field inhomogeneities aren't refocussed by the 180\textdegree\ pulse.
While the effect is typically small with standard sequences, any MR pulse sequence can be modified to introduce stronger sensitisation to Brownian motion, or diffusion weighting, in addition to the existing $T_1$ and $T_2$ based contrasts.
Most commonly this is achieved with a spin echo sequence and the addition of a pair of strong diffusion sensitisation gradient pulses of equal length and polarity separated by the 180\textdegree\ \gls{rf} refocussing pulse prior to echo generation and signal readout.

To illustrate the concept, we will consider the application of diffusion weighting along a single orthogonal direction $x$.
After slice selection and \gls{rf} excitation, a gradient of magnitude $G_d$ is applied along $x$ for time period $\delta$, followed by the 180\textdegree\ pulse and a second gradient pulse for a further duration $\delta$.
Consider a spin which is stationary along $x=x_1$ throughout the activation of these gradients.
Initially, it will gain a phase $\phi_1$ proportional to its position $x_1$ along the gradient $G_d$ according to $\gamma G_d \delta x_1$.
After magnetisation inversion and application of the second gradient, having not changed position $x=x_2=x_1$ it will accumulate the opposite phase $\phi_2 = -\phi_1$, reversing the effect of the first gradient and resulting in a net phase change of $0$ due to diffusion sensitisation.
Now consider a spin in net motion along $x$ which will consequently be subjected to different gradient magnitudes across the two time points.
It will experience a net dephasing of
\begin{equation}
  \Delta\phi = \gamma G_d (\int_0^{\delta} x(t) dt - \int_{\Delta}^{\Delta+\delta} x(t) dt)
\end{equation}
where $\Delta$ is the time separating onset of the two gradient pulses (Fig. \ref{fig:dwepi}).
When $\Delta\phi \neq 0$, refocussing will be incomplete, attenuating the spin echo amplitude accordingly.

Following diffusion sensitisation, spatial encoding and image formation proceed as in standard sequences, and those voxels in which diffusion is high will have have a high degree of diffusion weighting-induced dephasing and exhibit a corresponding signal attenuation.
In those voxels with low diffusion, phase coherence will remain relatively intact after diffusion sensitisation and signal loss will be correspondingly minimal.
The signal magnitude at echo time is determined in simplified terms and assuming free diffusion at the measured timescales by
\begin{align}
    S = S_0e^{-bD}
\end{align}\label{eq:S}
where $S_0$ is the baseline signal value without diffusion weighting, $D$ is now the effective, or \textit{apparent diffusion coefficient} (\glsentryshort{adc})\autocite{Basser1994,Beaulieu2002} (a quantity that depends on the molecular environment and includes both diffusion and other incoherent motion effects) and $b$ is the gradient factor summarising the effects of all gradient pulses, first described in \textcite{LeBihan1986} and since commonly termed the $b$-value.
The above description follows the spin echo pulse sequence proposed by Stejskal and Tanner in 1965\autocite{Stejskal1965}, one of the earliest \gls{dmri} sequence designs.
Modern \gls{dmri} sequences vary in the precise pattern of diffusion weighting gradients and refocussing pulses, so the $b$-value provides a sequence-independent summary of the effective diffusion weighting factor.
In the Stejskal-Tanner sequence, for example, we have $b = \gamma^2 G_d^2 \delta^2 (\Delta-\delta/3)$.
The choice of $b$-value is highly influential on the final image, and typically values in the range 600--3000~s~mm$^{-2}$ are used.
Higher $b$-values reflect stronger diffusion sensitisation, but also produce far noisier images, require longer echo times, and require more advanced scanner hardware and gradient instrumentation\autocite{Tournier2011}.
Values up to 1000~s~mm$^{-2}$ are more common in routine and clinical settings, balancing optimal diffusion dependent contrast with acceptable \gls{snr}, while higher values may be included in more advanced research methods\autocite{Roberts2007}, where the longer diffusion times provide higher sensitivity to tissue-specific diffusion processes and better angular resolution for the advanced imaging techniques described in Chapter \ref{chap:neuroimaging}.
Acquisitions may also combine multiple $b$-values in a so-called ``multi-shelled'' scheme, which can tease apart the effects of hindered and restricted diffusion compartments\autocite{Clark2002,Assaf2005}.

A single \gls{dmri} volume only applies a diffusion-weighting gradient of fixed strength along a single orientation, with all signal samples reflecting the diffusivity along that specific direction ($x$ in the description above).
As diffusion in much of the brain is anisotropic, it cannot be assumed in general that the diffusivity would be the same if observed along a different direction.
The simplest \gls{dmri} acquisition, which is employed routinely in diagnostic radiology, obtains three diffusion-weighted source image volumes with diffusion-weighting along orthogonal directions, along with the baseline $b_0$ image, which are combined to produce a \textit{trace-weighted} image:
\begin{equation}
  S_{tw} = S_0 e^{-b(D_{xx} + D_{yy} + D_{zz})/3}. \label{eq:trace}
\end{equation}
Alternatively, a map of \gls{adc} or mean diffusivity may be computed according to
\begin{equation}
  \text{ADC} = -b^{-1} ln (\frac{S_{tw}}{S_0}). \label{eq:adc}
\end{equation}
In modern acquisitions, including those required for the high angular resolution techniques we will encounter in Chapter \ref{chap:neuroimaging}, anywhere from six to 100s of directions are measured, and the number of scans may be additionally multiplied by repeating and averaging signals to boost \gls{snr}.
Consequently, \gls{dmri} scan times can be exceedingly long, from several minutes for clinical sequences into hours for advanced research studies, prompting efforts to reduce scan time by means of accelerated and parallel imaging techniques.

\begin{figure}[htb!]
  \includesvg[width=\textwidth]{chapter_1/dwepi.svg}
  \caption[Diffusion-weighted EPI pulse sequence diagram]{Diffusion-weighted \glsentrylong{epi} pulse sequence diagram, with diffusion gradients, drawn in dark grey to differentiate them from orthogonal slice, phase and frequency encoding, either side of a 180	extdegree\ \gls{rf} inversion pulse. All orthogonal gradients may be combined to produce diffusion weighting along an arbitrary direction. An entire slice of $k$-space is acquired from a single echo train, using a strong switching frequency encoding gradient $G_f$ to encode lines along $k_x$, combined with ``blipped'' phase encoding gradients $G_p$ to advance the signal through $k_y$.}
  \label{fig:dwepi}
\end{figure}

\subsection{Representing spherical functions}
\label{sec:sh}

Diffusion signal analysis and studying macroscopic \gls{wm} structure require describing quantities distributed in the three dimensional angular domain.
If the shape of the distribution is not generally known, then an arbitrary function may be represented in a suitable function space, defined by a set of orthogonal bases.
Analogous to the use of the basis vectors $\mathbf{e}_1=(1\,0\,0)^T, \mathbf{e}_2=(0\,1\,0)^T, \mathbf{e}_3=(0\,0\,1)^T$ to represent any other vector
$\mathbf{v}$ in $R^3$ as $\mathbf{v} =  (v_x\,v_z\,v_z)^T = v_x\mathbf{e}_1 + v_y\mathbf{e}_2 + v_z\mathbf{e}_3$,
so can basis functions be combined to represent any other function in that space.

\begin{SCfigure}[][hbt!]
  \includegraphics[width=0.6\textwidth]{chapter_1/SH.png}
  \caption[Spherical harmonic basis functions]{Visualisation of the first four orders (in rows) of the \glsentrylong{sh} basis functions. Positive and negative lobes are coloured grey and yellow respectively. Image adapted from the original by \citeauthor{Inigo.quilez2014}, licensed under \href{https://creativecommons.org/licenses/by-sa/3.0}{CC-BY-SA 3.0}, via Wikimedia Commons}
  \label{fig:sh}
\end{SCfigure}

Throughout this thesis, we will represent spherical functions using the \gls[noindex=false]{sh} basis defined on the 2-sphere $S^2$.
The solutions to Laplace's equation in spherical coordinates, the \glspl{sh} are widely used across physics, engineering and mathematics, and are given in complex form for non-negative integer orders $l = 0, 1, 2,...$ and integer phase factor $-l \leqslant m \leqslant l$ as

\begin{align}
  Y_l^m(\theta,\phi) = \sqrt{ \frac{2l+1}{4\pi} \frac{(l-m)!}{(l+m)!} } P_l^m(cos\phi)e^{im\phi}
\end{align}

where $P_l^m(x)$ are the associated Legendre polynomials.
In \gls{dmri} analysis, including for representing axon fibre orientations, a simplified set of \gls{sh} functions can be used.
We will consider only real valued functions, and under the assumption that diffusion processes are antipodally symmetric, only the even orders $l = 2 \mathbb{N} = 0,2,4,...$ are included\autocite{Descoteaux2006}.
This modified basis set is then given by
\begin{equation}
  Y_{lm} = \begin{cases}
              \sqrt{2} \; \Im (Y_l^{-m}) & \text{if } -l \leq m < 0 \\
              Y_l^0                      & \text{if } m=0 \\
              \sqrt{2} \; \Re (Y_l^m)    & \text{if } 0 < m \leq l
           \end{cases}, \label{eq:sh}
\end{equation}
where $l = 2\mathbb{N}$ and $\Re$ and $\Im$ denote the real and imaginary parts respectively.

An arbitrary function $f(\theta,\phi)$ defined on the sphere can then be defined as a linear combination of $Y_{lm}$:
\begin{equation}
  f(\theta,\phi) = \sum_{l=0}^{\infty}\sum_{m=-l}^{l} c_{lm}Y_{lm}(\theta,\phi) = \sum_j^{\infty} t_jY_j(\theta, \phi) \label{eq:shfun}
\end{equation}
with $j = \frac{1}{2}l(l+1) + m$.
In practical applications, only a finite number $N = \frac{1}{2}(l_{\text{max}}+1)(l_{\text{max}}+2)$ of coefficients $c_{lm}$ can be determined, up to a maximum order $l_{\text{max}}$.
A higher maximum order can represent more angular detail, however due to this necessary truncation our representations as a finite series can only ever be close approximations of the full function $f$.

\subsection{Echo planar imaging}
\label{sec:epi}

Aside from the cost and practicalities of the longer scans needed for \gls{dmri}, there is also the increased risk of motion artefacts as subjects tire or become uncooperative in the scanner.
To achieve orders of magnitude shorter scan times than with conventional sequences in which a single step of phase encoding (or a single line in $k$-space) is sampled per excitation, \gls{dmri} scans usually use a fast imaging technique called \gls[noindex=false]{epi}, in which all phase encoding steps are acquired after only a single \gls{rf} excitation.
In one example of an \gls{epi} pulse sequence, the phase and frequency encoding gradients are switched on to move the signal to a corner of $k$-space immediately prior to readout, then the readout gradient rapidly alternates polarity while the \gls{pe} gradient is ``blipped'' on each switch\autocite{Wielopolski1998}.
The result is a stepped zig-zag traversal of all of $k$-space during the readout of a single signal echo (Fig. \ref{fig:dwepi}).
Measuring enough samples in the extremely short time available before the signal decays requires very strong and rapidly switching gradients, placing significant demands on scanner hardware\autocite{Bowtell1998}.

Fitting the entire image readout into a single magnetisation decay period results in relatively long total acquisition times per excitation.
The pixel bandwidth in the \gls{pe} direction is the inverse of this total acquisition time, and the resulting low pixel bandwidth is associated with particular imaging artefacts.
Local differences in magnetic susceptibility within the sample cause field inhomogeneities, leading to off-resonance phase shifts and consequent geometric image distortions termed \textit{susceptibility artefacts} \autocite{Fischer1998}.
These are particularly strong at interfaces between substances with large susceptibility differences, such as the air--tissue boundaries around the sinuses in the human head, and can compromise the results of a neuroimaging experiment if not appropriately corrected for.
\clearpage{}
\clearpage{}\chapter{Diffusion MRI analysis}\label{chap:neuroimaging}

Having given a brief introduction to the principles of neuroanatomy and \gls{dmri} physics and image formation, we will turn now to the modern developments in applying \gls{dmri} to qualitative and quantitative analysis of \gls{wm} organisation.
This chapter will take us through early to state-of-the art uses of \gls{dmri} as a \gls{wm} imaging tool and outstanding challenges.

Though the tools for studying the structure and function of neural connectivity are numerous, \gls{dmri} remains the only available technique capable of investigating microscopic \gls{wm} structure of the entire human brain \textit{in vivo}.
Though the mechanism underlying a diffusion-weighted signal is well understood (see Section \ref{sec:dmri}), its correct interpretation hinges on an understanding of how diffusion is affected by the microstructural environment\autocite{LeBihan1995}, or even by just those features of the microstructural environment we are interested in measuring.
It is not obvious, for example, how much of diffusion within a voxel filled with myelinated axons can be best characterised as either hindered or restricted, and how properties such as axon density and diameter might contribute to the observed signal\autocite{Panagiotaki2012}.

Diffusion \gls{mri} image voxels are on the order of cubic millimetres, and in such a volume are contained thousands of individual axons, with average diameters of around 1$\mu m$\autocite{Liewald2014,Lampinen2019}.
What's more, rarely are all axons within a voxel uniformly aligned along a single direction.
Throughout the brain, fibre tracts mingle and intersect, bend and fan out\autocite{Jeurissen2013,Alexander2019}.
The physical diffusion processes in such complex configurations and multicellular environments cannot be known, only approximated and modelled through a choice of assumptions and simplifications.
In this there are, of course, many contrasting approaches, developed for different applications and the directed study of specific quantities and microstructural features.
Such properties of interest to \gls{dmri} analysis include cellular composition of \gls{gm}, axon density and myelination, intra- and extracellular water content, and diffusivity perpendicular to axonal orientation.
These scalar parameters can be estimated using a family of microstructural models known collectively as multi-compartment models, so named for their separation of the measured signal into isolated compartments representing, for example, intra-axonal or extracellular space, each with different diffusion patterns\autocite{Panagiotaki2012,Alexander2019} which could capture virtual cytohistological information and even serve as biomarkers for disease\autocite{Alexander2008}.
For the purposes of studying macroscopic brain connectivity and the course and organisation of individual \gls{wm} fibre tracts, the key information of interest to be determined from raw \gls{dmri} is the distribution of axonal orientations within each voxel described by a \gls[noindex=false]{fod}.

\section{Tissue microstructure and fibre orientation modelling}


The first approach for modelling fibre orientations came in the form of \gls[noindex=false]{dti}, which remains one of the dominant diffusion models in many applications today, particularly in clinical contexts.
The diffusion process within a \gls{wm} voxel is modelled as a three dimensional Gaussian distribution, whose covariance matrix is proportional to the \textit{diffusion tensor}\autocite{Basser1994,ODonnell2011} $\mathbf{D}$:
\begin{equation}
  \mathbf{D} = \begin{pmatrix}
                D_{xx} & D_{xy} & D_{xz}\\
                D_{yx} & D_{yy} & D_{yz} \\
                D_{zx} & D_{zy} & D_{zz}
                \end{pmatrix} \label{eq:dt}
\end{equation}
At least six unique directions are required to fully determine $\mathbf{D}$ (which is diagonally symmetric, i.e. $D_{ij} = D_{ji}$), although the diagonal components, sufficient for measuring \gls{adc} as previously described in (\ref{eq:trace}--\ref{eq:adc}), may be determined from only three orthogonal measurements.

By diagonalising $\mathbf{D}$, which is symmetric and positive-definite, three orthogonal eigenvectors and corresponding eigenvalues $\lambda_1 \geqslant \lambda_2 \geqslant \lambda_3$ can be computed.
Rotationally invariant indices derived from these tensor eigenvalues, including mean diffusivity (MD) and \gls[noindex=false]{fa}, provide scalar quantities which can give insight into underlying tissue properties, the latter being a measure of the degree of diffusion anisotropy given by
\begin{equation}
  FA = \sqrt{\frac{3}{2}}\frac{\sqrt{(\lambda_1 - \langle \lambda \rangle)^2 + (\lambda_2 - \langle \lambda \rangle)^2 + (\lambda_3 - \langle \lambda \rangle)^2}}{\sqrt{\lambda_1^2 + \lambda_2^2+ \lambda_3^2}}.
\end{equation}
The principal eigenvector $\bm{\lambda}_1$ is usually interpreted as the direction of fastest, or least restricted, diffusivity, although this holds only in voxels with a single population of straight, parallel axons.
If the \gls[noindex=false]{dt}, whose eigenvectors and eigenvalues can be represented in the form of an ellipsoid shaped surface of equal mean displacement, is used as a rudimentary approximation of the \gls{fod}, then $\bm{\lambda}_1$ is taken as the single peak orientation of axon fibres.
Depicting peak orientations is achieved through \gls[noindex=false]{dec} mapping, in which directions are mapped to colours in RGB space, where the $x$-axis, or medial-lateral / left-right direction, is assigned pure red, the $y$-axis, or posterior-anterior direction, is assigned pure green, and the $z$-axis, or inferior-superior direction, is assigned pure blue.
\Gls{dti} images are often depicted as \gls{dec} \gls{fa} (or ``colour \gls{fa}'') maps, where a voxel's colour is determined by $\bm{\lambda}_1$ and its brightness by the \gls{fa} value.

In fact, it is rare for a voxel of \gls{wm} to contain only a single uniformly oriented bundle of axons.
In so-called crossing fibre voxels, which are traversed by at least two distinctly oriented fibre populations, the \gls{dt} provides a woefully inaccurate or misleading picture:
A smaller population with a low signal contribution may be entirely unrepresented in the modelled \gls[noindex=false]{odf} peak, or else the effect is one of averaging all contributing populations such that none of the actual underlying fibre directions are indicated by the \gls{dt} eigenvectors (Fig. \ref{fig:cross}).

\begin{figure}
  \centering
  \includesvg[width=0.6\textwidth,pretex=\small\sffamily]{chapter_2/fods.svg}
  \caption[Fibre orientation distribution modelling: Comparison between diffusion tensor and constrained spherical deconvolution]{Popular \glsentryshort{dmri} modelling techniques produce different estimates of the underlying fibre orientations. In a voxel with a single population of fibres (top row), both the diffusion tensor and \glsentrylong{csd} derived \glsentrylong{fod} (\glsentryshort{fod}) capture the dominant fibre orientation. If two fibre bundles cross within a voxel (bottom row), the diffusion tensor does not accurately estimate any of the dominant fibre orientations, whereas the \glsentryshort{fod} is able to resolve both populations.}
  \label{fig:cross}
\end{figure}

The crossing fibres problem led to \gls{dti}, at least in research imaging, gradually giving way to higher-order fibre orientation modelling techniques which aim to account for more than one, or even an arbitrary number of distinctly oriented fibre populations\autocite{Alexander2005}.
These include some multi-compartment models which can include multiple directional intra-axonal compartments, such as the popular ``ball and sticks'' model\autocite{Behrens2003,Behrens2007}.
Alternatively, several approaches aim to retrieve an underlying orientation distribution with arbitrary number of peaks through transforming the raw signal \gls{odf} into a spherical distribution of diffusivity or correlates thereof.
Diffusion spectrum imaging (DSI)\autocite{Wedeen2008}, and the less data-demanding Q-ball imaging\autocite{Tuch2003,Tuch2004}, are examples of methods that reconstruct a distribution of water molecule displacement (\gls[noindex=false]{dodf}), with peaks aligned with the (presumed) fibre orientations.
Deconvolution methods, by contrast, aim to reconstruct the distribution of fibre orientations, beginning with the premise that a single population of parallel fibres will produce a characteristic \gls{dmri} signal profile.
Then, the observed signal $S(\theta,\phi)$ in a given voxel amounts to a convolution over the fibre orientation distribution function $F(\theta,\phi)$ with this single fibre ``response'' kernel $R(\theta)$:
\begin{align}
  S(\theta,\phi) = F(\theta,\phi) \otimes R(\theta)\label{eq:csd}
\end{align}
Solving (\ref{eq:csd}) for $F(\theta,\phi)$, represented in spherical harmonics basis as given by (\ref{eq:shfun}), involves computing the inverse operation:
A spherical \textit{de}convolution of the signal with the response function.
Early versions of this concept included one by \textcite{Anderson2005}, in which the response kernel was modelled as a \gls{dt}.
In another approach presented by \textcite{Tournier2004} $R(\theta)$ is estimated directly from the data, generally by averaging the signals from voxels with the highest diffusion anisotropy, and thus does not rely on a model of diffusion, although it does rely on the assumption that a single fibre population's response is uniform throughout the brain and for different fibre configurations.
To improve angular resolution and reduce noise sensitivity in the estimated \gls{fod}, a regularised implementation in which biophysically impossible negative $F(\theta,\phi)$ values are strongly penalised gave rise to the widely used \gls[noindex=false]{csd} method (Fig. \ref{fig:cross}).
\Gls{csd} can reconstruct \glspl{fod} resolving multiple crossing fibre populations with high angular resolution in a matter of seconds, from \gls{dmri} datasets with acquisition parameters achievable in routine imaging practice\autocite{Tournier2013}.
In addition to providing a means for estimating the \gls{fod} to a high degree of angular resolution, \gls{csd} also spurred exploration of new quantities relating to tissue microstructure, notably the interpretation of the \gls{fod} amplitude as a measure of intra-axonal volume fraction, or \gls[noindex=false]{afd} \autocite{Raffelt2012a}.
\Gls{afd} can be defined as a directional quantity or as a single scalar per voxel, obtained by integrating $F$ over the sphere.

By design, \gls{csd} presumes that the entire signal in a voxel can be explained by contributions from a number of fibre populations, each forming a highly restrictive environment in which diffusion is anisotropic.
Within pure and highly organised \gls{wm}, this is a reasonable assumption, however limitations become apparent outside of these areas.
As we have seen in previous sections, brain tissue covers a spectrum of cytoarchitectures and a corresponding diversity in diffusion environments.
At the typical resolutions of \gls{dmri} data, voxels may contain, in addition to axon fibres, signal contributions from \gls{csf} or \gls{gm}, which are typically characterised by isotropic and freer diffusion.
The result of such partial volume effects are highly noisy \gls{fod} estimates with spurious peaks, overestimation of \gls{fod} peak amplitudes and overestimation of \gls{afd} in affected voxels\autocite{Jeurissen2014}.
In addition, the original \gls{csd} method was designed only for data acquired with a single diffusion weighting $b$ factor (``shell''), unable to take advantage of the additional information contained in more advanced and increasingly popular multi-shell acquisitions.
To address these limitations, an extension proposed in \textcite{Jeurissen2014} and referred to as \gls[noindex=false]{msmt} \gls{csd} includes support for multiple $b$-value shells and separation of the signal into contributions from different tissue compartments, typically \gls{wm}, \gls{gm}, and \gls{csf}, each with their own characteristic response functions.
The resulting \gls{fod} estimates have higher angular precision than \gls[noindex=false]{ssst} \gls{csd}, with fewer noisy spurious lobes to confound downstream processing and interpretation.
Furthermore, separation of the isotropic signal contributions greatly improves the interpretation of \gls{fod} amplitude as a measure of \gls{afd}, as the amplitudes of each tissue's deconvolved \gls{odf} closely correspond to their respective tissue volume fractions\autocite{Jeurissen2014}.

\section{Streamline tractography}\label{sec:tractography}



Up to this point, we have only discussed the processing and analysis of \gls{dmri} data, and what it can or cannot reveal about \gls{wm} microstructure, at the level of individual image voxels.
However, the axons that form fibre tracts can traverse 100s of millimetres, and with the unique ability to measure fibre orientations in each voxel it wasn't long before this information was being exploited to reconstruct axonal connections in their entirety.
The basic principle is one of treating fibre orientations as a brain-wide vector field through which the paths of virtual neural fibres can be traced in a process called streamline tractography (or simply tractography, Fig. \ref{fig:tracking}).
It is vital to note here that the paths of individual \textit{in vivo} axons are entirely indiscernible from \gls{dmri}.
Tractography streamlines (or ``tracks'') are entirely abstract mathematical objects, each a collection of vertices, which aim to capture the \textit{potential} pathways of axons consistent with the observed \glspl{fod} representing an ensemble of thousands of axons.
Tractography is an immensely powerful and useful tool, and at the same time full of flaws due to this abstract and indirect nature.
All tractography algorithms consist at their core of the following steps:

\begin{lstlisting}[language=bash,label={lst:track},frame=single]
streamlines = []
while length(streamlines) < N do
  streamline = [get_seed_vertex()]
  STOP = false;
  while not STOP
    vertex = streamline[end]
    v = get_local_direction(vertex)
    new_vertex = vertex + step_size*v
    append(streamline, new_vertex)
    STOP = evaluate_stop(new_vertex)
  append(streamlines, streamline)
\end{lstlisting}

Once a predefined number N of streamlines which fulfil all selection criteria have been generated, tracking is terminated, and the resulting streamlines, each consisting of a set of vertices in 3D space, may be further analysed or visualised as required.
Volumetric images can also be generated by computing the number of streamlines traversing each voxel on a predetermined grid, a technique referred to as \gls[noindex=false]{tdi}\autocite{Calamante2010}.

\begin{figure}
  \includegraphics[width=\textwidth]{chapter_2/streamlines.png}
  \caption[Streamline tractography]{Streamline tracking involves tracing trajectories through a vector field of inferred dominant fibre orientations. Two example streamlines, from the \glsentrylong{cst} (blue) and \glsentrylong{cc} (red) are depicted in a coronal slice of the human brain. Figure reproduced from \textcite{Jeurissen2019} licensed under \href{https://creativecommons.org/licenses/by-nc/4.0/}{CC-BY-NC 4.0}.}
  \label{fig:tracking}
\end{figure}

Within the simple algorithm outlined above are a plethora of parameters and decisions which have transformative effects on the result.
They are apparent in the undefined functions such as \verb|get_local_direction()| or \verb|get_seed_vertex()| and the scalar parameters \verb|step_size| and \verb|N|.
The seed location, step length, conditions for terminating or entirely rejecting streamlines, interpolation of the surrounding vector field, number of streamlines to generate, are all choices to be made by the user (although in practice many parameters will be automatically determined or set to default values by the chosen algorithm).
There are two choices, though, that most fundamentally affect the tractography process and which feature most heavily in discussions on its use.

First is the choice of method for representing orientation information from the underlying data.
The earliest tractography algorithms were developed almost concurrently with \gls{dti}\autocite{Mori1998,Mori1999}, with the orientation vector field constructed from the principal eigenvectors of fitted diffusion tensors.
In the \gls[noindex=false]{fact} algorithm proposed by \textcite{Mori1999}, the local direction for each vertex is assigned from $\bm{\lambda}_1$ of the current voxel, and the stop criterion is a measure of neighbourhood fibre collinearity falling below a predefined threshold.
With only a single possible propagation direction at each location, which, as discussed above, may throughout much of the brain have little to do with any true axon orientations at that point, \gls{dt}-based tractography can only track fibre pathways with rather limited accuracy.
Streamlines may continue happily along a physiologically plausible path until encountering a region of intersecting tracts, at which point it may be prematurely terminated or diverted onto the trajectory of this intersecting tract, if a continuation of the current path is entirely unsupported by the principal eigenvector field.

It becomes clear when considering what we learned about neural connections in Section \ref{sec:hodology}, about compact fibre bundles diverging to distributed cortical targets, that a single dominant fibre direction at every point is incompatible with the dynamic organisation of \gls{wm} tracts, and the result is a tendency to reconstruct narrow and incomplete fibre bundles\autocite{Farquharson2013}.
A well-cited example of this limitation can be seen in reconstructions of the \gls[noindex=false]{cst}, which arises from the entire motor cortex from the apex down to the Sylvian fissure, but which is rendered by \gls{dt}-based algorithms only as a vertical pathway without any lateral projections.
Against this background, tractography based on higher-order fibre orientation models represents a vast improvement in the ability to contend with tracking complex configurations of \gls{wm}.
Now when two perpendicular tracts occupy the same voxel, the possible tracking direction is not limited to either that of the dominant bundle, such that tracking the smaller one is impossible, or of an average of the two, such that neither is properly represented.
However, with the flexibility of multi-peak distributions comes ambiguity, as when there are multiple distinct possible directions in which to propagate the streamline at any position, the decision of which direction to take becomes far more complex.
While tensor-based tractography may be particularly prone to false negatives, or neglecting certain pathways, multi-peak tractography can easily produce false-positives by hopping onto the paths of intersecting tracts.

The second significant distinction is between deterministic and probabilistic tracking approaches.
In deterministic tractography, there is only one single direction in which a streamline can be propagated from any given point, and two streamlines seeded in exactly the same location will be identical.
But the certainty implied by this deterministic approach is at significant odds with the reality that tractography operates in a domain and resolution far removed from that of individual axons.
Probabilistic tractography algorithms are here to acknowledge the uncertainty inherent in the tracking process.
Numerous probabilistic tracking algorithms have been developed, and while the end effect is essentially the same, whereby the next step direction is sampled from a probability distribution instead of deterministically selected, and seeding in the exact same location will not give rise to identical streamlines, there are two subtly different schools in what sort of uncertainty is being considered\autocite{Jeurissen2019}.
One considers the \textit{measurement} uncertainty of the calculated orientations.
Under this approach, the general direction to take is not under question, but the accuracy of that direction is.
It holds that, due to noise and inherent limitations in our measurement equipment and signal modelling, the fibre orientations can only be calculated with limited accuracy, and the tracking directions are sampled from a distribution reflecting this measurement uncertainty.
The probabilistic algorithm probtrackx\autocite{Behrens2007}, based on the ball-and-sticks fibre orientation model\autocite{Behrens2003}, is a notable example of this approach.

A second school takes the view that uncertainty in the choice of streamline step direction stems from the obscurity of the underlying physiological reality, and sampling a direction from the fibre \gls{odf} reflects that microstructural complexity.
Crucially though, tracking is not proceeding under any guidance relating to real biophysical connections, and though a streamline's direction in a given voxel may well be in accordance with real axons, whether that direction is appropriate in the context of the preceding steps of the same streamline is unresolvable.
In other words, \gls{fod}-based probabilistic tractography, of which first or second-order integration over \glspl{fod}  (iFOD1/iFOD2)\autocite{Tournier2012,Tournier2010} are notable examples, can capture the local dispersion of fibres in high detail, but that doesn't necessarily translate to long-range accuracy.
It is possible to constrain tractography according to broad heuristics about fibre tract geometries, but inevitably such simplifications will not be globally applicable.
For example, strategies to prevent streamlines from ``hopping'' onto intersecting, but not physically connected, pathways can include placing upper limits on the angle between successive steps under the expectation that most tracts will carry more or less straight on, but there are plenty of tracts in the brain with regions of high curvature, which become much harder to accurately reconstruct if the ``straight ahead'' constraint is too strict.

Due to streamline tractography's locally oriented and step-by-step nature, errors and missteps accumulate rapidly with little to no opportunity to correct them, resulting in some wildly implausible streamlines.
Attempts to address this blindness to biophysical reality are at the focus of much of modern tractography research\autocite{Bastiani2017,Rheault2019,Aydogan2021}, as the consequences of these ongoing challenges to connectivity research and neurology are substantial\autocite{Schilling2019, Yang2021, Grisot2021}.
Nevertheless, the significance of tractography to modern neuroscientific advancement cannot be overstated, as it has opened up new insights and visualisations of the whole brain \gls{wm} network \textit{in vivo} and in astonishing detail (Fig. \ref{fig:corwm}).

\begin{SCfigure}
  \includegraphics[width=0.5\textwidth]{chapter_1/coronalwm.png}
  \caption[dMRI white matter imaging and tractography]{Coronal section of the author's white matter, imaged with \glsentryshort{csd} \glsentryshort{dec} mapping and streamline tractography. The \glsentrylongpl{cst} are visible radiating from the cortex to converge in the internal capsules before descending through the anterior pons, perpendicular to the middle cerebellar peduncle fibres. Either side of the pons, the trigeminal nerves (cranial nerve V) are visible as small green dots.}
  \label{fig:corwm}
\end{SCfigure}

\section{Segmenting white matter tracts with tractography}

The functional division of \gls{wm} into distinct tracts is of great consequence to neuroscience, psychology and neurology in their efforts to analyse brain structure and function, and as we will see later, identifying tracts is also of vital importance in neurosurgery.
It follows that the spatial delineation of individual tracts is a key step in many \gls{dmri} analysis pipelines.

Streamline tractography was the first, and remains the dominant answer to this task.
Though the field is wide and the specific approaches numerous, we will outline the two main frameworks through which individual tract segmentations are derived using streamline tractography.
The first, sometimes dubbed ``virtual fibre dissection'', involves generating a large number (on the order of 10s of millions) of streamlines, usually covering the entire brain, followed by a selection process whereby streamlines are assigned to a tract of interest or discarded.
Streamline tracking proceeds virtually uninhibited, terminating only if a maximum length is reached or when leaving the \gls{wm} (as indicated by tissue segmentations\autocite{Smith2012} or an \gls{fod} amplitude threshold).
After tracking, one approach for selecting streamlines belonging to the target bundle is to use logical \glspl[noindex=false]{roi}, specifying inclusion volumes which must be visited and exclusion volumes to filter unwanted tracks.
These selection and exclusion \glspl{roi} encapsulates our \textit{a priori} neuroanatomical knowledge, and the resulting bundle, comprising only those streamlines fulfilling the criteria, represents the segmented tract (Fig. \ref{fig:tg_rois}).
The streamlines may be viewed as three-dimensional objects, or further processed into volumetric streamline density maps\autocite{Calamante2010} and thresholded binary segmentations.

Alternatives to \gls{roi}-based selection are clustering methods, which classify streamlines according to their proximity or similarity to each other, or other geometrical properties.
RecoBundles\autocite{Garyfallidis2018}, White Matter Analysis \autocite{ODonnell2017, ODonnell2007}, atlas based adaptive clustering \autocite{Tunc2014}, and example-based automatic tract labelling \autocite{Yoo2015} are all examples of data driven, group-wise streamline clustering and matching approaches.
They typically rely on registration of example data or streamline atlases based on which similar streamlines are recognised in the target data and labelled accordingly.
Streamline clustering methods have been shown to generate more consistent and reproducible results across subjects compared to \gls{roi}-based segmentation\autocite{Sydnor2018}.
Another recently proposed approach, named Classifyber, uses a learned linear classification of streamline features to label streamlines belonging to the target bundle in a new subject \autocite{Berto2021}.
In all clustering approaches, the necessary generation of whole brain tractograms in test subjects and the additional construction of example or reference tractography data present barriers to application, as well as, in some cases, long processing times and high memory requirements\autocite{Wasserthal2018}.

The whole brain approach is computationally extremely wasteful, as the vast number of streamlines generated will not even represent \textit{any} anatomically valid pathway through the brain, let alone one belonging to the tract of interest.
Furthermore, if streamlines are randomly seeded throughout the brain, then longer tracts covering a larger volume are more likely to be sampled, which, together with the tendency to continue straight along a ``path of least resistance'' at diverging or crossing fibres, results in inordinate overrepresentation of certain pathways\autocite{Smith2013}.
All this means that, after perhaps hours of tracking and billions of streamlines created, only a handful may be included in a final bundle reconstruction.

\begin{SCfigure}
  \includegraphics[width=0.5\textwidth]{chapter_2/tg_rois_glass.png}
  \caption[Virtual WM tract dissection with streamline tractography]{\Glsentrylong{wm} tracts are virtually dissected with streamline tractography and anatomically informed \glsentrylongpl{roi} (\glsentryshortpl{roi}). In this toy example, streamlines for the \glsentrylong{cst} are seeded in the cerebral peduncles (blue ring) and selected with an inclusion \glsentryshort{roi} in the posterior limb of the internal capsule (green ring). Streamlines following the paths of the \glsentrylong{cc} or cerebellar peduncles are excluded (red rings).}
  \label{fig:tg_rois}
\end{SCfigure}

The second approach may be called ``targeted tractography'', and involves only seeding streamlines in a tract-specific \gls{roi} and retaining those that fulfil anatomical selection criteria, provided as additional inclusion and exclusion regions (Fig. \ref{fig:tg_rois}), until a target number of streamlines have been selected.
A seed region can but does not necessarily have to be placed at one of the actual anatomical ends of the tract, and in some cases it makes more sense to seed from the middle of the tract and propagate bidirectionally, placing additional include regions at the ends to ensure complete coverage.
This approach does not mean that no streamlines are discarded (seeded fibres may be terminated before fulfilling all inclusion criteria, or stray into exclusion regions) but targeted seeding and selection certainly leads to a higher number of admissible streamlines being generated in far less computational time than in the whole brain approach, while discarding unwanted streamlines on the fly reduces storage requirements.
Targeted tractography is the more common approach particularly in applications where only a few or even just a single tract are relevant, such as in neurosurgery\autocite{Yang2021}.

Manual placement of \glspl{roi} in both whole-brain and targeted pipelines represents a significant intellectual burden, relies on expert anatomical knowledge and can be extremely time consuming, so it is sometimes automated by registering structural atlases and defining tracts in terms of logical relations to atlas structures, as in TractQuerier\autocite{Wassermann2016}, a similar proposed method using fuzzy logic\autocite{Delmonte2019}, and Tracula \autocite{Yendiki2011}, or pre-defined \glspl{roi}, as in XTRACT \autocite{Warrington2020}.
The former three are examples of methods that rely on comprehensive cortical parcellations, typically obtained with a software tool such as FreeSurfer\autocite{Desikan2006,FischlSalat2002} which can take many hours to run.
Seed placement may also be optimised using reference tracts, as in probabilistic neighbourhood tractography \autocite{Clayden2006,Clayden2009}.
In many scenarios, manual \gls{roi} placement remains the default method, particularly in clinical contexts where automatic \gls{roi} registration or segmentation may fail due to pathology.
Here, not only is a good understanding of the anatomy of a tract vital to produce high quality reconstructions, but the user will also need to understand the biases and pitfalls of their chosen \gls{fod} model and tractography algorithm to ensure proper interpretation and qualification of the results\autocite{Rheault2020,Rheault2022}.
Even to an experienced user, producing quality bundles is often time-consuming and tedious.
While modern research applications and increasingly more clinical applications almost exclusively favour probabilistic and multi-fibre \gls{odf} algorithms thanks to higher sensitivity to complex fibre configurations\autocite{Yang2021}, an inevitable trade-off is a high prevalence of false positive streamlines, representing either irrelevant or unphysical connections.
Filtering out these unwanted streamlines remains a considerable challenge\autocite{Jorgens2021}.
Attempts to reduce their creation in the first instance include injecting more anatomical priors into the tracking process, such as by modifying \glspl{fod} to favour the directions associated with the target tract\autocite{Rheault2019}, or by using directional \glspl{roi} for particularly tricky geometries \autocite{Chamberland2017}, or designing alternatives to the piece-wise linear tracking paradigm that aim to generate streamlines with more anatomical plausibility \autocite{Schomburg2017,Aydogan2021}.
Finally, due to a combination of the different computational methods available, and a general lack of consensus on the precise anatomical extents of many commonly reconstructed pathways, tractography suffers from notoriously low reproducibility\autocite{Schilling2021a}.

In view of these limitations, some in the field are continuing efforts to improve streamline tractography with novel tracking algorithms, finding new ways to incorporate anatomical priors, and developing more powerful streamline filtering, clustering and selection strategies.
Others are looking towards \gls{wm} segmentation solutions that do not rely on tractography at the point of application, but instead produce voxel-wise tract segmentations directly from \gls{dmri} or \gls{fod} data.

\section{Streamline-free white matter tract imaging}

There have been numerous works addressing the \gls{wm} tract identification task as a classic voxel-wise segmentation problem, utilising techniques including multi-label supervised clustering \autocite{Ratnarajah2014}, level-set and front propagation\autocite{Nazem-Zadeh2011, Hao2014}, and deep learning for direct segmentation from fibre orientation representations \autocite{Wasserthal2018,Li2020}.
Typically, direct methods require some number of samples with which to train a classifier, atlas, Bayesian model or neural network.

In \textcite{Hagler2009}, a fibre location and orientation atlas is created by averaging the \gls{dt} and tractography-derived information from multiple subjects and subsequently used to estimate the voxel-wise \textit{a posteriori} tract probability in a test subject.
As orientation information was encoded by averaging \gls{dt} principal eigenvectors across subjects this approach is not optimised for crossing fibre configurations.
The spatial probability was given by the averaged, normalised track density values from individual deterministic streamline tractography, although tracking biases discussed above mean that equating streamline density with likelihood of tract location is problematic\autocite{Rheault2019,Smith2013}.
\textcite{Bazin2011} also proposed a direct approach based on diffusion tensor-derived priors (``Diffusion-Oriented Tract Segmentation'', or DOTS) also based on \gls{dt} modelling.
Here the atlas orientation prior consisted of a single principal direction per voxel, and comparisons with the test subject data are made using Markov random field models and neighbouring tensor connectivity.

More recent developments have made use of advances in data science techniques including deep learning segmentation models, of which TractSeg\autocite{Wasserthal2018} and Neuro4Neuro\autocite{Li2020} are notable examples, using \gls{fod} peaks and diffusion tensors as inputs, respectively.
Deep learning-based approaches have the advantage of producing highly reproducible results in short processing time, without the need for template or atlas registration.
However, drawbacks of direct, deep learning-based methods which produce binary segmentations include a lack of explainability, and a dependence on large volumes of annotated training data which are labour-intensive to produce.
This limits their flexibility:
If a user requires a tract segmentation which is either anatomically different or not covered by an existing pre-trained model, then the necessary production of new training data and subsequent model training represents a high logistical and computational barrier.

Inference models trained on large volumes of healthy data may not be entirely robust to pathologies, particularly those causing significant topological changes.
In addition to healthy data, Neuro4Neuro\autocite{Li2020} was validated only in a dementia dataset, and TractSeg\autocite{Wasserthal2018} was qualitatively evaluated in schizophrenia and autism datasets in the original work, and subsequently in a tumour dataset with mostly successful results, with more complete segmentations in cases with minimally deforming tumours\autocite{Richards2021}.
In \textcite{Moshe2022}, the authors trained their own TractSeg model, on approximately 500 datasets, to segment the \gls{cst} in brain tumour patients.
The results were more reproducible than for the compared manual method, and obtained an average \gls{dice} of 0.64, almost 25\% worse than the performance in healthy data reported in the original TractSeg study (for the same tract).
The authors cite a lack of reliable and sufficient labelled training data as a reason for limiting their study to a single tract, despite the importance of other tracts in preoperative fibre mapping.
In summary, despite promising recent progress, there remains a lack of robust, streamline-free \gls{wm} imaging tools in which clinical applicability is a key feature, rather than simply an after-thought.
\clearpage{}
\clearpage{}
\chapter{Neurosurgery}\label{chap:neurosurgery}




Many types of interventions fall under the remit of cranial and spinal neurosurgery, including inserting electrodes for \gls{dbs}, diagnostic biopsies, vascular procedures, and insertion of \gls{csf} shunts.
In all cases, precision is paramount and tools for accurate navigation form vital components of the surgical workflow.
The following review, parts of which first appeared in abbreviated form in \textcite{Young2024}, will focus on some of the most complex and invasive procedures, involving craniotomy and removal of tumours and epileptogenic brain tissue.
In England between 2013--2018, oncological procedures were the third most common of all neurosurgery subspecialties comprising approximately 9\% of total, while functional neurosurgeries (including for epilepsy and deep brain stimulation) made up 8\%\autocite{Wahba2022}.
At \gls[noindex=false]{gosh} in London, a leading paediatric centre, 10\% of neurosurgical procedures between 2018--2022 were tumour-related, while 12\% were for epilepsy\autocite{gosh2023}.

Invasive brain tumour operations are both highly complex and variable in their neurophysiology, microbiology, treatment plans, and prognosis.
Such diversity presents a significant barrier to the development of image processing methods intended for generalised use in tumour patients\autocite{Bauer2013}.
Neoplasms occur throughout the brain, with the location having unique impact on surrounding structures and associated function.
A tumour's natural history and histopathology also play a large role in determining its effects on its environment.
Malignant gliomas, a category of tumours arising from glial cells, often have complex structures, with infiltrating components and peritumoural oedema blurring the distinction between tumour and non-tumour tissue\autocite{Weller2021}.
On the other hand, many non-malignant tumours, including most meningiomas and low-grade astrocytomas, are encapsulated, with clear demarcation from neighbouring brain tissues, which are displaced rather than infiltrated\autocite{Lu2004,Gerard2017}.

This project is concerned with the visualisation of cerebral white matter tracts, and therefore this review will focus on those indications and interventions in which damage to and navigation around such structures is of particular concern.
This is typically not the case for posterior fossa and suprasellar lesions, although there is growing interest in the role of the cerebellum in wider cognition and the brain functional network, and the surgical community is paying increasing attention on the effects of posterior fossa surgery on cerebellar tracts\autocite{Toescu2021,Skye2023}.
Brainstem tumours are often not candidates for surgical removal due to their eloquent location, with limited access and excessive risk to vital brainstem function.
The discussions in this section therefore apply primarily to supratentorial lesions in the cerebral hemispheres and thalamus, candidates for biopsy or resection via craniotomy.
The most common intracranial tumours are meningiomas, arising from the protective membranes surrounding the \gls{cns}, of which the majority are benign with good overall survival rates and relatively low-risk surgical treatment options\autocite{Rogers2015,Spena2022}.
Far more complex and controversial are decisions surrounding the surgical treatment of gliomas, those tumours originating in glial cells including astrocytes and oligodendrocytes which are \DIFaddbegin \DIFadd{among }\DIFaddend the most common malignant primary \gls{cns} neoplasms in both adults\autocite{Ostrom2015,Wanis2021} and children\autocite{Ostrom2015,Bauchet2009}.
While challenges regarding functional preservation and optimal surgical strategy are relevant to all intracranial surgeries, they are acutely highlighted within the context of glioma literature.
These tumours can embed themselves insidiously within the brain's functional architecture with devastating prognosis, challenging oncologists and surgeons with stark dilemmas in their bid to maximise both patient survival and quality of life.
Gliomas are classified into four \gls[noindex=false]{who} grades, commonly split into \gls[noindex=false]{lgg} (\gls{who} grades 1--2) and \gls[noindex=false]{hgg} (\gls{who} grades 3--4) to reflect differences is malignancy and prognosis.
There are many subtypes based on histological and genetic characteristics which are periodically updated\autocite{Louis2021}, but this overview will focus on the broad categories of \gls{hgg} and \gls{lgg}.

\section{Extent of resection}\label{sec:eor}

Complete removal of all pathological tissue, perhaps counterintuitively, is not always the surgical objective.
Though it may in many cases be the ideal outcome from an oncological perspective, this scenario would frequently be in conflict with other equally important outcome indicators, such as the preservation of surrounding brain structures and the patient's neurological wellbeing.
Successfully balancing these consequences is a central dilemma in neurosurgical practice, with the key measure being \gls[noindex=false]{eor}, the amount of tumour removed.
In theory, \gls{eor} is a straightforward concept, but in practice it is ill-defined and inconsistently reported, while remaining central to studies of surgical efficacy and outcomes.

Easily defined in oncology\footnote[2]{\gls{eor} is relevant to epilepsy surgery, although the terminology and calculations here are different. Epileptogenic centres cannot always be distinguished and measured on imaging, and functionally eloquent tissue may be the clear source of epileptic activity, and thus subject to removal.}
as either the absolute volume or relative percentage of tumour tissue removed, accurately and consistently determining \gls{eor} is very difficult.
It is often reported in terms of broad categories, the most common being biopsy, \gls[noindex=false]{str} or partial resection (PR), near total resection, \gls[noindex=false]{gtr} and supratotal resection\autocite{Wykes2021,Karschnia2021}.
\DIFdelbegin \DIFdel{There is also no general }\DIFdelend \DIFaddbegin \DIFadd{Until recently, no clear }\DIFaddend consensus on how these categories \DIFdelbegin \DIFdel{are defined }\DIFdelend \DIFaddbegin \DIFadd{should be defined existed}\DIFaddend , making comparison between \DIFaddbegin \DIFadd{past }\DIFaddend studies even more difficult\autocite{Karschnia2021}.
Many studies simply give very rough percentage values as estimated visually by the operating surgeon or a radiologist based on whether or not tumour residue is visible in the resection cavity or on a postoperative scan, with limited accuracy\autocite{Sanai2008,Martino2013,Lau2018,Sezer2020}.
Over time, the definitions for \gls{eor} have evolved with the availability of techniques for measuring it, and the current accepted standard for quantifying \gls{eor} is with volumetric measurement on pre- and postoperative imaging\autocite{Rincon-Torroella2019}, but here too practices are inconsistent\autocite{Wykes2021}.
Full volumetric analysis requires accurately segmenting the entire lesion, though sometimes \gls{eor} is calculated by simply taking the diameter of the lesion on a single or multiple slices, or with approximate ellipsoid segmentation\autocite{Sanai2008,Albuquerque2021}.
\DIFaddbegin \DIFadd{With the recent publication of new proposed volumetric resection categories in \textcite{Karschnia2021}, alongside evidence for improved prognostic value using this classification system \textcite{Karschnia2023}, the neurosurgical community will hopefully move towards more standardised reporting of tumour }\gls{eor} \DIFadd{in glioblastoma.
Even so, variations in volumetric analysis methods (e.g. segmentation software) are likely to continue to confound reliable comparisons between studies.
Furthermore, the categories defined in \textcite{Karschnia2021} are for glioblastoma resections only, while resection classification has yet to be standardised for many other tumour types, including medulloblastoma}\autocite{Thompson2018}\DIFadd{, one of the most common malignant childhood brain tumours.
}\DIFaddend 


Even manual delineation can be unreliable and inconsistent, especially for tumours with poorly defined borders and on postoperative imaging \autocite{Ertl-Wagner2009,Bo2017,Visser2019}.
Semi- or fully automatic segmentation improves reproducibility\autocite{Ertl-Wagner2009,Sezer2020} and modern algorithms are proving ever more accurate, although there are still challenges regarding computational performance and robust clinical translation\autocite{Angulakshmi2017,Wadhwa2019,Fawzi2021}.
Estimates of \gls{eor} can also be compromised by post-operative brain tissue shifting and obscuring the actual volume of resected tumour \autocite{Schucht2014a}, while microscopic tumour cell invasion means that complete resection, as viewed either on imaging or by intraoperative visual assessment, does not necessarily mean no tumour residue remains\autocite{Yordanova2017}.
Finally, while there is the most focus on reporting relative reductions in tumour volume as a percentage of original size, more recent studies have argued that absolute residual tumour volume is as, if not more relevant for determining postoperative outcomes\autocite{Ius2012,Rincon-Torroella2019,Smith2008,Karschnia2021}.

\section{Oncological and neurological outcomes: Necessarily in opposition?}

Inconsistent reporting of \gls{eor} is one factor complicating the study of its effects on clinical outcomes, even as there is widespread agreement on the importance of studying those effects\autocite{Rincon-Torroella2019,Wykes2021,Weller2021}.
Broadly speaking, \gls{gtr} has been shown to increase overall and progression free survival over \gls{str} across age groups in both high \autocite{Hatoum2022, Han2020, Adams2016, McCrea2015, Bloch2012, McGirt2009, Kramm2006} and low-grade \autocite{Keles2001, Pollack1995, Sanai2008} gliomas.
For \gls{lgg}, and especially in paediatric patients, \gls{gtr} has become the recommended standard of care, as complete resection leads to a lower rate of recurrence\autocite{Berger1994,Claus2005}.
In particular, maximal resection of \glspl{lgg} drastically reduces the risk of residual tumour evolving into \gls{hgg} (known as malignant transformation \autocite{Duffau2013,Hervey-Jumper2016,Rincon-Torroella2019}), though this is only a concern in adult patients, as malignant transformation in paediatric \glspl{lgg} is exceedingly rare\autocite{Collins2020}.
More recent voices have even argued for supratotal resection, beyond the margins of any abnormally enhancing areas on $T_1$-weighted and FLAIR $T_2$-weighted \gls{mri} scans, as reviewed in \textcite{deLeeuw2019}.
There is limited evidence, though controversial, to suggest that supratotal resection of \gls{who} grade 2 gliomas in adults is followed by fewer cases of malignant transformation and improved progression-free survival \autocite{Yordanova2011}.

However, due to a general lack of prospective randomisation and robust comparison with appropriately matched controls, drawing definitive conclusions from studies investigating the effects of \gls{eor} (or other surgical variables) on post-operative outcomes is contentious\autocite{deLeeuw2019,Keles2001}.
Results may be confounded by selection biases, for example, different tumour histological subtypes may lend themselves more or less easily to greater \gls{eor}, or arise more frequently in eloquent areas of the brain (which include cortex and subcortical \gls{wm} subserving language, motor, and sensory functions, as well as the thalamus, midline structures involved in memory processing, and the brain stem), where an aggressive surgical strategy is likely to be discounted\autocite{deLeeuw2019}.
Adult \glspl{lgg} tend to occur more frequently than \glspl{hgg} in highly eloquent cortical regions\autocite{Duffau2004}, indeed the control group for the supratotal \gls{lgg} study\autocite{Yordanova2011} mentioned above consisted of patients whose gliomas were located in eloquent brain areas, and who therefore underwent only \gls{gtr}.
One might therefore expect supratotal resection to be associated with worse postoperative neurological outcomes, and indeed \textcite{Rossi2019a} found higher probabilities of immediate postoperative deficits in supratotal versus total resection of \glspl{lgg}.
These were however significantly reversed at three month and one year follow-ups, and initial overall evidence suggests that neuropsychological outcomes are comparable between total and supratotal groups\autocite{Tabor2021}.

In adults with glioma, maximal safe resection, combined with adjuvant radio- and chemotherapy, has been the established standard of care for some time.
It has been less clear, however, whether the same should apply to children.
The most prevalent anatomical locations in which gliomas arise may differ between adults and children\autocite{Duffau2004}.
Thalamic gliomas, for example, are more frequent in children than in adults\autocite{Cinalli2018,Palmisciano2021,GomezVecchio2021}:
Adult gliomas are usually located in the hemispheres, mostly the frontal lobe, with only approximately 4--7\%\autocite{GomezVecchio2021,Larjavaara2007} situated in the thalamus, while as many as 19\% of paediatric \glspl{hgg} are thalamic\autocite{McCrea2015}.
There is also concern that oncological differences between adult and paediatric type gliomas preclude safe extrapolation of treatment plans from one patient group to the other\autocite{Jones2012,Greuter2021}.
In addition to paediatric tumours frequently arising in high-risk areas such as the thalamus and brain stem \autocite{Ostrom2015}, neurocognitive and functional preservation is an especially critical concern in children.
\DIFaddbegin \DIFadd{As the developing brain of young children is especially sensitive to radiation-induced neurocognitive injury, the use of high-dose adjuvant radiotherapy may be deferred for as long as possible in paediatric patients to avoid subsequent life-long impairment}\autocite{Padovani2012,Moxon-Emre2014}\DIFadd{.
As such, the importance of }\gls{eor} \DIFadd{in children gains additional significance where residual unresected tumour cannot be neutralised with adjuvant radiotherapy.
}\DIFaddend A meta-analysis published in 2022 analysed 37 articles to assess the association between \gls{eor} and survival in paediatric patients with \gls{hgg}\autocite{Hatoum2022}.
Notwithstanding the difficulties in consistently defining and reporting \gls{eor} as discussed above, the study found strong evidence for improved overall survival in \gls{gtr} over \gls{str} of gliomas located in the cerebral hemispheres, but no association between \gls{eor} and survival was observed in midline cases.
The authors emphasise that midline (thalamic and brain stem) gliomas are not often indicated for aggressive resection due to the elevated risk to critical neurological function, and the lack of observed association may stem from measurement biases, including lower sample size and the pooling of histologically distinct tumour types which may respond differently to treatment.
Moreover, no comparison was made for postoperative functional neurological outcomes, thus failing to capture the full picture of factors contributing to a decision to pursue radical surgery.

New postoperative neurological deficits occur in over one third of glioma surgeries \autocite{Zetterling2020a}, although most patients improve significantly over longer-term follow-up.
Unsurprisingly, higher chances of postoperative deficits were associated with higher \gls{eor} and with tumours situated in eloquent areas\autocite{Zetterling2020a}.
In \textcite{Gil-Robles2010}, authors argue for a more conservative resection margin in \gls{who} grade 2 gliomas to protect functional structures, yet current consensus recommends total resection in \gls{lgg} wherever possible\autocite{Rincon-Torroella2019,Albuquerque2021}. For the most malignant tumour types, even maximal resection combined with adjuvant therapy is rarely curative, and may only lead to a survival advantage of just a few months\autocite{Rincon-Torroella2019,Karschnia2023}.
Given the overall poor survival outcomes associated with aggressive gliomas, oftentimes the risks to quality of life and postoperative neurological function associated with pursuing \gls{gtr} outweigh any potential oncological benefits\autocite{Rahman2016,Tabor2021}.
In \glspl{hgg}, the justification for \gls{gtr} or even supratotal resection is weaker than in \gls{lgg}, given that it cannot secure long-term survival for affected patients.
Where radical resection carries no likely oncological benefit and is contraindicated by a high functional risk to eloquent areas, the goal of surgery may be conservative debulking of the lesion to relieve pressure on the brain and reduce current neurological symptoms.
With the widespread evidence of an oncological advantage associated with more extensive resection, physicians have increasingly advocated for \gls{gtr} as the standard treatment for \gls{lgg} and maximal safe resection for \gls{hgg} \autocite{Rincon-Torroella2019}.
But this comes with the caveat that tumours of lower malignancy are also often those found to be more operable, muddying the causal link between overall survival and extent of resection\autocite{Weller2021}.
Furthermore, post-operative neurological deficits, due to cortical and subcortical injury, themselves have a negative impact on overall survival, independently of differences in pre-operative symptoms \autocite{Trinh2013,Rahman2016,Rincon-Torroella2019}.
Hence even if only overall survival is considered as the measure of treatment success, the evidence that both neurological injury and un-resected tumour negatively impact survival highlights the dilemma of surgically treating tumours in eloquent brain regions\autocite{Rincon-Torroella2019,Duffau2004,Rahman2016}.
The European Association of Neuro-Onocology's recommendation, as of 2021, is that prevention of new neurological deficits should be prioritised over maximal extent of resection in the surgical treatment of gliomas\autocite{Weller2021}.

A further consideration on the feasibility of \gls{gtr} or supratotal resection is neuroplasticity\autocite{Duffau2005}.
Slowly growing, low-grade, or recurring tumours may lead to functional reorganisation of surrounding brain tissue\autocite{Takahashi2012,Southwell2016,Das2019} or compensatory recruitment of equivalent contralateral regions\autocite{Mitolo2022}, enabling the safe removal of a greater margin of tissue than would otherwise be accepted for eloquent areas\autocite{Rossi2019a}.
Current understanding of neuroplasticity and brain tumours is limited to a small number of case studies, and more systematic research into the mechanisms and robust detection of functional reorganisation are required before these findings can be put into routine clinical practice\autocite{Duffau2005,Abel2015,Satoer2017}.
Taken together with the emergence of a hodological framework for neurosurgery discussed in Section \ref{sec:hodology}\autocite{Sala2019}, improved study of neuroplasticity could gradually lead to wider applicability of total or supratotal resection without compromising on neurological function and postoperative quality of life.

Early neurological principles of rigid functional localisation formed the basis for the concepts of eloquence and tumour operability guiding neurosurgeons throughout much of recent decades.
The recent move towards a more individualised view has only been made possible through developments in imaging and functional monitoring tools, allowing clinical teams to adapt the surgical strategy to each unique brain-tumour system, rather than relying on received assumptions about functional organisation\autocite{Boerger2023}.
The next section will explore some of those advanced technologies instrumental in the planning and execution of state-of-the-art neurosurgical practice.



\section{Surgical planning and preoperative imaging}

Tumours can interact with their surroundings in a number of ways, depending on their nature and location.
Some tumours, including some \glspl{lgg} and meningiomas, are fully encapsulated and displace surrounding brain as they grow.
This strong demarcation between tumour and healthy tissue can facilitate surgical treatment and total removal of the tumour without undue risk to functioning neural tissue, but such lesions can cause neurological impairments as surrounding structures are stretched or compressed, leading to a recommendation for surgical removal of a tumour which poses an otherwise lower oncological threat.
Others cause almost no spatial displacement of brain tissue, with cancer cells instead invading the parenchyma and blurring the boundaries between disease and healthy brain.
Infiltrating tumours pose a particular surgical challenge due to the risk of surgical injury to eloquent tissue, and may only be conservatively debulked to relieve intracranial pressure and improve the effectiveness of adjuvant radiation or chemotherapy treatment.
In order to meet the goal of safely balancing maximal resection and functional preservation, the full tumour-brain interaction must be comprehensively mapped to determine the optimal resection margin.

Preoperatively, structural and functional non-invasive imaging are used for diagnosis and, if surgery is indicated, surgical planning.
At this stage the goal is to assess the spatial and functional relationships between involved and healthy tissues, map out a safe operative corridor to access the lesion, and determine the appropriate \gls{eor} under all considerations explored in the previous section.
Structural imaging with \gls{ct} and \gls{mri} provide critical anatomical information in high spatial detail.
Multi-contrast \gls{mri} examinations, including FLAIR, $T_1$-weighted and $T_2$-weighted imaging sequences, each provide unique contrasts for visualising different aspects of a tumour, such as necrotic and infiltrating regions, which can aid in determining tumour type, what \gls{eor} to aim for, or which region of the tumour to target for biopsy.
Angiography, detailed mapping of blood vessels with \gls{ct} and specialised \gls{mri} sequences, can also be employed for determining a tumour's vasculature and identifying major vessels involved\autocite{Kashimura2008,Kim2019}.
\Gls[noindex=false]{fmri} and navigated transcranial magnetic stimulation \autocite{WeissLucas2020} map out eloquent cortex lying in proximity to the lesion, including the motor, language, and sensory cortices, supplementing the purely structural data obtained from conventional \gls{mri} or \gls{ct}.
In epilepsy patients, \gls{eeg} may be employed to monitor activity in epileptogenic regions\autocite{Sarco2006}.

\Gls{dmri} is playing an increasingly important role for neurosurgical planning and navigation\autocite{Manan2022}, especially as focus moves from functional localisation towards viewing the brain as an interconnected network.
Tumour and \gls{wm} interactions are varied and can be difficult to distinguish on conventional contrast \gls{mri} alone.
Depending on the infiltrative nature of a lesion, \gls{wm} tracts may be displaced due to mass effect, invaded while remaining functionally intact, disrupted or destroyed, or experience a combination of effects\autocite{Essayed2017,DSouza2019,Manan2023}.
Diffusion \gls{dmri} can be instrumental in differentiating these circumstances and assessing tract integrity\autocite{Field2004,Manan2023}\DIFdelbegin \DIFdel{, but care must be taken to recognise how tumour effects may disturb diffusion patterns and impact the results}\DIFdelend \DIFaddbegin \DIFadd{.
Patterns of tract infiltration, displacement, and disruption can be described from the relative decreases and increases in }\gls{fa} \DIFadd{and }\gls{adc}\DIFadd{, together with changes in colour on }\gls{dec} \DIFadd{maps, as reviewed in \textcite{Manan2023}.
While }\gls{dmri} \DIFadd{can therefore reveal information about the tumour--}\gls{wm} \DIFadd{environment and inform surgical planning, it can also hinder accurate tract reconstruction, which often relies on expectations of normal diffusion patterns}\DIFaddend .
Peritumoural oedema, caused by \gls{bbb} break down and abnormal brain parenchyma fluid regulation\autocite{Ohmura2023}, and invading cells can lead to drastically altered diffusion signal measurements and reduction of anisotropy, complicating their interpretation and inhibiting accurate streamline tracking\autocite{Bulakbas2009,Nimsky2010,Kuhnt2013}.
Nevertheless, \gls{dti} and streamline tractography have brought dramatic improvements to neurosurgical planning, unlocking detailed visualisations of \gls{wm} tracts and their spatial relationships to the surgical target (Fig. \ref{fig:nav}) and even acting as a predictor of postoperative deficits with implications for preoperative patient counselling\autocite{Manan2022}.
This potential was recognised almost immediately, with \gls{dti} and early tractography quickly making their way into clinical practice in the early 2000s \autocite{Lee2001,Mori2002a,Nimsky2005}.
In the intervening years, research imaging has largely transitioned to the multi-fibre models and probabilistic algorithms described in Chapter \ref{chap:neuroimaging}, but in clinical practice tractography is frequently still based on \gls{dt} fibre orientation models \autocite{Toescu2020, Yang2021} and deterministic tracking algorithms.
As a notable example, popular neuronavigation platform provider Brainlab's iPlan\textregistered{} (single tensor)\autocite{Brainlab2012} and Elements (dual tensor)\autocite{Sollmann2020a} Fibre Tracking applications (Brainlab AG, Munich, Germany) use a version of the probabilistic \gls{fact} algorithm first proposed in \citeyear{Mori1999} \autocite{Mori1999}.

\begin{SCfigure}[][htb!]
  \includegraphics[width=0.5\textwidth]{chapter_2/neuronav.png}
  \caption[Streamline tractography for neurosurgical planning and navigation]{Idealised demonstration of tractography for neurosurgical planning and navigation in a paediatric patient with a left ependymoma (orange, outlined) involved with the \glsentrylong{or} (green) and \glsentrylong{cst} (blue). (Illustrative only, not a depiction of real clinical tractography.)}
  \label{fig:nav}
\end{SCfigure}

Regardless of the particular combination of fibre model, algorithm and tracking criteria, streamline tractography is compromised by weaknesses that can lead to flawed results or interpretations if not accounted for\autocite{Rheault2020, Schilling2022, Schilling2019}.
The same techniques and associated limitations for reconstructing individual \gls{wm} bundles already described apply here too, and can even be exasperated by additional tumour-related effects.
\Gls{dt}-based tractography, already afflicted by low sensitivity in healthy applications, often encounters particular difficulties tracking through oedema and areas of infiltration even where intact and functioning fibres may persist\autocite{Leclercq2010}, leading to missed connections and dangerous blind spots in the very regions at risk during surgery, where accurate navigation is most critical \autocite{Kuhnt2013,Ashmore2020}.
Meanwhile, the high propensity for false positive streamlines typical of probabilistic algorithms can be even more difficult to manage when tumour deformations disturb normal fibre orientations and inhibit accurate placement of \glspl{roi}\autocite{Yang2021}.

Perhaps clinical translation of multi-fibre probabilistic tractography has also been muted on account of its lower ease of use and practicality.
\Gls{dt} acquisitions can have as few as six diffusion-weighted directions, resulting in much shorter scan times compared to full \gls[noindex=false]{hardi} scans.
Deterministic tracking itself is rapid, and the placement of \glspl{roi} need not be as strict as with probabilistic tractography owing to a lower sensitivity to false positives \autocite{ODonnell2017}.
A general lack of availability of the necessary expertise and time limits neurosurgical centres' access to state-of-the-art tractography \autocite{Toescu2020}.
Until recently, commercially available neurosurgical navigation platforms have exclusively supported \gls{dt} modelling and deterministic tractography (a recent exception is the Medtronic Stealth\texttrademark{} S8 Tractography application (Medtronic, USA), which implements \gls{csd}-based tractography\autocite{Pozzilli2023} as well as \gls{dt}).
This lack of readily available alternatives in the neurosurgeon's workflow and certification for safe clinical use is undoubtedly a major factor in the persisting preference for deterministic methods in clinical practice.
Nonetheless, there is growing consensus that (pending appropriate regulatory approval) the clinical community ought to adopt probabilistic, non-\gls{dt} tractography\autocite{Yang2021, Beare2022, Petersen2017}.
There is evidence that this shift is gradually underway, at least in the context of presurgical planning\autocite{Toescu2020}, driven probably by a combination and feedback loop of growing demand and better availability and integration of advanced techniques into the clinical workflow.

\section{Neuronavigation and brain shift}

During the surgical procedure itself, multimodal information streams continue to guide the safest possible resection.
Functional monitoring with \gls[noindex=false]{des} is a crucial component of neurosurgical workflows and widely considered the gold standard for localising neural function after craniotomy.
Electrical current is applied to the cortical surface at increasing strengths\autocite{Saito2015}, and where stimulation elicits a functional response or disruption, the corresponding region is deemed eloquent.
Additionally, stimulation of subcortical white matter behind the resection cavity wall can be used to indicate when surgery should be halted as underlying eloquent structures are approached\autocite{Sala2019}.
\Gls{des} can be utilised in awake or asleep paradigms.
In the former, patients are awakened after craniotomy, and perform structured cognitive tasks involving those cortical hubs that may be at risk while undergoing electrical stimulation.
It is commonly considered for \gls{lgg} treatment within the UK, and to a lesser extent for \gls{hgg}\autocite{WykesV.2017}.
Language function is perhaps the most common target for awake stimulation as well as high-level motor tasks (such as playing an instrument) and vision\autocite{Mazerand2017}.
Awake surgery is technically complex, psycho-cognitively and emotionally demanding of the patient, and not universally tolerated\autocite{Nossek2013a,Wang2019}.
In very young children, awake surgery is rarely possible except in the most cooperative and resilient patients, and with appropriate preparation\autocite{Zolotova2022}.
\Gls{des} may also be performed in asleep patients, albeit limited to assessing motor and somatosensory function, where stimulation elicited somatosensory or motor evoked potentials (SSEPs, MEPs) can be measured in the patient's sensory cortex or muscles\autocite{Stone2019} (although the effects of anaesthesia can limit the sensitivity and accuracy of this approach\autocite{Stone2019,WeissLucas2020}).
There is also a considerable risk of intraoperative seizures, particularly in younger patients, which can lead to failure of awake functional mapping and potential increased risk of postoperative deficits\autocite{Nossek2013,Wang2019,Rigolo2020a}.
The neurological and oncological benefits associated with greater \gls{eor} discussed in Section \ref{sec:eor} have been achieved in large part with the help of awake functional mapping, primarily in adult patients.
It is therefore vital to develop and improve alternatives to cortical mapping to achieve maximal safe resection in all populations and especially in those patients who cannot undergo awake surgery, including some children.

Imaging and functional data acquired in preparation for surgery is not only instrumental to surgical planning, it also serves to guide the surgeon throughout the procedure, providing real-time multidimensional navigational information to supplement their live view through the surgical microscope.
Image guidance can involve simply displaying preoperative imaging and mapping in the theatre, while more advanced systems can also indicate the positions of surgical tools, or overlay imaging information on the microscope view.
This is achieved through stereotactic image guided surgery, which arrived with the introduction of frame-based systems in the later half of the 20th century, later largely giving way to frameless setups for craniotomies\autocite{Sandeman1995}, which are considered more time and cost effective\autocite{Sattur2019}.
In the former case, the patient's head is fixed within a stereotactic frame which guides the positioning of surgical tools, while in modern frameless systems the tools are tracked remotely, most commonly by an infrared camera system\autocite{Sattur2019}.
Fiducial markers, affixed either to the frame or patient, are detected on imaging and registered to the operating room coordinate system, allowing the tools' and patient's positions to be mapped and displayed on imaging in real time.
Intraoperative navigation with preoperative \gls{fmri} and \gls{dti} or tractography can improve \gls{eor} and preservation of critical cortical and subcortical function\autocite{Wu2007,Bello2008,Bello2010d}, particularly when combined with \gls{des} or awake surgery\autocite{Aibar-Duran2020}.
Where awake surgery is contraindicated or abandoned, preoperative functional mapping remains the only guidance available for higher cognitive functions, playing a crucial role in improving surgical care for patients who would not qualify for awake surgery.
\textcite{Rigolo2020a} found that preoperative \gls{fmri} guidance enabled safe resection of tumours or epileptic foci to continue after failed or incomplete \gls{des} function mapping, with no significant difference in postoperative morbidity.

Maximising \gls{eor} has further been significantly improved with the introduction of 5-aminolevulinic acid (5-ALA) guidance.
This compound is administered orally prior to surgery and is converted within cells to protoporphyrin IX (PPIX), which fluoresces when excited by short wavelength light.
Uptake of 5-ALA is highest in tumour cells\DIFaddbegin \DIFadd{, }\DIFaddend due to \gls{bbb} disruption, and the metabolic pathways producing PPIX are more active in tumour cells, allowing the surgeon to distinguish them from healthy tissue under the surgical microscope.
\DIFaddbegin \DIFadd{The fluorescence effect is particularly prominent in high-grade tumours, which are also often more difficult to distinguish from surrounding healthy tissue without the enhanced contrast provided by PPIX.
}\DIFaddend 5-ALA guided surgery results in improved \gls{eor} and higher progression free survival \DIFaddbegin \DIFadd{in high-grade gliomas}\DIFaddend , while maintaining preservation of functional tissue\autocite{Coburger2019,Golub2020}, and has seen widespread adoption in the surgical treatment of gliomas\autocite{Stummer2006} and inclusion in UK national care guidelines\autocite{NICE2021}.
\DIFaddbegin \DIFadd{Not all tumour types, however, can be better resected under 5-ALA guidance.
In particular, strong evidence for the usefulness of 5-ALA in common paediatric tumours, including }\glspl{lgg} \DIFadd{and medulloblastomas, is lacking, as are rigorous trials confirming it is safe to use in children}\autocite{Schwake2019}\DIFadd{.
As such, the 5-ALA remains useful in only limited subset of all brain tumour patients.
}\DIFaddend 

\begin{SCfigure}[][hp!]
  \includesvg[height=\textheight,pretex=\small\sffamily]{chapter_2/brain_shift.svg}
  \caption[Intraoperative brain shift]{Illustration of intraoperative brain shift and its effect on neuronavigation.
  \textbf{\sffamily a.} Preoperative imaging of a right \gls{who} grade 1 epidermoid lesion patient. Image is a $T_1$ weighted structural scan overlaid with \gls{csd}-derived \gls{dec} map. White arrowhead indicates medial displacement of the \gls{cst} (coloured blue/purple).
  \textbf{\sffamily b.} Intraoperative imaging with partially resected lesion. Brain shift has caused the \gls{cst} to relax laterally towards the craniotomy (white arrowhead).
  \textbf{\sffamily c.} Streamline tractography reconstructions of the \gls{cst} from preoperative (red) and intraoperative (green) \gls{dmri}, with areas of overlap in yellow. Note how the red streamlines give the impression of a tract further from the resection margin.}
  \label{fig:shift}
\end{SCfigure}

The dynamic conditions of brain surgery result in the unpredictable and often substantial movement, compression and deformation of tissue referred to as brain shift.
A range of factors contribute to this phenomenon, including \gls{csf} drainage, sagging due to gravity, decompression of tissue surrounding the resection cavity, swelling, craniotomy herniation and the effects of surgical instruments\autocite{Gerard2017}.
These factors may act in competing directions and combine in complex ways, for example swelling and tumour debulking can cause brain shift towards the craniotomy, while gravity and \gls{csf} drainage may have the opposite effect\autocite{Roberts1998}.
With the magnitude and direction of brain shift being so unpredictable, ranging from 1~mm to as much as 50~mm\autocite{Gerard2017}, accounting for it with predictive modelling is very difficult\autocite{Bayer2017b}.

Brain shift can affect the neurosurgeon's perception of the shape and location of the target lesion and invalidate preoperative imaging used for navigation\autocite{Nimsky2000}.
Many neurosurgeons rely on intuition to update their mental map of the surgical site throughout the procedure.
On more advanced neuronavigational platforms which integrate preoperative imaging and intraoperative data such as \gls{des} stimulation sites, brain shift can lead to misleading and inaccurate depictions of the spatial relationships between tumour and surrounding structures.
In particular, accurate localisation on image-guided navigation systems of deep tumour margins and the functionally eloquent structures beyond is significantly compromised by brain deformation (Fig. \ref{fig:shift}) and cannot be as easily mentally compensated for by the neurosurgeon as visible surface movements\autocite{Nimsky2000}.
Numerous techniques have been proposed to address the problem of brain shift\autocite{Bayer2017b} by dynamically adjusting preoperative imaging with patient specific deformation modelling.
Some rely entirely on preoperative imaging in combination with predictive modelling to simulate deformations, others include sparse or alternative modality intraoperative data, including sparse tracking of cortical surface features\autocite{Luo2019}, optical imaging of the cortical surface\autocite{Skrinjar2002,Audette2005,Fan2017} and intraoperative ultrasound \autocite{Letteboer2005,Reinertsen2007,Bucki2012,Machado2019}, to estimate brain shift and update preoperative imaging accordingly.
Yet \textcite{Yang2017a} found that \gls{wm} tract shift direction was largely independent of cortical surface shift.
Ultimately, the most accurate 3D patient anatomy information can only be obtained with full 3D structural imaging after brain shift has occurred.



\section{Intraoperative imaging}

To mitigate the effects of brain shift on neuronavigation accuracy, new structural and functional guidance information can be acquired intraoperatively.
Once more, different modalities offer different strengths and weaknesses.
Ultrasound imaging can probe into tissue beyond the surface, is safe, and can be used at the surgical table without needing to move the patient\autocite{Elmesallamy2019,Eljamel2016}.
Doppler ultrasound is particularly useful for detecting intra- and peritumoural vasculature\autocite{Steno2016}, although image quality is limited and can be difficult to compare with other imaging modalities such as \gls{mri}\autocite{Eljamel2016}.
Specialised \gls{ct} systems can also be utilised intraoperatively \autocite{Bayer2018}, but they cause additional patient exposure to ionising radiation which is to be avoided wherever possible.

\Gls[noindex=false]{imri} is becoming an increasingly common and valued addition to neurosurgical set-ups.
This includes low field ($<1$T) open bore systems which can be installed in the operating room, and allow for easy transfer of the patient into the scanner\autocite{Steinmeier1998,Senft2010}, as well as high-field (1.5--3T) systems which acquire far higher quality images with potentially more clinical utility\autocite{Makary2011} at the expense of practicality, as interrupting surgery for an extended scan session and safely transferring a patient from the operating table to inside the scanner bore in an adjoining room is a substantial logistical and medical undertaking\autocite{Senft2010,Giordano2016a,Sattur2019}.
Technical challenges notwithstanding, \gls{imri} is incredibly valuable for determining surgical margins and providing guidance after the effects of brain shift deformations have invalidated preoperative imaging.

Numerous works demonstrate the benefits of \gls{imri} for improving postsurgical outcomes in both tumour and epilepsy surgery, including greater \gls{eor}, fewer new postoperative deficits, greater postoperative seizure freedom, and reduced length of hospital stay in both adults and children
\autocite{Shah2012,Zhang2015a,Sacino2016,Rao2017c,Giordano2017,Lu2018a,Garzon-Muvdi2019,Leroy2019,Karsy2019,Golub2020,Hlavac2020,Englman2021}.
Even when using advanced image and surgical guidance, postoperative \gls{mri} may reveal tumour nodules unintentionally left in the brain, concealed in corners of the resection cavity.
In some cases early repeat surgery is required to remove residual disease, resulting in significant additional clinical burden to the patient, longer hospital stays, heightened risk of complications including wound infection\autocite{Tenney1985,Chang2003}, delays in the planning of adjuvant treatment, and greater financial expense\autocite{Shah2012}.
Intraoperative imaging enables these remaining tumour margins to be detected and fully resected within the initial operation\autocite{Sattur2019,Hlavac2020}.
For low-grade lesions, the use of \gls{imri} could mean the difference between a potentially curative procedure and one leaving the patient at risk for recurrence and/or reoperation\autocite{Shah2012}.
Where \gls{imri} indicates no need for further resection, it can replace the need to subject patients to an additional scan immediately following surgery, with all the discomfort and possible sedation, or in children, general anaesthesia, that it would entail.

While there is still some debate over the overall cost to benefit ratio of high-field \gls{imri} systems\autocite{Eljamel2016,Giordano2016a,Giussani2022}, and indeed high initial capital expense remains one of the primary barriers to their installation\autocite{NICE2021}, cost-effectiveness analyses have indicated that upfront investments are recouped by lifetime savings associated with shorter hospital stays, and improved postoperative recovery and survival\autocite{Giordano2016a,Sacino2018}.
Cost-effectiveness can be further increased with a dual use set-up, in which an \gls{imri} system is used both for intraoperative and routine diagnostic scanning\autocite{Giordano2016a}, as is the case at \gls{gosh}.
Further challenges of implementing \gls{imri} include longer operating times, equipment compatibility, and patient positioning for both navigated surgery and scanning\autocite{Giordano2017}.
It is worth noting that the strength of evidence supporting all \gls{imri} use and cost-effectiveness is still under debate \autocite{Jenkinson2018,Garzon-Muvdi2019,Caras2020}, and interpretation of \gls{imri} studies is confounded by selection bias and a lack of randomised control trials\autocite{Kubben2011}.



The majority of \gls{imri} sequences are $T_1$-weighted (with or without injected contrast agent enhancement), $T_2$-weighted, and other conventional acquisitions with good tumour tissue contrast\autocite{Kubben2011,Coburger2019}.
In line with this finding, the majority of literature reviewed surrounding \gls{imri} focusses on evaluating \gls{eor}, and in this regard the technique has gradually established itself as a clearly beneficial and in some centres indispensable surgical aid\autocite{Garzon-Muvdi2019,Hlavac2020}.
What has been less extensively studied is the potential for \gls{imri} to provide updated advanced functional neuronavigation.
Broadly speaking, this can be approached in two ways.
The first is to use conventional \gls{imri} to dynamically adjust preoperative functional and connectivity information (including \gls{fmri} and tractography) using deformable image registration and/or biomechanical brain shift modelling, the second is to directly acquire new advanced \gls{mri} sequences intraoperatively.
While the former may seem like the preferred choice, as one avoids having to acquire and process additional lengthy scans and further prolonging surgery interruption, achieving robust deformable registration in a reasonable timeframe is by no means trivial.
It is especially complicated given the significantly disturbed anatomy typically observed following craniotomy and complex brain shift, as well as the effects of air in the resection cavity not present on preoperative imaging.
Intraoperative \gls{fmri} has been reported in a very limited number of studies in \gls{dbs}\autocite{Hiss2015,Knight2015}, tumour resection\autocite{Roder2016a,Qiu2017a}, and neurovascular\autocite{Muscas2019} surgeries, but it's unlikely to find widespread use on account of practicality and an established preference for \gls{des} for cortical mapping.
By contrast, intraoperative \gls{dmri} has several potential advantages.

Use of tractography for neuronavigation is valued by many, but its accuracy diminishes with increasing brain shift, with precision most compromised during later stages of an operation, at the same time as the resection is potentially approaching critical subcortical \gls{wm}\autocite{Yang2019} (Fig. \ref{fig:shift}).
Some account for this by registering preoperative tractography or \gls{dti} \gls{dec} maps to intraoperative structural \gls{mri}\autocite{Nimsky2006a,Tamura2022}, which depends on robust and accurate registration\autocite{Beare2016}.
Alternatively, intraoperative acquisition of \gls{dti} and even \gls{hardi} scans has been garnering interest.
Early work by \textcite{Nimsky2005}$^,$\autocite{Nimsky2005a} demonstrated the feasibility of intraoperative \gls{dmri} and tractography for illustrating \gls{wm} tract shifting after tumour resection.
Safe resection of tumours and epilepsy foci aided by intraoperative reconstruction of motor\autocite{Maesawa2010,Javadi2017}, language\autocite{DAndrea2016,Li2021} and visual\autocite{Daga2012,Cui2015} pathways has been confirmed in subsequent studies.
Overall, a systematic review by \citeauthor{Aylmore} (in production)\autocite{Aylmore} found 26 articles reporting intraoperative \gls{dti} or tractography and associated outcome measures, with moderate suggestion of benefit to surgical success.

There are considerable technical considerations associated with intraoperative \gls{dmri}, even more so than with routine imaging.
Duration is of course one, with minimum possible scan time a priority for interrupted procedures.
Secondly, intraoperative \gls{dmri} can suffer from degraded image quality\autocite{Roder2019}, and distortion artefacts common in \gls{dmri} sequences utilising \gls{epi} are exasperated by the tissue--air interface at the craniotomy site\autocite{Elliott2020}, which can significantly diminish the accuracy of \gls{wm} tract localisation\autocite{Yang2022}.
Early studies suggested that intraoperative tract reconstructions may not be reliable on their own, but useful in combination with adjuvant functional mapping such as \gls{des}\autocite{Ostry2013}.
Nearly all current implementations of intraoperative tractography are limited to commercially available \gls{dt}-based deterministic algorithms, which have known limitations particularly in application to pathology.
Nevertheless, concrete evidence of benefits to postoperative outcomes with intraoperative advanced \gls{dmri} is mounting\autocite{DAndrea2012,Cui2015,Maesawa2010}, and in combination with what we already know about conventional \gls{imri} and preoperative imaging for \gls{wm} neuronavigation, it stands to reason that intraoperative \gls{dmri} could bring significant improvements to brain surgery if some of the technical and image processing limitations can be overcome.

All neuronavigational tools are complementary, each providing distinct streams of information, and where possible, combining techniques can increase the chances of achieving an optimal outcome.
For example, maximal safe resection with combined 5-ALA and \gls{imri} guidance is recommended by UK national clinical guidelines for surgical treatment of gliomas, which further advise considering the use of \gls{dti}, if available, and awake craniotomy to optimise safety and effectiveness\autocite{NICE2021}.
Reviews have found similar benefits to \gls{eor} from both \gls{imri} and 5-ALA guidance separately\autocite{Golub2020} and even better outcomes when combined\autocite{Nickel2018,Coburger2019}.
5-ALA provides the surgeon with a direct visualisation of tumour cells in the surgical view, but residual nodules may remain out of sight behind cavity convexities and only show up on \gls{imri}\autocite{SueroMolina2019}.
Combining awake surgery with \gls{imri} is also viable and beneficial in some patients \autocite{Motomura2017,Tuleasca2021}.
In summary, advanced functional mapping, presurgical planning, and intraoperative monitoring, where the appropriate resources and expertise are available\autocite{GeorgeZakiGhali2020}, all increase the operability of highly eloquent gliomas
\autocite{Bello2008,Krieg2013,DellaPuppa2013b,Magill2018}.
\clearpage{}
\clearpage{}\chapter{Summary and objectives}
\label{chap:problem}

We have seen how, along the long journey towards understanding and mapping the brain to advance neuroscience and neurology, \gls{mri} has played a pivotal role.
In particular, \gls{dmri} \gls{wm} imaging with high-order fibre orientation modelling and streamline tractography has brought us closer to comprehensively exploring the brain's structural connectivity pathways, at the same time as those pathways have achieved higher recognition as structures of interest to neurosurgical planning and navigation.
The unique challenges of the neurosurgical imaging environment, including limited resources and pathology-affected data, have so far held back effective clinical translation of the most advanced capabilities of \gls{dmri}.
These challenges are especially acute in intraoperative imaging, which has the potential to improve surgical outcomes through overcoming the destabilising effects of brain shift on image-guided surgery accuracy.
Considering the existing evidence for improved \gls{eor} with conventional contrast \gls{imri}, in combination with the proven beneficial contribution of preoperative \gls{dmri}, it is reasonable to hypothesise that the addition of intraoperative \gls{dmri} \gls{wm} imaging to the surgical workflow could significantly improve outcomes still further\autocite{Manan2022}.

Streamline tractography, as the dominant \gls{wm} imaging tool, has found widespread adoption for surgical planning while remaining error prone and easily affected by disease-specific image effects, including disturbed diffusion around tumour oedema and infiltration.
Direct voxel-based segmentation does not have the same sensitivity to error propagation and magnification inherent in the point-wise tracking process, but like tractography it still requires careful consideration of the anatomical criteria defining a tract, only these are required during a preparatory training phase, rather than at the point of application.
Data-hungry solutions including deep-learning models demonstrate impressive accuracy and speed, but often have limited generalisability without extensive retraining.
Notably, inference can fail in the presence of large tumour distortions, limiting applicability in neurosurgical practice.
Finally, a methodology dependent on large volumes of training data is arguably less than ideal for an application where the accuracy of that data may be subjective and change over time, as is the case with white matter tract anatomical definitions, which are evolving and lacking consensus, as we will find out in the next chapter.

This thesis describes the conceptualisation and implementation of a \gls{wm} tract imaging method tailored to neurosurgical application in light of all considerations laid out above.
The preceding sections reviewing the current state of the art motivate the following guiding principles and objectives for the proposed method:
\DIFdelbegin \DIFdel{Of course, applicability and robustness to clinical data and specifically brain tumour patient data featuring large spatial distortions is paramount.
Secondly, streamline }\DIFdelend \DIFaddbegin \begin{itemize}[itemsep=0pt,parsep=0pt]
\item[\DIFadd{--}] \DIFadd{Streamline }\DIFaddend tracking at the point of application is to be avoided, owing to its demands on resources and expertise and vulnerability to pathology.
Instead the rich orientation information available in \gls{dmri} acquisitions should be directly compared with \textit{a priori} tract feature knowledge to infer the tract's likely location in the target subject.
\DIFaddbegin \item[\DIFadd{--}] \DIFadd{Applicability and robustness to clinical data and specifically brain tumour patient data featuring large spatial distortions is paramount.
}\item[\DIFadd{--}] \DIFaddend More generally we require a pipeline compatible with the specific constraints of the intraoperative environment, including an acquisition--to--result timescale of under ten minutes, and minimal to no reliance on user input.
\DIFdelbegin \DIFdel{Finally, the }\DIFdelend \DIFaddbegin \item[\DIFadd{--}] \DIFadd{The }\DIFaddend training data requirements should be kept to a minimum, to allow flexibility and adaptability to new tracts or modified tract definitions in an environment where obtaining and curating high quality reference data is restricted.
\DIFaddbegin \end{itemize}
\DIFaddend 

The remainder of this thesis is divided into two parts.
A proposed methodology is presented next, consisting of a tract specific statistical atlas expressing priors on spatial and orientational distributions to guide inference in the target image, and a tumour deformation model for adaptability to patient-specific tumour distortions.
Subsequently, feasibility, applicability, and technical considerations are explored through a series of quantitative benchmark evaluations and case studies.
Along the way the above objectives will be continually invoked to frame and evaluate the proposed approaches.
\clearpage{}

\epart[\epigraph{I've spent my life trying to make things simpler. \\ Because I find ultimately that complicated doesn't reach the heart.}{Hans Zimmer}]{Tractfinder}
\clearpage{}\chapter{Tract orientation atlases and mapping}
\label{chap:atlas}

\DIFdelbegin \DIFdel{The chosen approach to }\DIFdelend \DIFaddbegin \DIFadd{To }\DIFaddend fulfil the aims laid out in Chapter \ref{chap:problem}\DIFaddbegin \DIFadd{, a framework loosely modelled on Bayesian statistics was initially proposed, in which the voxel-wise probability of identifying a tract in a given location is determined by neuroanatomically informed prior expectations and the }\gls{dmri}\DIFadd{-derived evidence supporting those expectations.
Some early explorations which formed the basis for the work covered in this chapter are described in \textcite{Young2020}.
The final method }\DIFaddend combines aspects of both traditional atlas-based and direct data-driven approaches to segmentation, and is named \textit{tractfinder}.
\DIFaddbegin 

\DIFaddend It is simultaneously clear that the task of \gls{wm} tract identification cannot be accomplished without incorporating anatomical priors, and that those priors cannot account for all individual differences.
Anatomical priors can be automatically incorporated for streamline tractography, as is done for several prior works including the \gls{fsl} tool XTRACT\autocite{Warrington2020} and the white matter query language (WMQL) framework\autocite{Wassermann2016}.
These approaches still rely on tractography as the means for parsing the local orientation information in the test subject, and consequently inherit many associated drawbacks such as the computational time and power required for whole brain tractography, or the difficulties tracking through oedema or around tumours.
As an alternative to tractography, in which the attribution of a voxel to a tract of interest is contingent on it being visited by a streamline which, based on its entire length, has been attributed to the tract, we will instead develop a voxel-wise approach, in which a voxel can be associated with a tract based simply on its local diffusion properties and global coordinates.

We'll thus consider a voxel's membership with a target tract to be determined by its location within the brain and local directional diffusion characteristics, and how well they align with prior expectations for the tract.
This chapter, much of which was previously described in \textcite{Young2024}, is concerned with the encoding of those expectations in the form of a tract-specific location and orientation atlas, and the voxel-wise comparison of those expectations with observed directional diffusion information in a new image.
The necessary intermediate step, of establishing spatial correspondence between the atlas and observed data, is the subject of Chapter \ref{chap:reg}.

\section{Theoretical motivation}

A tract orientation atlas (one is to be made per tract) aims to capture the typical spatial and orientational distribution of a bundle across a sample of healthy subjects.
In order to motivate its format and derivation, it is worth first considering the information channels aimed to be contained within the atlas and their interpretation.

First is the spatial distribution, i.e. the likelihood of finding the tract in a given voxel.
How should we quantify this likelihood?
At the individual level, this is sometimes conflated with or loosely equated to streamline density derived from tractography, when in reality the relative densities between different portions of a streamline bundle have far more to do with the biases of tracking algorithms than the underlying anatomy.
Instead, let's imagine we pick a single voxel in a brain scan and ask, ``are fibres of the \gls{cst} present in this voxel?''.
In attempting to answer this question, we encounter several sources of uncertainty.

\textit{a) Definitional uncertainty.}
While we can more or less agree that the \gls{cst} originates in the primary motor cortex in the precentral gyrus at one end, and forms the pyramids of the medulla at the other, its precise boundaries in the subcortical \gls{wm} are less obvious.
Within the corona radiata, for instance, where is the transition, if a clear one exists, between motor fibres and ascending sensory pathways?
This lack of highly detailed neurophysiological knowledge is our first source of uncertainty.

\textit{b) Inter-subject variability.}
Even if we could nail down the exact shape of a pathway in a hypothetical ideal brain, we would be left to contend with natural differences in brain shape and organisation.
How can we know if, in this particular individual, the transition from afferent to efferent fibres occurs slightly more posteriorly than on average?

\textit{c) Measurement uncertainty.}
Suppose we have overcome our doubts about the expected shape of the \gls{cst} in our specific individual.
Our voxel is in or near the tract's domain, but it also sits adjacent to grey matter.
The diffusion profile measured in this voxel is consistent with the presence of some fibres travelling in the expected direction, but only according to one estimate of the \gls{fod}, using \gls{ssst} \gls{csd}.
Using \gls{msmt} \gls{csd}, we obtain a different \gls{fod}, one which doesn't align with the expected \gls{cst} fibres, suggesting the data only weakly supports the presence of such fibres, and possibly only thanks to an artefact or noise.


Faced with this tangle of unknowns, a simplistic route was chosen.
At the individual level, we will decide upon a theoretical bundle structure that is best supported by known neuroanatomy (effectively eliminating the definitional uncertainty), and attempt to faithfully reconstruct that structure using probabilistic streamline tractography and careful manual filtering of implausible streamlines.
From this reconstruction we will derive a binary spatial segmentation, effectively assuming no measurement uncertainty for the purpose of constructing an atlas. By repeating this process and considering the differences in segmentations across a population of reference subjects, we will in doing so capture an approximation of the inter-subject variability.


The second channel of information to be conveyed by the atlas is the orientation distribution, or the spread of likely directions along which fibres of a bundle might be aligned at a given location.
This property, too, is subject to uncertainties as described above.
However, once a tract's location is known, or at least assumed, we can in most cases be fairly certain of its local orientations based on our knowledge of its global shape and trajectory.
In a multi-fibre voxel, the distinct population in the apparent \gls{fod} which corresponds to our tract of interest is assumed to be known.
In this case, tractography which probabilistically draws samples from that distribution will reconstruct some streamlines travelling through that voxel in a distribution of directions consistent with our expectations for that tract, alongside potential spurious streamlines traversing the voxel in the ``wrong'' direction.
After carefully ensuring, as much as is feasible, that only those consistent streamlines remain, we can retrieve the orientation distribution of the tract of interest in isolation from the other crossing populations using a technique called \glsreset{tod}\gls[noindex=false]{tod}\autocite{Dhollander2014} mapping.
After aggregating this information in a template space we should be left with a single spherical distribution per voxel expressing the range of possible orientations typical of the tract.
A narrow distribution may be found where the tract's orientation is highly consistent across all subjects, whereas a more spread-out distribution would reflect a wider inter-subject variability, which may be seen in regions of fanning or sharp turning.

To obtain such a mapping, a combination of streamline tractography and \gls{tod} imaging is used.
While tractography has significant limitations as discussed previously, it remains the standard way of segmenting \gls{wm} bundles from \textit{in vivo} \gls{dmri} data, and biases and errors can, with appropriate post-processing steps, be mostly corrected for.
In addition, tractography uniquely enables the extraction of fibre orientation information specific to the reconstructed bundle.
Being derived from carefully curated streamline tractography reconstructions, we can conceptualise the atlas as a store of those anatomical prior expectations which would otherwise be utilised to draw appropriate \glspl{roi} for targeted tractography.

\section{Tract definitions}

It is important that the atlases closely reflect the anatomical definitions most prevalent in the literature, and most useful for clinical applications.
To this end, an extensive literature search was conducted for each tract before confirming the \gls{roi} strategy that would in turn determine the shape of its normative atlas.
A combination of neuroanatomical studies using blunt dissection, functional studies using either non-invasive methods including \gls{fmri} or intraoperative functional stimulation were sought out as well as tractography studies detailing recommended \gls{roi} strategies.

There is as of yet no perfect method for determining the exact course of individual \gls{wm} fibres in the human brain.
A technique called tract-tracing, which involves the invasive injection of tracer compounds \textit{in vivo} and subsequent post-mortem histological analysis, is widely accepted as the gold standard, but only ethically feasible in animal studies, and there is disagreement and controversy surrounding the transferability of tract definitions between no-human primates and humans.\autocite{Becker2022,ThiebautdeSchotten2012}.
Blunt dissection in post-mortem specimens, greatly advanced through the fixation methods developed by Joseph Klingler (1888--1963) in the early 20th century\autocite{Agrawal2011}, is therefore the most direct technique available for studying global \gls{wm} anatomy in humans.
Though greatly advancing the study of \gls{wm} anatomy and widely used to evaluate the findings of tractography studies, post-mortem dissections, in which layers of \gls{wm} fibres are painstakingly separated and isolated, cannot flawlessly uncover the complete and intact extent of an entire bundle\autocite{Martino2010, Dick2012}, and its findings have also been the subject of rigorous debate (see, for example, \textcite{Giampiccolo2022a},\textcite{Becker2022},\textcite{Giampiccolo2022b}).
Nevertheless, post-mortem dissection and \textit{in vivo} functional evidence from intraoperative \gls{des} are preferred over conclusions drawn purely from non-invasive tractography for informing our tract definitions.
By relying wherever possible on such ``primary source'' evidence, it is hoped that the circularity of basing our understanding of a tract's anatomy on tractography studies alone can be avoided. For example, there are several studies describing the cortical terminations of streamlines in tract reconstructions and presenting them as evidence for connectivity\autocite{Conner2018,Hau2016}, when in reality inferring the existence of a physical connection based on tractography alone is highly inadvisable\autocite{Rheault2020}.

For projection tracts, such as the \gls{or}, the function and anatomy almost define each-other:
The \gls{or} transmits optical signals from the thalamus to the first point of contact in the cortex, the primary visual cortex.
With association tracts, on the other hand, one can observe a backwards and forwards between deducing possible cortical terminations based on observed function, e.g. particular functional deficits induced through stimulating the core of the tract, and theorising on possible functions based on cortical terminations observed from blunt postmortem dissections.

Once an anatomical definition was decided upon based on the balance of available evidence, tractography studies detailing recommended \gls{roi} strategies were compared.
Prior studies acted as guidance for the choice of \glspl{roi} used for each tract if a clear consensus consistent with the chosen definition emerged.
Due to the difficulties in constraining tractography streamlines as they approach the cortex, grey matter \glspl{roi} derived from cortical parcellations were used wherever practical.

As is invariably necessary for probabilistic tractography, additional exclusion \glspl{roi} are needed to filter false positive streamlines that are clearly not part of the tract.
These include, for example, midline exclusion \glspl{roi} to remove \gls{cst} streamlines straying into the corpus callosum, or a posterior \gls{roi} to exclude \gls{af} streamlines from propagating into the occipital lobe.
These exclusion \glspl{roi} will not be described, as their motivation is trivial, unless they result in reconstructions that differ notably from other anatomical definitions.
For example, some reconstructions of the \gls{cst} include parts of the afferent dorsal-column medial lemniscus system within the brainstem, whereas in tractfinder these fibres are explicitly excluded.
Where relevant, such distinctions are described below, and full details on all tractography \glspl{roi} are given in Appendix \ref{app:rois}.
Atlases have been created for the most commonly indicated pathways in neurosurgical planning and guidance, namely the \gls{cst}, \gls{af}, \gls{or} and \gls{ifof}.

\subsection{Optic radiation}

\begin{SCfigure}[][htb!]
  \includegraphics[width=0.5\textwidth]{chapter_3/45_or.png}
  \caption[Optic radiation]{Schematic reconstruction of the \glsentrylongpl{or}, shown from an inferior axial viewpoint. Fibres originating in the \glsentrylong{lgn} of the thalamus arch anteriorly forming Meyer's loop, before returning towards the visual centres in the occipital cortex.}
  \label{fig:or}
\end{SCfigure}

As a projection tract, the \DIFdelbegin %DIFDELCMD < \glsreset{or}[noindex=false]\gls{or} %%%
\DIFdelend \DIFaddbegin \glsreset{or}\gls[noindex=false]{or} \DIFaddend is among the more easily definable in terms of its end points.
Arising from the \glsreset{lgn}\gls[noindex=false]{lgn} of the thalamus, its fibres swoop around the lateral ventricle, some directly and some in an elaborate arc reaching as far forward as the tip of the lateral ventricle's temporal pole before curving back towards the occipital lobe in a formation named \glsreset{ml}\gls[noindex=false]{ml} (Fig. \ref{fig:or})\autocite{Sarubbo2015}, the extent and shape of which vary significantly between individuals\autocite{Ebeling1988,Yogarajah2009}.As it projects towards the most posterior reaches of the occipital lobe, the \gls{or} forms part of the \gls[noindex=false]{ss}, a confluence of fibres superficial to the lateral ventricle, comprising the antero-posteriorly oriented fibres of association and projection fibres of the occipital, temporal and frontal lobes\autocite{Maldonado2021}.
The fibres of the \gls{or}, conferring low-level visual data on contrast, colour and shapes, then terminate in the primary visual areas of the occipital cortex surrounding the calcarine fissure, a sulcus on the medial surface of the occipital lobe.

The \gls{or} is naturally an integral structure for visual function and thus quality of life.
With the extension of \gls{ml} deep into the temporal pole, this structure is at high risk of injury in common surgeries for treatment of epilepsy, often resulting in postoperative visual field deficits\autocite{Lacerda2020}.
Tumour infiltration of the \gls{or} as evidenced by preoperative tractography has been significantly correlated with worse postoperative visual outcomes\autocite{Soumpasis2023}.
Intraoperative mapping with subcortical \gls{des} can be used to assess for visual field deficits during surgery\autocite{Duffau2004a,Mazerand2017}, although \textcite{Shahar2018} have suggested that this may be of limited value if visual responses can only be elicited once the resection has already reached a critical proximity to the tract.

The \gls{or} is one of the more difficult tracts to reconstruct using streamline tractography, and there is significant variability in particular in the regions of the \gls{lgn} and \gls{ml}.
The \gls{lgn} is a small nucleus, its localisation on \gls{mri} not straightforward, resulting in often very generous \glspl{roi} including most of or even the entire ipsilateral thalamus.
In addition, due to the complex arrangement of \gls{wm} structures in the upper midbrain and thalamus regions, it is easy for streamlines to extend into the entire posterior thalamus and fornix and even descend into the brainstem.
This contributes to often broad \gls{or} segmentations in the thalamic portion at the start of the tract.
Secondly, the full anterior extent of Meyer's loop is often not comprehensively reconstructed by tractography, due to the extreme and tight curvature\autocite{Lilja2015,Chamberland2018}.

To achieve the best possible representation of this important pathway, the \gls{lgn} was segmented with precision in all subjects using the optic tracts and positioning relative to the cerebral peduncles for orientation.
Using the \gls{lgn} for seeding, an inclusion \gls{roi} was drawn around the \gls{ss} on the coronal plane.
Finally, to improve the coverage of Meyer's loop, an anterior waypoint was positioned just posterior to the anterior pole of the temporal horn.
Since not all of the \gls{or} takes the full scenic route, tractography was run with and without this additional anterior waypoint and the two sets of streamlines combined.

\subsection{Corticospinal tract}

\begin{SCfigure}[][htb!]
  \includegraphics[width=0.5\textwidth]{chapter_3/45_cst.png}
  \caption[Corticospinal tract]{Schematic reconstruction of the \glsentrylongpl{cst}, viewed coronally, showing efferent fibres originating in the primary motor cortices converging into a compact bundle in the internal capsule and descending into the spinal cord.}
  \label{fig:cst}
\end{SCfigure}

Voluntary movement is the cornerstone of life and our means of interacting with the outside world, without which motion, speech, and full vision would be impossible.
It stands to reason that an outsized portion of the central nervous system is dedicated to the precise innervation of muscles throughout the body, and that such a vast repertoire of motor function, which includes the fine control of vocal chords to automated touch-typing, is mediated by an intricate interplay of different systems including the basal ganglia circuits, cerebellum and the sensori-motor cortices of the cerebral hemispheres\autocite{Kandel2021}.
Within the primary motor cortex, situated in the precentral gyri, are the bodies of neurons whose axons project all the way into the spinal cord, where they synapse with the motor neurons of the \gls{pns} to control muscles\autocite{Ebeling1992}.
These axons form a critical pathway called the \glsentrylong{cst} (Fig. \ref{fig:cst}).
The motor cortex exhibits a somatotopic organisation, with correspondence between specific cortical areas and the muscles in different regions of the body, while recent evidence suggests that this somatotopy is lost in the spinal cord as descending fibres converge and overlap\autocite{Lemon2023}.
As with many cerebral functions, somatotopic representation in the brain is contralateral, such that the left motor cortex controls movement in the right side of the body and vice versa.
Standardised \gls{wm} atlases and tractography protocols varyingly describe the corticospinal and pyramidal tracts.
These two terms are often used interchangeably in tractography-oriented publications, while in anatomical terms they are distinct:
The pyramidal tracts are commonly defined as encompassing both the \glspl[noindex=false]{cst} and the corticobulbar (or corticonuclear) tracts, the latter of which synapses with cranial nerve nuclei of the medulla to control motion of the head, neck and face\autocite{Chenot2019}.
This despite the fact that only the \gls{cst} fibres continue to form the pyramids of the medulla, after which the pyramidal tract is named.
Since the two are not practically distinguishable on \gls{dmri} from the brainstem upwards, we will hereon refer only to the corticospinal tract.

Classically, the \gls{cst} was thought to arise solely from the primary motor cortex (M1), with supportive areas including supplementary motor and somatosensory areas mediating voluntary movement via interaction with M1.
More recent analysis has revealed that these secondary cortices also send projections directly via the \gls{cst}\autocite{Kandel2021}, with up to 35\% of neurons originating in non-motor (primary and supplementary) areas, although these figures are based on animal studies\autocite{Welniarz2017}.
Evidence in humans for non-primary motor cortex origins of \gls{cst} fibres is limited to tractography and lesion studies\autocite{Kumar2009,Jane1967}.
However, particularly with probabilistic tractography, it is near impossible to constrain streamlines exiting the internal capsule into the fanning corona radiata to only a narrow region of cortex, and of course tractography is unable to distinguish efferent and afferent pathways.
Thus streamline-based \gls{cst} segmentations often contain parts of the somatosensory cortex whether intentionally or not\autocite{Poulin2022a}.
In another conflation of descending and ascending pathways, the inclusion of the medial lemniscus, an ascending pathway in the brain stem distinct from the descending \gls{cst}, is frequently seen in \gls{cst} segmentations (usually as it is not explicitly excluded, rather than being actively included)\autocite{Wasserthal2018,Warrington2020,Poulin2022a}.

The tractfinder \gls{cst} atlas streamlines were obtained using FreeSurfer parcellations \autocite{Desikan2006,FischlSalat2002} of the primary motor cortex.
To boost coverage of the lateral precentral gyrus, which is typically neglected by tractography due to the difficulty in crossing the centrum semiovale and its preference for taking the straightest path, two sets of streamlines were generated.
First, streamlines were seeded from the entire precentral gyrus, with inclusion \glspl{roi} on the cerebral peduncles and mid-pons level, retaining the coordinates of the successful (included) seeds.
Then, a dilated mask of those coordinates was subtracted from the motor cortex mask, and the tracking was repeated from this modified low-count seed mask.
An exclusion \gls{roi} on the medial lemniscus in the mid-pons excluded afferent fibres at the brainstem level.

\subsection{Arcuate fasciculus}



\begin{SCfigure}[][htb!]
  \includegraphics[width=0.5\textwidth]{chapter_3/45_af.png}
  \caption[Arcuate fasciculus]{Schematic reconstruction of the \glsentrylong{af}, viewed sagittally, showing fibres arcing between the posterior temporal and inferior frontal lobes.}
  \label{fig:af}
\end{SCfigure}

The \glsentrylong{af} has long held the fascination of neuroscientists, anatomists and psychologists since its earliest descriptions in the 19th century\autocite{Burdach1822}.
Ever since, the nomenclature and anatomical description of the \gls[noindex=false]{af} has been the subject of much disagreement and confusion, with researchers variably dividing it into a series of subcomponents, disputing its cortical terminals and declaring it synonymous with either part or all of the \gls{slf}.
A comprehensive review of this jumble and its history can be read in \textcite{PortodeOliveira2021}.
Intertwined with this discussion on the \gls{wm} connections between (and within) frontal and temporal lobes are the ever-evolving models of language reception, processing and production in the human brain\autocite{Becker2022a}.
In particular, conflicts over the distinction between the \gls{slf} and \gls{af} and their respective terminations in the superior, middle or inferior temporal gyri are directly motivated by a desire to disentangle the distribution of different language associated functions in the temporal cortex and their processing by means of the dorsal and ventral language streams\autocite{Hickok2004,Friederici2013a,Kljajevic2014a,Giampiccolo2022a,Becker2022a}.

Even the core precondition for the classical definition of the \gls{af}, namely the existence of a direct connection between the inferior frontal gyrus (loosely, Broca's area) and the posterior superior temporal gyrus (Wernicke's area), has been called into question\autocite{Dick2012,Giampiccolo2022a}.
While we cannot dispute the existence of a core bundle of fibres arching around the Sylvian fissure, no-one can seem to agree on where exactly those fibres are headed, in either direction.
Within the temporal lobe, attempts to definitively disentangle the \gls{af} from other temporal lobe tracts have fallen short\autocite{Becker2022}, and with regards to the frontal terminations, there is disagreement over the relationships between the various dorsal stream components and the premotor cortex and inferior frontal gyrus\autocite{Kljajevic2014a,Giampiccolo2022a}.

Numerous lengthy articles and thesis chapters have been devoted to this discussion, and it is beyond the scope of this short review to attempt a definitive summary.
Instead, for the purposes of this report, we will consider only the narrowest and perhaps most canonical definition of the \gls{af}:
That of a bundle connecting the posterior-superior aspect of the temporal lobe with the inferior frontal gyrus (Fig. \ref{fig:af}), with a critical role in speech production\autocite{Baldo2015}.
By this definition, we will exclude longer extensions into the anterior temporal cortex\autocite{Giampiccolo2022a} and any frontal gyri other than the inferior frontal gyrus, and reject the tripartite paradigm which includes additional subdivisions connecting to Geschwind's territory of the parietal lobe\autocite{Catani2005,Martino2013a}.
Streamlines were seeded from an \gls{roi} covering the \gls{af} on a coronal slice at the level of the central sulcus, where it appears on colour \gls{fa} as a distinctive green triangle, with an additional axial inclusion \gls{roi} placed on the descending fibres, posterior to the tip of the Sylvian fissure.


\subsection{Inferior fronto-occipital fasciculus}



\begin{SCfigure}[][hbt!]
  \includegraphics[width=0.5\textwidth]{chapter_3/45_ifof.png}
  \caption[Inferior fronto-occipital fasciculus]{Schematic reconstruction of the \glsentrylong{ifof}, viewed sagittally, showing association fibres passing through the narrow external capsule and temporal stem to connect the occipital and frontal lobes}
  \label{fig:ifof}
\end{SCfigure}

Even compared to the \gls{af}, whose canonical description, however since disputed, was at least anchored in a strongly motivated connection between two well described cortical regions and an attached clear linguistic function, study of a direct \gls{wm} fibre connection between the occipital and frontal lobes has an unsettled history\autocite{Forkel2014a}.
Thanks to a confusing mix of attempting to directly compare simian and human anatomy\autocite{Schmahmann2007,ThiebautdeSchotten2012,Mandonnet2018,Sarubbo2019}, the historical study of and extrapolation from atypically developed or lesioned brains\autocite{Schmahmann2007,Forkel2014a}, and the difficulty by any means of following long-range connections in the brain
\autocite{Martino2010}, the \glsentrylong{ifof} is an exceedingly difficult tract to pin down from available literature\autocite{Sarubbo2019,Weiller2021}.

The most widely cited functional role of the \gls[noindex=false]{ifof} is one of semantic processing and fluency as part of the wider language network, supported by lesion studies\autocite{Ille2018b,Almairac2015} and the eliciting of semantic paraphasia by \gls{des} along its entire course\autocite{Duffau2013a,Herbet2017,Voets2017,Vigren2020a}, while the connections to the occipital lobes point more specifically to lexical and visual-semantic involvement
\autocite{Martino2010,Rollans2017,Rollans2018}.
From a surgical perspective, damage to and infiltration of the \gls{ifof} has been associated with transient or permanent semantic deficits and is commonly dislocated or infiltrated by tumours\autocite{Almairac2015,Voets2017,Altieri2019,Binding2023}.
Situated within the temporal lobe alongside the \gls{or} and the dense language association network, the \gls{ifof} is at risk of injury in temporal lobectomy surgeries common in the treatment of epilepsy\autocite{Baran2020,Shah2022,Binding2023}.

The \gls{ifof} is frequently described as the key pathway underpinning a proposed ventral stream of language processing\autocite{Duffau2013a,Rollans2018,Voets2017,DavidPoeppel2012} (inspired by the dual-stream ``what'' and ``where'' model of visual processing), even though early works on this dual-stream framework for language make practically no mention of the occipital cortex\autocite{Hickok2004,DavidPoeppel2012,Kummerer2013}.
Others conflate it with the ``extreme capsule fibre system''\autocite{Friederici2013a,Zhang2018} (analogous to the structure described in non-human primates\autocite{Mandonnet2018,ThiebautdeSchotten2012}), said to connect the temporal and frontal cortices\autocite{Kummerer2013},
or otherwise declare the \gls{ifof}'s origins as being, at least partly, in the temporal lobe\autocite{Bajada2015a}.
This casual blurring of functional and anatomical postulation and evidence has led to a fair degree of uncertainty surrounding the cortical, in particular the posterior\autocite{Martino2010,Forkel2014a,Weiller2021}, terminations of the \gls{ifof} and how it ties in with the wider models of cerebral language networks
\autocite{Duffau2013a, Mandonnet2018, Rollans2018, Friederici2013a}.
The cognitive functions residing in the frontal cortex are often nebulously defined and difficult to study and pin down, making the anatomical definition of pathways associated with those functions correspondingly difficult.
Beyond the commonly cited semantic functional role, researchers have also speculated, often based on at best tenuous evidence, on roles for the \gls{ifof} that include reading and writing functions associated with parts of the parietal cortex\autocite{Motomura2014}, sensorimotor processing\autocite{Martino2010} and planning\autocite{Sarubbo2013}, and involvement in goal-oriented behaviour\autocite{Conner2018}.
\textcite{Sarubbo2013}, \textcite{Martino2010}, and \textcite{Rollans2018}, considering terminations in the parietal lobe in addition to occipital regions, have all proposed possible subdivisions of the \gls{ifof} into language and non-language related functions.


We will look for an anatomical consensus among post-mortem dissection studies aimed at uncovering the anatomical path of the \gls{ifof}, considering first the posterior and subsequently the frontal terminations.
In their detailed dissection studies, \textcite{Martino2010} and \textcite{Sarubbo2013} confirmed the extrastriate occipital cortex, in particular the middle occipital gyrus, as primary posterior targets, while lateral occipital lobe terminations were also studied in \textcite{Palejwala2020}.
In addition to the largely agreed upon occipital regions, additional terminations have been found in the parietal and temporal cortices, in particular the superior parietal lobe, fusiform and inferior temporal gyri in \textcite{Martino2010}, and the lingual gyrus in \textcite{Sarubbo2013}.
The lingual gyrus and cuneus are cited as \gls{ifof} origination regions in \textcite{Palejwala2021}.
However, it must be pointed out that the \gls{ifof} fibres have not been unambiguously distinguished from other parietal-occipital bundles, including the \gls{or} and temporal association tracts.
For example, the lingual gyrus is cited in \textcite{Sarubbo2013} as one of three ``main apparent origins'' of the \gls{ifof}, albeit to a limited extent, while \textcite{Martino2010} asserted that any fibres connected to the lingual gyrus belonged to the optic radiations.


After coursing through the temporal lobe above the roof of the temporal horn of the lateral ventricle\autocite{Martino2010,Kljajevic2014a}, the \gls{ifof} converges into a narrow bundle to squeeze, alongside the \gls[noindex=false]{uf}, through the temporal stem and into the frontal lobe (Fig. \ref{fig:ifof})\autocite{Martino2010,Sarubbo2013}.
\textcite{Martino2010} maintained that ``at the level of the frontal operculum, the fibres of the \gls{ifof} strongly intersect with the terminal branches of other long association fasciculus [\textit{sic}]'', making dissection extremely difficult, and indeed not possible for that particular study.
Nevertheless, \textcite{Sarubbo2013} have conducted the only dissection study to my knowledge analysing in detail the frontal connections specifically of the \gls{ifof}.
Other studies, e.g. \textcite{Burks2017}, focus on a particular cortical region and study all connections to it.
The disadvantage with such studies is in the aforementioned difficulty in simultaneously keeping the fibres of multiple bundles intact and traceable at once.
Connections to the inferior frontal gyrus, the seat of Broca's area, were confirmed in \textcite{Sarubbo2013} as well as \textcite{Hau2016}, and are consistent with a linguistic function.
\textcite{Sarubbo2013} also found terminations in the middle frontal gyrus (and more specifically the dorsolateral prefrontal cortex) ``in all specimens''.
Finally, the orbitofrontal cortex, which is considered to play an important role in decision making, and frontal pole terminations were found in \textcite{Sarubbo2013} as well as in \textcite{Burks2017}.

Taken together, the dissection studies confirm (i.e. reported in all or most specimens) primarily posterior terminations in the superior, middle, and inferior occipital gyri and anterior terminations in the middle and inferior frontal gyri, frontal pole, and orbitofrontal cortex.
The evidence is less clear for wider connections to the parietal, posterior temporal and superior frontal cortices, so these will be excluded from the tractfinder atlas.
Streamlines were seeded in the temporal stem (extreme capsule) and selected according to cortical targets derived from the FreeSurfer (v4.5) Destrieux atlas\autocite{Destrieux2010} (2009 version) parcellation.\footnote[2]{frontal labels: 1\{1,2\}1\{01,05,15,54,12,13,14,53,63,24,65\}; occipital labels: 1\{1,2\}1\{02,19,20,58,59\}}

\section{Streamline tractography and filtering}\label{sec:atlasmethods}

An existing dataset of \gls{hardi} acquisitions from 16 healthy adult volunteers (``EEG, fMRI and NODDI dataset''\autocite{Clayden2020}, available online at osf.io/94c5t) was reused in the creation of reference bundles for developing the atlases.
\Gls{dmri} scans were acquired at 2.5~mm isotropic voxel size with three shells at $b=2400$~s~mm$^{-2}$ (60 directions), $b=800$~s~mm$^{-2}$ (30 directions), $b=300$~s~mm$^{-2}$ (9 directions), and a single $b=0$ image with no diffusion weighting.
In addition, a high resolution anatomical $T_1$-weighted scan with 1~mm isotropic voxel size was acquired in each subject.
The \gls{dmri} was preprocessed to remove noise, bias field, susceptibility distortions and motion artefacts using tools provided by the MRtrix3 (v3.0\_RC3) and FMRIB (v6.0.5) software libraries.
\Gls{fod} images where produced using multi-shell multi-tissue \gls{csd}\autocite{Jeurissen2014,Tournier2019} and response functions estimated using the Dhollander algorithm\autocite{Dhollander2016}.
Then each bundle of interest was reconstructed in both hemispheres using probabilistic streamline tractography with iFOD2 \autocite{Tournier2010} and an \gls{roi} strategy based on anatomical landmarks as described above and in Appendix \ref{app:rois}.

After streamline generation, each streamline bundle was transformed into MNI space using affine registration implemented in the \gls{fsl} Linear Image Registration Tool\autocite{Jenkinson2002} ``FLIRT'' between the subject's $T_1$-weighted image and the MNI152 $T_1$ template\autocite{Fonov2011}.
Affine registration rather than non-linear, was used for this step to capture individual anatomical variation and minimise unrealistic warping of streamlines from local registration errors or overfitting.
With all subject streamlines aggregated in MNI space, manual filtering of streamlines was performed according to the definitions previously laid out for each tract (Fig. \ref{fig:filter}), to remove not only ``volumetric false positives'', which depart from the accepted volume of the tract, but also ``orientational false positives'' (OFPs), which remain entirely within the tract volume but are at least in part aligned with a different, intersecting bundle.
Examples of such OFPs are depicted in Figure \ref{fig:ofps}. Such streamlines have little effect on any volumetric applications of the reconstruction, e.g. via a track density depiction.
However, their removal is important for the construction of the orientation atlases, which summarise the orientational distribution of streamlines on a voxel-wise basis.
Filtering was performed in DSI studio (v2021\_04, \url{https://dsi-studio.labsolver.org/})\autocite{Yeh2021a}, which enables the filtering of streamlines based on angle of intersection with a cutting plane.

\begin{figure}[htb!]
  \centering
  \includesvg[width=\textwidth,pretex=\small\sffamily]{chapter_3/ofps.svg}
  \caption[Orientational false positive streamlines]{Orientational false positives (OFPs) are streamlines that remain within the target bundle volume, but partially follow the path of a different tract.
  \textbf{\sffamily a.} Example orientational false positive streamlines, isolated from a \gls{cst} bundle, viewed from a superior vantage. Ascending streamlines following the CST abruptly turn along an anterior-posterior direction to follow the course of the intersecting superior longitudinal fasciculus, before again turning back towards the motor cortex.
  \textbf{\sffamily b.} Atlas \glspl{tod} before filtering out OFPs, with green lobes that do not correspond the expected \gls{cst} orientations.
  \textbf{\sffamily c.} Atlas \glspl{tod} after filtering, showing only directions corresponding to the \gls{cst}. A = anterior, P = posterior.}
  \label{fig:ofps}
\end{figure}

The percentage of streamlines filtered for each tract and summarised reasons for removal are presented in Table \ref{tab:filt}.
What is immediately apparent is the large discrepancy in the number of streamlines filtered between tracts, an indication of how much easier it is to reconstruct some tracts over others.
For the \gls{cst}, thanks to the well targeted cortical seeding and structurally highly distinct pathway, it was only necessary to remove a small percentage of streamlines, mostly OFPs in the centrum semiovale.
By contrast, approximately half of all \gls{ifof} streamlines were discarded, thanks to the tight association of multiple tracts in the \gls{ss} resulting in overwhelming contamination from abutting structures, in particular the \gls{af} and \gls{vof}.
The absolute number of retained streamlines is of little consequence to subsequent steps in creating the atlas, after ensuring full coverage of the tract in every training subject.

\begin{table*}[t]
  \caption[Streamline filtering statistics]{\DIFdelbeginFL \DIFdelFL{Streamline }\DIFdelendFL \DIFaddbeginFL \DIFaddFL{Total streamline counts (all atlas subjects combined) before and after bulk }\DIFaddendFL filtering\DIFdelbeginFL \DIFdelFL{statistics}\DIFdelendFL . Abbreviations: \acrolist{af,crp,cc,cst,ec,slf,sfof,vof}}
  \label{tab:filt}
  \small
  \begin{tabularx}{\textwidth}{llllll X}\toprule
      &      & Original & Filtered & Reduction & Reasons for discarding \\
   \midrule
  CST & left & 148833 & 145300 & 2.37\% & \multirow{3}{0.42\textwidth}{Contamination from: AF, SLF, SFOF, CC, CrP} \\
   & right & 144759 & 139019 & 3.97\% &  \\
   & total & 293592 & 284319 & 3.16\% &  \\ \addlinespace
  AF & left & 61922 & 49778 & 19.61\% & \multirow{3}{0.42\textwidth}{Contamination from:  EC, CST, CC Overextension into: Motor, anterior temporal, and superior frontal cortex} \\
   & right & 61834 & 43027 & 30.42\% &  \\
   & total & 123756 & 92805 & 25.01\% &  \\ \addlinespace
  OR & left & 123842 & 99984 & 19.26\% & \multirow{3}{0.42\textwidth}{Contamination from: Tapetum of CC, SLF} \\
   & right & 122534 & 109265 & 10.83\% &  \\
   & total & 246376 & 209249 & 15.07\% & \\ \addlinespace
  IFOF & left & 80000 & 44224 & 44.72\% & \multirow{3}{0.42\textwidth}{Contamination from: Tapetum of CC, VOF, superior frontal cortex} \\
   & right & 80000 & 31753 & 60.31\% &  \\
   & total & 160000 & 75977 & 52.51 \% & \\ \toprule
 \end{tabularx}
\end{table*}


\begin{figure}
  \makebox[\linewidth][c]{\includegraphics[width=0.28\textwidth]{chapter_3/cst_removed.png}\includegraphics[width=0.28\textwidth]{chapter_3/or_removed.png}\includegraphics[width=0.28\textwidth]{chapter_3/af_removed.png}\includegraphics[width=0.28\textwidth]{chapter_3/ifo_removed.png}}
  \makebox[\linewidth][c]{\includegraphics[width=0.28\textwidth]{chapter_3/cst_filtered.png}\includegraphics[width=0.28\textwidth]{chapter_3/or_filtered.png}\includegraphics[width=0.28\textwidth]{chapter_3/af_filtered.png}\includegraphics[width=0.28\textwidth]{chapter_3/ifo_filtered.png}}
  \caption[Atlas streamline filtering]{False positive streamlines filtered from each tract (top row), aggregated from all subjects, and final set of selected streamlines (bottom row). Reproduced from \textcite{Young2024}}\label{fig:filter}
\end{figure}


\section{TOD mapping}

After aggregate filtering, the retained streamlines were re-separated into individual subject bundles from which \gls{tod} maps were computed as described in \textcite{Dhollander2014} and implemented in MRtrix3 \autocite{Tournier2019}.
\Gls{tod} mapping is the generalisation of \gls{tdi} into the angular domain, creating a 5D spatio-angular representation of streamline tracks on a voxel-wise basis.
The \gls{tod} image is represented in modified \gls{sh} basis \autocite{Descoteaux2006} using only even orders up to a maximum order $l_{max}=8$, meaning each image consists of 45 coefficients, denoted $t_j$, per voxel.
The distribution is described by those coefficients and the modified \gls{sh} basis functions $Y_{l,m}$ defined in (\ref{eq:sh}) \autocite{Descoteaux2006} as

\begin{align}
  T(\theta, \phi) = \sum_{l=0}^{l_{max}} \sum_{m=-l}^l t_{l,m} Y_{l,m}(\theta, \phi) = \sum_j t_jY_j(\theta, \phi).
\end{align}

The individual \gls{tod} images, which combine track density information in an amplitude component as well as angular distribution information in the higher order \gls{sh} components, at this stage still contain significant density bias, with exaggerated differences in magnitude between the core bundle portions and fanning extremities owing to tractography's tendency towards early termination outside of the densest collinear tract regions \autocite{Smith2013,Calamante2015,Rheault2020}.
The purpose of the atlas is to capture only the likelihood of a tract's presence in any given voxel (spatial prior) and, in the case that it is present, its expected orientation (orientational prior).
If the spatial prior is to be determined by considering the spatial variation of the tract between subjects, then the only information needed for each individual subject is a binary visitation map for the bundle, and orientational data.
Thus to remove the streamline density component, the \gls{tod} maps for each subject are normalised.
The spherical integral of each \gls{sh} basis function $Y_{l,m}$ is
\begin{align}
  \int_{\Omega} Y^m_l(\theta, \phi) = \begin{cases}
   \sqrt{4\pi} & \text{ if } l=m=0\\
   0 & \text{ otherwise. }
  \end{cases}
\end{align}
Using the sum and constant rules of integration, the spherical integral of $T(\theta,\phi)$ is
\begin{align}
  \int_{\Omega} T(\theta,\phi) = t_0 \sqrt{4\pi},
\end{align}
where $t_0$ is the first \gls{sh} coefficient for $l=m=0$. Thus to remove density information the \gls{tod} map is normalised to unit integral as
\begin{align}
  \widetilde{T}(\theta, \phi) = \frac{T(\theta,\phi)}{\sqrt{4\pi} t_0}.
\end{align}
After each individual \gls{tod} map has been normalised in MNI space, the resulting image contains only information about the tract's streamline orientations, and none about the number of streamlines passing through a given voxel in the original reconstruction.

\begin{SCfigure}
  \includesvg[width=0.5\textwidth]{chapter_3/tod_mean.svg}
  \caption[TOD averaging to create and atlas]{Averaging the \glsentrylong{tod} contributions from all subjects (smaller cutouts) produces a smooth map of voxel-wise orientation distributions (large cutout), illustrated here in the anterior point of Meyer's loop, a part of the \glsentrylong{or} with significant inter-subject variation. Adapted from \textcite{Young2024}}
  \label{fig:todmean}
\end{SCfigure}

Finally, the mean of all individual normalised \gls{tod} maps is computed to produce the final population tract \gls{tod} atlas (Fig. \ref{fig:todmean}).
Averaging all maps results in distributions that reflect all possible ranges of a tract's orientations in each voxel (Fig. \ref{fig:todmean}), while the first \gls{sh} coefficient of the atlas will reflect the proportion of training subjects in which the tract was present in a given voxel.
Outlier voxels visited by streamlines in only a single subject's reconstruction will contribute little weight to the final atlas.
This atlas can then be registered to a target subject for further processing (Fig. \ref{fig:atlases}).



\begin{figure*}[htb!]
  \centering
    \includegraphics[width=0.45\textwidth,align=c]{chapter_3/cst_tod.png}\includegraphics[width=0.55\textwidth,align=c]{chapter_3/or_tod.png}
    \includegraphics[width=0.55\textwidth,align=c]{chapter_3/af_tod.png}\includegraphics[width=0.45\textwidth,align=c]{chapter_3/ifo_tod.png}
  \caption[Tract orientation atlases for the CST, AF, OR and IFOF]{Tract orientation atlases, composite images projected onto coronal (CST), axial (OR, IFOF) and sagittal (AF) views in MNI152 reference space. Reproduced from \textcite{Young2024}}\label{fig:atlases}
\end{figure*}
\clearpage{}
\clearpage{}\section{Tract mapping}
\label{sec:mapping}

The methodological description of the preceding section was concerned with capturing prior information about tracts.
Assuming this prior information can be brought into anatomical alignment with the target image (this problem is tackled in Chapter \ref{chap:reg}), it remains to compare those priors with the information present in the target diffusion data itself.
This process, of comparing a prior with the information in the acquired data and computing a similarity measure from them, will be generally referred to as ``tract mapping'', and encompasses both spatial and angular components.

Frequently, general atlas-based segmentation techniques rely entirely on accurate registration of the atlas to the novel data, with the registered atlas forming the final segmentation without any further refinement.
This strategy is strongly dependent on the accuracy of registration and assumes a one-to-one correspondence can be found between the template and subject anatomies.
The approach taken here is a different one.
Rather than expecting perfect and robust registration, which is a particularly tall order for clinical applications, we instead make use of the rich information obtainable in \gls{dmri} to refine the initial estimate indicated by a ``close enough'' registration.

The tract atlas intentionally conveys a degree of spatial tolerance to account for individual variations in tract location, with the final mapping step acting to refine the estimate according to observed local information in the target image.
The latter, derived from raw \gls{dmri} data, could conceivably take any form.
For example, one could take the log transform of the normalised signal, and compare the resulting angular diffusivity function with the atlas expectations. However, given that conceptually the \gls{tod} represents a tract specific \gls{fodf}, the most intuitive comparison can be made against the full \glspl{fod}, as modelled using \gls{csd}, observed in the target image.

\subsection{Inner product}\label{sec:ip}

Consider two frequency distributions defined over the same domain.
A straightforward method for quantifying the degree of similarity between them is to multiply the two functions together, where the resulting product will be high wherever both the compared functions were high, and low elsewhere.
By integrating that product over the entire domain we obtain a measure of total congruence across both functions.
This is the functional analogue to taking the dot product (or inner product) of two vectors, which is a commonly seen similarity metric employed in statistical and machine learning contexts.
We define the inner product of two functions $f$ and $g$ over the closed interval $[a \geq x \geq b]$ as

\begin{align}
  \langle f,g \rangle = \int_a^b f g dx.
\end{align}

In our scenario, we are comparing two spherical functions defined over the angular domain parameterised by the azimuthal ($\phi$) and polar ($\theta$) angles, $F(\theta, \phi)$ and $T(\theta, \phi)$.
Both are represented in the modified \gls{sh} basis (\ref{eq:sh}) introduced in Section \ref{sec:sh}:
\begin{align}
  F(\theta, \phi) = \sum_{l=0}^{l_{max}} \sum_{m=-l}^l f_{l,m} Y_{l,m}(\theta, \phi) = \sum_j f_jY_j(\theta, \phi),
\end{align}
defined by a vector of \gls{sh} coefficients $\mathbf{f} = [f_0, ..., f_j, ..., f_N]^T$ where $N = \frac{1}{2}(l_{max}+1)(l_{max}+2)$.
Using the distributive law for the product of sums and and the sum rule for integration, we can simplify the inner product of the two functions $\langle F(\theta, \phi), T(\theta, \phi) \rangle$ as follows:

\begin{align}
  \begin{split}
    \langle F(\theta, \phi), T(\theta, \phi) \rangle
    =& \int_0^{\pi} \int_0^{2\pi} F(\theta, \phi) T(\theta, \phi) sin(\theta) d\theta d\phi \\
    = & \int_0^{\pi} \int_0^{2\pi} (\sum_j f_jY_j(\theta, \phi)) (\sum_k t_kY_k(\theta, \phi)) sin(\theta) d\theta d\phi \\
    = & \int_0^{\pi} \int_0^{2\pi}  (\sum_{j=0}^N (\sum_{k=0}^N f_jY_j t_kY_k )) sin(\theta) d\theta d\phi \\
    = & \sum_{j=0}^N (\sum_{k=0}^N f_j t_k \int_0^{\pi} \int_0^{2\pi}  Y_j Y_k sin(\theta) d\theta d\phi ).
  \end{split}\label{eq:ip1}
\end{align}
The spherical harmonics form an orthonormal system, such that the inner product of two basis functions $Y_{l_1,m_1}$ and $Y_{l_2,m_2}$ is
\begin{equation}
  \langle Y_{l_1,m_1}, Y_{l_2,m_2} \rangle
    = \int_0^{\pi} \int_0^{2\pi} Y_{l_1,m_1}(\theta, \phi) Y_{l_2,m_2}(\theta, \phi) sin(\theta) d\theta d\phi
    = \delta_{m_1, m_2} \delta_{l_1, l_2},
\end{equation}
where $\delta_{i,j}$ is the Kronecker delta defined as $\delta_{i,j} = 1 \text{ if } i = j \text{ else } 0$.
Using this property, (\ref{eq:ip1}) simplifies to
\begin{align}
  \langle F(\theta, \phi), T(\theta, \phi) \rangle = \sum_{k,j} f_j t_k \delta_{jk}.
\end{align}\label{eq:ip2}
Thus for two distributions represented by a vector of their spherical harmonic coefficients, the inner product of their functions is equal to the inner product of the two coefficient vectors.
The resulting image can be regarded as a pseudo-probability map of tract location, in arbitrary and dimensionless units with values typically in the range 0--0.5.
This measure of similarity, computed voxel-wise for the registered \gls{tod} atlas and subject \gls{fod} images, is the mapping used in tractfinder and all analyses presented in Chapters \ref{chap:eval}--\ref{chap:applications}.
It is favoured for its computational simplicity and relatively intuitive meaning, although alternative comparison approaches are also explored below.

The tract atlases and \gls{fod} data contain both scalar and directional information, all of which contribute to the final inner product value.
The combined effects of these factors means that a lower mapped value could be driven by a low degree of overlap in the angular domain (``the fibre orientations here to not match what would be expected for the tract''), low amplitude in the atlas (``tract is unlikely to be found here''), or low amplitude in the \gls{fod} (``there is not much white matter signal here'').
In \gls{sh} terms, the scalar component is the total amplitude, or spherical integral, of the \gls{odf} and is given by the first \gls{sh} coefficient ($Y_{0,0}$ term).
In the atlas, this amplitude represents the spatial probability, or the likelihood across the training population of a tract being located within a specific voxel.
Within the target \gls{fod} image, the amplitude is not tract specific, but rather reflects the \gls{afd} and by extension the likelihood that \textit{any} tract is present within a voxel, rather than grey matter or \gls{csf}.
However, it is crucial to note that this assumption only holds under specific circumstances, depending on the strength of diffusion weighting of the input data ($b$-value(s)) and \gls{csd} algorithm used to model the \glspl{fod}.

For $b$-values lower than around 2000--3000~s~mm$^{-2}$, the total diffusion signal is higher in \gls{gm} than in \gls{wm}, and the \gls{fod} amplitude does not provide a good estimate of \gls{afd} when using \gls{ssst} \gls{csd}.
This results in far more extensive segmentations based on the inner product in particular producing higher values within the cortex.
Therefore when using lower $b$-value acquisitions (which remain the norm in clinical applications), \gls{msmt} \gls{csd} should be used to estimate the volume fraction of different tissues and thus achieve better \gls{wm} specificity.
If only one diffusion-weighted shell (in addition to $b_0$) is available, then at most two tissues can be modelled using \gls{msmt} \gls{csd}.
The choice of whether to use either \gls{gm} or \gls{csf} as the second response function should be made as appropriate to the situation: If good specificity between \gls{wm} and \gls{gm} is desired, then those tissues should be used.
If, on the other hand, the tract of interest is adjacent to \gls{csf}, then the \gls{wm}+\gls{gm} strategy should be avoided, as the \gls{wm} \gls{fod} amplitude will be excessively attenuated in partial volume voxels with \gls{csf} (see Section \ref{sec:ismrmdiff} for further discussion).

Of further note is the behaviour in voxels containing multiple crossing fibre populations.
Given two voxels with equal \gls{fod} integrals (or equal \gls{afd}), in one containing two fibre populations that total \gls{afd} will be distributed among two \gls{fod} peaks, compared with one where the total \gls{afd} is contained within a single peak.
The presence of crossing fibres will therefore tend to reduce the total inner product amplitude, even if one of the \gls{fod} fibre populations aligns well with the atlas.
If the objective is to obtain a likelihood score for a particular tract, regardless of whether or not the tract is sharing a voxel with another fibre population, then the inner product framework, which slightly penalises crossing fibre voxels over single fibre ones, is inadequate (see also Section \ref{sec:fixel}). 

\subsection{Alternative similarity metrics}

The inner product is a simple comparison metric for the similarity of two vectors or distributions, each being equally weighted.
There are plenty of other possible similarity measures which could be computed, each with slightly different interpretations.
One option considered (although not strictly speaking a similarity metric) is the Kullback-Leibler (KL) divergence, a type of statistical distance for determining the information gained by approximating a measured reference distribution $Q$ with an estimated model distribution $P$, and is given by
\begin{equation}
  D_{KL}(P||Q) = \sum_x P(x) \ln \frac{P(x)}{Q(x)}. \label{eq:kl}
\end{equation}

KL divergence is an asymmetric quantity, meaning $D_{KL}(P||Q) \neq D_{KL}(Q||P)$, where $D_{KL}(Q||P)$ can be called the reverse KL divergence.
It is of interest for our purposes because $D_{KL}$ can be used to quantify how well an observed bimodal distribution $Q$ can be approximated with a unimodal model $P$.
This is similar to the problem of considering a \gls{tod} prior with a single orientation peak and a crossing fibre \gls{fod}, in which one of the crossing fibre populations aligns with the prior, and the other is of a different, irrelevant tract.
In this scenario, it would be useful to measure how well the \gls{tod} represents \textit{one} of the \gls{fod} lobes, while ignoring the contributions of $F$ where $T$ is zero.
This is effectively what the reverse KL divergence $D_{KL}(T||F) = \sum_x T(x) \ln \frac{T(x)}{P(x)}$ would represent, as the weighting by $T$ suppresses any contributions to the sum where $T$ is zero, regardless of the value of $F$.

After initial investigations, however, the use of a KL divergence-based mapping hasn't been further pursued for a few reasons.
Firstly, the interpretation of the computed value is difficult, as higher similarity is indicated by lower divergence, and the values can be infinitely high.
Secondly, as $D_{KL}$ is defined for probability distributions (with unit integral), it could be only used for measuring angular agreement (after normalising $F$ and $T$), leaving the question of how to reintroduce the spatial information in the \gls{tod} and \gls{fod} amplitudes.
Prototyping experiments indicated that in practice, little useful information could be conferred from considering the KL  divergence over the inner product or other options (see Section \ref{sec:fixel}).
Finally, an analytical expression for the multiplication of two arbitrary \gls{sh} functions, which would be needed to compute the metric for continuous spherical distributions, is not known for the real spherical harmonics used in \gls{dmri} analysis, meaning it can currently only be calculated discretely, requiring the distributions to be densely sampled at great computational cost.

\subsection{Fixel analysis}\label{sec:fixel}

As described in Section \ref{sec:ip}, the inner product value between an estimated orientation distribution representing a mixture of fibre populations and the expected orientation distribution for a specific isolated population will depend on a range of interacting components.
To disentangle the competing effects in crossing fibre voxels, we consider an extension of the inner product analysis using the concept of fixels.
A portmanteau of ``fibre'' and ``voxel'', a fixel describes a sub-voxel element representing a single fibre population\autocite{Raffelt2015,Raffelt2017}.
A voxel can contain any number of fixels, and each can be analysed in isolation with fixel-level properties such as peak orientation, dispersion and fibre density.
By segmenting a full \gls{fod} into individual lobes and computing properties on those sub-distributions, we can separate the effects of different fibre populations within a single voxel.

The fixel framework has been adapted here to extend the inner product to consider individual \gls{fod} lobes in isolation.
\Gls{fod} segmentation is achieved as described in \textcite{Smith2013} by sampling the \gls{sh} distribution on a dense set of directions (\textgreater 1000).
Each of these direction samples is binned into a fixel based on its neighbourhood using a fast-marching level set algorithm, resulting in each fixel being represented by a collection of directional samples and the associated amplitudes of the original \gls{fod} (Fig. \ref{fig:fixip}f--g).
Each fixel is then converted to \gls{sh} representation via a linear least-squares fit to the sampled amplitudes, producing separate \gls{sh} distributions $I_p(\theta, \phi)$ per voxel for each fibre population $p$.

The amplitude of the fixel \glspl{odf} will reflect that of the corresponding lobe in the original full multi-fibre distribution, and thus will always be smaller or equal to that of the original \gls{fod}.
However, we are interested in whether the fibre population of interest is present in a voxel based on orientation alone, regardless of its relative weight to the other fibres occupying the same voxel.
The location probability, in other words, should be influenced only by the expected location probability $t_0$ and the overall white matter signal integral $f_0$.
To remove the relative weighting effect of crossing fibres, we re-normalise each fixel \gls{odf} to the overall \gls{fod} amplitude:
\begin{align}
  \widetilde{I} = f_0\frac{I}{i_0}
\end{align}
where the fixel \gls{odf} is $I(\theta, \phi) = \sum_j i_j Y_j(\theta,\phi)$.
Then we compute the inner product of the atlas \gls{tod} and each fixel \gls{odf} $I_p$ separately, before taking the maximum value as the result:
\begin{align}
  P = \operatorname*{argmax}_k \langle \widetilde{I}_k(\theta,\phi), T(\theta,\phi) \rangle
\end{align}

The result is improved tract sensitivity in areas of crossing fibres where there is still sufficient evidence that the tract is present in the voxel, and less ambiguity in the interpretation of a low mapped value.
The fixel inner product framework was tested in a synthetic image known as the Fibre Cup phantom, after a tractography challenge held for the 12th international conference on Medical Image Computing and Computer-Assisted Intervention (MICCAI) in 2009\autocite{Fillard2011}.
The phantom features seven bundles arranged in a way that aims to mimic the range of configurations found in the brain, including sharp curves, multi-way crossings, kissing and fanning fibres.
A digital recreation of the original data and accompanying ground truth streamlines are openly available\autocite{NITRC}.
Figure \ref{fig:fixip} demonstrates the effect of using the fixel framework to reconstruct a tract segmentation with consistent intensity values across crossing fibre regions.
To address the concern that renormalising each $I_p$ could excessively amplify noise, lending equal voice to tiny \gls{fod} lobes that make practically no contribution to the total voxel signal, we can place a minimum threshold on the lobe amplitude for a single fixel during the \gls{fod} segmentation process.

\begin{figure}[tbh!]
  \centering
  \includesvg[width=1.08\textwidth,pretex=\small\sffamily]{chapter_3/fixel.svg}
  \caption[Fixel inner product]{Demonstration of fixel-wise inner product using the synthetic Fibre Cup Phantom.
  \textbf{\sffamily a.} Seven ground truth streamline bundles intersect throughout the phantom.
  \textbf{\sffamily b.} The brown tract intersects with three other parallel bundles. The crossing fibres are denser and therefore dominate the \gls{fod} reconstruction (\textbf{\sffamily d.}), resulting in a drop in tractfinder amplitude.
  \textbf{\sffamily c.} By computing the inner product with each fixel separately, weighted by the overall \gls{fod} amplitude, and taking the maximum value, a smooth tract reconstruction is achieved.
  \textbf{\sffamily e.} \glspl{fod} in a second crossing area, showing the brown and purple tracts intersecting at an oblique angle.
  \textbf{\sffamily f.} First segmented fixel.
  \textbf{\sffamily g.} Second segmented fixel.
  \textbf{\sffamily h.} \Gls{tod} atlas for the brown tract.}\label{fig:fixip}
\end{figure}
\clearpage{}
\clearpage{}\section{Atlas data requirements}\label{sec:ntrain}

In segmentation methods which ``learn'' patterns from seen data to apply to unseen data, the volume and range of training samples influences the prediction accuracy and generalisability.
For complex deep learning models, which have many thousands of network parameters to learn, the amount of training data required to achieve accurate and stable performance can be immense, posing a particular barrier to the use of such models in applications where suitably annotated data is scarce.
In the case of the deep learning tool TractSeg\autocite{Wasserthal2018}, for example, 105 subjects in total were used for cross-validation training, with each fully trained model having seen randomly sampled data slices from 63 unique subjects.
In tractfinder, the number of subjects used to construct each atlas influences the amount of inter-subject anatomical variation reflected in the spatial and orientational components.
It is to be expected that, save for extreme outliers, the additional information gained from adding more training subjects would reach a point of saturation.

To investigate this, in work which first appeared in \textcite{Young2023}, an experiment was conducted whereby the number of subjects included in atlas construction was varied, and the effect on segmentation accuracy compared.
For this purpose the TractSeg \gls{hcp} bundles were used in order to enable objective evaluation against reference segmentations defined in the same manner as the training data, and direct comparison with TractSeg itself.
Using the same train--test data split as described in \textcite{Wasserthal2018b}, subsets of 1, 3, 5, 10, 15 and 30, as well as the full set of 63 training subjects were randomly selected, from which separate \gls{tod} atlases where constructed.
Tractfinder maps were then generated in the 42 test subjects using each of the different subset atlases and compared with the available reference segmentations using the \gls{dice} \DIFdelbegin \DIFdel{and density correlation metrics }\DIFdelend (see Section \ref{sec:metrics} for details).

\begin{figure}[htb!]
    \centering
    \makebox[\linewidth][r]{\includegraphics{chapter_3/ntrain.pdf}}
    \caption[Tractfinder performance against number of atlas training subjects]{Comparison of segmentation performance using different numbers of atlas training subjects. Results are grouped by tract, colour represents number of training subjects. The IFO and \glsentrylong{or} are in places indistinguishable. \acrolist{af,cst,or,ifof}}
    \label{fig:ntrain}
\end{figure}

When using only a single subject's normalised \gls{tod} map as an ``atlas'', mean \glspl{dice} ranged from 0.65 to 0.71 for the \gls{ifof} and \gls{cst} respectively (Fig. \ref{fig:ntrain}).
The maximum increase in mean \gls{dice} between the 15 and 63 subjects atlases was 0.00835, for the \gls{cst}, representing only a 1\% increase from the lower score of 0.759.
Across all tracts and both comparison metrics, differences in performance between the different atlases above 10 training samples were consistently negligible.
These results indicate that additional atlas subjects beyond a minimum number of around 10--15 do little or nothing to improve tractfinder results.
This can be interpreted as the extra training subjects offering minimal additional information on inter-subject variability, as a lot of this variability is already smoothed out due to affine (instead of diffeomorphic) co-registration of training subjects into template space.

The effects of additional training data may present differently if the atlases are constructed with non-linear co-registration of training subjects. There are two sources of inter-subject variability wrapped up in the atlas:
The first is the global anatomical variability including skull shape and differences in cortical surface geometry, the second is variability in position and shape of the tract itself.
Theoretically, diffeomorphic registration of training subjects would eliminate the first of these effects (global variability), leaving only the tract specific variation.
However, such an atlas would necessitate subsequent applications in new target subjects to also utilise diffeomorphic registration between subject and template space, as the atlas would contain no ``allowance'' for global variability, expecting perfect alignment with a target image.
Requiring diffeomorphic registration at the point of application would greatly inhibit the robustness and speed of tractfinder, and is therefore not the preferred approach.

The issue of deep learning's data requirements has been attracting increased attention.
A search on the online publication database Web of Science for \spverb|(few OR single OR one) AND shot AND learning|  (excluding results about single-shot echo planar imaging) revealed a sharply increasing trend in publication volume within the field of medical imaging (Fig. \ref{fig:pubs}).
A recent example, by \textcite{Liu2023a} and building on previous work in \textcite{Lu2021}, looked at using single-shot learning to train a deep neural network, an extension of the TractSeg architecture, for \gls{wm} tract segmentation.
They trained a network on 60 of the 72 total TractSeg tracts, and applied data-augmentation and transfer learning techniques to adapt the model to segment the remaining 12 tracts using only a single exemplar training dataset for those novel tracts, which included both the \gls{cst} and \gls{or}.

Considering only their best results (which differed depending on how many novel tracts the network was extended to) for these two tracts, they achieved a mean \gls{dice} of 0.719 for the \gls{cst} and 0.624 for the \gls{or}, equally good or worse than for tractfinder ``trained'' even on only a single dataset (Fig. \ref{fig:ntrain}).
While the computational time spent on network training is not disclosed in either \textcite{Liu2023a} or \textcite{Wasserthal2018}, \textcite{Berto2021} re-trained their own TractSeg model as a benchmark comparison for the streamline clustering method Classifyber and reported GPU-accelerated training times of 3--7 hours.
Even so, one-shot training of a deep neural network appears to offer no improvement over simply registering a segmentation from one subject to another.

\begin{SCfigure}[][h!]
  \centering
  \includegraphics{chapter_3/pubs.pdf}
  \caption[Publications on few-shot learning]{Publication records by year including the term ``single/one/few shot learning'' (or similar) on the database Web of Science.}
  \label{fig:pubs}
\end{SCfigure}
\clearpage{}

\clearpage{}\chapter{Aligning atlas and data}
\label{chap:reg}

Each tract atlas is constructed in a standard template space, an averaged, idealised brain image in standardised reference coordinates, and before being used to estimate the location of a tract in a new subject it must first be aligned with that individual's anatomy.
In order to achieve this, some sort of registration strategy is needed, with the most appropriate registration approach depending on the type of application.
In healthy subjects, it is sufficient to use affine registration to transform the atlas from template into subject space, before the inner product (or whichever comparison is used) is calculated.
However, in some patient data, more advanced processing may be necessary to ensure satisfactory alignment with target anatomy.
The following sections, some elements of which have been previously published in \textcite{Young2022}, will describe the registration methods used in tractfinder for different scenarios, including the use of tumour deformation modelling to account for space-occupying lesions.

\section{Registration from standard space}
\label{sec:reg1}

There are two main categories of image registration.
In global registration, a single transform applies to the entire image, preserving its topology.
Different degrees of freedom define some commonly used terms:
Rigid body registration comprises just translation and rotation, while affine registration further includes scaling and shearing (up to 12 degrees of freedom on $\mathbb{R}^3$).
Following purely algebraic definition, a linear transform does not include translation, however in practice (at least in the medical imaging sphere) ``linear'' and ``affine'' are often used interchangeably.
Non-linear, deformable, or diffeomorphic registrations compute local deformations on a voxel-by-voxel basis, and therefore have orders of magnitude more degrees of freedom.
They are useful for fine-grained alignment of local structures, but many algorithms are unstable and prone to converging on local minima, making them difficult to successfully integrate into automated pipelines.

As we saw in Chapter \ref{chap:atlas}, a certain degree of inter-subject variability, including both tract-specific and global anatomical differences, is a feature of the atlas due to the use of affine registration to combine the training data in template space.
We mirror this relationship by allowing the atlas to be registered to a new target dataset using affine registration alone, with the subsequent comparison with native \gls{dmri} data acting to correct slight misalignments due to anatomical variability.
Computing a non-linear registration between atlas and subject at the point of application is undesirable principally due the lack of robust, stable and generalisable algorithms that are open source.
Maintaining practicality is a key requisite of tractfinder, and various non-linear registration tools were found during prototyping to either be too unstable (requiring manual adjustment of parameters in difficult cases such as some clinical scans) or having too long a computation time\autocite{Visser2020}.

As we shall see in the range of evaluations presented in Chapters \ref{chap:eval}--\ref{chap:applications}, the use of affine registration does not result in significant segmentation inaccuracies or errors, thus there is no credible incentive to favour the use of non-linear registration at the cost of increased processing time and potential instabilities.
Indeed, \textcite{Visser2020} showed that subject--to--standard registration accuracy of the tumoural region in low- and high-grade gliomas is not significantly improved using non-linear registration across a range of different packages, concluding there is little to justify the additional time cost and lack of robust automation.
Nevertheless, the trade-offs associated with relying on affine registration must be acknowledged.
Particularly in brains with modest deformations or other lesions, the accuracy of anatomical alignment can be impacted, and there follows the risk of suboptimal spatial and/or orientational alignment, resulting in missed areas (false negatives) or erroneously included regions (false positives).

The risk of misalignment is especially high for small structures, such as the fornix or anterior commissure, both narrow bundles which are reliably difficult to segment.
For these cases, the construction of the atlas with spatial inter-subject smoothing is advantageous, as even with slight misalignment there is still a good chance that enough of the atlas will overlap with the structure in the target image to achieve detection.
However, this same feature may also lead to false positives in some cases, such as when two distinct tracts run in parallel, oriented the same way in relative spatial proximity.
An example of this scenario can be seen at the temporoparietal fibre intersection area, where at least seven different identified bundles converge\autocite{Martino2013}.
Here the vertical portion of the arcuate fasciculus lies lateral to the sagittal stratum containing antero-posteriorly oriented association fibres (including the \gls{or}), which in turn lies lateral to the tapetum of the corpus callosum.
The similar orientations of the \DIFdelbegin \DIFdel{the }\DIFdelend \gls{af} and tapetum in this region, together with their close proximity, could lead to fibres of one being wrongly attributed to the other if the atlas is too broad (Fig. \ref{fig:tpfia}).

\begin{SCfigure}[][h!]
  \includesvg[width=0.6\textwidth,pretex=\ttfamily\small]{chapter_4/tpfia.svg}
  \caption[Atlas misalignment with linear registration]{Example of potential for atlas misalignment. The \glsentrylong{af} (A) and tapetum (T) are proximal and parallel at the temporoparietal fibre intersection area. Linearly registered right \glsentrylong{af} atlas \glsentrylongpl{tod} may overlap with the tapetum (arrowhead).}
  \label{fig:tpfia}
\end{SCfigure}

Fortunately, such misalignment effects are small and unlikely to impact the overall segmentation quality, in part because only the margins of the atlas are likely to ``spill'' into neighbouring tracts, and the low spatial probability will result in very low or sub-threshold mapped values.
To confirm this assessment, Figure \ref{fig:nrr} shows tractfinder segmentation similarity scores compared with probabilistic targeted tractography when using either affine (FMRIB's Linear Image Registration Tool\autocite{Jenkinson2002}) or non-linear (ANTs registration package Symmetric Normalisation algorithm\autocite{Tustison2013,Avants2011}) atlas registration in 71 healthy subjects (\textit{TractoInferno} dataset, see Section \ref{sec:data} for more details).
Note that the atlases themselves are unchanged from those previously described and used in subsequent analyses, meaning the generation of the atlases still involves only affine registration between training subjects.
There is no discernible difference in the scores, indicating that non-linear registration does not improve the final output.
This comparison originally appeared as supplementary material in \textcite{Young2024}.

\begin{figure}[hbt!]
  \makebox[\linewidth][r]{\includegraphics{chapter_4/registration_2.pdf}}
  \caption[Comparing linear and non-linear atlas registration]{Difference in tractfinder performance when using either affine or non-linear atlas registration, compared with targeted \glsentrylong{roi} tractography, in the \textit{TractoInferno} dataset of healthy subjects. For the \glsentrylong{dice} a threshold of 0.05 applies.}\label{fig:nrr}
\end{figure}

In conclusion, affine registration between template and subject space is preferred in most applications, including for healthy subjects and clinical subjects without large space-occupying lesions.
The next sections will address the scenarios where patient anatomical topology differs significantly from atlas space:
In the presence of deforming tumours, or intraoperative brain shift.
\clearpage{}
\clearpage{}\section{Tumour deformation modelling}
\label{chapterlabel3}

Tract orientation atlases represent the expected orientation and location of a tract in typical healthy subjects.
In the previous section we asserted that, for subjects with little structural divergence from the norm, linear registration is sufficient for aligning the atlas and target data.
However, in cases with large mass-effect, affine registration becomes clearly inadequate, as the distances between brain structures' expected and actual locations are simply too great.
In order to correct for displacement of \gls{wm} tracts due to space-occupying lesions, the atlas will need to be deformed more dramatically before comparing with the native \gls{fod} map.

Anatomical non-correspondence between subject and template images caused by space-occupying lesions poses a substantial challenge to the use of atlas-based \gls{wm} segmentation methods in clinical subjects.
While nonlinear deformation tools can produce accurate registration in clinical images, they require usually manual adjustment of many input and regularisation parameters on a case by case basis, and a robust automation of this process is not available to best knowledge.
More significantly, in the case of registration between a normative template and a scan with a brain tumour, a fundamental assumption of many registration algorithms, that of topological equivalence between the two images, does not hold\autocite{Zacharaki2009}.
The core of the problem is that one image---the template---depicts the anatomy of an average, healthy brain, and the other that of a diseased brain affected by a tumour, presenting two potential scenarios.
Either the lesion exists \textit{in addition} to all the structures and tissue expected of a healthy brain, which have been displaced or compressed to accommodate it, or it has \textit{replaced} brain landmarks or rendered them otherwise unrecognisable through the effects of oedema or differing MR properties between tumour and healthy brain tissue.
In either case, a non-linear registration algorithm is tasked with finding a mapping between two images with different sets of anatomical landmarks, with finding anatomical correspondence between tissue in one image which is nonexistent in the other.
The result of this violation of the topological equivalence assumption is often ridiculously contorted images, and particularly in the peritumoural zone, which is rather problematic for neurosurgical applications.

Deformable registration alone is thus largely insufficient for handling the anatomical mismatch problem\autocite{Elazab2018, Visser2020}.
It has long been proposed that an acceptable registration between atlas and tumour data can only be obtained with an additional step of artificially implanting or modelling a representation of the tumour in the atlas\autocite{Cabezas2011,Mang2020}.
This can be in the form of a seeded atlas deformation\autocite{Dawant2002}, where a small seed is placed in the atlas before deformable registration, which acts as a region of tissue which can be warped to match the full tumour in patient space.
Alternatively, the seed can be artificially ``grown'' using a biophysical or mathematical model of tumour proliferation to simulate deformation, followed by registration of the deformed atlas to the patient image\autocite{Cuadra2004, Zacharaki2009}.
Many proposed tumour deformation models aim to achieve highly accurate modelling of tumour growth dynamics and the effects on surrounding tissues, by taking into account elastic tissue properties and microscopic tumour growth modelling (see \textcite{Elazab2018} for a comprehensive review).
Generally models will consider one or a coupling\autocite{Clatz2005,Hogea2007,Prastawa2009} of tumour cellular proliferation and infiltration into surrounding tissues using a reaction diffusion or similar framework
\autocite{Tunc2021,Scheufele2019b,Elaff2018},
or the biomechanical forces acting between tissues\autocite{Mohamed2006,Hogea2007a,Zacharaki2009}.
The resulting algorithms are often mathematically complex and implemented using finite element methods\autocite{Elazab2018}, require optimisation of tumour parameters through problem inversion or by other means \autocite{Mohamed2006, Zacharaki2009, Mang2020} and take anywhere between one and 36 hours to run, even on high performance computing setups\autocite{Zacharaki2009,Bauer2012,Gooya2012,Bauer2013,Mang2012}.
This is entirely reasonable for studies in which accuracy is a far greater priority than speed.
Typical applications of tumour deformation modelling include intra-patient longitudinal studies of tumour growth, and inter-patient registration and spatial normalisation for atlas-based segmentation or statistical analysis across patient populations (see \textcite{Bauer2013} and \textcite{Cabezas2011} for overviews).
Given the time constraints of intraoperative imaging and the practical constraints of the computing capacity which can reasonably be assumed to be available in a clinical setup, the aim for this project was to achieve an estimate of tract displacement with low computational complexity.

The tract orientation atlas described in the previous chapter provides a degree of spatial tolerance that alleviates the need for voxel-perfect registration and deformation, allowing the implementation of a minimal deformation algorithm.
The idea is to obtain a deformation model which is simple enough to compute using few temporal and computational resources, and use it in combination with affine registration, as before, to achieve spatial alignment between atlas and patient space.
To this end we will interest ourselves solely in the macroscopic spatial effects of tumour growth:
What effect does the presence of a tumour have on the physical position of, and the fibre orientations within, a given volume unit of tissue?
Of course, this question cannot be fully answered without considering the complicated factors described above, such as whether a tumour is encapsulated or infiltrating.
Nevertheless, as all models are wrong, and some are useful, the utility of our model will be measured by its ease of computation and accurate capturing of tract displacement.
Whether performance in the latter criteria is satisfactory will be measured through the resulting improvement in tract mapping as compared with non-atlas methods such as streamline tractography (see Section \ref{sec:btcd}).
We will consider a radial model of tumour deformation, previously described in \textcite{Young2022}, assuming the tumour has expanded outwards from a central seed, and that surrounding tissue is displaced along the same radial directions.
\textcite{Cuadra2004} similarly used a radial expansion assumption, but for modelling the interior tumour region (rather than outside the tumour), while an optical flow algorithm was implemented for image matching outside the tumour.

\subsection{Development of a radial deformation model}

We begin with a radial deformation model described by \textcite{Nowinski2005}.
Their motivation was remarkably similar:
The rapid deformation of a morphological brain atlas to aid the interpretation of brain anatomies affected by tumour mass effect.
The required model inputs are the segmentations of the tumour and brain volumes.
We define the direction $\mathbf{\hat{e}}$, which is the unit vector along the line connecting a point $P(x,y,z)$ anywhere within the brain to the tumour centre of mass, $S$.
This is the direction along which we assume the tissue at that point to be shifted by the tumour:
Radially outward from the tumour centre.
Along $\mathbf{\hat{e}}$ we also define $D_p$ as the distance  $\|\overrightarrow{SP}\|$, $D_b$ as the distance from $S$ to the brain surface\DIFaddbegin \DIFadd{, }\DIFaddend and $D_t$ as the distance from $S$ to the tumour surface (Fig. \ref{fig:virtue}).

\begin{figure}[htp]
  \centering
  \includesvg[inkscapelatex=true]{chapter_4/virtue_vars.svg}
  \caption[Tumour deformation model variables]{Graphical schema of the variables defined in the radial deformation model}
  \label{fig:virtue}
\end{figure}

Then for a point in the original image $P = (x,y,z)$ the transformed location in the deformed image $P' = (x',y',z')$ is
\begin{align}\label{eq:forwardP}
  P' = f(P) = P + \mathbf{\hat{e}}kD_ts.
\end{align}
The amount of displacement $\Delta P = kD_ts$ is thus determined by $D_t$, a scale factor $0<s \leq 1$ and a spatially varying displacement factor $k(P)$.
In \textcite{Nowinski2005}, $k$ is a linear function of $D_p$: $k = 1-\frac{D_p}{D_b}$. \footnote[2]{In the original \textcite{Nowinski2005} article, the deformation is described in reverse, as a shrinking model, and the variables there look a little different. They are consistent with the formulation used here, which has been chosen for ease of conceptualisation. Note that both forward and reverse models are required for different types of image transformation, and will be derived later.}
This can be conceptualised as a displacement force radiating from the centre of the tumour and decaying linearly with distance, reaching 0 only at the brain boundary \DIFdelbegin \DIFdel{.
}\DIFdelend \DIFaddbegin \DIFadd{(Fig. \ref{fig:k}).
}\DIFaddend However, initial experimentations with this model revealed that such a linearly decaying force doesn't do well at capturing the displacement fields observed in real tumour cases.
The elasticity and compressibility of brain tissue means that the radial force is absorbed by surrounding tissue more rapidly than a linear function accounts for.
Even with very large tumours, it is common for parts of the brain some distance from the tumour surface to experience no displacement at all, suggesting a  more rapidly decaying function would be a more appropriate choice for $k$.

An exponentially decaying function captures this well, while remaining easily computable, close-form\DIFaddbegin \DIFadd{, }\DIFaddend and invertible.
We begin with a function in the form \DIFdelbegin \DIFdel{$k(P) \propto e^{-\lambda \frac{D_p}{D_t}}$}\DIFdelend \DIFaddbegin \DIFadd{$k(P) \propto e^{-\lambda \frac{D_p}{D_b}}$}\DIFaddend .
There are two boundary conditions:
Points on the brain surface should not be displaced ($k(D_p = D_b) = 0$) and points at the centre of the tumour should be displaced by exactly $D_t$ ($k(D_p = 0) = 1$).
Note that the latter boundary condition is an assumption reflecting a fully encapsulated tumour, where no normal tissue remains inside the final tumour boundary after displacement.
Solving for these boundary conditions gives us a normalisation constant:
\begin{align}
  k &= a e^{-\lambda x} + c &\text{ where } x = D_p / D_b \nonumber \\
  k(x=0)=1 \longrightarrow 1 &= a e^{-0} + c = a + c \nonumber \\
  k(x=1)=0 \longrightarrow 0 &= a e^{-\lambda} + c \nonumber \\
  -c &= (1-c) e^{-\lambda} \nonumber \\
  c &= \frac{e^{-\lambda}}{e^{-\lambda} - 1} \label{eq:c}
\end{align}
giving
\begin{align}\label{eq:forwardk}
  k(P) = (1-c)e^{-\lambda \frac{D_p}{D_b}} +c.
\end{align}

Equations (\ref{eq:forwardP}) and (\ref{eq:forwardk}) describe the forward deformation transform $P'=f(P)$, which maps a point in the original image $P$ to a new position $P'$ in the deformed image.
Forward warping works well for continuous valued data such as streamlines:
As the transformed image is also defined in continuous coordinates, each vertex can be ``pushed'' to its exact location in the warped output.
When transforming discrete image data defined on a pixel (or voxel) grid, however, the transformed coordinate $P'$ will not generally correspond to a grid point, resulting in voxels in the transformed image with no assigned value (``holes''), or voxels being assigned values from multiple overlapping mapped points.
To produce the transformed image, the value of $P'$ has to be distributed among the neighbouring voxel grid points (within a predefined kernel) using a process called ``splatting''\autocite{Niklaus2020}, where each voxel is assigned the value weighted by its distance to $P'$ (Fig. \ref{fig:warp}).
As each voxel value will be determined by a weighted sum of all transformed values, this method requires an intermediate buffer to store all transformed points and distributed weights before the final voxel values can be determined, and filling techniques may be required to fill in any holes.

\begin{figure}[h!]
  \centering
  \includesvg[width=0.7\textwidth]{chapter_4/warping.svg}
  \caption[Forward and reverse image warping]{Forward image warping $f(P)$ maps from original to transformed coordinates, where the transformed value is distributed among neighbourhood grid points in a process called ``splatting'' (orange). Reverse image warping uses the inverse transform $f^{-1}(P')$ to determine the source position in the original image, with the appropriate value interpolated from the neighbourhood of $P$ (blue).}
  \label{fig:warp}
\end{figure}

It is therefore usually preferable, if the inverse transform $P = f^{-1}(P')$ is known, to use reverse warping, in which each grid point value in the transformed image is ``pulled'' from the corresponding continuous-valued point in the source image (Fig. \ref{fig:warp}).
As with forward warping, the reverse mapped point $f^{-1}(P')$ will not generally fall exactly on a grid point in the source image, so the appropriate value is interpolated from the neighbourhood.
Reverse warp convention is preferred in medical image manipulation packages to smoothly deform (and resample) a gridded image, and so we need to obtain the inverse mapping $P = f^{-1}(P')$.
The inverse function for the linear model (as formulated in \textcite{Nowinski2005}) is
\begin{align}
  P = P' - \mathbf{\hat{e}}(1-\frac{D_{p'}-D_t}{D_b-D_t})D_ts.
\end{align}
To obtain the inverse mapping for the exponential model, we solve equation (\ref{eq:forwardP}) for $P$, using the exponential $k(P)$ given by (\ref{eq:forwardk}) to get:
\begin{align}
  P = P' - \mathbf{\hat{e}}(D_t s c - \frac{D_b}{\lambda}\mathcal{W}_0(\frac{-\lambda D_t s (1-c) e^{-\lambda(D_{p'}-D_tsc)/D_b}}{D_b}))
\end{align}
where $\mathcal{W}_0(y)$ is the principal branch of the lambert $\mathcal{W}$ function, defined as the inverse function of $ y(x) = xe^x $ for $x,y \in \mathbb{R}$.

The most appropriate value for the exponential decay parameter $\lambda$ will depend on characteristics of the specific lesion being modelled.
For example, smaller lesions (20--30~mm diameter) typically displace tissue only in their immediate surroundings, with distant tissue remaining virtually unmoved.
In such cases, a higher value of $\lambda$ ($\geq 3$), indicating stronger fall-off in displacement force, would be appropriate.
In any case, in order to keep the transforms well-behaved, we need to enforce the boundary condition that every point $P$ in the source image that is within the tumour perimeter ends up strictly outside the tumour in the eventual deformed image.
In other words,
\begin{equation}\label{eq:lambdabound}
  k(P) \geq 1 - \frac{D_p}{D_t} = g(P)
\end{equation}
must hold for all $P$ (Fig. \ref{fig:k}).
Given that the gradient of $k$ is strictly decreasing and $g(P) = 1 - \frac{D_p}{D_t}$ is linear, it is sufficient to set
\begin{align*}
  \frac{d}{dP}\bigg\rvert_{D_p=0}k(P) &= \frac{d}{dP}\bigg\rvert_{D_p=0}g(P) &\text{ where } \frac{dk}{dP} &= -\frac{\lambda}{D_b}(1-c)e^{-D_p/D_b} &\text{ and } \frac{dg}{dP} &= -\frac{1}{D_t}
\end{align*}
We solve for $\lambda$:
\begin{align*}
  -\frac{\lambda_{max}}{D_b}(1-c) &= -\frac{1}{D_t} \\
  \lambda_{max} &= \frac{D_b}{D_t (1-c)}
\end{align*}
where $c$ is itself a function of $\lambda$ as defined in (\ref{eq:c}). This value can be determined iteratively, or with the expression
\begin{equation}
  \lambda_{max} = \mathfrak{Re} \left[ \mathcal{W}_0(-\frac{D_b}{D_t}e^{-D_b/D_t}) \right] +\frac{D_b}{D_t}.
\end{equation}
Thus for strictly non-infiltrating lesions, we set $\lambda \leq \lambda_{max}$ to satisfy equation (\ref{eq:lambdabound}), where $\lambda_{max}$ is used as the default value if none is specified (referred to as ``adaptive $\lambda$'').
Note that $\lambda_{max}$ varies throughout the brain, being a function of the relative distances to brain and tumour surfaces for each specific $P$.

\begin{SCfigure}[][h!]
  \includegraphics{chapter_4/k.pdf}
  \caption[Deformation factor k]{Deformation factor $k$ as a function of $D_p$. $\lambda$ must be small enough such that $k_{\lambda}$ is strictly above the line $1-(\frac{D_p}{D_t})$ (dashed line). An exponential $k$ with $\lambda_{max}$ is plotted in solid black, compared with a linear $k$ as proposed in \textcite{Nowinski2005} (dotted line).}
  \label{fig:k}
\end{SCfigure}

The tumour deformation model is implemented in Python, and full execution takes on average 1~minute for a 208\x{}256\x{}256 voxel image.
If lookup tables for $ D_t$ and $D_b$ are precomputed and saved, then subsequent executions of the model (e.g. with different values for $\lambda$ and $s$, as appropriate for a given tumour) take less than 10~seconds, as long as the tumour and brain segmentations remain unchanged.

\subsection{Limitations}

Due to its mathematical simplicity and crudeness in assumptions about tissue mechanics, the model described above naturally comes with several limitations, of which two in particular have a noticeable effect on real clinical applications and generalisability.
One is the inbuilt assumption of full tumour encapsulation, as expressed in the boundary condition that all existing tissue in the template brain end up outside the tumour perimeter in the deformed image.
Through this assumption we have explicitly excluded the modelling of infiltrative tumours, which may distort existing structures but not necessarily wholesale shunt them outside the tumour boundary.

The difficulty with modelling tumour infiltration under the current framework is in the appropriate demarcation of the lesion in the subject scan.
We can broadly categorise tumour regions into three components based on their interaction with the non-tumour environment:
A tumour core, non-infiltrating and possibly cystic or necrotic, an infiltrating component, and finally peritumoural oedema.
These distinctions are not enough to predict the mass effect of a tumour.
Some tumours that are entirely infiltrative exert no visually perceivable mass effect (although infiltration and oedema will always act to raise intracranial pressure), while others do.
In a lesion comprising both infiltrating and non-infiltrating parts, the latter may well be the only cause of mass effect, and readily demarcated, but not necessarily.
Hence even if it were practical to more finely segment a tumour, separately labelling oedema, infiltration and core tumour components, exploiting this information to more accurately model the full mass effect would be by no means straightforward, and perhaps not possible in a way that generalises to all tumours without resorting to the in-depth modelling described in the introduction to this section.

This is not to say that the tumour deformation model in its present form cannot handle any form of infiltrative tumour (indeed, we will see examples in Chapter \ref{chap:applications} of such cases), for example by adjusting the scale parameter $s$ if the segmentation partially includes infiltrative tumour.
However, tumours which are more or less wholly infiltrating, for example cases 4 and 5 in Figure \ref{fig:fa_hist}, cannot be modelled.
An alternative form of $k(P)$ could be considered, such as the radial polynomial models used in optics and digital imaging to describe and correct for lens distortions\autocite{Zhang2000a}.
The magnification effect near the centre of a radial distortion field is comparable with the distending of tissue seen in some infiltrative tumours, see for example case 5 in Figure \ref{fig:fa_hist}.
If using different deformation models and parameters for infiltrating and encapsulated tumours, then an means for automatically determining which model applies in a given case would be required to keep user input to a minimum.
This could be achieved by comparing the \gls{fa} values inside the tumour with those in the brain as a whole, as an indication of whether anisotropic \gls{wm} is present within the tumour boundary (Fig. \ref{fig:fa_hist}).
In-depth exploration of such additional distortion models was not pursued for this project, principally because the relevant clinical cases are far less likely to be indicated for surgical resection owing to the grave risk to infiltrated structures.
Nevertheless, accurate \gls{wm} imaging could still be informative in assessing the lesion and informing treatment plans such as radiotherapy targeting\autocite{Jena2005,Berberat2014}, so the extension of this framework to include robust handling of infiltration could be an interesting avenue for future work.

\begin{figure}[htb!]
  \makebox[\linewidth][r]{\includegraphics{chapter_4/tumour_fa.pdf}}
  \caption[Intra-tumoural FA indicates infiltration]{A series of paediatric brain tumour cases from retrospective \glsentrylong{gosh} data. For each subject an axial slice through the tumour centre is displayed, with \glsentrylong{dti}-derived \glsentrylong{dec} \glsentrylong{fa} maps overlaid. Patients 4 and 5 have infiltrating tumours centred on the left and right basal ganglia respectively. Histograms depict the \glsentrylong{fa} values within the entire brain volume (including the tumour) and tumour volume only. The two infiltrating tumours, 4 and 5, have higher overall \glsentrylong{fa} values than the rest of the brain, indicating the presence of infiltrated anisotropic white matter within the tumour.}
  \label{fig:fa_hist}
\end{figure}

\begin{figure}[htb!]
  \includesvg[width=\textwidth,pretex=\small\sffamily]{chapter_4/ventricles.svg}
  \caption[Ventricle deformation examples]{Examples of ventricle deformation under tumour pressure, and tumour deformation modelling in template image. Arrows indicate ventricle compression not fully captured in the deformation model. \textbf{\sffamily a.} \glsentrylong{btc} sub-PAT03. \textbf{\sffamily b.} \glsentrylong{btc} sub-PAT26}
  \label{fig:ventricles}
\end{figure}

A second key limitation of the present deformation model relates to the characteristics of the surrounding tissues rather than the tumour itself.
The displacement of a position $P$ in the brain depends only on the distances between it, the tumour, and the brain surface.
Otherwise, the brain is treated as a homogenous mass, when in reality it comprises regions with very different mechanical properties.
This becomes most apparent when we observe the deformations involving the ventricles.
The ventricles, forming a part of the wider \gls{csf} network, are far more compressible than \gls{wm} or \gls{gm}, allowing them to often absorb to a large extent the compressive forces exerted by the tumour and causing them to collapse more than the adjacent tissue.
These tissue-dependent effects are completely omitted from the presented model, leading to mis-estimations of the deformations surrounding ventricles in some cases (Fig. \ref{fig:ventricles}).
Controlling the magnitude of deformation adjacent to the tumour by adjusting the deformation decay constant $\lambda$ can mitigate this affect to an extent, however this increases the need for case-by-case manual parameter adjustment, which we wish to avoid.
Ultimately, accurate registration throughout the brain cannot be achieved with such a simple model, as indeed was never the aim.
Future work could investigate the feasibility of combining the radial deformation with adaptable tissue-dependent elasticity, modelled on previous similar works within the registration literature\autocite{Rohde2003,Duay2004}.
\clearpage{}
\clearpage{}\section{Intra-patient registration and brain shift}

Given the target application of intraoperative \gls{wm} imaging, it would be entirely remiss to not discuss the additional registration issues brought about by brain shift.
Predictably, the sheer variability in direction, magnitude and extent of brain shift means that a universally elegant and robust solution cannot be found.
Instead, we will discuss some common intraoperative scenarios and how tractfinder would be applied to them.

One major source of brain shift is tumour debulking.
As tumour is removed through the craniotomy, the surrounding tissue may collapse and the associated mass effect be correspondingly reduced.
For this scenario, tumour deformation can be utilised using a preoperative segmentation of the tumour and adjustment of the tumour scale parameter $s$, which if set to a value below 1 will virtually decrease the tumour radius, mimicking the debulking effect.
A case study demonstrating this is presented in Section \ref{sec:case}.
The necessary steps (that couldn't be precomputed preoperatively) in such a scenario would be affine registration between pre- and intraoperative subject space, re-computation of a tumour deformation field using preoperative tumour segmentation and $s<1$, atlas deformation, and finally tract mapping with an intraoperative \gls{fod} image.
The additional steps of pre- to intraoperative registration and tumour deformation would add less than five minutes to the total processing time.

Of course, tumour debulking does not always result in intraoperative brain shift that looks like a simple reduction in mass effect.
Often, a certain degree of tissue inertia means that surrounding structures don't instantly relax back into a more ``normal'' position, and the effects of gravity, \gls{csf} drainage, herniation, or intracranial pressure changes will then have a far more drastic influence on overall brain shift than the removal of tumour tissue.
In these cases, tumour deformation modelling cannot be leveraged to account for brain shift, and different registration strategies will be required.
Affine registration with anisotropic scaling may be sufficient for brain shift principally characterised by sagging or compression due to gravity.

Finally, there are those cases in which topological differences between pre- and intraoperative scans are so great that neither affine registration nor adjusted tumour deformation can sufficiently align the data.
In these subjects, advanced deformable registration is necessary.
The task of intraoperative registration to account for brain shift is a well studied one, although in most cases the approach is to deform preoperative data (such as streamlines) to continually provide accurate neuronavigation after brain shift\autocite{Clatz2005,Archip2007,Wittek2007,Archip2008}.
This requires highly accurate registration, as there is no intraoperative \gls{dmri} acquisition to inform tract identification.
For tractfinder, it would be preferable to impose stricter regularisation to increase registration stability and minimise the creation of unphysical distortions (e.g. surrounding the site of resection), since any consequent minor misalignments can be handled by the atlas smoothness and comparison with local diffusion data.
In individual case studies, non-linear registration has successfully been leveraged in this way (Fig. \ref{fig:nrrex}).
However, this is still only possible in an experimental setting and involves significant trial-and-error in determining the appropriate registration input parameters, and it remains the subject of future work to achieve robust generalisation and automation.

\begin{figure}[hb!]
  \centering
  \includesvg[width=\textwidth,pretex=\small\sffamily]{chapter_4/nrr_gosh.svg}
  \caption[Nonlinear registration for intraoperative MRI example]{Example of non-linear registration applied to a \glsentrylong{gosh} \glsentrylong{imri} case.
  \textbf{\sffamily a.} Intraoperative scan with colour \glsentrylong{fa} map overlaid to enhance \glsentrylong{wm} contrast, registered to preoperative scan using rigid registration. Outline of \glsentrylong{wm} segmentation of preoperative scan is overlaid in white, demonstrating substantial brain shift away from the craniotomy (*). Arrow highlights shifting of the ipsilateral external capsule. Dotted line indicates position of coronal plane in right image.
  \textbf{\sffamily b.} Tractfinder map of the \glsentrylongpl{cst} after non-linear registration between pre- and intraoperative scans, using the Fast Free-Form Deformation algorithm\autocite{Modat2010} from the NiftyReg package (\url{http://cmictig.cs.ucl.ac.uk/wiki/index.php/NiftyReg}).}
  \label{fig:nrrex}
\end{figure}

\section{Summary}

All atlas-based image segmentation or analysis techniques have to contend with the problem of aligning said atlas with the target image, and tractfinder is no exception.
In this chapter we have considered the available solutions to this problem, their respective advantages and drawbacks, and applicability to various practical scenarios.

By designing a fuzzy atlas which allows for inter-subject variability to provide an initial estimate for a tract's expected features in the target image, and relying on additional subject-specific information to refine that estimate, we can avoid the need for voxel-perfect atlas alignment, as long as the registered atlas fully covers the domain of the target tract.
In healthy and structurally normal data, affine registration with 12 degrees of freedom is fully sufficient to achieve this coverage.
We saw in Section \ref{sec:reg1} that non-linear registration is unlikely to provide any significant improvement in performance, and only decrease speed and practicality.

Where space occupying lesions are concerned, however, simple affine registration can leave the registered tract atlas entirely misaligned with the corresponding subject anatomy.
In such cases, more advanced atlas deformation is necessary to account for the effects of tumours, while still keeping the compute complexity to an acceptable minimum for clinical implementation.
A simple radial deformation algorithm is proposed to specifically account for mass effect after initial affine registration, which successfully models gross tract displacements.

Finally, we saw how the stringent assumptions of the tumour expansion model do not adequately support the modelling of infiltrating tumours, and how the complexities of brain shift may call for more involved registration tactics, including non-linear algorithms and others developed specifically for intraoperative registration.
In the next chapters we will put the methodological components together and assess the proposed techniques' performance in a series of quantitative evaluations and practical applications.
\clearpage{}

\epart[\epigraph{Unfortunately, nature seems unaware of our intellectual need for convenience and unity, and very often takes delight in complication and diversity.}{Santiago Ramón y Cajal, 1906 Nobel Lecture}]{Results and applications}
\clearpage{}\chapter{Methodological evaluation}\label{chap:eval}

The following sections set out to provide an evaluation of the tractfinder method from a range of perspectives.
A common complaint in the \gls{wm} imaging field is over the lack of reliable ground truth information for fibre tract identification.
This goes deeper than the limitations of our scientific instruments:
Micrometer-resolution \gls{mri} or infallible tractography algorithms are not the answer.
As was explored in Chapter \ref{chap:atlas}, our understanding of \gls{wm} anatomy, connection, and function is neither fixed nor universal, meaning for many tracts a single definition that can be agreed upon by all does not even exist in a given moment, nor one to last over time.
We can nonetheless arrive at a reasonable assessment of tractfinder's accuracy, reliability and applicability by making justifiable assumptions and comparing it, quantitatively and visually, with benchmark methods in a variety of datasets and case studies.

The missing ground truth problem comes to the fore when comparing different methods, as we will see in Section \ref{sec:validation}, but the results are still instructive in assessing their respective characteristics.
To balance this issue, a further analysis directly comparing tractfinder against the deep learning method TractSeg by constructing new \gls{tod} atlases from the TractSeg model training data is presented in Section \ref{sec:tractseg}.
These results originally appeared in \textcite{Young2024}.
We'll further attempt to quantitatively evaluate the effect of tumour deformation modelling on improving segmentation performance in affected patients.

Subsequently, in Chapter \ref{chap:applications}, we'll move on from similarity scores and box plots to explore the practical applicability of tractfinder, considering the technical aspects of bringing it to the bedside, and illustrating the strengths and remaining roadblocks through a series of case studies.

\section{Datasets and image processing}
\label{sec:data}

A range of datasets are considered in which to compare the tractfinder approach against alternative tract segmentation methods, covering adult and paediatric, healthy and clinical populations (Tab. \ref{tab:datasets}).
Each dataset and any dataset-specific preprocessing is described in the following subsections.
In addition, the following default preprocessing steps were applied to all scans, unless otherwise specified:\footnote[2]{software versions: MRtrix3 v3.0.2--3.0.3 (\url{https://www.mrtrix.org/}), \gls{fsl} v.6.0 (\url{https://fsl.fmrib.ox.ac.uk/})}

\textit{Brain masking}\autocite{Tournier2019}
using a heuristic algorithm based on thresholding the mean of each diffusion-weighted shell described in \textcite{Dhollander2016}, implemented in MRtrix3\autocite{Tournier2019} as \verb|dwi2mask|.

\textit{Affine registration} between subject space and MNI152\autocite{Fonov2011} template space using the \gls{fsl} Linear Image Registration Tool (FLIRT)\autocite{Jenkinson2001,Jenkinson2002}.

\textit{\Gls{csd}}. Depending on the dataset and/or application, two different versions of \gls{csd} applied: \gls{ssst} \gls{csd}\autocite{Tournier2007,Tournier2019} (``original flavour'') and \gls{msmt} \gls{csd}\autocite{Jeurissen2014} restricted to \gls{wm} and \gls{gm} tissue compartments for single-shelled acquisitions (those with a single nonzero $b$-value),
and \gls{msmt} \gls{csd} with three tissue compartments (\gls{wm}, \gls{gm}, \gls{csf}) for multi-shelled acquisitions (see Section \ref{sec:ismrmdiff} for further context on these choices).
In all cases, response functions were obtained using the Dhollander unsupervised 3-tissue response function estimation algorithm\autocite{Dhollander2016,Dhollander2019}.
All \gls{csd} was performed using the MRtrix3 image processing software package\autocite{Tournier2019}.

\subsection{HCP}

The first large healthy adult dataset (``HCP49'') comprised 49 scans from the WU-Minn \gls[noindex=false]{hcp} Young Adult S1200 data release (\url{https://www.humanconnectome.org/study/hcp-young-adult/document/1200-subjects-data-release}) \autocite{VanEssen2013}.
The original data comprises high resolution $T_1$-weighted and diffusion-weighted \gls{mri} acquired on a modified Siemens 3T Skyra scanner.
Raw diffusion data contains three diffusion-weighted shells at $b=$ 1000, 2000 and 3000~s~mm$^{-2}$ with 90 directions each and 18 $b=0$ volumes\autocite{Sotiropoulos2013}.
These images have been preprocessed as documented in \textcite{Glasser2013} and \textcite{Sotiropoulos2013},
and additionally for this analysis they were downsampled to 2.5~mm isotropic voxels (from the original resolution of 1.25~mm isotropic) and a subset of 60 optimally distributed directions at $b=1000$~s~mm$^{-2}$ were extracted, emulating a typical clinical acquisition in spatial and angular resolution.

A different set of 105 \gls{hcp} subjects (``HCP105'') were used in \textcite{Wasserthal2018} to train and test the deep learning segmentation method TractSeg.
The curated TractSeg reference streamline bundles for these subjects are publicly available\autocite{Wasserthal2018b} and were used in a separate analysis to directly compare tractfinder and TractSeg.
For these subjects, the same preprocessing steps were applied as in the set of 49 subjects, with the exception of selecting a subset of 30 instead of 60 $b=1000$~s~mm$^{-2}$ directions, to match the \textit{clinical quality} data described in \textcite{Wasserthal2018}.

\subsection{TractoInferno}

The recently released \textit{TractoInferno} database (v1.1.1, available at \url{https://openneuro.org/datasets/ds003900/versions/1.1.1})\autocite{Poulin2022a,Poulin2022}, created as an open dataset to support the training and comparison of machine learning tractography algorithms, contains diffusion and $T_1$-weighted \gls{mri} scans for 284 subjects pooled from several studies, accompanied by reference streamline tractography bundle reconstructions produced using four different tracking algorithms and the RecoBundlesX automatic streamline clustering method \autocite{Garyfallidis2018,Rheault2020a}, followed by semi-automatic quality control.
Of the 284 subjects included in the full \textit{TractoInferno} database, the 80 subjects with tractography of all of the \gls{cst}, \gls{or}, \gls{ifof} and \gls{af} were selected.
Nine subjects were excluded from the final analysis due to \DIFdelbegin \DIFdel{inadequate }\DIFdelend \DIFaddbegin \DIFadd{poor }\DIFaddend non-linear registration performance resulting in failed targeted \gls{roi} tractography (see Section \ref{sec:methods}), leaving a final 71 subjects.
\DIFaddbegin \DIFadd{Excluding these subjects was preferable to manually tweaking the }\gls{roi} \DIFadd{registration to ensure all data was processed equally and reproducibly.
Note that the non-linear registration in question was independent from the linear registration required to align the tractfinder atlases to each subject.
}\DIFaddend Diffusion acquisition parameters and preprocessing steps are described in detail in \textcite{Poulin2022} and summarised in Table \ref{tab:datasets}, and all data was additionally resampled to 2.3~mm isotropic voxels, the lowest resolution present in the dataset and one in line with the other datasets.

\subsection{Clinical (GOSH \& NHNN)}

To validate tractfinder in real clinical patient imaging, a dataset of 15 individual scans from eight different subjects and two different institutions was collated.
They include four adult glioma subjects acquired in 2009 at the \gls[noindex=false]{nhnn}, London (cases 4 and 5 from \textcite{Mancini2022}, others unpublished data),
three paediatric subjects from \gls[noindex=false]{gosh} (each with one preoperative and one intraoperative scan),
a mock “intraoperative” scan on a healthy adult volunteer acquired with the \gls{gosh} intraoperative \gls{dti} protocol and using a simulated intraoperative setup (flex-coils wrapped around the head instead of a head coil, head significantly displaced from scanner isocenter etc.),
and a partridge in a pear tree.
The three \gls{gosh} paediatric patients all had low-grade astrocytoma (pilocytic astrocytoma, diffuse astrocytoma, and pleomorphic xanthoastrocytoma).
Two of the \gls{nhnn} subjects had oligodendroglioma, histology types for the other two tumours was not available.
For full acquisition details see Table \ref{tab:datasets}.
All clinical scans used in the systematic comparison with benchmark methods involved non-deforming tumours, in the sense that any lesions did not appreciably displace \gls{wm} structures from their typical positions.
Further \gls{gosh} and \gls{nhnn} scans with substantial tumours were analysed on a case-by case basis, as the low sample size and high tumour heterogeneity precluded a systematic quantitative evaluation.

Each \gls{dmri} scan was minimally preprocessed with \gls[noindex=false]{mppca} denoising\autocite{Veraart2016, Cordero-Grande2019} Gibbs-ringing correction\autocite{Kellner2016} and bias field correction\autocite{Zhang2001, Smith2004}, as implemented in MRtrix3 \autocite{Tournier2019}.
Preoperative scans additionally had eddy current and motion distortion correction\autocite{Andersson2016a, Smith2004} applied, while this step was omitted for intraoperative scans to maintain a clinically realistic timeline.
No \gls{epi} distortion correction was performed, as it is frequently omitted from clinical pipelines due to lack of requisite reverse phase encoding or field map information and long processing times\autocite{Yang2022}.

\subsection{BTCD}
\label{sec:data_btcd}

The \gls[noindex=false]{btc} dataset\autocite{Aerts2020a,Aerts2022a,Aerts2022} comprises a series of pre- and postoperative structural, functional, and diffusion-weighted \gls{mri} scans of glioma and meningioma patients.
Of the total 25 patients in the original dataset, ten with macroscopic, non-infiltrating lesions were selected for further validation of tractfinder in tumour patients.
In one of the ten subjects (sub-PAT22), only a preoperative session is available, giving a total of 19 unique scan sessions used for this dataset.
The data consists of high quality \gls{hardi} acquisitions, including reverse phase encoding for susceptibility distortion correction, and structural $T_1$-weighted images.
In addition, tumour segmentation masks for each preoperative scan are available in the original dataset, which could be used for tumour deformation modelling.
The ten selected subjects included four with tumours large enough (three meningioma and one anaplastic astrocytoma) to warrant the use of deformation modelling to improve atlas alignment.
Of these four subjects, one had a large midline frontal meningioma affecting both hemispheres, and for this subject all tracts were considered ``ipsilateral'' tracts.
In all other subjects, the tracts were labelled ipsilateral or contralateral if they were in the same or opposite hemisphere as the tumour respectively.

Diffusion \gls{mri} data was preprocessed with \gls{mppca} denoising, bias field correction and \gls{epi} distortion correction (eddy for CUDA 9.1, part of \gls{fsl} v.6.0.4), followed by \gls{msmt}-\gls{csd} as previously described.
Given the disruption caused by the space-occupying and in cases infiltrating lesions, manual tractography using multi-\gls{roi} targeting alone did not produce acceptable results.
Instead, the streamlines were manually filtered to remove spurious and implausible streamlines using the same process as was used when curating the atlas training data.
These filtered bundles are used as reference segmentations for the 10 \gls{btc} subjects.

\begin{landscape}
\begin{table}[t]
  \caption[Benchmark evaluation datasets]{Overview of acquisition parameters for the datasets included in quantitative benchmark evaluation. \dag Resampled from original, see text for details.}
  \label{tab:datasets}
  \footnotesize
  \begin{tabularx}{\linewidth}{l c c c c c c c c} \toprule
             & \multicolumn{2}{c}{GOSH} & \multicolumn{2}{c}{NHNN} & \multicolumn{2}{c}{BTC\autocite{Aerts2018, Aerts2020a}} & \gls{hcp}\autocite{Sotiropoulos2013, Glasser2013} & \textit{TractoInferno}\autocite{Poulin2022} \\
             & pre-op.   & intra-op.      & pre-op. & intra-op.        & pre-op. & post-op.       & & \\
  \midrule $n$ subjects & 3          & 4                & 4      & 4                               & 10    & 9                         & 49         & 71     \\[1em]
  Age range (y)  & \multicolumn{2}{c}{paediatric (3--12, n=3)} & \multicolumn{2}{c}{adult (30--39)} & \multicolumn{2}{c}{adult (39--74)} & adult (22--35) & adult (18--75) \\
             &            & adult (n=1)      &        &                                 &       &                           &            &         \\[1em]
  Diagnosis  & \multicolumn{2}{c}{low-grade astrocytoma (n=3)}  & \multicolumn{2}{c}{oligodendroglioma (n=2)}& \multicolumn{2}{c}{meningioma (n=4)} & healthy & healthy \\
              &           & healthy (n=1)  & \multicolumn{2}{c}{other tumour (n=2)}     & \multicolumn{2}{c}{glioma (n=5)}  &            & \\[1em]
  $b$-values (s~mm$^2$) & 800 (n=1)   & 1000           & 1000     & 1000                       & \multicolumn{2}{c}{0, 700, 1200, 2800} & 1000       & 1000 (n=68) \\
   & 1000, 2200 (n=2) &          &          &                            &            &                           &            & 700 (n=3) \\[1em]
  $n$ directions & 15 (n=1)     & 30             & 64 (n=3) & 30 (n=3)                   & \multicolumn{2}{c}{8, 16, 30, 50}      & 60\dag     & 21--128 \\
           & 60, 60 (n=2) &                & 61 (n=1) & 3\x{}12 (n=1)               &            &                           &            & \\[2em]
  Voxel size (mm) & 1.75\x{}1.75\x{}2.5 (n=1) & 2.5 (n=1) & 2.5 & 2.5\x{}2.5\x{}2.7 & \multicolumn{2}{c}{2.5} & 2.5\dag    & 2.3\dag \\
                  & 2\x{}2\x{}2.2 (n=2)       & 2.3 (n=3) & & & & & & \\[1em]
  Scanner & Philips Ingenia 1.5T (n=1)  & Siemens  & Siemens & Siemens & \multicolumn{2}{c}{Siemens} & Siemens 3T & variable\\
          &  Siemens Prisma 3T (n=2)    &  Vida 3T  & Trio 3T  & Espree 1.5T                & \multicolumn{2}{c}{Trio 3T}           & ``Connectome Skyra” & \\
          &                          &                &          &                            &                   &                   &   & \\ \bottomrule
  \end{tabularx}
\end{table}
\end{landscape}
\clearpage{}
\clearpage{}
\section{Validation: Methods}

Segmentation tasks are commonly validated against reference data using a single numeric score of similarity or accuracy.
In the absence of a ground truth for \textit{in vivo} \gls{wm} tracts, what should instead by the measure by which a tract imaging method is considered ``accurate''?
Perhaps the only reasonable performance target can be to reproduce as closely as possible the results which would be obtained by the current standard, which in most cases would be streamline tractography.
However, given the very large degree of variability between tractography algorithms and tract reconstruction approaches, even this is a poorly defined reference point.
False positive streamlines can unfairly skew accuracy scores, while manually editing reference bundles to remove them is likely to introduce bias.

Given these challenges, we'll aim to draw as rounded a picture as possible of the differences and characteristic features of each method through different volumetric and distance-based similarity metrics.
The purpose of this validation is not to determine which method is best, as indeed cannot be determined without a reliable reference point, but to highlight the ways in which they are similar, and their strengths and weaknesses.

\subsection{Benchmark methods}
\label{sec:methods}

We'll consider three alternative tract segmentation approaches and compare each with tractfinder: Probabilistic streamline tractography, representing the current standard, the deep learning direct segmentation technique TractSeg, and a ``naive'' atlas registration.
There are numerous alternative techniques, as detailed in Chapter \ref{chap:neuroimaging}, many based on streamline clustering or classification, or automatic \gls{roi} registration, including White Matter Analysis (WMA)\autocite{ODonnell2017}, RecoBundles \autocite{Garyfallidis2018}, Classifyber\autocite{Berto2021}, and Tracula\autocite{Yendiki2011}.
We will not be comparing tractfinder with all of these alternatives because such a comparison would hardly provide any additional useful information:
In those methods which rely on a predefined atlas of streamlines or \glspl{roi}, the anatomical assumptions embedded in those predefinitions would influence the similarity metrics more than any methodological differences, as indeed they do for the comparisons with TractSeg that follow.
Furthermore, many of those alternatives have already been systematically compared with TractSeg\autocite{Wasserthal2018,Berto2021}, and so in comparing tractfinder with the latter we can infer relative performance against them.

\paragraph*{Streamline tractography}

Targeted probabilistic streamline tractography (iFOD2 algorithm\autocite{Tournier2010}, from MRtrix3\autocite{Tournier2019} v3.0.3)  was run in each scan using a multi-\gls{roi} approach (see Appendix \ref{app:rois} for \gls{roi} details),
with tractography input \glspl{fod} derived from \gls{msmt} \gls{csd}\autocite{Jeurissen2014} (using \gls{wm} and \gls{gm} tissue compartments in single-shelled data).
In the clinical dataset, \glspl{roi} were placed manually for each subject.
For the 184 \gls{hcp} and \textit{TractoInferno} subjects, manual \gls{roi} placement was infeasible.
Instead the same \glspl{roi} were drawn in MNI152 template space aided by the \gls{fsl} \gls{hcp}-1065 \gls{dti} template\autocite{FSLATLAS} and transformed to subject space using non-linear registration
(\gls{hcp} data includes MNI transformation warps, while warps were computed for the \textit{TractoInferno} data using the ANTs registration package v2.4.2 (http://stnava.github.io/ANTs/) \autocite{Tustison2013,Avants2011}.

The anatomical definitions informing the manual tractography paradigm were the same as for creating the respective atlases, however, to avoid dependence on resource-intensive tools for cortical parcellation, only \gls{wm} \glspl{roi} could be used.
The result is generally more extensive cortical terminations into gyri and sulci, for example in the \gls{af} and \gls{ifof}, not otherwise included in the atlases due to either the use of cortical \glspl{roi} or extensive manual filtering, while the core subcortical portions match the definitions reflected in the atlases.
The same \gls{wm} seed and inclusion \glspl{roi} used for the atlases were adopted, along with additional exclusion \gls{roi} where necessary.
For a full description of all tractography \glspl{roi} and any differences between atlas streamlines and validation manual tractography, see Appendix \ref{app:rois}.
This manual tractography (using the same anatomical definitions as the atlases) is subsequently abbreviated to ``\gls{tg}'', while the reference \textit{TractoInferno} bundles (produced using RecoBundlesX, see Section \ref{sec:data}) are referred to as ``\gls{tgr}''.
Since tractfinder is a voxel-based method, streamlines were converted to voxel-based representation via \gls{tdi}\autocite{Calamante2010}, producing maps of streamline density per voxel.

\paragraph*{TractSeg}

TractSeg \autocite{Wasserthal2018} is a deep convolutional neural network model for tract segmentation which produces volumetric segmentation for 72 tracts directly from fibre orientation distribution peak directions (TractSeg v2.3--2.6, available at \url{https://github.com/MIC-DKFZ/TractSeg}).
There are two models available: One trained on modified streamline reconstructions using TractQuerier \autocite{Wassermann2016} as described in \textcite{Wasserthal2018} (``\gls{tsd}''), and a second trained on streamline density maps output by \gls{fsl}'s XTRACT \autocite{Warrington2020} application using automatic predefined \gls{roi} registration and probtracx probabilistic tractography\autocite{Behrens2007} (``\gls{tsx}'').
Here both versions are compared, as they feature significant differences in anatomical tract definitions.
Input peaks for all TractSeg results were derived from the \gls{ssst} \gls{csd} \glspl{fod} for all single-shelled data and from the \gls{msmt} peaks for all multi-shelled data in accordance with how the TractSeg models were trained.

\paragraph*{Atlas registration}

As well as the full tractfinder method (atlas alignment and inner product), we'll also consider a ``naive'' tract atlas approach, using only the density component (first \gls{sh} coefficient) of the linearly registered tract atlases without any comparison with the native \glspl{fod}.
This amounts to a segmentation based on prior spatial expectation only, without taking into account the diffusion data or orientation expectations, and is comparable with segmentation approaches that rely entirely on registration, such as common cortical parcellation tools.
Comparing results with simple atlas registration between structural image data is instructive for determining the added value of utilising \gls{dmri} data to inform segmentation, which may be relevant to future investigations into the benefits of acquiring intraoperative \gls{dmri} in addition to conventional tissue contrast \gls{mri} sequences.

\subsection{Comparison metrics}\label{sec:metrics}


In all datasets, each technique was compared against all others, rather than designating any single technique as ``ground truth'', with the exception of the \textit{TractoInferno} dataset, where the published streamlines were treated as an independent reference.
Several comparison metrics are computed, to capture different aspects of agreement between segmentations.
The \gls[noindex=false]{dice} \autocite{Dice1945} is a popular, symmetric measure of segmentation similarity given by
\begin{align}
  DSC &= \frac{2 |A \cap B|}{|A| + |B|}
\end{align}
for two binary voxel sets $A$ and $B$.
Since the \gls{dice} is a measure for binary segmentations, it requires thresholding of continuous-valued images such as track density maps and the pseudo-probability maps produced by tractfinder, which poses some problems.
Firstly, the conversion from continuous-valued to binary representation introduces a high degree of ambiguity over the appropriate choice of threshold value.
While the simplest and least ambiguous approach may be to include all non-zero valued voxels in the segmentation, this makes little sense in practice.
In the case of tractography, a small number of stray false positive streamlines can substantial increase the thresholded volume, and in the case of TractSeg, very few voxels actually have an inference probability of 0.
The thresholds used wherever binary segmentations are concerned are given in Table \ref{tab:thresh}.
Secondly, the binary nature of a \gls{dice} discounts the additional confidence information present in density and probability maps.
Assigning equal weight to the segmentation voxel with the highest overall value and one only marginally above threshold doesn't provide a fair assessment of a method's performance and leads to tract maps which are visually harder to interpret.

\begin{table}[h!]
  \caption{Intensity thresholds for binary comparison measures.}
  \label{tab:thresh}
  \small
  \begin{tabularx}{\textwidth}{>{\raggedright\arraybackslash}X l >{\raggedright\arraybackslash}X}
    \toprule
    Method    & Threshold value & Reasoning \\
    \midrule
    Tractfinder   & 0.05 & empirically determined \\
    Tractography (streamline density) & 10 & enough to exclude obvious false positives \\
    Reference tractography (\textit{TractoInferno} only) & 0 & Assume no false positives (curated dataset) \\
    Atlas         & 0.1 & double the tractfinder value, as the inner product roughly halves the atlas values \\
    TractSeg      & 0.5 & consistent with default TractSeg settings \\ \bottomrule
  \end{tabularx}
\end{table}

In addition to the binary \gls{dice} measure, we therefore utilise the density correlation as a measure of volumetric agreement between two continuous valued segmentations, which is simply the Pearson correlation coefficient between the two sets of voxel values, and has been used in other studies to compare streamline density volumes\autocite{Radwan2022, Schilling2021a}.

\begin{SCfigure}[50][htbp!]
  \centering
  \includegraphics[width=0.3\textwidth]{chapter_5/segmentations_distance_euc.pdf}
  \caption[Bundle distance calculation]{Illustration of regions involved in calculating the bundle distance metric. The light grey area is $A\setminus B$, dark grey is $B\setminus A$, and in each grey voxel is written its minimum Euclidean distance to the black intersection. Distance sign is negative outside of the reference segmentation ($B$ in this example) and positive inside. To compute the bundle distance $BD(A,B)$ (Eq. \ref{eq:bd}), the mean minimum absolute distance is taken across all 17 voxels in the two grey areas $BD(A,B) = (14+4\sqrt{2}+3\sqrt{5})/17 = 1.55$. To compute the signed bundle distance $BD_s(A,B)$ (Eq. \ref{eq:bds}), the signed distances relative to the reference set $B$ are used: $BD_s(A,B) = (2-2\sqrt{2}-\sqrt{5})/17 = -0.18$.
  The \glsentrylong{dice} for these two segmentations would be $DSC = 2*4/(13+12) = 0.32$}
  \label{fig:BD}
\end{SCfigure}

As well as the volumetric overlap and density metrics, the volumetric bundle adjacency as defined in \textcite{Schilling2021a} is also measured.
However, to avoid confusion with the streamline-based bundle adjacency\autocite{Radwan2022, Garyfallidis2012, Rheault2022} metric previously defined in \textcite{Garyfallidis2012},
and to give more intuitive meaning to the obtained values, we will refer to it as bundle distance ($BD$).
It is computed by taking the mean of minimum distances from every non-overlapping voxel, in each segmentation, to the closest voxel in the other segmentation (Fig. \ref{fig:BD}):
\begin{align}
  BD(A,B) &= \frac{\sum_{i \in A\setminus B} d_i(B) + \sum_{i \in B\setminus A} d_i(A)}{|A\Delta B|} \label{eq:bd},
\end{align}
where $| \cdot |$ denotes set cardinality, $\setminus$ denotes the set difference, and $d_i(X)$ is the Euclidean distance transform (defined relative to the foreground of segmentation $X$, i.e. $d_i(X) = 0$ when $i \in X $ and $d_i(X) = |\overrightarrow{ij}|$ when $i \not\in X$ and where $j \in X$ is the voxel in $X$ closest to voxel $i$)  of segmentation $X$ at voxel $i$.
Finally, to give a sense of whether the boundary of one segmentation is primarily within or outside that of a second segmentation, we'll also consider the \textit{signed} bundle distance ($BD_s$).
This metric is asymmetric, with $BD_s (A,B) = -BD_s(B,A)$, and is defined as

\begin{align}
  BD_s(A,B) &= \frac{\sum_{i \in A\setminus B} - d_i(B) + \sum_{i \in B\setminus A} d_i(A)}{|A\Delta B|}. \label{eq:bds}
\end{align}


\section{Pairwise benchmark evaluation}
\label{sec:validation}

In the following section, we will review the results of quantitatively comparing each of the methods described in Section \ref{sec:methods}, in the HCP49, clinical, and TractoInferno datasets, including material which has been accepted for publication in \textit{Human Brain Mapping}\autocite{Young2024}.
Given the large disparities in tract anatomical definitions between methods, the results presented in this section can be regarded as a quantitative performance evaluation against benchmark methods.
In a subsequent section (\ref{sec:tractseg}) we will see how the two methods tractfinder and TractSeg compare when both are trained on the same data.

Segmentation results results for a representative subject from the \textit{TractoInferno} dataset are shown in Figures \ref{fig:lb.cstor}--\ref{fig:lb.afifof}.
Tractfinder maps typically have values ranging from 0 to 0.5 (in arbitrary units, derived from the magnitudes of fibre and atlas \glspl{odf}).
Due to the combined effects of \gls{odf} amplitude and orientation information, a low tract map value can have several causes:
a) The \gls{fod} amplitude is low, indicating low evidence for \gls{wm} tissue in the voxel in question;
b) The atlas amplitude is low, indicating low prior likelihood of the tract being present in that location;
c) The peak orientations between the \gls{fod} and atlas are poorly aligned.

Thus combining information from the atlas and data-derived \glspl{fod} improves the tract map estimation over the ``raw'' registered atlas in both the spatial and orientational domain. For example, the \gls{tod} atlases have poor definition of gyri and sulci, due to the effect of averaging over many subjects and linear registration. The reduced overall \gls{fod} amplitude in \gls{gm} corrects this non-specificity. And in regions where different \gls{wm} structures lie in close proximity, where the atlas can erroneously predict the likely presence of the tract, and \gls{fod} amplitude is high, the lack of orientational agreement discounts the presence of the tract of interest in that location.

\subsection{Quantitative results}\label{sec:quant}

\begin{SCfigure}
  \includegraphics{chapter_5/score_mats_group.pdf}
  \caption[Pairwise DSC and density correlations for the \textit{TractoInferno}, HCP, and clinical datasets]{Pairwise \glsentrylong{dice} and density correlations for the \textit{TractoInferno} (top) HCP49 (middle) and clinical (bottom) datasets.
  \DIFaddbegin \DIFadd{Upper triangles show density correlation means (background colour) and distributions (white boxplots), lower triangles show the binary }\glspl{dice} \DIFadd{for each pairwise method comparison.
  }\DIFaddend For each \DIFdelbegin \DIFdel{pair of methods}\DIFdelend \DIFaddbegin \DIFadd{combination}\DIFaddend , the metric distributions across all subjects are shown for \DIFdelbegin \DIFdel{each tract }\DIFdelend \DIFaddbegin \DIFadd{the four tracts separately }\DIFaddend (from left to right: \glsentryshort{af}, \glsentryshort{cst}, \glsentryshort{ifof}, \glsentryshort{or})\DIFdelbegin \DIFdel{, with background colours corresponding to mean score value}\DIFdelend . \acrolist{af, cst, or, ifof, tf, at, tsx, tsd, tgr}
  \label{fig:dscmats}}
\end{SCfigure}

In a realistic clinical context, our target segmentation is represented not by an independently determined and verifiable ground truth, but by the results of whatever approach would normally be taken to produce tract reconstructions for surgical guidance in the absence of any suitable alternative.
That benchmark is targeted multi-\gls{roi} streamline tractography, against which we need to evaluate tractfinder.
At the same time, quantitative evaluation using tractography as a reference is problematic as with typical use tractography will produce many false positives that are easily mentally discounted by an experienced viewer, but which will confound quantitative accuracy metrics.

This is one reason why the density correlation metric is particularly useful for comparing methods in this task:
False positive streamlines are more likely to be apparent in areas of low streamline density, and if the compared segmentation correspondingly predicts a low probability in the same areas, then this will be consistent with a high correlation value.
The density correlation thus helps illustrate the cases where the choice of threshold may have a disproportionate influence on subsequent binary comparisons.
For example, in the \gls{hcp} dataset and for the \gls{cst}, mean \gls{dice} was $0.69$ between tractfinder and tractography and $0.51$ between TractSeg XTRACT and tractography (a difference of $0.18$).
For the same two comparisons, the density correlations differed only by $0.04$ ($0.63$ and $0.59$) respectively, indicating strong agreement about areas with high and low tract probabilities, with slight differences in cutoff value likely contributing to the larger disparity in \gls{dice} scores (Fig. \ref{fig:dscmats}).

The signed bundle distance gives an indication of the nature of disagreement between two techniques where other metrics show little difference.
For example, in the \textit{TractoInferno} dataset and for the \gls{af}, mean bundle distance to reference tractography was 5.06~mm for the naive registered atlas and a very similar 5.07~mm TractSeg DKFZ at (Fig. \ref{fig:combobox}). However, the signed bundle distances for those same two comparisons were +1.65~mm and -2.60~mm respectively (similar values were also found for the \gls{hcp} dataset, Fig. \ref{fig:hcpbox}).
This indicates that, while if only considering the bundle distance metric, both TractSeg and the atlas appear to agree to a similar degree with tractography, TractSeg actually systematically over-segments the \gls{af} (relative to tractography), while the naive atlas segmentation tends towards under-segmentation.

\begin{figure}[h!]
  \centering
  \makebox[\linewidth][r]{\includegraphics{chapter_5/TI_boxplots.pdf}}
  \caption[Comparison results with \textit{TractoInferno} reference streamlines]{All methods compared against the \textit{TractoInferno} reference streamlines. \acrolist{af, cst, or, ifof, tf, at, tsx, tsd, tg}}
  \label{fig:combobox}
\end{figure}

\DIFaddbegin \DIFadd{It should be noted that this analysis does not attempt to define minimum metric scores acceptable for clinical use, as there are no standards for }\gls{wm} \DIFadd{tract segmentation accuracy given the difficulties in defining a ground truth.
The }\gls{dice}\DIFadd{, for example, does not allow for prioritisation of either sensitivity or specificity, as these are judgements that could currently only be made by an interpreting radiologist on a case-by-case basis.
Furthermore, a single score for an entire metric does not capture information about spatially varying accuracy.
If the reconstruction where accurate new the tumour, but not in the brain stem due to the limitations of linear registration, then the resulting overall accuracy score would not fully reflect the clinical usefulness of the reconstruction.
}

\DIFaddend With all this in mind, we find that tractfinder does indeed reliably perform well against targeted streamline tractography across all metrics, with the exception of the \gls{af} and a lesser extent the \gls{ifof} when measured on \gls{dice} or distance metrics.
There are a couple possible factors which may account for the worse performance for the \gls{af}.
One is the difficulty in consistently reconstructing association tracts using only \gls{wm} based \glspl{roi} (as opposed to cortical \glspl{roi} which necessitate more extensive image preprocessing), which are unable to adequately constrain the streamlines as they fan out into numerous cortical regions.
Supporting this theory, tractfinder returned the highest standard deviation values for the \gls{af} out of all four tracts across all metrics except the signed bundle distance.
It should be noted that, compared to tractography, TractSeg also returns higher standard deviations for the \gls{af}, pointing to an inter-subject variability in tractography segmentations, rather than an high degree of inconsistency in tractfinder performance.
We can also see that the signed bundle distance is significantly higher on average for the \gls{af} and \gls{ifof} (Fig. \ref{fig:hcpbox}), meaning tractfinder typically under-segments these tracts relative to tractography, which is to be expected with large numbers of streamlines fanning to adjacent cortical termini not included in the curated tract atlas.
Another potentially contributing factor is the \gls{af}'s unique shape, which features a tight bend around the top of the Sylvian fissure.
Alignment of an atlas with this shape is particularly sensitive to individual anatomical variations which may also partly explain the lower average accuracy for this tract.

\begin{SCfigure}
  \includegraphics{chapter_5/hcp_box.pdf}
  \caption[Signed bundle distances to streamline tractography, HCP dataset]{Signed bundle distances for all methods compared against targeted tractography in the \glsentryshort{hcp} dataset. \acrolist{af, cst, or, ifof, tf, at, tsx, tsd}}
  \label{fig:hcpbox}
\end{SCfigure}

Notwithstanding the slightly worse accuracy for the association tracts on balance we can see that tractfinder returns consistently strong agreement with tractography across all metrics.
We also observe a low degree of variability in tractfinder scores against benchmark methods (Fig. \ref{fig:dscmats}), indicating consistent performance.
There is little variation in results across subjects, and the overall patterns also remain consistent between the different datasets, both healthy and clinical, which feature a range of acquisition parameters and ages.

As a final comment on reproducibility, we note that the results in Figure \ref{fig:combobox} agree with values published in \textcite{Wasserthal2018}.
There, a mean \gls{dice} of between 0.58 and 0.67 across all tracts was reported for RecoBundles evaluated against the TractQuerier derived reference bundles.
Measured \glspl{dice} between TractSeg DKFZ and the TractoInferno reference bundles (which were generated using RecoBundlesX) ranged between 0.45 and 0.66 across the four tracts studied, demonstrating consistency in the two methods between different datasets.

\begin{figure}[hp!]
  \begin{subfigure}{\textwidth}
    \makebox[\linewidth][r]{\includesvg[width=\textwidth,pretex=\sffamily\scriptsize]{chapter_5/LB_AF.svg}}
  \end{subfigure}
  \begin{subfigure}{\textwidth}
    \makebox[\linewidth][r]{\includesvg[width=\textwidth,pretex=\sffamily\scriptsize]{chapter_5/LB_IFOF.svg}}
  \end{subfigure}
  \caption[AF and IFOF tract segmentations in all benchmark evaluation methods]{Segmentations for the left \glsentrylong{af} (top) and \glsentrylong{ifof} (bottom) in a representative subject from the \textit{TractoInferno} dataset, shown on slices at 5~mm intervals and as volumes in the last column. Segmentations have been thresholded according to Table \ref{tab:thresh}, and colour scales are drawn once for each method category (tractfinder, tractography and TractSeg).}
  \label{fig:lb.afifof}
\end{figure}
\begin{figure}[htb!]
  \centering
  \begin{subfigure}{\textwidth}
    \makebox[\linewidth][r]{\includesvg[width=\textwidth,pretex=\sffamily\scriptsize]{chapter_5/LB_CST.svg}}
  \end{subfigure}
  \begin{subfigure}{\textwidth}
    \makebox[\linewidth][r]{\includesvg[width=\textwidth,pretex=\sffamily\scriptsize]{chapter_5/LB_OR.svg}}
  \end{subfigure}
\end{figure}
\clearpage\addtocounter{figure}{-1} {\DIFdelbegin %DIFDELCMD < \captionof{figure}[CST and OR tract segmentations in all benchmark evaluation methods]{\textit{[Previous page]} Segmentations for the the projection tracts \glsentrylong{cst} (top) and \glsentrylong{or} (bottom) in a representative subject from the \textit{TractoInferno} dataset shown on slices at 5~mm intervals and as volumes in the last column. Segmentations have been thresholded according to Table \ref{tab:thresh}, and colour scales are drawn once for each method category (tractfinder, tractography and TractSeg). \label{fig:lb.cstor}}
%DIFDELCMD < %%%
\DIFdelend \DIFaddbegin \captionof{figure}[CST and OR tract segmentations in all benchmark evaluation methods]{\textit{[Previous page]} Segmentations for the projection tracts \glsentrylong{cst} (top) and \glsentrylong{or} (bottom) in a representative subject from the \textit{TractoInferno} dataset shown on slices at 5~mm intervals and as volumes in the last column. Segmentations have been thresholded according to Table \ref{tab:thresh}, and colour scales are drawn once for each method category (tractfinder, tractography and TractSeg). \label{fig:lb.cstor}}
\DIFaddend }

\paragraph*{Tract variability}

Visual assessment reveals that systematic differences in the shapes of the segmented tracts account for a large part of the discrepancy between methods.
Again, this is most apparent in the association tracts, where anatomical definitions differ widely (Fig. \ref{fig:lb.cstor}--\ref{fig:lb.afifof}).
For example, TractSeg DKFZ includes extensive coverage of the frontal and temporal lobe in its \gls{af} segmentations, including parts of the primary motor cortex.
\Glspl{dice} between different methods are lower across the board for the \gls{af} owing to these anatomical disagreements.
Similarly, the \gls{ifof} reconstructions also display strong variability, owing in part to disagreements in its precise definition as reviewed in Chapter \ref{chap:atlas}.
Conversely in the \gls{cst}, which has a relatively well agreed-upon structure, segmentations have much higher volumetric agreement between methods, with the exception of TractSeg XTRACT, which does not include the lateral projections (Fig. \ref{fig:lb.cstor}).
Agreement in the \glspl{or} is somewhere in between, with slightly lower \glspl{dice} in tractfinder compared against the two TractSeg methods, which tend to include more thalamus and a lesser extent of Meyer's loop.
These differences highlight the difficulty in assessing the ``accuracy'' of \gls{wm} segmentation methods given limited consensus on the precise anatomical definitions of different pathways.

A further consideration is the effect of segmentation volume on \gls{dice} values.
This metric is more sensitive to segmentation errors in smaller volumes, and conversely segmentations of large structures tend to score higher.
We can see this particularly well in TractSeg DKFZ's results when compared with reference tractography in the TractoInferno data, which was produced using the streamline clustering tool RecoBundlesX\autocite{Garyfallidis2018,Rheault2020a}.
In Figure \ref{fig:combobox}, TractSeg scores notably high against the TractoInferno reference bundles for the \gls{ifof} and \gls{af}, both association bundles whose cortical terminations are poorly defined.
As reviewed in Chapter \ref{chap:atlas}, contemporary controversies surrounding the \gls{af} include proposals that it terminates in the motor or pre-motor cortex instead of Broca's area in the inferior frontal gyrus, and that its temporal terminations extend beyond Wernicke's area into the superior, middle and inferior temporal gyri\autocite{Dick2012,Giampiccolo2022a}.
Both TractSeg DKFZ\autocite{Wasserthal2018c} and the RecoBundlesX atlas\autocite{Rheault2021} have adopted a more extensive definition of the \gls{af}, selecting streamlines visiting any of the inferior frontal, middle frontal, and precentral gyri, and all of the temporal lobe, producing large bundles and favourable \gls{dice} results.
This contrasts with the bundle distance and density correlation metrics, where the differences between TractSeg (DKFZ) and other methods are no more pronounced for the \gls{af} than in the other tracts.


\paragraph*{TractoInferno reference bundles}



Initially intending to make use of the published \textit{TractoInferno} streamlines as an unbiased reference for comparison, upon closely examining the results it became apparent that they were unsuited for this purpose.
Regardless of tract, metric, or compared methods, the \textit{TractoInferno} reference streamlines yielded variable results with large numbers of outliers (Fig. \ref{fig:combobox}).
Further investigation into these outliers revealed numerous subjects with incomplete or highly asymmetric bundles, this despite strict quality control \autocite{Poulin2022a}.
For example, in several cases, \gls{or} streamlines only reach the superior portion of the occipital lobe (Fig. \ref{fig:duds}).
In others, the right \gls{af} is significantly smaller than the left (Fig. \ref{fig:duds}).

\begin{figure}[h!]
  \centering
  \includegraphics[width=\textwidth]{chapter_5/badti.png}
  \caption[Incomplete bundles in the \textit{TractoInferno} reference data]{Incomplete and conspicuously asymmetric \glsentrylong{af} (top row) and \glsentrylong{or} (bottom row) bundles in the \textit{TractoInferno} dataset. Examples shown from subjects 1078, 1117, 1114, 1131, 1222, 1170, and 1088.}
  \label{fig:duds}
\end{figure}

Given these inconsistencies, a critical analysis on the relative performances of the different segmentation approaches against the \textit{TractoInferno} reference streamlines was ruled out.
The \textit{TractoInferno} dataset features 284 subjects in total (71 were used for the present analysis), and was published with the intention of providing a large and high quality dataset of reference streamlines explicitly for the purpose of training deep learning models for improved tractography and other data-intensive applications.
Of the full dataset, only 80 subjects were found to have all four tracts studied here available, and a query of the public database\autocite{Poulin2022a} indicates that of the 30 reconstructed bundles, only 24 bundles were retained on average per subject after quality control.
The inconsistent quality of these bundles highlights the difficulty in producing streamline tractography bundles with a high degree of anatomical fidelity and inter-subject consistency in so many individuals.
The reliance on such large datasets thus presents a significant challenge to deep machine learning methods and demonstrates the advantage of an approach like tractfinder, as was seen in Section \ref{sec:ntrain}.

\paragraph*{Clinical applicability (GOSH \& NHNN)}

The benchmark analysis included clinical scans with non-deforming lesions, meaning the orientation atlas could be registered to the target image using only affine registration without the need for tumour deformation modelling.
For results in clinical scans featuring deforming lesions, see Sections \ref{sec:btcd} and \ref{sec:imri}. Overall, quantitative results in the clinical (\gls{gosh} \& \gls{nhnn}) dataset were highly consistent with those seen in the healthy datasets.
Although this dataset contained a mixture of adult and paediatric patients, scanned at two different hospitals, when the results were split on hospital/age group (paediatric or adult), no appreciable difference in results was observed.
Equally, no systematic difference in segmentation scores was observed between intraoperative and preoperative datasets.

Two example clinical subjects, one adult and one paediatric, are shown in Figures \ref{fig:lb.nh}--\ref{fig:lb.gosh} and demonstrate key observations in the clinical applicability of the compared methods.
In Figure \ref{fig:lb.nh}, a sagittal view displays the interaction between the surgical resection cavity of a frontal tumour and the \gls{cst}.
Tractfinder maps portions of the \gls{cst} in close proximity to the resection site, as does manual tractography, where the TractSeg segmentations are far more conservative and even incomplete, potentially missing \gls{cst} locations influenced by oedema, brain shift, or other tumour effects.
Even though parts of the precentral sulcus remain some distance from the resection site, the slight brain shift may have affected TractSeg's ability to fully recognise the \gls{cst}.
Intraoperative imaging of a paediatric tumour patient in Figure \ref{fig:lb.gosh} reveals the ipsilateral \gls{or} immediately deep to the resection cavity.
Minimal brain shift and diffusion disturbance from this small craniotomy have enabled good reconstruction of the affected tract by all techniques.
The anterior extent of Meyer's loop has additionally been captured by tractography and tractfinder, but is absent from both TractSeg results.
Further examples of tract reconstructions in intraoperative imaging can be viewed in Figure \ref{fig:tumours}, panels c and e.

\begin{figure}[htb!]
  \makebox[\linewidth][r]{\includesvg[width=\textwidth,pretex=\sffamily\scriptsize]{chapter_5/LB_NHNN_5.svg}}
  \caption[CST segmentations in all benchmark evaluation methods for an adult intraoperative case in the clinical dataset]{Intraoperative imaging for an adult right frontal \glsentryshort{who} grade 2 oligodendroglioma patient from the clinical dataset, with right \glsentrylong{cst} segmentations (thresholded according to Table \ref{tab:thresh}) shown in coronal slices at 5~mm intervals and as full volumes in the right-most column. White arrowheads indicate the precentral sulcus.}
  \label{fig:lb.nh}
\end{figure}
\begin{figure}[htb!]
  \makebox[\linewidth][r]{\includesvg[width=\textwidth,pretex=\sffamily\scriptsize]{chapter_5/LB_GOSH_3.svg}}
  \caption[OR segmentations in all benchmark evaluation methods for a paediatric intraoperative case in the clinical dataset]{Intraoperative imaging for a paediatric temporal diffuse low-grade astrocytoma patient, with bilateral \glsentrylong{or} segmentations (thresholded according to Table \ref{tab:thresh}) shown in axial slices at 5~mm intervals and as full volumes in the right-most column.}
  \label{fig:lb.gosh}
\end{figure}



\subsection{Processing times}

Each pipeline step for the \textit{TractoInferno} dataset was timed separately to give a full processing time breakdown, presented in Table \ref{tab:time}.
Atlas transformation and inner product computation time per subject for all four tracts and both hemispheres was 24~$\pm$~5~s, plus 1--2~min for \gls{msmt}-\gls{csd} and 20~s for MNI registration, with an average total time of just under 3~min for the entire tractfinder pipeline (Tab. \ref{tab:time}).
For TractSeg, mean processing time (for all tracts, 72 for DKFZ and 23 for XTRACT, both hemispheres) was 4~$\pm$~1~min, plus 15--20~s for \gls{ssst}-\gls{csd}.

For manual streamline tractography in the clinical datasets, processing time was not explicitly measured, due to the high variability that comes with manual \gls{roi} drawing (between 10--25 minutes for all tracts in a single subject, although anecdotally this varies significantly between operators).
\Gls{hcp} and \textit{TractoInferno} tractography was run on a high performance computing cluster, taking approximately ten seconds per tract (single hemisphere), using 36 CPU cores, and additionally up to 2~minutes for non-linear \gls{roi} registration (Table \ref{tab:time}).
However, since the time taken depends greatly on several factors, including number of streamlines to select and streamline acceptance rate (often low in brains with pathology due to oedema, deformation etc.), a precise time analysis for manual tractography is not provided here.

\begin{landscape}
\begin{table}[ht!]
  \caption[Measured preprocessing times for different tract segmentation pipelines]{Measured processing times (mean and standard deviation) for each step and average total for each pipeline, measured in the \textit{TractoInferno} dataset. Note that the tractography pipeline was partially run on a high performance computing cluster, so the reported total time is not representative of a typical setup. Further note that for the present study, tractography \glsentrylongpl{roi} were drawn once for the whole dataset, whereas for clinical datasets manual \glsentryshort{roi} delineation will have to be repeated for each subject. \dag Desktop iMac with 4 GHz Quad-Core Intel Core i7 \ddag High performance computing cluster, one node per subject, 36 Intel(R) Xeon(R) Gold 6240 CPU @ 2.60 GHz cores per node.}
  \label{tab:time}
  \small
  \begin{tabularx}{\linewidth}{>{\raggedright}X >{\centering}X ^>{\sffamily}c ^>{\sffamily}c ^>{\sffamily}c ^>{\sffamily}c}
    \toprule
    \rowstyle{\rmfamily}
    Step & Processing time (s, per subject) & Tractfinder & TractSeg & Atlas & Tractography \\
    \midrule
    \dag Brain masking & 3 $\pm$ 2 &\x{}&\x{}&\x{}&\x{}\\
    \dag Affine MNI registration & 20 $\pm$ 4 &\x{}&  &\x{}&  \\
    \dag Response function & 5 $\pm$ 3 &\x{}&\x{}&\x{}&\x{}\\
    \dag MSMT CSD & 110 $\pm$ 55 &\x{}&  &\x{}&\x{}\\
    \dag SSST CSD + peaks estimation & 18 $\pm$ 8 &  &\x{}&  &  \\
    \dag Atlas transformation + inner product\newline(4 tracts, 2 hemispheres) & 24 $\pm$ 5 &\x{}&  & (\x{}) &  \\
    \dag TractSeg DKFZ / XTRACT (72 / 23 tracts) & 240 $\pm$ 60 &  &\x{}& & \\
    \dag Manual \gls{roi} delineation\newline(once for whole dataset) & 1200 & & & &\x{}\\
    \ddag Non-linear \gls{roi} registration + tractography (4 tracts, 2 hemispheres) & 334 $\pm$ 163 & & & &\x{}\\ \addlinespace
    \rowstyle{\bfseries\rmfamily}
    Total &  & 2:42~min & 4:25~min & \textless2:42~min & \textgreater 27:32~min \\ \bottomrule
  \end{tabularx}
\end{table}
\end{landscape}


\section{Two methods, same training data}\label{sec:tractseg}

A key component of tractfinder is the precise definition of each tract and careful filtering of atlas training streamlines.
Despite the availability of public streamline datasets including the \textit{TractoInferno} data and the TractSeg training and validation bundles, none were found to adequately fulfil the predefined anatomical criteria, and thus new training streamlines were generated for tractfinder.
While the anatomical fidelity of the resulting atlases represents a particular strength, it does make objective comparison between benchmark methods, which may be based on different references, difficult.
In order to produce a direct and objective comparison with TractSeg, a new set of tract atlases were created from the TractSeg DKFZ reference bundles\autocite{Wasserthal2018b} in the HCP105 dataset, using the same split into training and testing subjects as used in the deep learning model (published version).
TractSeg was trained on 63 subjects and tested on the remaining 42, and here the same was done for tractfinder (where ``training'' means constructing the atlas from individual bundles as described in Chapter \ref{chap:atlas}), even though tractfinder atlases can be constructed from just 10-15 subjects (see Section \ref{sec:ntrain}).
Since the original TractSeg paper includes 72 tracts (including both left and right instances of non-commissural tracts and multiple subdivisions of tracts such as the corpus callosum or anterior thalamic radiation) and the performance varied substantially across tracts, tractfinder was run for all tracts for completeness as well as the four (\gls{af}, \gls{cst}, \gls{ifof}, \gls{or}) studied thus far.
\DIFaddbegin \DIFadd{For this experiment, a tractfinder binarisation threshold of 0.01 applies, to more closely approximate the behaviour of TractSeg, which favours larger segmentation volumes.
}\DIFaddend 

\begin{figure}[htb!]
  \makebox[\linewidth][r]{\includegraphics{chapter_5/ts_test_box.pdf}}
  \caption[Direct comparison between tractfinder and TractSeg, all metrics]{Scores for tractfinder (using atlases constructed from the TractSeg training data) and TractSeg validated on the TractSeg reference bundles.}
  \label{fig:ts_atlas}
\end{figure}

In those four main tracts, tractfinder scores equally well or better on all volumetric overlap and distance metrics with the exception of the \gls{dice}, where in particular the scores for the \gls{ifof} and \gls{or} are lower (Fig. \ref{fig:ts_atlas}).
These results clearly demonstrate that for clinically relevant large \gls{wm} tracts, tractfinder is an accurate alternative to TractSeg while maintaining additional advantages in flexibility, robustness and explainability.
The \gls{dice} scores and density correlations for all 72 tracts are shown in Figure \ref{fig:ts_all_tracts}, analogous to Figure 6 in \textcite{Wasserthal2018}.
Neither method convincingly outperforms the other, although TractSeg does score higher when measured by \gls{dice} on more tracts.

An important difference between the two lies in how the fibre orientations, which serve as input to the two methods (as full \glspl{fodf} for tractfinder and as peaks for TractSeg) are computed.
For single shell data, \gls{msmt} \gls{csd} has been used for tractfinder in all datasets and all tracts for consistency, using \gls{wm} and \gls{gm} compartments, to reduce the amount of noise and spurious \gls{wm} signal in non-\gls{wm} areas, most notably the cortex (see also Section \ref{sec:ismrmdiff}).
This is motivated by the need for spatial information with a high specificity for \gls{wm} with which to compare the tract-specific spatial prior in the atlas.
Unfortunately, using only two tissue compartments for \gls{msmt} \gls{csd} in this manner can produce inaccurate reconstructions at tissue boundaries, in particular the subsuming of weak \gls{wm} signal contributions in \gls{csf} partial volume voxels, leading to missing \gls{wm} \glspl{odf} in these areas.
This is the case for tracts like the anterior commissure (CA) and fornices (FX\_left, FX\_right) which are small structures in places directly adjacent to or partially surrounded by \gls{csf}, the reason behind the notably low overlap scores for these structures in Figure \ref{fig:ts_all_tracts}.
Tractfinder based on \gls{ssst} \gls{csd} is much more sensitive to these small structures, at the expense of producing more extensive (less specific) overall tract maps elsewhere.

\begin{figure}[h!]
  \includegraphics{chapter_5/ts_scatter.pdf}
  \caption[Direct comparison between tractfinder and TractSeg, in all tracts]{\Glsentrylong{dice} and density correlation scores for tractfinder and TractSeg using the same training and testing streamlines from the HCP105 dataset. Mean over all subjects is shown for each tract. For abbreviations see \textcite{Wasserthal2018}}
  \label{fig:ts_all_tracts}
\end{figure}

\begin{figure}
  \includesvg[width=\textwidth,pretex=\sffamily\small]{chapter_5/tumours.svg}
  \caption[Example tractfinder results in tumour patients]{Tractfinder maps for selected \glsentryshort{btc} and clinical subjects. Tumours or resection cavities are marked with asterisks (*), where a double asterisk (**) indicates that tumour deformation modelling was applied.
  \textbf{\sffamily a.} Right \gls{or} mapped in \gls{btc} subject PAT16 with a right fronto-temporal anaplastic astrocytoma (\DIFdelbeginFL %DIFDELCMD < \gls{who} %%%
\DIFdelendFL \DIFaddbeginFL \glsentryshort{who} \DIFaddendFL grade 2--3).
  \textbf{\sffamily b.} Right \gls{cst} mapped in \gls{btc} subject PAT26 with a right temporal anaplastic astrocytoma (\DIFdelbeginFL %DIFDELCMD < \gls{who} %%%
\DIFdelendFL \DIFaddbeginFL \glsentryshort{who} \DIFaddendFL grade 3).
  \textbf{\sffamily c.} Right \gls{or} mapped in \gls{nhnn} subject 6, intraoperative scan, with a right temporal oligodendroglioma (\glsentryshort{who} grade 2).
  \textbf{\sffamily d.} Right \gls{af} mapped in \gls{btc} subject PAT03 with a right parietal meningioma (\glsentryshort{who} grade 1).
  \textbf{\sffamily e.} Left \gls{af} mapped in \gls{gosh} subject 4, intraoperative scan, with a left temporal pleomorphic xanthoastrocytoma (\glsentryshort{who} grade 2).
  \textbf{\sffamily f.} Left \gls{ifof} mapped in \gls{btc} subject PAT07 with left temporal ependymoma (\glsentryshort{who} grade 2)}
  \label{fig:tumours}
\end{figure}

\section{The effect of deformation modelling}
\label{sec:btcd}

The \gls{btc} dataset, with its range of tumour types, high quality \gls{dmri} scans and tumour masks, provides an ideal arena to evaluate the effect of tumour deformation modelling on segmentation accuracy.
Four out of the ten selected \gls{btc} subjects had tumours substantial enough to warrant deformation modelling, which was done using the exponential deformation factor model with adaptive $\lambda$ for all subjects.
Results were compared with manual targeted tractography with additional streamline filtering as described in Section \ref{sec:data_btcd}, which was necessary to ensure a high quality baseline in challenging data with disruptive tumour effects.
Selected tract segmentations in the \gls{btc} dataset are shown in Figure \ref{fig:tumours} (panels a, b, d, and f), alongside additional examples from the \gls{nhnn} and \gls{gosh} clinical data.
Across the whole dataset, including both deforming and non-deforming tumours, tractfinder produced consistently high volumetric similarity scores compared with manual tractography, while an average signed bundle distance close to 0 for all tracts but the \gls{or} indicate consistently high agreement in segmentation margins (Fig. \ref{fig:btcd_box}).

\begin{figure}[htb!]
  \makebox[\linewidth][r]{\includegraphics{chapter_5/btcd_box.pdf}}
  \caption[Similarity scores against tractography for tractfinder and TractSeg in the BTC dataset]{Similarity scores for tractfinder and TractSeg DKFZ in the \glsentryshort{btc} dataset, compared with filtered manual tractography.}
  \label{fig:btcd_box}
\end{figure}

Figure \ref{fig:btcd_def} shows how deformation modelling improves segmentation accuracy in most cases, measured with density correlation, although the sample size is small, and there is considerable variation.
In two subjects, there is no clear overall improvement (average across all tracts) with the addition of deformation modelling.
For subject PAT11 (highest scores in Figure \ref{fig:btcd_def}), the superior frontal location of the tumour barely affected any of the studied tracts, hence the overall high score and imperceptible improvement with the addition of deformation modelling.
In subject PAT26, which in Figure \ref{fig:btcd_def} is the only subject showing an average slight decrease in \gls{dice} for ipsilateral tracts with the addition of deformation modelling, the \gls{cst} and \gls{af} saw increased scores, while the \gls{ifof} and \gls{or} segmentations were slightly worse with deformation modelling than without.
In this case, the deformation modelled in the direct vicinity of the temporal lobe tumour was exaggerated, leading to slightly worse detection of the \gls{ifof} and \gls{or}.
At a greater distance from the tumour, however, the deformation modelling accurately matched the patient anatomy, leading to improved detection of more distant tracts, the \gls{af} and \gls{cst} (Fig. \ref{fig:tumours}, panel b).

\begin{SCfigure}[][htb!]
  \includegraphics{chapter_5/btcd_defchange.pdf}
  \caption[Effect of deformation modelling in the BTC dataset]{Effect of deformation modelling on segmentation accuracy, compared with manually filtered targeted tractography. Each large datapoint represents the average across all tracts for a single subject, and is coloured according to the tumour side. Small datapoints represent individual tracts. In the one subject with a midline tumour, all tracts are considered ipsilateral.}
  \label{fig:btcd_def}
\end{SCfigure}
\clearpage{}
\clearpage{}\chapter{Practical application}
\label{chap:applications}

In previous chapters, we have, in large part, focussed on testing and evaluating tractfinder in relatively idealised environments and datasets.
Much of the quantitative evaluation in Chapter \ref{chap:eval} was concerted with research-quality acquisitions of healthy subjects, while even the \gls{btc} data are multi-shell \gls{hardi} scans.
We will next consider some of the realities of real-world clinical data and environments, discussing technical limitations and illustrating the technique's applicability in a selection of case studies.

\section{Technical considerations}\label{sec:technical}

The technical feasibility of tractfinder for intraoperative application depends not only on features of the method itself in terms of speed and reliability, but also on its reliance on a certain minimum raw image quality.
Intraoperative scanning must occur under significant time pressure, and the concessions made to achieve this will have profound consequences for the image quality and subsequent image processing.
In general, two strategies exist for reducing \gls{dmri} scan time:
The first is to reduce the information content of the acquisition, which could mean lower spatial resolution (larger voxels), lower angular resolution (fewer diffusion-weighted directions), shorter diffusion weightings (lower $b$-values) or fewer unique $b$-values (``shells'').
The second is to use accelerated imaging techniques to achieve the (notionally) same image parameters in shorter time.
These two possibilities are explored in the following sections, beginning with an investigation into the effect of diffusion-weighting scan parameters on tract segmentations, in work first published in \textcite{Young2022a} and since expanded to include the \gls{ifof}.

\subsection{Data requirements}\label{sec:ismrmdiff}

It is important to assess the applicability of image processing techniques developed with research quality data in acquisitions more typical of a clinical setting.
Advanced \gls{dmri} processing techniques may sometimes set constraints on the input data, such as a recommended or required minimum number of diffusion-weighted volumes (angular resolution), $b$-values, or spatial resolution.
If it can be demonstrated that an image processing pipeline produces excellent results in a high quality research dataset, which can be interpreted with confidence, then the question follows:
Were one to acquire a lower quality dataset of the same subject and perform the same data processing, how comparable would the resulting segmentation be to that of the high quality scan?
This is the question explored in the following study, which makes use of the high quality \gls{hcp} \gls{dmri} data to determine the minimum acceptable data requirements to obtain successful segmentation, comparing different numbers of direction samples, $b$-values and post-processing strategies, as well as the affects of decreasing data quality on segmentation stability for three different methods (tractfinder, tractography and TractSeg; see Section \ref{sec:methods} for further details).



The downsampled HCP49 dataset (see Section \ref{sec:data} for full details) was used for this study.
In addition to the preprocessing previously described, for each subject, the following five subsampled diffusion schemes were extracted from the full acquisition (Tab. \ref{tab:subschemes}):
60 directions each at $b=1000$~s~mm$^{-2}$ and $b=2000$~s~mm$^{-2}$ (120 directions total, ``DWI-1”), 30 directions each at $b=1000$~s~mm$^{-2}$ and $b=2000$~s~mm$^{-2}$ (60 directions total, ``DWI-2”), 60 directions at $b=1000$~s~mm$^{-2}$ (``DWI-3”), 30 directions at $b=1000$~s~mm$^{-2}$ (``DWI-4”) and 12 directions at $b=1000$~s~mm$^{-2}$ (``DWI-5”).

\begin{table}
  \centering
  \begin{tabular}{c c c} \toprule
    Identifier & $b=1000$~s~mm$^{-2}$ & $b=2000$~s~mm$^{-2}$ \\
    \midrule
    DWI-1 & 60 & 60 \\
    DWI-2 & 30 & 30 \\
    DWI-3 & 60 &    \\
    DWI-4 & 30 &    \\
    DWI-5 & 12 &    \\ \bottomrule
  \end{tabular}
  \caption[Subsampled diffusion schemes for data stability analysis]{Subsampled diffusion schemes, number of directions}
  \label{tab:subschemes}
\end{table}

Four different tract segmentations were produced in each subject using the three different approaches described in Section \ref{sec:methods}: Tractfinder, targeted \gls{roi}-based, probabilistic streamline tractography, and TractSeg (DKFZ and XTRACT models).
Segmentations were compared for the three tracts most commonly reconstructed in neurosurgical applications:
The \gls{cst}, \gls{af}, \gls{or}, and \gls{ifof}.
All compared tract segmentation methods are predicated on \glspl{fod} modelled from diffusion data with \gls{csd}, although tractography and TractSeg could both use fibre orientations estimated via other techniques such as BedpostX\autocite{Behrens2007}.
For the two-shelled datasets, \gls{msmt} \gls{csd} can be run to separate the signal contributions from \gls{wm}, \gls{gm} and \gls{csf} and thus obtain an optimal \gls{wm} \gls{fod} image free of noisy extraneous signal in non-\gls{wm} regions.
For the single-shelled datasets, full, direct \gls{msmt} \gls{csd} with all three tissue compartments is not possible.
Instead, the following approaches were compared: 1) Modelling two tissue types with \gls{msmt}.
\Gls{msmt} \gls{csd} is performed twice, once modelling \gls{wm} and \gls{gm} compartments, and once modelling \gls{wm} and \gls{csf} compartments; 2) Standard \gls{ssst} \gls{csd}.

To investigate the effects of different data quality and modelling on a given method, the \gls{dice} \autocite{Dice1945} was computed between each segmentation in datasets DWI-2--5 and the corresponding segmentation of the same method in DWI-1 using \gls{msmt} \gls{csd} (considered the benchmark segmentation).
This ``self-similarity'' approach is intended to answer the question put forward above:
How close to the results produced from an ``ideal'' acquisition can be achieved from a lower quality scan.
This comparison was made for each combination of diffusion scheme and \gls{csd} approach.



\begin{figure}[htb!]
  \centering
  \includesvg[width=\textwidth,pretex=\sffamily\footnotesize]{chapter_6/self_dice.svg}
  \caption[Self-similarity scores for data stability analysis]{Self Dice scores for all tracts and methods. Each datapoint represents a comparison with the corresponding segmentation in DWI-1, the highest quality acquisition scheme. Outlined markers represent the mean over all subjects.}
  \label{fig:self_dice}
\end{figure}

Self-similarity results are plotted in Figure \ref{fig:self_dice}.
Each datapoint corresponds to the segmentation produced for a given \gls{dmri} direction scheme, method and pipeline \textit{compared against} the result obtained from the highest quality scheme (DWI-1).
The consistency in segmentation results differed with both data quality and \gls{csd} approach.
In particular, a high sensitivity to \gls{csd} technique is apparent in the tractography and tractfinder results, with particularly large differences between \gls{msmt} \gls{wm}+\gls{csf} / \gls{ssst} and \gls{msmt} \gls{wm}+\gls{gm}, which can be attributed to the inclusion of \gls{gm} signal in the \gls{wm} \gls{fod} reconstructions.
In the first two approaches, the \gls{gm} compartment is not as strongly suppressed from the \gls{wm} reconstruction, resulting in high \gls{wm} \gls{fod} amplitudes in the cortex and subcortical \gls{gm}, leading in turn to more extensive segmentations:
In tractography via the further propagation of streamlines into \gls{gm}, and in tractfinder by increasing the inner product values.

TractSeg is appears less sensitive to differences in \gls{csd} reconstruction.
When considering only the best results (\gls{msmt} in DWI-2 and, in the single-shelled datasets, \gls{msmt} \gls{wm}+\gls{gm} for tractography and tractfinder and \gls{ssst} for TractSeg, Fig. \ref{fig:self_dice}), tractography displays the greatest instability with decreasing data quality, with the similarity score between DWI-5 \gls{msmt} \gls{wm}+\gls{gm} and DWI-1 \gls{msmt} falling as low as 0.78 (subject and cerebral hemisphere mean) for the \gls{af}.
TractSeg is trained on \gls{ssst} \gls{csd} in single-shelled data as well as multi-shelled data, explaining why it does best with either \gls{ssst} or \gls{msmt} \gls{wm}+\gls{csf} in the single-shell results.

Probabilistic tractography's comparatively high sensitivity to acquisition and processing pipelines is consistent with the well-established reproducibility and noise sensitivity problems associated with tractography.
Meanwhile, voxel-wise segmentation methods are not susceptible to error propagation along the tract and are more robust to lower angular resolution:
When the number of directions in a multi-shelled acquisition was halved (DWI-2), average tractfinder self-similarity was 0.98 or above, while for tractography it was substantially lower at around 0.9.
Overall, these results motivate the use of \gls{msmt} \gls{wm}+\gls{gm} for tractfinder in all single-shelled acquisitions, with the caveat that particular care must be taken when interpreting segmentations in \gls{wm}/\gls{csf} partial volume areas.
Understanding the behaviour of different tract segmentation techniques when applied to varying qualities of \gls{dmri} acquisition and post-processing approaches is important if segmentation methods developed in research settings are to be consistently and reproducibly applied to clinical quality acquisitions.
It is useful to know how comparable segmentation results in a single-shelled, 30 direction dataset are to those one might have obtained with a much higher angular resolution dataset and optimal post-processing.
This is particularly relevant for longitudinal studies, or when comparing different acquisitions of the same subject in a clinical context (for example, for monitoring disease progressions or post-operative changes).

\subsection{Fast imaging and artefacts}

Because high angular resolution \gls{dmri} acquisitions involve acquiring many volumes of data, they would have prohibitively long scan times without the use of \gls{epi} (see Section \ref{sec:epi}).
Despite its undeniable benefits, \gls{epi} produces problems of its own, including the geometric image distortions that commonly arise at boundaries between substances with different magnetic susceptibilities, typically at tissue/air interfaces.
For intraoperative scanning, this poses a particular problem in which precisely the site of interest, the operative field and surgical cavity, can suffer from the strongest distortions, affecting nearby structures (such as \gls{wm} tracts) that are of neuronavigational importance\autocite{Yang2022}.
In most applications, \gls{epi} distortion artefacts are tolerable as they can be fairly robustly corrected for during image preprocessing.
\DIFaddbegin \DIFadd{Smaller distortions may also have little effect on tractfinder if their effects can be incorporated into the image transform computed by linear registration to atlas space.
}\DIFaddend However, not only can the distortions around resection cavities be especially severe due to the large exposed area of brain surface (Fig. \ref{fig:epi}), neither can they be easily corrected for intraoperatively, as the currently available tools take on the order of tens of minutes to hours to run\DIFaddbegin \DIFadd{, or alternatively require the aquision of susceptibility field maps, adding to total scan times}\DIFaddend .

\begin{figure}[h!]
  \includegraphics[width=\textwidth]{chapter_6/epidist.pdf}
  \caption[EPI distortion artefacts]{Examples of strong susceptibility distortions (arrowheads) on intraoperative \glsentrylong{epi} acquisitions obtained on the Siemens MAGNETOM Vida system at \glsentrylong{gosh}. Effects this extreme are not present in all cases, but neither are they uncommon.}
  \label{fig:epi}
\end{figure}

Typical \gls{dmri} acquisitions are based on \gls[noindex=false]{ssepi}, in which an entire slice's worth of data is sampled in a single excitation-readout sequence.
As an alternative to \gls{ssepi}, multi-shot sequences have been proposed.
These include interleaved and \gls[noindex=false]{rsepi} sequences which, in general, only traverse a segment of $k$-space per echo train, increasing the bandwidth along the phase-encoding direction (and thus reducing field inhomogeneity induced phase accumulation) while increasing scan time\autocite{Wang2018}.
RESOLVE (readout segmentation of long variable echo-trains), a Siemens product, is one example of a commonly used \gls{rsepi} sequence for high quality three-scan trace weighted images with high in-plane resolution.

While RS sequences can be combined with acceleration techniques including generalised autocalibrating partially parallel acquisition (GRAPPA) and simultaneous multi-slice (SMS), they nevertheless result in significantly longer scan times, which, in clinical applications, must be weighed against any gains in image quality and distortion reduction.
For example, in \textcite{Wang2018} authors compared interleaved and \gls{rsepi} sequences with \textit{in vivo} scan times of around ten minutes to acquire only four slices of 4~mm thickness!
In \textcite{Elliott2020}, \gls{ssepi} was compared with a RESOLVE sequence with 15 segments, six diffusion-weighted directions, and a scan time of 10:34 minutes.
For comparison, the clinical \gls{ssepi} \gls{dti} sequence currently used intraoperatively at \gls{gosh} is a 30 direction $b=1000$~s~mm$^{-2}$ acquisition lasting just 4:57~minutes with 2.3~mm isotropic voxels.



\begin{table}
  \caption[RS-EPI test scan acquisition parameters]{\Glsentrylong{rsepi} test scan parameters}
  \label{tab:rsepi}
\footnotesize
  \begin{tabularx}{\textwidth}{l >{\raggedright\arraybackslash}X >{\raggedright\arraybackslash}X >{\raggedright\arraybackslash}X >{\raggedright\arraybackslash}X >{\raggedright\arraybackslash}X} \toprule
    & RESOLVE & Routine clinical \gls{dti} & Test scan 1: 7seg 6\x $b$800 & Test scan 2: 3seg 12\x $b$1000 & Test scan 3: 3seg 20\x $b$1000 \\
  \midrule
   $n$ directions & 3 & 30 & 6 & 12 & 20 \\
   b values & 0, 800 & 0, 1000 & 0, 800 & 0, 1000 & 0, 1000 \\
   scan time (min) & 4:53 & 4:00 & 4:57 & 7:30 & \DIFdelbeginFL \DIFdelFL{not recorded }\DIFdelendFL \DIFaddbeginFL \DIFaddFL{\textless5:33 }\DIFaddendFL \\
   in-plane & 1.198\x{}1.198 & 2.3\x{}2.3 & 2.3\x{}2.3 & 2.5\x{}2.5 & 2.3\x{}2.3 \\
   resolution (mm) & & & & & \\
   slice thickness~(mm) & 4 & 2.3 & 2.1 & 2.5 & 2.3 \\
   size (voxels) & 192\x{}192\x{}35  & 96\x{}96\x{}62 & 96\x{}96\x{}47 & 88\x{}88\x{}58 & 96\x{}96\x{}58 \\
   FOV (mm)  & 230\x{}230  & 220\x{}220 & 220\x{}220 & 220\x{}220 & 220\x{}220 \\
   TR (ms) & 6480 & 6600 & 6570 & 8960 & 3850 \\
   $n$ segments & 7 & 1 & 7 & 3 & 3 \\
   GRAPPA & 2 & 2 & 2 & 2 &  \\
   SMS &  &  &  &  & 2 \\ \bottomrule
  \end{tabularx}
\end{table}

We set out to explore the feasibility of setting up a \gls{rsepi} \gls{dti} sequence for intraoperative use at \gls{gosh}.
An (approximately) isotropic voxel size is generally considered optimal for \gls{dti} modelling and tractography \autocite{Vos2011, Neher2013}, although tractography with high in plane resolution and thick slices is certainly possible.
Additionally, a sufficient number of diffusion-weighted directions for crossing fibre modelling (e.g \gls{csd}) would be ideal, potentially as few as 12 directions given the results from Section \ref{sec:ismrmdiff}.
In the end, we tested three different \gls{dti} sequences as well as several different three-scan trace RESOLVE sequences (not discussed here), all acquired from a healthy adult volunteer on the intraoperative Siemens MAGNETOM Vida (3T) system at \gls{gosh}.
The acquisition parameters of the three test sequences, along with the routine \gls{gosh} \gls{dmri} sequences for comparison, are shown in Table \ref{tab:rsepi}.

For each scan, minimal preprocessing was conducted using MRtrix3\autocite{Tournier2019} tools:
\Gls{mppca} denoising and bias field correction, followed by \gls{dt} modelling and \gls{dec} mapping, and \gls{ssst} \gls{csd}.
For \gls{csd}, $l_{max}$ was set to 8 except for the six direction scan where $l_{max}=6$ was used.
No distortion correction was applied.
Finally, the left \gls{cst} was reconstructed with a simple probabilistic tractography protocol, using the internal capsule as a seed region and an inclusion \gls{roi} at brain stem level, along with exclusion \glspl{roi} for common false positives.

\begin{figure}[htb!]
  \centering
  \includesvg[width=\textwidth,pretex=\small\sffamily]{chapter_6/rsepi.svg}
  \caption[RS-EPI test scan results]{Baseline \glsentrylong{ssepi} diffusion sequence and three \glsentrylong{rsepi} test sequences
    \textbf{\sffamily a.} Baseline (30\x{}$b$1000 \glsentrylong{ssepi} \glsentrylong{dti})
    \textbf{\sffamily b.} 7seg 6\x{}$b$800
    \textbf{\sffamily c.} 3seg 12\x{}$b$\DIFdelbeginFL \DIFdelFL{b1000
    }\DIFdelendFL \DIFaddbeginFL \DIFaddFL{1000
    }\DIFaddendFL \textbf{\sffamily d.} 3seg 20\x{}$b$1000}\label{fig:rsepi}
\end{figure}

Based on visual assessment alone, the degree of distortion reduction is difficult to evaluate for all compared sequences (Fig. \ref{fig:rsepi}).
It is unclear whether, in the case of intraoperative scans with high levels of distortion at the resection site, the amount of distortion improvement would be comparatively greater than is apparent in a healthy scan with minimal distortion to begin with.
The third test scan additionally used SMS acceleration which appears to have significantly degraded and even corrupted the scan, even as Siemens has advertised the use of SMS with RESOLVE on compatible systems
\footnote[2]{minimum software version syngo MR XA11B. Source: \url{https://www.siemens-healthineers.com/en-uk/magnetic-resonance-imaging/options-and-upgrades/clinical-applications/syngo-resolve} retrieved 06-Oct-2023}.


\begin{figure}[htb!]
  \centering
  \includesvg[width=\textwidth,pretex=\small\sffamily]{chapter_6/rsepi_fod_tg.svg}
  \caption[RS-EPI test scan fibre orientation and CST reconstructions]{\Glsentrylongpl{fod} at the centrum semiovale and left \glsentrylong{cst} tractography streamlines.
  \textbf{\sffamily a.} Baseline (30\x{}$b$1000 \glsentrylong{ssepi} \glsentrylong{dti})
  \textbf{\sffamily b.} 7seg 6\x{}$b$800
  \textbf{\sffamily c.} 3seg 12\x{}$b$1000
  \textbf{\sffamily d.} 3seg 20\x{}$b$1000}
  \label{fig:rsepi-fod}
\end{figure}

Figure \ref{fig:rsepi-fod} demonstrates the \glspl{fod} and tractography reconstructions obtained from each sequence.
The six direction scan resulted in poorly resolved \glspl{fod} and correspondingly noisy streamlines, while the third test scan which was degraded due to the use of SMS produced entirely unusable \glspl{fod}.
Most promising are the results obtained from the 12 direction scan, despite the low angular resolution, although there is certainly still a lot of noise.
The findings presented in Section \ref{sec:ismrmdiff} indicate that a 12 direction scan, while far from ideal, could still be used to obtained reliable tract segmentations (Fig. \ref{fig:ssepi2}).
It is simply a question of priority:
Is the achievable reduction in susceptibility distortion worth the implications of lower angular resolution?
This is a question that cannot be answered from such exploratory data, and indeed the answer is likely to be case-specific, given that the amount of distortion varies significantly from scan to scan depending on the specific patient and surgical setup.
While further exploration of distortion reduction was not a priority of this project, it remains an interesting avenue for future research, and these preliminary findings provide an indication that \gls{rsepi} for intraoperative imaging remains an open possibility.

\begin{figure}[hbt!]
  \centering
  \begin{subfigure}{0.4\textwidth}
    \includegraphics[width=\textwidth]{chapter_6/2_glass_tg.png}
  \end{subfigure}\begin{subfigure}{0.4\textwidth}
    \includegraphics[width=\textwidth]{chapter_6/2_glass_tf.png}
  \end{subfigure}
  \caption[RS-EPI 12 direction test scan CST reconstructions with tractography and tractfinder]{Tractfinder reconstruction (right) of the left \glsentrylong{cst} in the 12 direction, three segment \glsentrylong{rsepi} scan compared with streamline tractography (left).}
  \label{fig:ssepi2}
\end{figure}
\clearpage{}
\clearpage{}\section{Clinical case studies}

Having discussed some of the technological challenges of intraoperative image acquisition and processing, we will now turn our minds to the other side of the camera and explore the complex reality of tumour radiology and image guided neurosurgery.
In this section, the use of tractfinder in the presence of both significant tumour deformation and brain shift is illustrated in an \gls{imri} case from the \gls{nhnn}, first described in \textcite{Young2022}.
A further example, also from the \gls{nhnn}, demonstrates the effect of peritumoural oedema on tract imaging.
Finally, Section \ref{sec:case} will look at two examples of some of the most challenging neurosurgical procedures, involving paediatric tumours of the thalamus, in \gls{imri} cases taking place at \gls{gosh}.

\subsection{Intraoperative brain shift}
\label{sec:imri}

When considering the application of tractfinder to intraoperative imaging, we need to take brain shift into account, which is unpredictable:
Differing effects stem from drainage of fluid, pressure changes, tumour debulking and gravity.
Nevertheless, the aim is to achieve accurate intraoperative tract segmentation while avoiding the need to perform additional tumour segmentation intraoperatively.

As the atlas is designed to be fairly inclusive, with the inner product acting to correct small spatial inaccuracies, it is possible in some cases where brain shift is minimal to reuse the preoperative tumour deformation field.
In cases of significant tumour debulking, the deformation field can be recomputed from the preoperative tumour segmentation by adjusting the value of $s$ in the deformation expression (\ref{eq:forwardP}) to simulate a reduction in tumour volume.
This scenario is demonstrated in Figure \ref{fig:shrink}, showing the resection of a large temporal epidermoid cyst in an adult patient, the same case depicted in an earlier demonstration of brain shift and tractography in Figure \ref{fig:shift}.
Intraoperative imaging shows a significant reduction in cyst volume and the adjacent \gls{cst} has shifted accordingly, resulting in significant discrepancy between preoperative tract reconstruction and intraoperative tract position.
Even so, the pathway is still shifted medially compared to its normal position (which can be clearly seen in comparison with the contralateral tract), such that tumour deformation modelling is still required for accurate segmentation.
By reusing the preoperative lesion segmentation and setting $s=0.8$ (effectively modelling a 20\% smaller tumour), the resulting deformation field is able to capture the approximate location of the shifted tract, without the need for additional tumour segmentation on the intraoperative structural imaging.
Adjusting the value of $s$ and reusing preoperatively computed values of $D_t$ and $D_b$ avoids time and resource-intensive intraoperative lesion segmentation, brain shift modelling, or non-linear registration.

\begin{figure}[h!]
  \centering
  \includesvg[width=\textwidth,pretex=\sffamily]{chapter_6/shrink.svg}
  \caption[Intraoperative tumour deformation and brain shift modelling]{Demonstration of successful pre- and intraoperative tract imaging using only preoperative tumour segmentation (blue outline). Setting $s=0.8$ (green outline) simulates intraoperative decompression and models accurate atlas deformation to align with intraoperative anatomy after brain shift. The resulting \glsentrylong{cst} tractfinder segmentations are in strong agreement with streamline tractography. An earlier version of this figure was first published in \textcite{Young2022}.}
  \label{fig:shrink}
\end{figure}

\subsection{Tumour deformation and oedema}

In this example case of an \gls{nhnn} patient with a tumour embedded within the right precentral gyrus (tumour histology information is unavailable for this subject) we will consider the combination of tumour mass effect and peritumoural oedema on tract reconstruction.
Not only is there substantial distortion to the \gls{cst} and cortical topology, the tumour is also surrounded with oedema, which is affecting diffusion measurements.
Due to the large free water compartment within the \gls{wm}, the usual \gls{msmt} \gls{csd} approach with \gls{wm} and \gls{gm} compartments is unsuitable in this situation.
Instead, \gls{wm} and \gls{csf} compartments were modelled, allowing modelling of \gls{wm} \glspl{fod} within oedema while reducing noise slightly compared to \gls{ssst} \gls{csd}.
After \gls{fod} modelling, the \gls{cst} was reconstructed using probabilistic tractography, tractfinder with tumour deformation modelling, and TractSeg (DKFZ).

In Figure \ref{fig:oedema}, panel a, fibre orientations clearly indicate the presence of \gls{csd} fibres circumventing the tumour, yet the presence of oedema has inhibited their detection in both the tractography (panel d) and TractSeg (panel e) reconstructions.
The deformed tract atlas, by contrast, aligns well with the \gls{fod} orientations (panel b) resulting in a successful tractfinder segmentation  (panel c).
This example highlights a particular advantages of tractfinder:
All three methods utilised the same \glspl{fod} as the basis for identifying the tract, which clearly depict the presence of \gls{wm} fibres within the oedematous zone, yet the combination of crossing \gls{af} fibres, altered diffusion, and tract displacement presented a significant barrier to streamlines, with few penetrating to the \gls{cst} beyond.
Whether TractSeg neglected the same region because of deformation, oedema, or a combination of both is difficult to determine.
Meanwhile, the deformed atlas predicted the presence of fibres within the affected region and their expected orientation, both of which are confirmed by the \gls{fod} estimates, resulting in appropriate detection of the tract of interest.
Note how the atlas probability is also high within the tumour itself:
Its prediction alone is not sufficient for accurate segmentation, and a comparison with the local diffusion data is required to correct for imperfect atlas alignment.

\begin{figure}
  \includesvg[width=\textwidth,pretex=\sffamily]{chapter_6/oedema.svg}
  \caption[Tumour deformation modelling and oedema]{Adult patient with a tumour in the right precentral gyrus.
  \textbf{\sffamily a.} \Glspl{fod} computed with \gls{msmt} \gls{csd} (\gls{wm}+\gls{csf} compartments).
  \textbf{\sffamily b.} Deformed \gls{cst} orientation atlas.
  \textbf{\sffamily c.} Tractfinder map of the \gls{cst}, thresholded at 0.05.
  \textbf{\sffamily d.} Probabilistic tractography streamlines\DIFdelbeginFL \DIFdelFL{, \textbf{e.} }\DIFdelendFL \DIFaddbeginFL \DIFaddFL{.
  \textbf{\sffamily e.} }\DIFaddendFL TractSeg segmentation probability, thresholded at 0.5.
  \textbf{\sffamily f.} Coronal and sagittal slices indicating the location of panels \textbf{\sffamily a--e}.}
  \label{fig:oedema}
\end{figure}

\clearpage
\subsection{A tale of two paediatric thalamic gliomas}\label{sec:case}

From mid 2021 to late 2023, nine tumour resections were carried out at \gls{gosh} with \gls{imri} guidance that also included multi-directional diffusion-weighted imaging (that is, not trace-weighted or \gls{adc} imaging).
Throughout this period, neurosurgical and radiological staff were continually learning how best to integrate the \gls{imri} facility into their practice.

The cases included two posterior fossa tumours, two midline (thalamic) gliomas and five supratentorial hemispheric tumours.
Of these, \gls{imri} guidance with diffusion tractography is potentially the most useful for thalamic tumours, given their complex functional environment and difficult surgical access.
Both low-grade \autocite{Wong2016} and high-grade\autocite{Dorfer2021} thalamic tumours may be candidates for maximal safe resection, although infiltrating lesions cannot be safely resected completely without serious risk of neurological detriment to the patient.
With advances in surgical navigation, maximal safe resection has become an important consideration in thalamic tumours once considered inoperable\autocite{Souweidane1996,Puget2007,Steinbok2016,Grewal2019,Sunderland2021}.
Children in particular\autocite{Ferroli2023} experience better overall survival rates when thalamic tumours are more radically resected compared to subtotal resection or biopsy\autocite{Cinalli2018}.
The Alder Hey Children's Hospital in Liverpool, UK experienced an increase in substantial resection ($<1.5$~cm$^3$ residual) rates from 37\% to 94\% with the introduction of \gls{imri} navigation, without any associated increase in postoperative morbidity\autocite{Sunderland2021}.
Similarly, a review of 38 thalamic tumour patients (paediatric and adult) treated at the Chinese PLA General Hospital in Beijing, China found that the use of \gls{imri} increased \gls{gtr} rates from 42\% to 68\%\autocite{Zheng2016}.
Preservation of the posterior limb of the internal capsule, containing the \gls{cst}, during resection is critical, with motor deficits being the most common functional symptoms of thalamic tumours\autocite{Puget2007, Zheng2016, Palmisciano2021}.
As the specific interactions between thalamic tumours and surrounding \gls{wm} are complex and unpredictable, as we will see in the following case studies, \gls{dmri} \gls{wm} imaging is especially valuable in these cases for safely planning and conducting resections\autocite{Celtikci2017}.

The following will look at two of the \gls{gosh} diffusion \gls{imri} cases in detail, beginning with the most recent, considering the imaging and tumour features, perioperative clinical presentation, and neurology.
Hopefully this will provide a balanced perspective into the challenges and potential of intraoperative \gls{dmri} for such cases.
Note that all of the following image processing was performed retrospectively in a research capacity, and was not involved in any clinical decision making.

\paragraph*{Patient 8: High grade glioma}

The first of the two patients presented at seven years of age with a diffuse midline astrocytoma (\gls{who} grade 4) in the left thalamus.
A strong degree of mass effect called for tumour deformation modelling, so the lesion was segmented on the preoperative structural scan (Fig. \ref{fig:8p}).

\begin{figure}[htb!]
  \centering
  \includegraphics[width=\textwidth]{chapter_6/case_studies/gosh_8_preop_t1.png}
  \includegraphics[width=\textwidth]{chapter_6/case_studies/gosh_8_preop_def.png}
  \caption[GOSH iMRI patient 8, preoperative imaging]{Preoperative $T_1$-weighted imaging (top) of patient 8 with high-grade left thalamic glioma. Bottom: Tumour deformation modelling in a co-registered MNI template image. No \glsentrylong{dmri} acquired preoperatively was available.}
  \label{fig:8p}
\end{figure}

The diffuse and infiltrative nature of this tumour made identifying the posterior capsule, thalamus and tumour margins exceedingly difficult.
During the \gls{imri} session, a \gls{dti} scan was requested by the radiologist after extensive discussion of the already acquired conventional contrast scans, from which the extent of internal capsule infiltration was indiscernible (Fig. \ref{fig:8i}).
\Gls{dti} \gls{dec} maps showed a largely intact and slightly displaced internal capsule with extensive tumour infiltration, so the decision was made not to attempt further resection of that part of the tumour.
Roughly half of the lesion was resected, predominantly in the ventral and medial parts.
Postoperatively the patient's right side weakness was slightly improved and the patient received adjuvant radiotherapy after confirmation of high-grade histology with H3K27M mutation.
Over the following weeks the patient's hemiparesis persisted but improved slowly, consistent with the decompression of the \gls{csd} and the tumour's infiltrative nature.

\begin{figure}
  \centering
  \includegraphics[width=\textwidth]{chapter_6/case_studies/gosh_8_iop_t1.png}
  \includesvg[width=\textwidth]{chapter_6/case_studies/gosh_8_iop_fa.svg}
  \caption[GOSH iMRI patient 8, intraoperative imaging]{Intraoperative imaging showing transcallosal approach to patient 8's thalamic glioma. Top: $T_1$-weighted imaging. Bottom: \glsentrylong{dti} \glsentrylong{dec} \glsentrylong{fa} map. Arrowheads indicate partially infiltrated and displaced internal capsule.}
  \label{fig:8i}
\end{figure}

Identification of the internal capsule and assessment of its condition on intraoperative imaging may have been significantly simplified with advanced tract-specific \gls{dmri} analysis.
Interpreting colours on the directional \gls{fa} map is complicated by crossing fibres and disturbed diffusion caused by infiltration and oedema.
What is unclear is the state of the neuronal fibres in the infiltrated portion: Are they intact, distorted, or destroyed?
Visualising the fibre \glspl{odf} using \gls{csd} gives us further insight (Fig. \ref{fig:8i_fod}).

\begin{figure}
  \centering
  \includegraphics[width=0.33\textwidth]{chapter_6/case_studies/gosh_8_close_tensor.png}\,\includegraphics[width=0.33\textwidth]{chapter_6/case_studies/gosh_8_close_csd.png}\,\includegraphics[width=0.33\textwidth]{chapter_6/case_studies/gosh_8_close_atlas.png}
  \caption[GOSH iMRI patient 8, intraoperative imaging, internal capsule detail]{Magnified coronal view of partially resected tumour and internal capsule and modelled fibre orientations.
  \textbf{\sffamily a.} \Glsentrylong{dt}.
  \textbf{\sffamily b.} \Glsentrylong{csd} \glsentrylongpl{fod} (\glsentrylong{msmt} \glsentrylong{csd} with \glsentrylong{wm} and \glsentrylong{csf} compartments).
  \textbf{\sffamily c.} Deformed \glsentrylong{tod} \glsentrylong{cst} atlas. While the main body of the tract is well defined in all three images, the boundaries between intact fibres, infiltrated \glsentrylong{wm}, and tumour are indiscernible.}
  \label{fig:8i_fod}
\end{figure}

Using a tumour deformation field modelled from the preoperative scan, tractfinder reconstructed the corticospinal tract in very close agreement with probabilistic streamline tractography (Fig. \ref{fig:8i_tf}).
Both indicate a largely intact bundle slightly laterally displaced, although neither can necessarily rule out the presence of infiltrated fibres closer to the tumour, as the expected resulting decrease in anisotropy would produce fewer streamlines through that region and reduced tractfinder probability.
Surprisingly, TractSeg was unable to successfully detect the \gls{cst} in the affected hemisphere, despite the relatively manageable amount of distortion.
It is unclear what led to failure in this particular subject, as TractSeg has shown some success in cases with similar tumour deformations\autocite{Moshe2022}.

\begin{figure}
\includegraphics[width=\textwidth]{chapter_6/case_studies/gosh_8_iop_tg.png}
  \includegraphics[width=\textwidth]{chapter_6/case_studies/gosh_8_iop_tf.png}
  \caption[GOSH iMRI patient 8, intraoperative imaging, corticospinal tract reconstructions]{Coronal slices in 5~mm increments with \glsentrylong{cst} reconstructions on intraoperative imaging for patient 8. Top: Probabilistic streamline tractography, Bottom: Tractfinder \glsentrylong{cst} with tumour deformation modelling based on preoperative tumour segmentation.}
  \label{fig:8i_tf}
\end{figure}

\paragraph*{Patient 5: Low grade pilocytic astrocytoma}

A second patient presented with a low-grade pilocytic astrocytoma in the left thalamus at 23 months of age with progressive right-sided weakness.
On preoperative \gls{mri} examination, a sliver of tissue within the tumour, presumed to be a part of the \gls{cst}, was evident, while the rest of the tract was displaced (Fig. \ref{fig:5p}).
This being a low-grade and well-encapsulated tumour, confirming the location of the \gls{cst}, divided into an intratumoural and a displaced portion, was particularly difficult in this case.

\begin{figure}[htb!]
  \centering
  \includesvg[width=\textwidth]{chapter_6/case_studies/gosh_5_preop_t1.svg}
  \includesvg[width=\textwidth]{chapter_6/case_studies/gosh_5_preop_fa.svg}
  \caption[GOSH iMRI patient 5, preoperative imaging]{Preoperative radiological presentation of \glsentrylong{gosh} patient 5, with a low-grade astrocytoma of the left thalamus. Top: $T_1$-weighted scan. Bottom: \glsentrylong{dti} \glsentrylong{dec} \glsentrylong{fa} map. Arrowheads indicate the \glsentrylong{cst}, partially located within the tumour.}
  \label{fig:5p}
\end{figure}

A preoperative \gls{dmri} sequence with \gls{dec} visualisation was able to confirm that the strip of tissue inside the tumour indeed constituted part of the \gls{cst} (Fig. \ref{fig:5p}).
Streamline tractography also depicted the \gls{cst} as both within and displaced posteriorly around the tumour.
The part of the pathway involved was at the cerebral peduncle level, which even in controls appears as a relatively narrow bundle.
Consequently, the tractfinder atlas was unable to account for both the tumour-engulfed section, which remained spatially relatively unmoved from a normal position, and the posteriorly displaced portion simultaneously.
Basic tract mapping reconstructed the first part, while additional tumour deformation modelling enabled detection of the second part (Fig. \ref{fig:5p_cst}).
Of course, such an \textit{ad hoc} solution would be entirely impractical and difficult to interpret, compared to tractography which is better at exploring the available pathways regardless of prior anatomical expectations.
As such, this is a case in which tractfinder was unsuited to the complex anatomy and tumour deformation effects at hand.

\begin{figure}[htb!]
  \centering
  \includegraphics[width=\textwidth]{chapter_6/case_studies/gosh_5_preop_tg_5mm.png}
  \includegraphics[width=\textwidth]{chapter_6/case_studies/gosh_5_preop_tf_5mm.png}
  \caption[GOSH iMRI patient 5, preoperative imaging, corticospinal tract reconstruction]{Reconstruction of the left \glsentrylong{cst} on preoperative imaging for patient 5. Top: Streamline tractography \glsentrylong{tdi} map, thresholded at ten streamlines. Bottom: Tractfinder with and without deformation modelling merged into a single segmentation.}
  \label{fig:5p_cst}
\end{figure}

Due to this complicated involvement of the internal capsule, a debulking resection under \gls{imri} guidance was indicated to relieve pressure on the \gls{cst}.
After a large part of the tumour was removed, the patient was brought through to \gls{imri} (Fig \ref{fig:5i}) to confirm maximal safe resection with some residual tumour remaining.
There was concern of an area of infarction involving the \gls{cst}, however the patient's hemiparesis, after experiencing worsened motor deficit immediately postoperatively, improved over the following days after decompression of the motor fibres.
The patient went on to receive adjuvant chemotherapy and showed further improvement over longer term follow-up, their hemiparesis returning to preoperative levels.

\begin{figure}[htb!]
  \centering
  \includegraphics[width=\textwidth]{chapter_6/case_studies/gosh_5_iop_t1.png}
  \includegraphics[width=\textwidth]{chapter_6/case_studies/gosh_5_iop_fa.png}
  \caption[GOSH iMRI patient 5, intraoperative imaging]{Intraoperative $T_1$-weighted (top) and \glsentrylong{dec} \glsentrylong{fa} (bottom) for patient 5, showing left craniotomy and surgical corridor through the temporal lobe}
  \label{fig:5i}
\end{figure}

\begin{figure}[htb!]
  \centering
  \includegraphics[width=\textwidth]{chapter_6/case_studies/gosh_5_iop_tg_5mm.png}
  \includegraphics[width=\textwidth]{chapter_6/case_studies/gosh_5_iop_tf_5mm.png}
  \caption[GOSH iMRI patient 5, intraoperative imaging, corticospinal tract reconstruction]{Reconstruction of the \glsentrylong{cst} on intraoperative imaging using tractography (top) and tractfinder (bottom) with standard affine registration.}
  \label{fig:5i_cst}
\end{figure}

Reconstruction of the \gls{cst} on intraoperative imaging again proved difficult.
Here, significant brain shift away from the craniotomy site prevented accurate atlas registration, resulting in a mismatch in anatomical alignment between atlas and target image (Fig \ref{fig:5i_cst}).
Non-linear registration, with harsh penalties on strong local deformation to reduce overfitting around the resection surface, could mitigate this, but would be impractical for routine intraoperative use.
\clearpage{}

\epipage{It is good to have an end to journey towards, but it is the journey that matters, in the end.}{Ursula K. Le Guin, \textit{The Left Hand of Darkness}}
\clearpage{}\chapter{Conclusions}
\DIFdelbegin %DIFDELCMD < \label{chapterlabel6}
%DIFDELCMD < %%%
\DIFdelend 

The growing availability of advanced \gls{mri} capabilities in health centres is bringing attention to the shortcomings of current image processing techniques in fully exploiting potential benefits for patients.
In a small internal survey of five neuroradiologists and neurosurgeons at \gls{gosh}, all respondents expressed that they would find a tool for intraoperative imaging of \gls{wm} bundles after brain shift to be either slightly (2/5) or very (3/5) useful, while confirming reservations about the reliability or accuracy of tractography.
Against this background, this thesis has set out to explore the current capabilities of \gls{dmri} to map brain \gls{wm} for surgical planning and neuronavigation, and propose a novel technique to fulfil the requirements for rapid and robust \gls{wm} tract detection.
\DIFdelbegin \DIFdel{Guided by the objectives set out in Chapter \ref{chap:problem}, the }\DIFdelend \DIFaddbegin \DIFadd{The }\DIFaddend proposed pipeline, called tractfinder, involves constructing tract-specific orientation atlases which are compared with a subject's \gls{fod} image, to achieve direct voxel-wise segmentation with incorporated \textit{a priori} anatomical knowledge.
\DIFaddbegin \DIFadd{Chapter \ref{chap:problem} set out four specific objectives to guide the developement process, which have each been addressed as follows:
}

\begin{itemize}


\item[\DIFadd{--}]\DIFadd{\textit{Streamline tracking at the point of application is to be avoided, owing to its demands on resources and expertise and vulnerability to pathology.
Instead the rich orientation information available in \gls{dmri} acquisitions should be directly compared with} a priori \textit{tract feature knowledge to infer the tract's likely location in the target subject.}
}

\DIFaddend Each atlas is constructed from meticulously filtered training streamlines based on available consensus neuroanatomical definitions which are then mapped to voxel-wise orientation distributions using \gls{tod} imaging and averaged over the training population (Chapter \ref{chap:atlas}).
Upon registration with the target image, the tract's location is estimated by comparing the voxel-wise tract orientation and spatial priors with the \gls{fod} modelled from \gls{dmri} data, which can be achieved by taking the inner product of the two spherical distributions.
\DIFaddbegin \DIFadd{Thus although streamline tractography plays a role in the one-time creation of the atlases, it is not employed in subsequent application to subject data.
}\DIFaddend If large tumour deformation effects are present in the image, then the atlas is adjusted accordingly with an exponential radial deformation model (Chapter \ref{chap:reg}) prior to computing the tract map.
Through detailed evaluation against benchmark methods presented in Chapter \ref{chap:eval}, tractfinder has been shown to produce consistent and accurate segmentations at a standard comparable with streamline tractography and deep learning, issues with variable tract definitions and reference data quality notwithstanding.

\DIFaddbegin \item[\DIFadd{--}]\DIFadd{\textit{Applicability and robustness to clinical data and specifically brain tumour patient data featuring large spatial distortions is paramount.}
}

\DIFadd{The addition of tumour deformation modelling introduces explicit patient-specific lesion awareness unique to tractfinder among state-of-the-art }\gls{wm} \DIFadd{tract reconstruction techniques.}\autocite{Coenen2005a}
\DIFadd{Even so, challenges remain in the translation to real-world clinical application.
Given that tumour infiltration and oedema affect diffusion measurements and thus all downstream processing and modelling, we cannot claim that these effects do not impact tract mapping using tractfinder.
In addition, while steroid medications, commonly administered to control oedema, are known to influence $T_2$-w and }\gls{dmri} \DIFadd{signals}\autocite{Steens2005,Moll2020}\DIFadd{, little is understood about their effects on }\gls{dmri}\DIFadd{-based }\gls{wm} \DIFadd{connectivity imaging or tractography}\autocite{Coenen2005a}\DIFadd{.
Indeed, it has been hypothesised that by decreasing peri-tumoural oedema and thereby increasing }\gls{fa} \DIFadd{in affected areas, steroids may serve to improve fibre imaging quality}\autocite{Coenen2005a}\DIFadd{, though without further investigation the influence of steroid treatment on }\gls{dmri} \DIFadd{parameters should be treated as unknown and possibly significant.
}

\DIFadd{However, a voxel-wise approach is not susceptible to a compounding of errors in the same way streamline tractography is, where oedema in one part of the tract can derail tracking for the entire bundle, including parts not directly affected by infiltration.
Where disturbed diffusion does result in lower tractfinder values, they can be interpreted in the context of other imaging and even provide clinically useful information about }\gls{wm} \DIFadd{integrity.
By contrast, we have seen examples of the deep learning method TractSeg failing to fully recognise tracts disturbed by tumours (Sections \ref{sec:quant}, \ref{sec:case}), displaying a lack of explainability that is unacceptable for effective clinical translation.
In this way, the requirement of a clinically applicable and robust method is achieved, although this has yet to be rigorously verified in a prospective study (see Section \ref{sec:future}).
}

\item[\DIFadd{--}]\DIFadd{\textit{A pipeline compatible with the specific constraints of the intraoperative environment, including an acquisition--to--result timescale of under ten minutes, and minimal to no reliance on user input.}
}

\DIFaddend Several design choices in the tractfinder pipeline are supported by the objective to keep point-of-application processing time and user-interaction to a minimum.
The anatomical priors, which would for tractography either be drawn by hand or automatically via cortical parcellation or deformable registration, are provided by the tract atlases which also account for a degree of inter-subject variability.
As we saw in Chapter \ref{chap:reg}, only linear registration is required to align atlas and subject data, which is faster and more robust than the non-linear or deformable registration that would be necessary for accurate segmentation using registration alone without comparison with subject diffusion data.
It is also applicable to challenging clinical data without need for manually adjusting registration parameters, where non-linear algorithms can fail to reach stable convergence.
Using the \gls{sh} basis for representing orientation distribution data allows for efficient inner product computation, and could flexibly support alternative comparison metrics.
Overall, the full pipeline can be run in under five minutes for a single tract including minimal preprocessing, with the only potential need for user input being the choice of tumour deformation model parameter $\lambda$, if the default adaptive value produces suboptimal results.

\DIFdelbegin \DIFdel{Given that tumour infiltration and oedema affect diffusion measurements and thus all downstream processing and modelling, we cannot claim that these effects do not impact tract mapping using tractfinder.
However, a voxel-wise approach is not susceptible to a compounding of errors in the same way streamline tractography is, where oedema in one part of the tract can derail tracking for the entire bundle, including parts not directly affected by infiltration.
Where disturbed diffusion does result in lower tractfinder values, they can be interpreted in the context of other imaging and even provide clinically useful information about }%DIFDELCMD < \gls{wm} %%%
\DIFdel{integrity.
By contrast, we have seen examples of the deep learning method TractSeg failing to fully recognise tracts disturbed by tumours (Sections \ref{sec:quant}, \ref{sec:case}), displaying a lack of explainability that is unacceptable for effective clinical translation.
In this way, the requirement of a clinically applicable and robust method is achieved, although this has yet to be rigorously verified in a prospective study (see Section \ref{sec:future}).
}\DIFdelend \DIFaddbegin \item[\DIFadd{--}]\DIFadd{\textit{The training data requirements should be kept to a minimum, to allow flexibility and adaptability to new tracts or modified tract definitions in an environment where obtaining and curating high quality reference data is restricted.}
}\DIFaddend 

\DIFdelbegin \DIFdel{A final stipulation was to keep reference data requirements for atlas creation to a minimum, to accommodate evolving neuroanatomical definitions and the difficulty in producing high quality reconstructions.
}\DIFdelend With only 16 training subjects used for the original atlases, and subsequent analysis presented in Section \ref{sec:ntrain} indicating that as few as ten may be sufficient, tractfinder has a significant advantage over more data-intensive statistical and deep learning alternatives.
\DIFaddbegin \DIFadd{This minimal reference data dependence should enable tractfinder to accommodate evolving neuroanatomical definitions and the difficulty in producing high quality reconstructions.
}\DIFaddend 

The work described in this thesis came about during a time when increasingly sizeable contingents of the research and medical communities are focussing minds on the potential of machine learning techniques to disrupt hitherto intractable problems.
In the \gls{wm} imaging space, machine learning is gaining traction both for direct segmentation methods and as a means for finally overcoming the fundamental roadblocks in streamline tractography which have entangled researchers for years.
Against this backdrop of excitement for new algorithmic and big-data possibilities, proposals for a new atlas-based technique have been met with some skepticism.
Nevertheless, just as streamline tractography has long delivered astonishing benefits while simultaneously remaining unable to expel its sensitivity-specificity trade-off gremlins, so too are the difficulties of bringing deep learning solutions to real clinical translation beginning to show.
Fulfilling the need for large volumes of accurately annotated data may be easy where the resources for producing said data are freely available, or at least justified if the application is a well-defined and static problem.
Tract segmentation is neither of those things:
Producing the ground truth reference annotated data is burdensome, and the likelihood that efforts may need duplicating as our understanding of \gls{wm} anatomy evolves is high.
We have seen this in Section \ref{sec:quant} with the \textit{TractoInferno} dataset, which represents a substantial contribution to the machine learning tractography research community, but which nonetheless contains a number of poor-quality samples undetected by rigorous quality control.
Work on single-shot and transfer learning as discussed in Section \ref{sec:ntrain} has underscored the need for methods which can easily be retrained or extended to support new tracts as and when they become relevant.
In addition, a key concern in the translation of complex ``black box'' models is that clinical decision making must remain traceable and transparent, including when aided by computers.
In this light, a flexible atlas, which can be re-trained if needed using only a handful of exemplar datasets and which leads to intuitive and interpretable results can play a unique role alongside machine learning.

\DIFaddbegin \end{itemize}

\DIFaddend \section{Limitations and future directions}\label{sec:future}

The reality of achieving reliable \gls{wm} neuronavigation with intraoperative \gls{dmri} is not simply a matter of replacing tractography with a ``better'' automated tract segmentation pipeline.
As we have seen, the challenges of brain shift and tumour mass effect are substantial and will unlikely be solved with a single one-size-fits-all approach, given the degree of patient heterogeneity.

Exponential radial tumour deformation modelling can go a long way towards extending the applicability of atlas-based tract mapping to cases involving mass effect, however, the results must be inspected carefully as the effects of many tumours are not well captured by such a simplistic model.
It works particularly well for encapsulated tumours and for tumours with a relatively simple shape and location:
Lesions growing in the cortex or subcortical \gls{wm} tend to displace tissue around them in a fairly predictable way which is captured in a radial model.
By contrast, tumours of the diencephalon, midbrain, or hindbrain can produce deformations in non-radial directions, owing to the complex arrangement of surrounding structures and their biomechanical relationships to one another.
In addition, a notable shortcoming of the radial deformation model is the lack of awareness of different brain tissues' elasticities, especially in the ventricles.
The ability of the fluid filled ventricles to absorb a sizeable amount of displacement force and prevent the mass effect from propagating further through the brain than it otherwise might in a medium of uniform elasticity, is what frequently causes atlas registration inaccuracies in tumour cases.
This shortcoming should be the first to be addressed in any future modification and improvements to the tumour deformation model.
A further significant simplification is the strict non-infiltration assumption, in which all healthy tissue is fully displaced to beyond the tumour boundary.
Indeed this assumption is necessary to ensure the formula for the displacement factor $k$ remains well-behaved and invertible.
It may be argued that explicitly modelling tumour infiltration is of lower priority, as tumours infiltrating eloquent tracts are not generally candidates for total resection, as we saw in the case of patient 8 in Section \ref{sec:case}, where resection of the infiltrating tumour portion was abandoned to protect the \gls{cst}.
We also saw that tumour deformation modelling can still be effective in such cases, if only the solid component is segmented, or a scale factor $s<1$ is employed to reduce the effective tumour radius.
In this way, \gls{dmri}-based \gls{wm} mapping could play a role together with direct stimulation functional monitoring in subtotal resections and biopsies in high risk locations close to critical \gls{wm}.

\Gls{imri} is a relatively new technique, and as additional time spent scanning under general anaesthetic and with an open craniotomy carries potential risks for the patient, acquiring supplementary sequences with no clear or confirmed clinical benefit may be unethical.
Particularly with regards to intraoperative \gls{dmri}, a standard of care or guidelines for its use based on large-cohort trials has yet to be established.
As a consequence, while the new \gls{imri} system at \gls{gosh} has been used extensively since installation, the acquisition of diffusion sequences has remained infrequent, leading to a lack of available modern data with which to rigorously validate tractfinder.
Two of the available \gls{gosh} datasets were discussed in detail in Section \ref{sec:case}, while two more where included in the quantitative analysis of Chapter \ref{chap:eval}, providing illustrative insights into the use of tractfinder in real-world scenarios.
There are also outstanding challenges to obtaining consistently high quality diffusion images intraoperatively within a short-enough scan time, as discussed in reference to \gls{epi} artefacts and accelerated imaging in Section \ref{sec:technical}.
A further aspect of clinical uptake which was not addressed in this thesis is acceptance of a volumetric, voxel-based intensity map where radiologists and neurosurgeons may prefer visualising tracts in three dimensions as streamline bundles.
Addressing this potential barrier to translation, for example through a combination of the two techniques or improved volumetric visualisation strategies, should be considered as part of future investigations.

\DIFdelbegin \DIFdel{These limitations }\DIFdelend \DIFaddbegin \DIFadd{Finally, as discussed in Section \ref{sec:quant}, future studies should include discussions on how tractfinder's performance can be evaluated in a clinically meaningful way.
Different validation metrics may convey information of varying interpretability and utility to clinical decision making, while appropriate thresholds of acceptability are to be determined under guidance from neuroradiologists and surgeons.
Future evaluations should ideally additionally include direct comparison with intraoperative }\gls{des} \DIFadd{of subcortical tracts, which is currently the gold standard for identifying eloquent }\gls{wm} \DIFadd{intraoperatively.
}

\DIFadd{These limitations and outstanding questions }\DIFaddend are to be addressed in an upcoming follow-on prospective study, funded by a grant from Children with Cancer UK (Ref: CwC2022\textbackslash 100006), which will evaluate tractfinder against conventional tractography in a series of children undergoing brain tumour surgery with \gls{imri} at \gls{gosh} and assess its clinical applicability in a range of tumour histological types and locations.
It is hoped that the methodologies and technical considerations presented in this work can contribute to the wider exploitation and adoption of advanced \gls{dmri}-based \gls{wm} imaging in neurosurgical practice, bringing to bear the full potential of modern technological and neuroscientific developments to the benefit of patients.

\chapter*{List of publications and outputs}\addcontentsline{toc}{chapter}{List of publications and outputs}

\subparagraph*{Peer-reviewed journal articles}
\begin{itemize}
  \item[] \fullcite{Young2022}
  \item[] \fullcite{Young2024}
\end{itemize}
\subparagraph*{Articles in production}
\begin{itemize}
  \item[] \fullcite{Aylmore}
\end{itemize}
\subparagraph*{International conferences, accepted abstracts}
\begin{itemize}
  \item[] \fullcite{Young2022b}
  \item[] \fullcite{Young2022a}
  \item[] \fullcite{Young2023}
\end{itemize}

\subparagraph*{Software and data}
\begin{itemize}
  \item[] Tractfinder external MRtrix3 module, available at: \url{github.com/fionaEyoung/tractfinder}
  \item[] Tract orientation atlases and training streamlines, available at: \url{https://doi.org/10.5281/zenodo.10149873}
\end{itemize}

\noindent A follow-on prospective two-year study applying the techniques presented in this thesis has been granted funding by Children with Cancer UK (Grant Ref: CwC2022\textbackslash100006, Title:  A tailored image guidance approach for children undergoing surgery for brain tumours).
\clearpage{}

\clearpage{}\phantomsection
\appendix
\addtocontents{toc}{\protect\setcounter{tocdepth}{0}}

\chapter{Tractography parameters and ROI protocols}
\label{app:rois}

Default parameters as documented for the \verb|tckgen| command of MRtrix3 (release version 3.0.3, available at \url{https://mrtrix.readthedocs.io/en/3.0.3/reference/commands/tckgen.html}) (including \verb|-select 5000 -algorithm iFOD2|) were used for all tractography:

\begin{center}
\begin{tabular}{ l l }\toprule
  Parameter & Value \\
 \midrule
 Algorithm      &   iFOD2\autocite{Tournier2010} \\
 Number of streamlines selected &   5000 \\
 Maximum angle  &   45\degree  \\
 Step size & $0.5 \, \mathsf{x}$ voxel size \\
 \glsentryshort{fod} amplitude threshold & 0.1 \\ \bottomrule
\end{tabular}
\end{center}


In addition, the parameter \verb|-seed_unidirectional| was included for \gls{or} reconstructions, to ensure streamlines are propagated from a single direction out of the \gls{lgn}.


\section{ROI definitions}
\label{sec:rois}

The following ROI strategies were used for atlas constructions and subsequent validation tractography (differences between the two specified where applicable).
Visualisations of each ROI are shown on MNI152 template in Figures \ref{fig:rois.af}, \ref{fig:rois.cst} and \ref{fig:rois.or}.

\subsection{Arcuate fasciculus}

\begin{description}
  \item[Seed] White matter medial of angular gyrus, visible on coronal views of colour \gls{fa} maps as a ``green triangle'', drawn on the coronal plane.
  Level of coronal plane selected from sagittal view by locating the central sulcus (Fig. \ref{fig:rois.af}, arrow).
  \item[Include] Descending section of the arcuate fasciculus, drawn on the axial plane
  \item[Exclude] Exclusion ROIs targeting: midline, superior fronto-occipital fasciculus, ipsilateral cerebral penduncles, sagittal stratum, corona radiata and external capsules.
\end{description}

The following publications were reviewed to inform the above ROI strategy: \textcite{Brown2014a},\textcite{Catani2002},\textcite{Catani2005},\textcite{Chen2015c},
\textcite{Eluvathingal2007},\textcite{Kamali2014},\textcite{Martino2013a},\textcite{Nucifora2005},
\textcite{Parker2005},\textcite{Bain2019},\textcite{Talozzi2018}

\begin{figure*}[h]
  \centering
    \includegraphics[width=\textwidth]{appendix/AF_include.png}
    \includegraphics[width=\textwidth]{appendix/AF_exclude.png}
  \caption[Arcuate fasciculus tractography ROIs]{Seed (yellow), inclusion (green) and exclusion (red) \glspl{roi} for the arcuate fasciculus. Arrow indicates central sulcus, landmark for seed ROI.}
  \label{fig:rois.af}
\end{figure*}

\subsection{Corticospinal tract}

Corticospinal tract tracography strategy differed between the atlas creation and general tractography applied to new subjects.

\begin{description}
  \item[Seed (atlas)] For the orientation atlas, Freesurfer cortical parcellations were used to obtain more complete coverage of the motor cortex via the following process:
  \begin{enumerate}
    \item Seed in precentral gyrus and output successful seed location
    \item Generate binary mask from successful seed locations, subtract from precentral gyrus mask to create second seed mask
    \item Re-run tractography with second seed mask to cover rest of precentral gyrus
  \end{enumerate}
  \item[Seed (general)] Posterior limb of the internal capsule, drawn on three consecutive axial slices
  \item[Include] Posterior limb of the internal capsule (if not used for seed), cerebral penduncles, CST in mid-pons
  \item[Exclude] Cerebellar peduncles (drawn on coronal slice), medial lemniscus (drawn on axial slice), midline, superior fronto-occipital fasciculus,
\end{description}

The following publications were reviewed to inform the above ROI strategy: \textcite{Ciccarelli2006},\textcite{Han2010},\textcite{Hattingen2009a},
\textcite{Niu2016},\textcite{Radmanesh2015},\textcite{Reich2006},
\textcite{Rosenstock2017},\textcite{Szmuda2021},\textcite{Vargas2013}

\begin{figure*}[h]
  \centering
    \includegraphics[width=\textwidth]{appendix/CST_include.png}
    \includegraphics[width=\textwidth]{appendix/CST_exclude.png}
  \caption[Corticospinal tract tractography ROIs]{Seed (yellow), inclusion (green) and exclusion (red) \glspl{roi} for the corticospinal tract}
  \label{fig:rois.cst}
\end{figure*}

\subsection{Inferior fronto-occipital fasciculus}

\begin{description}
  \item[Seed] Temporal stem, between anterior tip of Meyer's loop and descending portion of the uncinate fasciculus
  \item[Include (atlas)] Posterior: inferior, middle, and superior occipital gyri and middle and superior occipital sulci (Freesurfer (v4.5) Destrieux atlas\autocite{Destrieux2010} (2009 version) parcellation labels 1\{1,2\}1\{02,19,20,58,59\}).
  Anterior: frontal pole, middle and inferior frontal gyri and sulci, orbital gyrus and sulci (Freesurfer labels 1\{1,2\}1\{01,05,15,54,12,13,14,53,63,24,65\})
  \item[Include (general)] Frontal lobe coronal slice, anterior to genu of the corpus callosum
  \item[Exclude] Coronal slice on frontal lobe at the level of the central sulcus, coronal slice on tip of anterior temporal lobe
\end{description}

The following publications reviewed to inform the above ROI strategy: \textcite{Martino2010},\textcite{Sarubbo2013},\textcite{Hau2016},
\textcite{Catani2008},\textcite{Wakana2007},\textcite{Wu2016}

\begin{figure*}[h]
  \centering
    \includegraphics[width=\textwidth]{appendix/IFO_include.png}
    \includegraphics[width=\textwidth]{appendix/IFO_exclude.png}
  \caption[Inferior fronto-occipital fasciculus tractography ROIs]{Seed (yellow), inclusion (green) and exclusion (red) \glspl{roi} for the inferior fronto-occipital fasciculus}
  \label{fig:rois.ifo}
\end{figure*}

\subsection{Optic radiation}

\begin{description}
  \item[Seed] Lateral geniculate nucleus (LGN; drawn on axial planes)
  \item[Include] Sagittal stratum (drawn on coronal plane)
  \item[Exclude] Coronal slice anterior of and axial slice inferior of most anterior point of lateral ventricles, axial slice at level of superior reach lateral ventricles, splenium of corpus callosum, fornix
\end{description}

The following publications reviewed to inform the above ROI strategy:
\textcite{Yogarajah2009},\textcite{Hofer2010},\textcite{Dayan2015}

\begin{figure*}[h]
  \centering
    \includegraphics[width=\textwidth]{appendix/OR_include.png}
    \includegraphics[width=\textwidth]{appendix/OR_exclude.png}
  \caption[Optic radiation tractography ROIs]{Seed (yellow), inclusion (green) and exclusion (red) regions of interest for the optic radiation}
  \label{fig:rois.or}
\end{figure*}
\clearpage{}

\addcontentsline{toc}{chapter}{Bibliography}
{\setstretch{1.0}
\printbibliography
}
\chapter*{Colophon}

This document was typeset in Libertinus Serif (a fork of Linux Libertine), {\sffamily Source Sans Pro} and {\ttfamily Courier} typefaces, with \LaTeX\ and Bib\TeX, and using the UCL \LaTeX\ \href{https://github.com/UCL/ucl-latex-thesis-templates}{thesis template} created by Ian Kirker.
Original vector graphics were produced in \href{https://inkscape.org/}{Inkscape}, Keynote, in Python using the \href{https://matplotlib.org/}{Matplotlib} package, and in MATLAB, and typeset using the \verb|svg| package.
Original raster graphics were produced with Inkscape and \verb|mrview| from the \href{https://www.mrtrix.org/}{MRtrix3} software package, and edited in \href{https://www.gimp.org/}{GIMP} 2.1 and with the \href{https://imagemagick.org/index.php}{ImageMagick} software suite.

\end{document}
