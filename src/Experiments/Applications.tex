\section{Applications}

\note{qualitative demos, clinical cases with intraop deformation }

\note{More qualitative validation in as many sample datasets as possible, including intraoperative data, tumour deformation and debulking cases. Show specific cases where differs from e.g. tractography or TractSeg in specific areas like peritumoural oedema. Also analyse failure cases.}

\note{Would be nice to also have e.g. epilepsy cases in here?}

\subsection{Tumour deformations}

\note{dump from IJCARS manuscript}

The target application is in intraoperative imaging.
The main difference therein is the need to account for brain shift, which is unpredictable: differing effects stem from drainage of fluid, pressure changes, tumour debulking and gravitational sag.
Nevertheless, we aim to achieve intraoperative tract segmentation while avoiding the need to perform additional tumour and / or resection cavity segmentation intraoperatively.

As the atlas is designed to be spatially inclusive, with the inner product acting to correct small spatial inaccuracies, it is possible in some cases where brain shift is minimal to reuse the properative tumour deformation field.
In cases of significant tumour debulking, the deformation field can be recomputed from the preoperative tumour segmentation by adjusting the value of $s$ to simulate a reduction in tumour volume.

This scenario is demonstrated in Figure \ref{fig:shrink}, showing the resection of a large temporal epidermoid cyst. There is significant reduction in cyst volume and the adjacent corticospinal tract has shifted accordingly, however by reusing the preoperative lesion segmentation and setting $s=0.8$, the resulting deformation field is able to capture the rough location of the shifted tract. By only adjusting the value of $s$ and reusing preoperatively computed values of $D_t$ and $D_b$, we can avoid time and resource-intensive intraoperative lesion segmentation, brain shift modelling or non-linear registration.

\begin{figure*}[h!]
  \centering
  \includegraphics[width=0.8\textwidth]{chapter_4/figure_4.png}
  \caption{Example results in intraoperative image using scaled preoperative tumour segmentation. Blue outline: Tumour segmentation. Green outline: effective tumour boundary with $s=0.8$ used for intraoperative segmentation}
  \label{fig:shrink}
\end{figure*}

\begin{figure*}[h!]
  \centering
  \includegraphics[width=\textwidth]{chapter_4/figure_5_revised.png}
  \caption{Sample results in 4 different clinical subjects. First column: linearly registered tract atlas (spatial component only). Second column: atlas after tumour deformation. Third column: Final tract map. Fourth column: Track density image from streamline tractography, where intensity corresponds to streamline count per $(2.5mm)^3$ voxel (thresholded at 10 streamlines).}
  \label{fig:res}
\end{figure*}

In the \gls{btc} dataset, tumour deformation modelling successfully improved tract and subject alignment and consequently tract mapping in all preoperative cases where it was necessary.
Examples in \note{figure ??} demonstrate how tractfinder including tumour deformation produces tract margins closer to those indicated by tractography, in contrast to the deep learning method TractSeg which appears to still rely on a learned healthy anatomical configuration without generalising to displaced tracts.

\section{iMRI case studies}

From \note{early 2022 to mid 2023}, 8 \note{verify!} tumour resections where carried out at Great Ormond Street Hospital for Children with intraoperative MRI guidance including multi-directional diffusion weighted imaging.
Throughout this period, neurosurgical and radiological staff were continually learning how best to integrate the iMRI facility into their practice.

The subjects included two posterior fossa tumour, two midline (thalamic) glioma and four hemispheric tumour patients.
Of these, thalamic tumours are the indications with potentially the most to gain from advanced intraoperative guidance, given their complex functional environment and difficult access.
Indeed, such tumours have widely been considered inoperable and have only recently seen surgical treatment thanks in part to changing attitudes\autocite{Souweidane1996,Puget2007} advances in neuronavigation.\autocite{Sunderland2021}
Paediatric patients in particular\autocite{Ferroli2023} experience better overall survival rates when thalamic tumours are more radically resected compared to subtotal resection or biopsy.\autocite{Cinalli2018a}
The Alder Hey Children's Hospital in Liverpool, UK experienced an increase in substantial resection \note{define} rates from 37\% to 94\% with the introduction of intraoperative MRI navigation, without any associated increase in postoperative morbidity.\autocite{Sunderland2021}
Similarly, a review of 38 thalamic tumour patients (paediatric and adult) treated at the Chinese PLA General Hospital in Beijing, China found that the use of iMRI increased GTR rates from 42\% to 68\%.\autocite{Zheng2016}
Motor deficits are the most common functional symptoms of thalamic tumours,\autocite{Puget2007, Zheng2016, Palmisciano2021} and preservation of the posterior limb of the internal capsule, containing the corticospinal tract, is a major concern when resecting them.
\note{to cite: Grewal2019, Wong2016, Kis2014, Dorfer2021, Steinbok2016, Celtikci2017}

\note{add some sort of disclaimer about how this is all hearsay based on MDT / scanner room notes?}

The first of the two patients harboured a low-grade pilocytic astrocytoma in the left thalamus and presented with progressive right side weakness.
The unusual radiological presentation showed part of the CST travelling through the tumour, while the rest of the tract was displaced.
Due to this complicated involvement of the internal capsule a debulking surgery under iMRI guidance was indicated.
Confirming the location of the CST, divided as it apparently was into an intratumoural and a displaced portion, was particularly difficult in this case.
A partial resection was completed, and as the intraoperative frozen section \note{??} was reported as appearing high grade, the decision was made not to pursue extensive resection, given the risk of functional injury and the unlikelihood of achieving curative resection for a high grade lesion \note{????}.
There was concern of an area of infarction involving the CST, however the patient, after experiencing worsened motor deficit immediately postoperatively, approved to preoperative levels of the following days.
\note{complete with imaging analysis for patient 5: imaging features preop and intraop, and tractography / tractfinder reconstructions}

The second thalamic tumour case study involves a \note{X} year old with a diffuse midline astrocytoma (WHO grade IV) also in the left thalamus.
The diffuse and infiltrative nature of this tumour made identifying the posterior capsule, thalamus and tumour margins exceedingly difficult.
Intraoperative DTI was requested by the radiologist after extensive discussion of the already acquired conventional contrast scans, from which the extent of PLIC infiltration was indiscernible.
Colour FA showed a largely intact and slightly displaced internal capsule with extensive tumour infiltration, so the decision was made not to attempt further resection of that part of the tumour.
In the end, roughly 50\% of the lesion was resected, predominantly in the ventral and medial parts.
Postoperatively the patient's right side weakness was slightly improved and the patient received adjuvant radiotherapy \note{did they? only have this down as planned} after confirmation of high-grade histology.
\note{complete with imaging analysis for patient 8: imaging features preop and intraop, and tractography / tractfinder reconstructions}
