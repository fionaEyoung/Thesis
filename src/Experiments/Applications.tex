\section{Intraoperative brain shift}
\label{sec:imri}

When considering the application of tractfinder for intraoperative imaging, we need to take brain shift into account, which is unpredictable: differing effects stem from drainage of fluid, pressure changes, tumour debulking and gravitational sag.
Nevertheless, we aim to achieve intraoperative tract segmentation while avoiding the need to perform additional tumour and / or resection cavity segmentation intraoperatively.

As the atlas is designed to be spatially inclusive, with the inner product acting to correct small spatial inaccuracies, it is possible in some cases where brain shift is minimal to reuse the properative tumour deformation field.
In cases of significant tumour debulking, the deformation field can be recomputed from the preoperative tumour segmentation by adjusting the value of $s$ in (\ref{eq:forwardP}) to simulate a reduction in tumour volume.

This scenario is demonstrated in Figure \ref{fig:shrink}, showing the resection of a large temporal epidermoid cyst.
There is significant reduction in cyst volume and the adjacent corticospinal tract has shifted accordingly, however by reusing the preoperative lesion segmentation and setting $s=0.8$, the resulting deformation field is able to capture the rough location of the shifted tract.
By only adjusting the value of $s$ and reusing preoperatively computed values of $D_t$ and $D_b$, we can avoid time and resource-intensive intraoperative lesion segmentation, brain shift modelling, or non-linear registration.

\begin{figure*}[h!]
  \centering
  \includegraphics[width=0.8\textwidth]{chapter_4/figure_4.png}
  \caption{Example results in intraoperative image using scaled preoperative tumour segmentation. Blue outline: Tumour segmentation. Green outline: effective tumour boundary with $s=0.8$ used for intraoperative segmentation}
  \label{fig:shrink}
\end{figure*}

\begin{figure*}[h!]
  \centering
  \includegraphics[width=\textwidth]{chapter_4/figure_5_revised.png}
  \caption{Sample results in 4 different clinical subjects. First column: linearly registered tract atlas (spatial component only). Second column: atlas after tumour deformation. Third column: Final tract map. Fourth column: Track density image from streamline tractography, where intensity corresponds to streamline count per $(2.5mm)^3$ voxel (thresholded at 10 streamlines).}
  \label{fig:res}
\end{figure*}

\note{add section about oedema: nhnn subject 5 tractography vs tractseg vs tractfinder}

\clearpage
\section{iMRI case studies}\label{sec:case}

From \note{early 2022 to late 2023}, 9 \note{verify!} tumour resections where carried out at Great Ormond Street Hospital for Children with intraoperative MRI guidance including multi-directional diffusion weighted imaging.
Throughout this period, neurosurgical and radiological staff were continually learning how best to integrate the iMRI facility into their practice.

The subjects included two posterior fossa tumour, two midline (thalamic) glioma and \note{five} hemispheric tumour patients.
Of these, thalamic tumours are the indications with potentially the most to gain from advanced intraoperative guidance, given their complex functional environment and difficult access.
Indeed, such tumours have widely been considered inoperable and have only recently seen surgical treatment thanks in part to changing attitudes\autocite{Souweidane1996,Puget2007} advances in neuronavigation.\autocite{Sunderland2021}
Paediatric patients in particular\autocite{Ferroli2023} experience better overall survival rates when thalamic tumours are more radically resected compared to subtotal resection or biopsy.\autocite{Cinalli2018a}
The Alder Hey Children's Hospital in Liverpool, UK experienced an increase in substantial resection \note{define} rates from 37\% to 94\% with the introduction of intraoperative MRI navigation, without any associated increase in postoperative morbidity.\autocite{Sunderland2021}
Similarly, a review of 38 thalamic tumour patients (paediatric and adult) treated at the Chinese PLA General Hospital in Beijing, China found that the use of iMRI increased GTR rates from 42\% to 68\%.\autocite{Zheng2016}
Motor deficits are the most common functional symptoms of thalamic tumours,\autocite{Puget2007, Zheng2016, Palmisciano2021} and preservation of the posterior limb of the internal capsule, containing the corticospinal tract, is a major concern when resecting them.
\note{to cite: Grewal2019, Wong2016, Kis2014, Dorfer2021, Steinbok2016, Celtikci2017}

The following will look at two of the GOSH diffusion iMRI cases in detail, considering the imaging features, tumours and peri-operative clinical presentation and neurology.
Hopefully this will provide a unique an balance perspective into the challenges and potential of intraoperative \gls{dmri} for such cases.

\subsection{Patient 5: Low grade pilocytic astrocytoma}

\note{add some sort of disclaimer about how this is all hearsay based on MDT / scanner room notes?}

The first of the two patients harboured a low-grade pilocytic astrocytoma in the left thalamus and presented at 23 months old with progressive right side weakness.
The unusual radiological presentation showed a sliver of tissue within the tumour, presumed to be a part of the CST, while the rest of the tract was displaced (Fig. \ref{fig:5p}).

\begin{figure}
  \centering
  \includegraphics[width=0.8\textwidth]{case_studies/5_preop.png}
  \caption{Preoperative radiological presentation of patient 5, with a low-grade astrocytoma of the left thalamus. Top: T1, bottom: DTI FA. Arrow heads indicate the CST, partially travelling through the tumour.}
  \label{fig:5p}
\end{figure}

Confirming the location of the CST, divided as it appeared to be into an intratumoural and a displaced portion, was particularly difficult in this case.
A \gls{dmri} sequence with colour FA visualisation was able to confirm that the strip of tissue inside the tumour indeed constituted part of the CST.
Streamline tractography also reconstructed the CST as both within and displaced posteriorly around the tumour.
The part of the pathway involved is at the cerebral peduncle level, which in healthy cases is a relatively narrow bundle.
Consequently, the tractfinder atlas was unable to account for both the tumour-engulfed section, which remained spatially relatively unmoved, and the posteriorly displaced portion simultaneously.
Basic tract mapping reconstructed only the first part, while additional tumour deformation modelling enabled detection of the second part (Fig. \ref{fig:5p_cst}).
Of course, such a hacked-together reconstruction would be entirely impractical and difficult to interpret, compared to tractography which is better at exploring the available pathways regardless of anatomical prior expectations.
As such, this is a case for which tractfinder is entirely unsuited.

\begin{figure}
  \centering
  \includegraphics[width=0.8\textwidth]{case_studies/5_preop_tf.png}
  \caption{Reconstruction of the corticospinal tract on preop imaging. Top: track density imaging, bottom: tractfinder (combined with and without deformation modelling), with tractography outlined in white for comparison.\note{redo with only unilateral?}}
  \label{fig:5p_cst}
\end{figure}

Due to this complicated involvement of the internal capsule a debulking resection under iMRI guidance was indicated to relieve pressure on the \gls{cst}.
After a large part of the tumour was resected, the patient was brought through to \gls{imri} (Fig \ref{fig:5i}).
Here, again, reconstruction of the \gls{cst} proved difficult.

\begin{figure}
  \centering
  \includegraphics[width=0.6\textwidth]{case_studies/5_iop.png}
  \caption{Intraoperative T1w (top) and diffusion colour FA (bottom) for patient 5, showing left craniotomy and surgical corridor through the temporal lobe}
  \label{fig:5i}
\end{figure}

\begin{figure}
  \centering
  \includegraphics[width=0.8\textwidth]{case_studies/5_iop_tf.png}
  \caption{Reconstruction of the CST on intraoperative imaging using tractography (top) and tractfinder (bottom) with standard affine registration.}
  \label{fig:5i_cst}
\end{figure}

This time, significant brain shift away from the craniotomy prevented accurate atlas registration, again resulting in a mismatch in anatomical alignment between atlas and target image (Fig \ref{fig:5i_cst}).
Non-linear registration, with harsh penalties on strong local deformation to reduce overfitting around the resection surface, could mitigate this, but would be impractical for routine use.

A partial resection was completed, and as the intraoperative frozen section \note{??} was reported as appearing high grade, the decision was made not to pursue extensive resection, given the risk of functional injury and the unlikelihood of achieving curative resection for a high grade lesion. \note{????}
There was concern of an area of infarction involving the CST, however the patient, after experiencing worsened motor deficit immediately postoperatively, approved to preoperative levels of the following days.
The patient went on to receive adjuvant chemotherapy and showed further improvement to their hemiparesis compared to their preoperative condition.

\subsection{Patient 8: High grade glioma}

The second thalamic tumour case study involves a seven year old with a diffuse midline astrocytoma (WHO grade IV) also in the left thalamus.
A high degree of mass effect called for tumour deformation modelling, so the lesion was segmented on the preoperative structural scan (Fig. \ref{fig:8p}).

\begin{figure}
  \centering
  \includegraphics[width=0.7\textwidth]{case_studies/8_preop.png}
  \caption{Preoperative T1 weighted scan (top) of patient 8 with high grade left thalamic glioma. Bottom row shows tumour deformation modelling applied to a coregistered MNI template image.}
  \label{fig:8p}
\end{figure}

The diffuse and infiltrative nature of this tumour made identifying the posterior capsule, thalamus and tumour margins exceedingly difficult.
Intraoperative DTI was requested by the radiologist after extensive discussion of the already acquired conventional contrast scans, from which the extent of PLIC infiltration was indiscernible (Fig. \ref{fig:8i}).

\begin{figure}
  \centering
  \includegraphics[width=0.7\textwidth]{case_studies/8_iop.png}
  \caption{Intraoperative imaging showing transcollosal approach to thalamic glioma. Top: T1w, bottom: DTI colour FA. White arrowhead on DTI indicates partially infiltrated and displaced internal capsule.}
  \label{fig:8i}
\end{figure}

Colour FA showed a largely intact and slightly displaced internal capsule with extensive tumour infiltration, so the decision was made not to attempt further resection of that part of the tumour.
In this situation, identification of the internal capsule and assessment of its condition could have been significantly assisted with more advanced \gls{dmri} analysis.
The smearing of colours on the directional FA map make it difficult to discern what affect is leading to the change from the expected indigo colour of a healthy internal capsule.
What is unclear is the state of the neuronal fibres in the infiltrated portion: are they intact but surrounded with oedema/tumour tissue, or simply in a rotated orientation, or destroyed?
Visualising the fibre \glspl{odf} using \gls{csd} gets us closer to the answer (Fig. \ref{fig:8i_fod}).

\begin{figure}
  \centering
  \begin{subfigure}{0.3\textwidth}
    \centering
    \includegraphics[width=\textwidth]{case_studies/8_iop_dt.png}
    \caption{Diffusion Tensor}
    \label{fig:8i_dt}
  \end{subfigure}%
  \begin{subfigure}{0.3\textwidth}
    \centering
    \includegraphics[width=\textwidth]{case_studies/8_iop_csd.png}
    \caption{CSD FOD}
    \label{fig:8i_csd}
  \end{subfigure}%
  \caption{Partial view of partially resected tumour and internal capsule on coronal slice and two different fibre orientation models: diffusion tensor and constrained spherical deconvolution. The multi-fibre FODs make it easier to identify where the descending fibres a likely to remain intact. \note{specify CSD type}}
  \label{fig:8i_fod}
\end{figure}

Using the tumour deformation field modelled from the preoperative scan, tractfinder reconstructs the corticospinal tract in very close accordance with probabilistic streamline tractography.
Both indicate a largely intact bundle slightly laterally displaced, although neither can necessarily rule out the presence of infiltrated fibres closer to the tumour, as the expected resulting decrease in anisotropy would produce fewer streamlines through that region and reduced tractfinder probability.
In the end, roughly 50\% of the lesion was resected, predominantly in the ventral and medial parts.
Postoperatively the patient's right side weakness was slightly improved and the patient received adjuvant radiotherapy \note{did they? only have this down as planned} after confirmation of high-grade histology.
Over the following weeks the patient's hemiparesis persisted, consistent with the infiltrative nature of the tumour.
\note{show tractseg here!}

\begin{figure}
  \includegraphics[width=\textwidth]{case_studies/8_iop_tf.png}
  \caption{Coronal slices in 5mm increments from intraoperative imaging for patient 8. Tractfinder CST map is overlaid along with tractography track density image outline in white for comparison (\note{100 streamline isosurface})}
  \label{fig:8i_tf}
\end{figure}
