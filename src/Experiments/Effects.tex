\chapter{Practical application}
\label{chap:applications}

% \note{\normalsize Chapter overview: More qualitative validation in as many sample datasets as possible, including intraoperative data, tumour deformation and debulking cases. Show specific cases where differs from e.g. tractography or TractSeg in specific areas like peritumoural oedema. Also analyse failure cases.}
% \note{Would be nice to also have e.g. epilepsy cases in here?}
In previous chapters, we have, in large part, focussed on applying and evaluating tractfinder in relatively idealised environments and datasets.
Much of the quantitative evaluation in Chapter \ref{chap:eval} was concerted with research-quality acquisitions of healthy subjects, while even the \gls{btc} data are multi-shell \gls{hardi} scans.
We will next consider some of the realities of real-world clinical data and environments, discussing technical limitations and illustrating the technique in a selection of case studies.

\section{Technical considerations}\label{sec:technical}

The technical feasibility of tractfinder for intraoperative application depends not only on the capabilities of the method itself in terms of speed and reliability, but also on the achievable image quality.
Intraoperative scanning must occur under significant time pressure, and the concessions made to achieve this will have profound consequences for the image quality and subsequent image processing.
In general, two strategies exist for reducing \gls{dmri} scan time:
The first is to reduce the information content of the acquisition, which could mean lower spatial resolution (larger voxels), lower angular resolution (fewer diffusion-weighted directions), shorter diffusion weightings (lower $b$-values) or fewer unique $b$-values (``shells").
The second is to use accelerated imaging techniques to achieve the (notionally) same image parameters in shorter time.
These two possibilities are investigated in the following sections, beginning with an investigation into the effect of diffusion-weighting scan parameters on tract segmentations, in work first published in \textcite{Young2022a}.

\subsection{Data requirements}\label{sec:ismrmdiff}

It is important to assess the applicability of image processing techniques developed with research quality data in acquisitions more typical of a clinical setting.
Advanced \gls{dmri} processing techniques may sometimes set constraints on the input data, such as a recommended or required minimum number of diffusion weighted volumes (angular resolution), $b$-values or spatial resolution.
If it can be demonstrated that an image processing pipeline produces excellent results in a high quality research dataset, which can be interpreted with confidence, then the question follows:
Were one to acquire a lower quality dataset of the same subject, and perform the same data processing, how comparable would the resulting segmentation be to that of the high quality scan?
This is the question explored in the following study, which makes use of the high quality \gls{hcp} \gls{dmri} data (Section \ref{sec:data}) to determine the minimum data requirements to obtain successful segmentation, comparing different numbers of direction samples, $b$-values and post-processing strategies, as well as the affects of decreasing data quality on segmentation stability for three different methods (tractfinder, tracotgraphy and TractSeg; see Section \ref{sec:methods} for further details).

% \paragraph*{Data and methods}

49 preprocessed \gls{dmri} scans from the WU-Minn \gls{hcp} Young Adult S1200 data release were spatially resampled to 2.5mm isotropic voxel size (from the original of 1mm isotropic), a resolution far more typical in both clinical and research settings.
Then, for each subject, the following five subsampled diffusion schemes were extracted from the full acquisition (Tab. \ref{tab:subschemes}):
60 directions each at $b=1000s/mm^2$ and $b=2000s/mm^2$ (120 directions total, ``DWI-1”), 30 directions each at $b=1000s/mm^2$ and $b=2000s/mm^2$ (60 directions total, ``DWI-2”), 60 directions at $b=1000s/mm^2$ (``DWI-3”), 30 directions at $b=1000s/mm^2$ (``DWI-4”) and 12 directions at $b=1000s/mm^2$ (``DWI-5”).

\begin{table}
  \centering
  \begin{tabular}{c c c} \toprule
    Identifier & $b = 1000$ & $b=2000$ \\
    \midrule
    DWI-1 & 60 & 60 \\
    DWI-2 & 30 & 30 \\
    DWI-3 & 60 &    \\
    DWI-4 & 30 &    \\
    DWI-5 & 12 &    \\ \bottomrule
  \end{tabular}
  \caption{Subsampled diffusion schemes, number of directions}
  \label{tab:subschemes}
\end{table}

Four different tract segmentations were produced in each subject using the three different approaches described in Section \ref{sec:methods}: Tractfinder, targeted ROI-based, probabilistic streamline tractography, and TractSeg (DKFZ and XTRACT models).
Segmentations were compared for the three tracts most commonly reconstructed in neurosurgical applications:
The \gls{cst}, \gls{af} and \gls{or} (this study was conducted before an atlas for the \gls{ifof} was completed).
All tract segmentation methods are predicated on \glspl{fod} modelled from diffusion data with \gls{csd}, although tractography and TractSeg could both use fibre orientations estimated via other techniques such as BedpostX\autocite{Behrens2007}.
For the two-shelled datasets, \gls{msmt} \gls{csd} can be run to separate the signal contributions from \gls{wm}, \gls{gm} and \gls{csf} and thus obtain an optimal \gls{wm} \gls{fod} image free of noisy extraneous signal in non-\gls{wm} regions.
For the single-shelled datasets, full, direct \gls{msmt} \gls{csd} with all three tissue compartments is not possible.
Instead, the following approaches were compared: 1) Modelling two tissue types with \gls{msmt}.
\gls{msmt} \gls{csd} is performed twice, once modelling \gls{wm} and \gls{gm} compartments, and once modelling \gls{wm} and \gls{csf} compartments; 2) Standard \gls{ssst} \gls{csd}.

To investigate the effects of different data quality and modelling on a given method, the \gls{dice} \autocite{Dice1945} was computed between each segmentation in datasets DWI-2--5 and the corresponding segmentation of the same method in DWI-1 using three-tissue \gls{msmt} \gls{csd} (considered the benchmark segmentation).
This ``self-similarity" approach is intended to answer the question put forward above:
How close to the results produced from an ``ideal" acquisition can be achieved from a lower quality scan.
This comparison was made for each combination of diffusion scheme and \gls{csd} approach.

% \paragraph*{Findings}

\begin{figure}
  \centering
  \includegraphics{chapter_4/self_dice.png}
  \caption{Self dice scores \note{caption, legend}}
  \label{fig:self_dice}
\end{figure}

Self-similarity results are plotted in Figure \ref{fig:self_dice}.
Each datapoint corresponds to the segmentation produced for a given \gls{dmri} direction scheme, method and pipeline \textit{compared against} the result obtained from the highest quality scheme (DWI-1).
The consistency in segmentation results differed significantly with both data quality and CSD approach.
In particular, a high sensitivity to \gls{csd} technique is apparent in the tractography and Tractfinder results, with particularly large differences between \gls{msmt} \gls{wm}+\gls{csf} / \gls{ssst} and \gls{msmt} \gls{wm}+\gls{gm}, which can be attributed to the inclusion of \gls{gm} signal in the \gls{wm} \gls{fod} reconstructions.
In those two approaches, the \gls{gm} compartment is not as strongly suppressed from the \gls{wm} reconstruction, resulting in high \gls{wm} \gls{fod} amplitudes in the cortex and subcortical \gls{gm}, leading in turn to more extensive segmentations:
In tractography via the further propagation of streamlines into \gls{gm}, and in Tractfinder by increasing the inner product values.

TractSeg, while not immune, appears less sensitive to differences in \gls{csd} reconstruction.
When considering only the best results (\gls{msmt} in DWI-2 and, in the single-shelled datasets, \gls{msmt} \gls{wm}+\gls{gm} for tractography and Tractfinder and \gls{ssst} for TractSeg, \ref{fig:self_dice}), tractography displays the greatest instability with decreasing data quality, with the similarity score between DWI-5 \gls{msmt} \gls{wm}+\gls{gm} and DWI-1 \gls{msmt} falling as low as 0.78 (subject and cerebral hemisphere mean) for the \gls{af}.
TractSeg is trained on \gls{ssst} \gls{csd} in single-shelled data as well as multi-shelled data, explaining why it does best with either \gls{ssst} or \gls{msmt} \gls{wm}+\gls{csf} in our single-shell results, and has learned to not pay too much attention to fibre orientation peak magnitudes.

The comparatively high sensitivity of probabilistic tractography to acquisition and processing pipelines is consistent with the well-established reproducibility and noise sensitivity problems associated with tractography.
Meanwhile, voxel-wise segmentation methods are not susceptible to error propagation along the tract and are more robust to lower angular resolution:
When the number of directions in a multi-shelled acquisition was halved (DWI-2), average tractfinder self-similarity was 0.98 or above, while for tractography it was substantially lower at around 0.9.
Although tractfinder with \gls{ssst} \gls{csd} exhibits lower self-similarity with three-tissue \gls{msmt} \gls{csd}, scores remained steady across all single-shelled schemes.
Overall, these results motivate the use of \gls{msmt} \gls{wm}+\gls{gm} for tractfinder in all single-shelled acquisitions, with the caveat that particular care must be taken when interpreting segmentations in \gls{wm}/\gls{csf} partial volume areas.
Understanding the behaviour of different tract segmentation techniques when applied to varying qualities of \gls{dmri} acquisition and post-processing approaches is important if segmentation methods developed in research settings are to be consistently and reproducibly applied to clinical quality acquisitions.
It is useful to know how comparable segmentation results in a single-shelled, 30 direction dataset are to those one might have obtained with a \gls{hardi} dataset and state-of-the-art post-processing.
This is particularly relevant for longitudinal studies, or when comparing different acquisitions of the same subject in a clinical context (for example, for monitoring disease progressions or post-operative changes).

\subsection{Fast imaging and artefacts}

Because high angular resolution \gls{dmri} acquisitions involve acquiring many volumes of data, they would have prohibitively long scan times without the use of \gls{epi} (see Section \ref{sec:epi}).
Despite its undeniable benefits, \gls{epi} produces problems of its own, including the geometric image distortions that commonly arise at boundaries between substances with different magnetic susceptibilities, typically at tissue/air interfaces.
For intraoperative scanning, this poses a particular problem in which precisely the site of interest, the operative field and surgical cavity, can suffer from the strongest distortions, affecting nearby structures (such as \gls{wm} tracts) that are of neuronavigational importance\autocite{Yang2022}.
In most applications, \gls{epi} distortion artefacts are tolerable as they can be fairly robustly corrected for during image preprocessing.
However, not only can the distortions around resection cavities be especially severe due to the large exposed area of brain surface (Fig. \ref{fig:epi}), neither can they be easily corrected for intraoperatively, as the currently available tools take on the order of tens of minutes to hours to run.

\begin{figure}[h!]
  \includegraphics[width=\textwidth]{chapter_6/epidist.pdf}
  \caption{Examples of strong susceptibility distortions (arrowheads) on intraoperative \gls{epi} acquisitions obtained on the Siemens MAGNETOM Vida system at \gls{gosh}}
  \label{fig:epi}
\end{figure}

Typical \gls{dmri} acquisitions are based on \gls{ssepi}, in which an entire slice's worth of data is sampled in a single excitation-readout sequence.
As an alternative to \gls{ssepi}, multi-shot sequences have been proposed.
These include interleaved and \gls{rsepi} sequences which, in general, only traverse a portion of $k$-space per echo train, increasing the bandwidth along the phase-encoding direction (and thus reducing field inhomogeneity induced phase accumulation) while increasing scan time\autocite{Wang2018}.
RESOLVE (readout segmentation of long variable echo-trains), a Siemens product, is one example of a commonly used \gls{rsepi} sequence for high quality three-scan trace weighted images with high in-plane resolution.

While RS sequences can be combined with acceleration techniques including generalised autocalibrating partially parallel acquision (GRAPPA) and simultaneous multislice (SMS), they nevertheless result in significantly longer scan times, which, in clinical applications, must be weighed against any gains in image quality and distortion reduction.
For example, in \textcite{Wang2018} authors compared interleaved and \gls{rsepi} sequences with \textit{in vivo} scan times of around ten minutes, while only acquiring four slices of 4mm thickness!
In \textcite{Elliott2020}, \gls{ssepi} was compared with a RESOLVE sequence with 15 segments, six diffusion weighted directions, and a scan time of 10:34min.
For comparison, the clinical \gls{ssepi} \gls{dti} sequence currently used intraoperatively at \gls{gosh} is a 30 direction $b=1000s/mm^2$ acquisition lasting just 4:57min with 2.3mm isotropic voxels.

We set out to explore the feasibility of setting up a \gls{rsepi} \gls{dti} sequence for intraoperative use at \gls{gosh}.
An (approximately) isotropic voxel size is generally considered optimal for \gls{dti} modelling and tractography \autocite{Vos2011, Neher2013}, although tractography with high in plane resolution and thick slices is certainly possible.
Additionally, a sufficient number of diffusion weighted directions for crossing fibre modelling (e.g \gls{csd}) would be ideal, potentially as few as 12 directions given the results from Section \ref{sec:ismrmdiff}.
In the end, we tested three different DTI sequences as well as several different three-scan trace RESOLVE sequences (not discussed here), all acquired from a healthy adult volunteer on the intraoperative Siemens MAGNETOM Vida (3T) system at \gls{gosh}.
The acquisition parameters of the three test sequences, along with the standard \gls{gosh} \gls{dmri} sequences for comparison, are shown in Table \ref{tab:rsepi}.

% [1] Yang, J. Y.-M. *et al.* Assessment of intraoperative diffusion EPI distortion and its impact on estimation of supratentorial white matter tract positions in pediatric epilepsy surgery. *NeuroImage: Clinical* **35**, 103097 (2022).  https://www.sciencedirect.com/science/article/pii/S2213158222001620
% [2] Wang, Y. *et al.* A comparison of readout segmented EPI and interleaved EPI in high-resolution diffusion weighted imaging. *Magnetic Resonance Imaging* **47**, 39–47 (2018). DOI: 10.1016/j.mri.2017.11.011
% [3] Elliott, C. A. *et al.* Intraoperative acquisition of DTI in cranial neurosurgery: readout-segmented DTI versus standard single-shot DTI. *Journal of Neurosurgery* **133**, 1210–1219 (2020). http://dx.doi.org/10.3171/2019.5.JNS19890
% [4] https://archive.ismrm.org/2011/1945.html
% [5] https://www.researchgate.net/publication/236022704_Analysis_of_tractography_biases_introduced_by_anisotropic_voxels

\begin{table}
  \caption{RS-EPI test scan parameters}
  \label{tab:rsepi}
  % \setlength{\tabcolsep}{0.3em}
  \footnotesize
  \begin{tabularx}{\textwidth}{l >{\raggedright\arraybackslash}X >{\raggedright\arraybackslash}X >{\raggedright\arraybackslash}X >{\raggedright\arraybackslash}X >{\raggedright\arraybackslash}X} \toprule
    & \gls{gosh} VIDA “clinical” (RESOLVE) & \gls{gosh} VIDA Clinical \gls{dti} & Test scan 1: 7seg 6xb800 & Test scan 2: 3seg 12xb1000 & Test scan 3: 3seg 20xb1000 \\
  \midrule
   $n$ directions & 3 & 30 & 6 & 12 & 20 \\
   b values & 0, 800 & 0, 1000 & 0, 800 & 0, 1000 & 0, 1000 \\
   scan time (min) & 4:53 & 4:00 & 4:57 & 7:30 & not recorded \\
   in-plane & 1.198\x{}1.198 & 2.3\x{}2.3 & 2.3\x{}2.3 & 2.5\x{}2.5 & 2.3\x{}2.3 \\
   resolution (mm) & & & & & \\
   slice thickness~(mm) & 4 & 2.3 & 2.1 & 2.5 & 2.3 \\
   size (voxels) & 192\x{}192\x{}35  & 96\x{}96\x{}62 & 96\x{}96\x{}47 & 88\x{}88\x{}58 & 96\x{}96\x{}58 \\
   FOV (mm)  & 230\x{}230  & 220\x{}220 & 220\x{}220 & 220\x{}220 & 220\x{}220 \\
   TR (ms) & 6480 & 6600 & 6570 & 8960 & 3850 \\
   $n$ segments & 7 & 1 & 7 & 3 & 3 \\
   GRAPPA & 2 & 2 & 2 & 2 &  \\
   SMS &  &  &  &  & 2 \\ \bottomrule
  \end{tabularx}
\end{table}

For each scan, minimal preprocessing was conducted using MRtrix3\autocite{Tournier2019} tools:
\Gls{mppca} denoising and bias field correction, followed by \gls{dt} modelling and \gls{dec} mapping, and \gls{ssst} \gls{csd}.
For \gls{csd}, $l_{max}$ was set to 8 except for the six direction scan where $l_{max}=6$ was used.
No distortion correction was applied.
Finally, the \gls{cst} was reconstructed with a simple probabilistic tractography protocol, using the internal capsule as a seed region and an inclusion \gls{roi} at brain stem level, along with exclusion \glspl{roi} for common false positives.

\begin{figure}[h!]
  \centering
  \includegraphics[width=\textwidth]{chapter_6/rsepi.png}
  \caption{Baseline \gls{ssepi} diffusion sequence and three \gls{rsepi} test sequences. From top: baseline (30xb1000 SSEPI DTI), 7seg 6xb800, 3seg 12xb1000, 3seg 20xb1000}
  \label{fig:rsepi}
\end{figure}

Based on visual assessment alone, the degree of distortion reduction is difficult to evaluate for all compared sequences (Fig. \ref{fig:rsepi}).
It is unclear whether, in the case of intraoperative scans with high levels of distortion at the resection site, the amount of distortion improvement would be comparatively greater than is apparent in a healthy scan with minimal distortion to begin with.
The third test scan additionally used SMS acceleration which appears to have significantly degraded and even corrupted the scan, even as Siemens has advertised the use of SMS with RESOLVE on compatible systems
\footnote[2]{minimum software version syngo MR XA11B. Source: \url{https://www.siemens-healthineers.com/en-uk/magnetic-resonance-imaging/options-and-upgrades/clinical-applications/syngo-resolve} retrieved 06-Oct-2023}.


\begin{figure}[h!]
  \centering
  \begin{subfigure}{0.5\textwidth}
    \includegraphics[width=\textwidth]{chapter_6/0_fod_tg.png}
  \end{subfigure}%
  \begin{subfigure}{0.5\textwidth}
    \includegraphics[width=\textwidth]{chapter_6/1_fod_tg.png}
  \end{subfigure}
  %
  \begin{subfigure}{0.5\textwidth}
    \includegraphics[width=\textwidth]{chapter_6/2_fod_tg.png}
  \end{subfigure}%
  \begin{subfigure}{0.5\textwidth}
    \includegraphics[width=\textwidth]{chapter_6/3_fod_tg.png}
  \end{subfigure}
  \caption{\glspl{fod} at the centrum semiovale and full \gls{cst} tractography streamlines. From left to right, top to bottom: baseline (30xb1000 SSEPI DTI), 7seg 6xb800, 3seg 12xb1000, 3seg 20xb1000}
  \label{fig:rsepi-fod}
\end{figure}

Figure \ref{fig:rsepi-fod} demonstrates the \glspl{fod} and tractography reconstructions obtained from each sequence.
The six direction scan results in poorly resolved \glspl{fod} and correspondingly noisy streamlines, while the third test scan which was degraded due to the use of SMS produced entirely unusable \glspl{fod}.
Most promising are the results obtained from the 12 direction scan, despite the low angular resolution, although there is certainly still a lot of noise.
The findings presented in \ref{sec:ismrmdiff} indicate that a 12 direction scan, while far from ideal, could still be used to obtained reliable tract segmentations (Fig. \ref{fig:ssepi2}).
It is simply a question of priority:
Is the achievable reduction in susceptibility distortion worth the implications of lower angular resolution?
This is a question that cannot be answered from such exploratory data, and indeed the answer is likely to be case-specific, given that the amount of distortion varies significantly from scan to scan depending on the specific patient and surgical setup.
While further exploration of distortion reduction was not a priority of this project, it remains an interesting avenue for future research, and these preliminary findings provide an indication that \gls{rsepi} for intraoperative imaging remains an open possibility.

\begin{figure}
  \centering
  \begin{subfigure}{0.4\textwidth}
    \includegraphics[width=\textwidth]{chapter_6/2_glass_tg.png}
  \end{subfigure}%
  \begin{subfigure}{0.4\textwidth}
    \includegraphics[width=\textwidth]{chapter_6/2_glass_tf.png}
  \end{subfigure}
  \caption{Tractfinder reconstruction of the left \gls{cst} in the 12 direction, three segment \gls{rsepi} scan compared with streamline tractography.}
  \label{fig:ssepi2}
\end{figure}
