\section{Data requirements}

\note{Basically the findings from ISMRM diffusion workshop with different data inputs.}

\section{Atlas training volume}

In segmentation methods relying on seen data from which to ``learn" patterns to apply to unseen data, the volume and range of training data influences the prediction accuracy and generalisability. \note{says who?}
For complex deep learning models, which have many thousands of network parameters to learn, the amount of training data required to achieve accurate and stable performance can be immense, posing a particular barrier to the use of such models in applications where suitably annotated data is scarce.
In the case of TractSeg, for example, 105 subjects in total were used for cross-validation training, which each fully trained model having seen 63 unique subjects \note{also mention augmentation?}.

In tractfinder, the influence of the number of subjects used to construct each atlas is on the amount of inter-subject anatomical variation reflected in the spatial and orientational components.
It is to be expected that, save for extreme outliers, the additional information gained from adding more training subjects would reach a point of saturation.

To investigate this, an experiment was conducted whereby the number of subjects included in atlas construction was varied, and the effect on segmentation accuracy compared.
For this purpose the TractSeg HCP reference bundles were used.
Using the same train - test data split as described in \note{??}, subsets of 15 and 30, as well as the full 63 training subjects were selected, from which separate TOD atlases where constructed.
Tractfinder tract maps were then generated in the 43 test subjects using each of the different subset atlases and compared with the reference segmentations using the DSC and density correlation metrics.
