\chapter{Methodological evaluation}

The following sections set out to provide an evaluation of the tractfinder method from a range of perspectives.
A common complaint in the realm of white matter imaging is the lack of reliable ground truth for fibre tract identification.
The truth is, there is no such ground truth.
This goes deeper than the limitations of our scientific instruments: micrometer-resolution \gls{mri} or infallible tractography algorithms are not the answer.
As was explored in chapter \ref{chap:atlas}, our \note{knowledge} of white matter anatomy, connection and function is neither fixed nor universal, such that a single definition that can be agreed upon by all does not even exist for a single moment, let alone one to last over time.

We can nonetheless arrive at a reasonable assessment of tractfinder's accuracy, reliability and applicability by making justifiable assumptions and comparing it, quantitatively and visually, with benchmark methods in a variety of datasets and case studies.
The missing ground truth problem comes to the fore when comparing different methods, as we will see in section \ref{sec:validation}, but the results are still instructive in assessing their respective characteristics.
To balance this issue, we provide a further analysis directly comparing tractfinder against the deep learning method TractSeg by constructing new \gls{tod} atlases from the TractSeg model training data.
We'll further attempt to quantitatively evaluate the effect of tumour deformation modelling on improving segmentation performance in affected patients.

Finally, we'll move on from Dice scores and box plots to explore the practical applicability of tractfinder, considering the technical aspects of bringing it to the bedside, and illustrating the potentials and \note{remaining roadblocks} through a series of case studies.

\section{Datasets and image processing}
\label{sec:data}

A range of datasets are considered in which to compare the tractfinder approach against alternative tract segmentation methods.
Each dataset and any dataset-specific preprocessing is described in the following subsections.
In addition, the following default preprocessing steps were applied to all scans, unless otherwise specified:\footnote[2]{software versions: MRtrix3 v3.0.2-3.0.3 (\url{https://www.mrtrix.org/}), \gls{fsl} v.6.0}

\textit{Brain masking}\autocite{Tournier2019}
using a heuristic algorithm based on thresholding the mean of each diffusion weighted shell described in \textcite{Dhollander2016}, implemented in MRtrix3\autocite{Tournier2019} as \verb|dwi2mask|.

\textit{Affine registration} between subject space and MNI152\autocite{Fonov2011} space using \gls{fsl} Linear Image Registration Tool\autocite{Jenkinson2001,Jenkinson2002}

\textit{\Gls{csd}}. Depending on the dataset and / or application, two different versions of \gls{csd} applied: \gls{ssst} \gls{csd} (``original flavour")\autocite{Tournier2007,Tournier2019} and \gls{msmt} \gls{csd}\autocite{Jeurissen2014} restricted to \gls{wm} and \gls{gm} tissue compartments for single-shelled acquisitions (those with a single nonzero $b$-value),
and \gls{msmt} \gls{csd} with three tissue compartments (\gls{wm}, \gls{gm}, \gls{csf}) for multi-shelled acquisitions.
In all cases, response functions were obtained using the Dhollander unsupervised 3-tissue response function estimation algorithm.\autocite{Dhollander2016,Dhollander2019}
All \gls{csd} processing was conducted using the MRtrix3 image processing software package\autocite{Tournier2019}.

\subsection{HCP}

The first large healthy adult dataset comprised 49 scans from the WU-Minn \gls{hcp} Young Adult S1200 data release (\url{https://www.humanconnectome.org/study/hcp-young-adult/document/1200-subjects-data-release}) \autocite{VanEssen2013}.
The original data comprises high resolution T1 weighted and diffusion weighted \gls{mri} acquired on a modified Siemens 3T Skyra scanner.
Raw diffusion data contains three diffusion weighted shells at $b=$ 1000, 2000 and 3000 $s/mm^2$ with 90 directions each and 18 $b=0$ volumes.\autocite{Sotiropoulos2013}
These images have been preprocessed as documented in \textcite{Glasser2013,Sotiropoulos2013},
and additionally for this analysis they were downsampled to 2.5 mm isotropic voxels (from the original resolution of 1.25 mm isotropic) and a subset of 60 optimally distributed directions at $b=1000 s/mm^2$ extracted, emulating a typical clinical acquisition in spatial and angular resolution.

A different set of 105 HCP subjects were used in the training and testing of the deep learning segmentation method TractSeg.\autocite{Wasserthal2018}
The curated TractSeg reference streamline bundles for these subjects are publicly available\autocite{Wasserthal2018b} and were used in a separate set of analyses to directly compare tractfinder and TractSeg.
For these subjects, the same preprocessing steps were applied as in the set of 49 subjects, with the exception of selecting a subset of 30 instead of 60 $b=1000s/mm^2$ directions, for comparability with the \textit{clinical quality} data using in \textcite{Wasserthal2018}.

\subsection{TractoInferno}

The recently released \textit{TractoInferno} database (v1.1.1, available at \url{https://openneuro.org/datasets/ds003900/versions/1.1.1}),\autocite{Poulin2022} created for the training of machine learning tractography approaches, contains diffusion and T1-weighted MRI scans for 284 subjects pooled from several studies, accompanied by reference streamline tractography reconstructions.
\note{using RecoBundlesX, clustering method \autocite{Garyfallidis2018,Rheault2020a}}
Of the 284 subjects included in the full \textit{TractoInferno} database, the 80 subjects with tractography of the \gls{cst}, \gls{or}, \gls{ifof} and \gls{af} were selected.
Nine subjects were excluded from the final analysis due to inadequate non-linear registration performance resulting in failed targeted ROI tractography (see section \ref{sec:methods}), leaving a final 71 subjects.
Diffusion acquisition parameters and preprocessing steps are described in \textcite{Poulin2022}, and all data was additionally resampled to 2.3 mm isotropic voxels, the lowest resolution present in the dataset and one in line with clinical acquisitions.

\subsection{Clinical (GOSH \& NHNN)}

To validate tractfinder in real clinical patient scans, a dataset of 15 individual scans from eight different subjects from two different institutions was collated.
They include four adult glioma subjects acquired in 2009 at the National Hospital for Neurology and Neurosurgery, London (NHNN) (cases 4 and 5 from \textcite{Mancini2022}, others unpublished data),
three paediatric subjects from Great Ormond Street Hospital, London (GOSH) (each with one preoperative and one intraoperative scan),
a mock “intraoperative” scan on a healthy adult volunteer acquired with the GOSH intraoperative \gls{dti} protocol and using simulated intraoperative setup (flex-coils wrapped around the head instead of a head coil, head significantly displaced from scanner isocenter etc),
and a partridge in a pear tree.
For acquisition details see Table \ref{tab:datasets}.
All clinical scans used in the systematic comparison with benchmark methods involved non-deforming tumours, in the sense that any lesions did not appreciably displace white matter structures from their typical positions.
Further GOSH and NHNN scans with substantial tumours were analysed on a case-by case basis, as the low sample size and high tumour heterogenity precluded a systematic quantitative evaluation.

Each \gls{dmri} scan was minimally preprocessed with Marchenko-Pastur principal component analysis denoising\autocite{Veraart2016, Cordero-Grande2019} Gibbs-ringing correction\autocite{Kellner2016} and bias field correction,\autocite{Zhang2001, Smith2004} as implemented in MRtrix3 \autocite{Tournier2019}.
Preoperative scans additionally had eddy current and motion distortion correction\autocite{Andersson2016a, Smith2004} (MRtrix3 v3.0.3 and FMRIB Software Library (FSL, \url{https://fsl.fmrib.ox.ac.uk}) v6.0) applied, while this step was omitted for intraoperative scans to maintain a clinically realistic timeline.
No EPI distortion correction was performed, as it is frequently omitted from clinical pipelines due to lack of requisite reverse phase encoding or field map information and long processing times.\autocite{Yang2022}

%%%%%%%%%%%%%%%%%%%%%%%%%%%%%%%%%%%%%%%%%%%%%%%%%%%%%%%%%%%%%%%%%%%%%%%%%%%%%%%%
\begin{landscape}
\begin{table}[t]
  \caption{Overview of acquisition parameters for the datasets included. \dag Resampled from original, see text for details.}
  \label{tab:datasets}
  \small
  \begin{tabularx}{0.9\linewidth}{l l l l l l l l l} \toprule
             & \multicolumn{2}{c}{GOSH} & \multicolumn{2}{c}{NHNN} & \multicolumn{2}{c}{BTC\autocite{Aerts2018, Aerts2020a}} & HCP\autocite{Sotiropoulos2013, Glasser2013} & \textit{TractoInferno}\autocite{Poulin2022} \\
             & pre-op   & intra-op      & pre-op & intra-op        & pre-op & post-op       & & \\
  \midrule%
  %           GOSH pre    GOSH intra          NHNN pre      NHNN intra                  BTC pre  BTC post                   HCP           TractoInferno
  n          & 3          & 4                & 4      & 4                               & 10    & 9                         & 49         & 71     \\[1em]
  age range (y)  & \multicolumn{2}{c}{paediatric (3-12, n=3)} & \multicolumn{2}{c}{adult (30-39)} & \multicolumn{2}{c}{adult (39-74)} & adult (22-35) & adult (18-75) \\
             &            & adult (n=1)      &        &                                 &       &                           &            &         \\[1em]
  indication  & tumour    & tumour (n=3)   & \multicolumn{2}{c}{oligodendroglioma (n=2)}& \multicolumn{2}{c}{meningioma (n=4)} & healthy & healthy \\
              &           & healthy (n=1)  & \multicolumn{2}{c}{other tumour (n=2)}     & \multicolumn{2}{c}{glioma (n=5)}  &            & \\[1em]
  b values  & 800 (n=1)   & 1000           & 1000     & 1000                       & \multicolumn{2}{c}{0, 700, 1200, 2800} & 1000       & 1000 (n=68) \\
  (s/mm$^2$) & 1000, 2200 (n=2) &          &          &                            &            &                           &            & 700 (n=3) \\[1em]
  n dirs   & 15 (n=1)     & 30             & 64 (n=3) & 30 (n=3)                   & \multicolumn{2}{c}{8, 16, 30, 50}      & 60\dag     & 21-128 \\
           & 60, 60 (n=2) &                & 61 (n=1) & 3 x 12 (n=1)               &            &                           &            & \\[2em]
  voxel size & 1.75\textsf{x}1.75\textsf{x}2.5 (n=1) & 2.5 (n=1) & 2.5 & 2.5\textsf{x}2.5\textsf{x}2.7 & \multicolumn{2}{c}{2.5} & 2.5\dag    & 2.3\dag \\
  (mm)       & 2\textsf{x}2\textsf{x}2.2 (n=2)       & 2.3 (n=3) & & & & & & \\[1em]
  scanner & Philips Ingenia 1.5T (n=1)  & Siemens Vida 3T & Siemens & Siemens & \multicolumn{2}{c}{Siemens} & Siemens 3T & variable\\
          &  Siemens Prisma 3T (n=2)  &                & Trio 3T  & Espree 1.5T                & \multicolumn{2}{c}{Trio 3T}           & ``Connectome  & \\
          &                          &                &          &                            &                   &                   & Skyra”  & \\ \bottomrule
  \end{tabularx}
\end{table}
\end{landscape}
%%%%%%%%%%%%%%%%%%%%%%%%%%%%%%%%%%%%%%%%%%%%%%%%%%%%%%%%%%%%%%%%%%%%%%%%%%%%%%%%

\subsection{BTCD}

The \gls{btc} dataset comprises a series of pre- and postoperative scans of glioma and meningioma patients undergoing craniotomy for tumour resection.
Of the total 25 patients in the original dataset, 10 with macroscopic, non-infiltrating lesions were selected for the validation of tractfinder in tumour patients.
In one of the 10 subjects (sub-PAT22), only a preoperative session is available, giving a total of 19 unique scan sessions for this dataset.
The data consists of high quality \gls{hardi} acquisitions, including reverse phase encoding for susceptibility distortion, and structural T1w images.
In addition, tumour segmentation masks for each preoperative scan are available in the original dataset, which can be used for tumour deformation modelling in those patients with strong tumour deformation.
The 10 selected subjects include 4 with tumours large enough (3 meningioma and 1 anaplastic astrocytoma) to warrant the use of tumour deformation modelling to improve atlas alignment.
Of these 4 subjects, one had a large frontal meningioma situated on the midline affecting both hemispheres, and for this subject all tracts were considered ``ipsilateral" tracts.
In all other subjects, the tracts were labelled ipsilateral or contralateral if they were in the same or opposite hemisphere as the tumour respectively.

\gls{dmri} data was preprocessed with denoising, bias field correction and \gls{epi} distortion correction (eddy for CUDA 9.1, part of FSL v.6.0.4).
\gls{msmt}-\gls{csd}, tractfinder, TractSeg and manual tractography of the \gls{af}, \gls{cst}, \gls{ifof} and \gls{or} were performed as previously described.
Given the disruption caused by the space-occupying and in cases infiltrating lesions, manual tractography using multi-ROI targeting alone did not produce acceptable results.
Instead, the bundles were manually filtered to remove spurious and implausible streamlines using the same process as was used when curating the atlas training data.
These filtered bundles are used as reference segmentations for the 10 BTCD subjects.

\subsection{Fibercup phantom}

While testing and validating on \textit{in vivo} human data is irreplaceable, the complexities of such data, as well as the lack of definitive ground truth for fibre bundle reconstructions in unknown anatomies, can make results difficult to interpret.
This is where synthetic phantom data, with known physical parameters, are valuable for protoyping and validating methods.
Within the white matter imaging space there is a synthetic phantom known as the Fibre Cup phantom after a tractography challenge held for the 12th international conference on Medical Image Computing and Computer-Assisted Intervention (MICCAI) in 2009.\autocite{Fillard2011}
The phantom features seven bundles arranged in a way that aims to mimic the range of configurations found in the brain, including sharp curves, multi-way crossings, kissing and fanning fibres.
A digital recreation of the original data and accompanying ground truth streamlines are openly available.\autocite{NITRC}
The Fibre Cup phantom was used extensively in prototyping and proof-of-concepts for tractfinder, including testing alternative atlas-data comparison approachs such as the KL-divergence and fixel-decomposed inner product.
