\chapter{Tract Orientation Atlas}
\label{chap:atlas}

\note{Somehow also include the ``mapping" part in the chapter title?}

The chosen approach to fulfill the aims laid out in \ref{sec:problem} combines aspects of both traditional atlas-based and direct data-driven approaches to segmentations.
It is clear that the task of \gls{wm} tract identification cannot be accomplished without incorporating anatomical priors, and that those priors cannot account for individual differences.
The incorporation of anatomical priors can be automated for streamline tractography, as is done for several prior works including the FSL tool XTRACT and the white matter query language framework.\autocite{Wassermann2016}
These approaches still rely on tractography as the means for parsing the local orientation information in the test subject, and consequently inherit many associated drawbacks such as the computational time and power required for whole brain tractography, or the difficulties tracking through oedema or around tumours.
As an alternative to tractography, in which the attribution of a voxel to a tract of interest is contingent on it being visited by a streamline which, based on its entire length, has been attributed to the tract, we will instead develop a voxel-wise approach, in which a voxel can be associated with a tract based on its local diffusion properties and global coordinates.

We'll thus consider a voxel's membership with a target tract to be determined by it's location in the brain and local directional diffusion characteristics, and how well they align with prior expectations for the tract.
This chapter is concerned with the encoding of those expectations in the form of a tract-specific location and orientation atlas, and the voxel-wise comparison of those expectations with the captured directional diffusion information.
The necessary intermediate step, of establishing spatial correspondence between the atlas and subject data, will be the subject of Chapter \ref{chap:reg}.

\section{Theoretical motivation}

The tract orientation atlas aims to capture the typical spatial and orientational distribution of the bundle across a sample of healthy subjects.
In order to motivate the atlas creation steps, it is worth first considering the information channels aimed to be contained within the atlas and their interpretation.

First is the spatial distribution, i.e. the likelihood of finding the tract in a given voxel.
This is sometimes conflated with or loosely equated to streamline density derived from tractography, however this interpretation is highly flawed and biased.\note{improve this sentence}
Instead, we will model, at the individual level, the spatial distribution of the tract as a binary variable.
This approach of course does not account for any measurement uncertainty, and effectively assumes that the location of the tract can be determined with absolute certainty.

In fact, if we pick a voxel in the brain and ask

Thought experiment: we pick a voxel in a brain scan and ask, ``are fibres of the corticospinal tract present in this voxel".
In attempting to answer this question, we encounter several sources of uncertainty.
a) Definitional uncertainty.
While we can more or less agree that, at one end, it forms the pyramids of the medulla, and at the other it terminates in the primary motor cortex in the precentral gyrus, for the intermediate course the boundaries are less obvious.
Within the corona radiata, for instance, where is the boundary, if there is one at all, between the motor fibres and the ascending sensory pathways?
This lack of highly detailed neurophsyiological description leads to our first source of uncertainty.
b) Inter-subject variability.
Even if we could nail down, in a hypothetical ideal brain, the exact shape of a pathway, we would be left to contend with the individual differences in brain shape and organisation.
How can we know if, in this particular subject, the transition from afferent to efferent fibres occurs slightly more posteriorly than on average?
c) Measurement uncertainty.
Suppose we have overcome our doubts about the expected shape of a tract in a particular individual.
Our voxel is in or near the tract's domain, but it also sits adjacent to grey matter.
The diffusion profile measured in this voxel is also consistent with the presence of some fibres travelling in the expected direction, but only according to one estimate of the \gls{fod}, using \gls{ssst} \gls{csd}.
Using \gls{msmt} \gls{csd}, we get a different \gls{fod}, one which doesn't align with the expected \gls{cst} fibres, suggesting that data only weakly supports the presence of such fibres, and possibly only thanks to an artefact of noise.
This measurement and modelling uncertainty, incidentally, is what is captured in the probabilistic tractography algorithm probtrackx.\autocite{Behrens2007}

Faced with this tangle of unknowns, we have chosen a simplistic route.
At the individual level, we will decide upon a theoretical bundle structure that is best supported by known neuroanatomy, and attempt to faithfully reconstruct that structure using probabilistic streamline tractography and careful manual filtering of implausible streamlines.
From this reconstruction we will derive a binary spatial segmentation. \note{disregarding all those uncertainties?}
By repeating this process and considering the differences in segmentations across a population of reference subjects, we will in doing so capture an approximation of the inter-subject variability.

% its true we cant trust the mri signal so theres some uncertainty associated with that but given that we don't have a definitive definition of where the tract is even in a perfect situation with zero measurement uncertainty, trying to disentangle the ``we don't know if this voxel contains the tract necessarily because the signal might be noisy" and ``we don't know if the tract is here because we don't actully have a specific definition of the tract's anatomy" is sort of pointless. So we might as well decide on a definition and, whereever that definition is supported by the data we have via the presence of anatomically plausible streamlines, we'll define that voxel as having the tract.

The second channel of information conveyed in the atlas is the orientation distribution, meaning the spread of likely directions along which the fibres of the bundle might be travelling at a given location.
This property, too, is subject to uncertainties as described above.
However, once a tract's location is known, or at least assumed, we can in most cases be fairly certain of its local orientations based on our knowledge of its global shape and trajectory.
In a multi-fibre voxel, the distinct population in the apparent \gls{fod} which corresponds to our tract of interest is assumed to be known.
In this case, tractography which pobabilistically draws samples from that distribution will reconstruct some streamlines travelling through that voxel in a distribution of directions consistent with our expectations for that tract, alongside potential spurious streamlines traversing the voxel in the ``wrong" direction.
After carefully ensuring, as much as is feasible, that only those consistent streamlines remain, we can retrieve the orientation distribution of the tract of interest in isolation from the other crossing populations using \gls{tod} mapping.
The objective is to create a map in a template space capturing, at each location, the range of possible orientations the tract can take on as a single spherical distribution.
A narrow distribution may be found where the tract's orientation is highly consistent across all subjects, whereas a more spread-out distribution would reflect a wider range of possible orientations, which may be seen in regions of fanning or sharp turning (Fig. \note{fig:mean}).

To obtain such a mapping, a combination of streamline tractography and \gls{tod} mapping \autocite{Dhollander2014} is used.
While tractography has significant limitations as discussed previously, it remains the standard way of segmenting white matter bundles from \textit{in vivo} dMRI data, and biases and errors can, with appropriate post-processing steps, be at least partially corrected for.
In addition, tractography uniquely enables the extraction of orientation information specific to the reconstructed bundle, which would not be possible from a binary voxel-wise segmentation.
Being derived from carefully curated streamline tractography reconstructions, we can conceptualise the atlas as a store of anatomical prior expectations which would otherwise be utilised to draw appropriate ROIs for tractography in a target image.

\section{Streamline tractography and filtering}

A dataset of 16 healthy adult HARDI acquisitions (``EEG, fMRI and NODDI dataset",\autocite{Clayden2020} available online at osf.io/94c5t) was used in the creation of reference bundles for developing the atlas.
In each subject, the bundle of interest was reconstructed in both hemispheres using probabilistic streamline tractography with iFOD2 \autocite{Tournier2010} and a consistent ROI strategy based on anatomical landmarks broadly agreed upon in prior works. (See \note{sec:tractography} for full details on tractography parameters and ROI strategies used for each tract.)



After streamline generation, each streamline bundle was transformed into MNI space using affine registration implemented in FMRIB's Linear Image Registration Tool\autocite{Jenkinson2002} between the subject's T1-weighted image and the MNI152 T1 template.\autocite{Fonov2011}
Affine registration rather than non-linear, was used for this step to capture individual anatomical variation and minimise unrealistic warping of streamlines from local registration errors or overfitting.
With all subject streamlines aggregated in MNI space, manual filtering of streamlines was performed (Fig. \note{fig:atlases}) to remove not only ``volumetric false positives", which depart from the accepted volume of the tract, but also ``orientational false positives" (OFPs), which remain entirely within the tract volume but are at least in part aligned with a different, intersecting bundle.
An example of such OFPs are depicted in supplementary Figure \note{fig:ofp}. Such streamlines have little effect on any volumetric applications of the reconstruction, e.g. via a track density depiction.
However, their removal is vital for the construction of the orientation atlas, which summarises the orientational distribution of streamlines on a voxel-wise basis.
Filtering was performed in DSI studio (v2021\_04, \url{https://dsi-studio.labsolver.org/})\autocite{Yeh2021a}, which enables the filtering of streamlines based on angle of intersection with a cutting plane.
The percentage of streamlines filtered for each tract and summarised reasons for removal are presented in Table \ref{tab:filt}.

%%%%%%%%%%%%%%%%%%%%%%%%%%%%%%%%%%%%%%%%%%%%%%%%%%%%%%%%%%%%%%%%%%%%%%%%%%%%%%%%
\begin{table*}[t]
  \caption{Streamline filtering statistics. Abbreviations: \acrolist{af,crp,cc,cst,ec,slf,sfof}}
  \label{tab:filt}
  \small
  \begin{tabularx}{\textwidth}{llllll X}\toprule
   &  & Original & Filtered & Difference & Reduction & Reasons for discarding \\
   \midrule
  CST & left & 148833 & 145300 & 3533 & 2.37\% & \multirow{3}{=}{Contamination from: AF / SLF, SFOF, CC, CrP} \\
   & right & 144759 & 139019 & 5740 & 3.97\% &  \\
   & total & 293592 & 284319 & 9273 & 3.16\% &  \\ \addlinespace
  AF & left & 61922 & 49778 & 12144 & 19.61\% & \multirow{3}{=}{Contamination from:  EC, CST, CC Overextension into: Motor, anterior temporal, and superior frontal cortex} \\
   & right & 61834 & 43027 & 18807 & 30.42\% &  \\
   & total & 123756 & 92805 & 30951 & 25.01\% &  \\ \addlinespace
  OR & left & 123842 & 99984 & 23858 & 19.26\% & \multirow{3}{=}{Contamination from: Tapetum of CC, SLF} \\
   & right & 122534 & 109265 & 13269 & 10.83\% &  \\
   & total & 246376 & 209249 & 37127 & 15.07\% & \\ \toprule
 \end{tabularx}
\end{table*}
%%%%%%%%%%%%%%%%%%%%%%%%%%%%%%%%%%%%%%%%%%%%%%%%%%%%%%%%%%%%%%%%%%%%%%%%%%%%%%%%


\subsection{Tract definitions}

\subsubsection{Optic radiation}

\subsubsection{Corticospinal tract}

\subsubsection{Arcuate fasciculus}

\subsubsection{Inferior fronto-occipital fasciculus}


\section{TOD mapping}

\note{Registration to MNI space and \gls{tod} mapping. Includes the whole spiel about normalisation.}


After aggregate filtering, the retained streamlines were re-separated into individual subject bundles and the \gls{tod} was computed from the individual bundles as described in \textcite{Dhollander2014} and implemented in MRtrix3. \autocite{Tournier2019}
\gls{tod} mapping is the generalisation of track density imaging into the angular domain, creating a 5D spatio-angular representation of streamline tracks on a voxel-wise basis.
The \gls{tod} image is represented in modified \gls{sh} basis \autocite{Descoteaux2006} using only even orders up to a maximum order $l_{max}=8$, meaning each image consists of 45 coefficients, denoted $t_j$, per voxel.
The distribution is described by those coefficients and the modified \gls{sh} basis functions $Y_{l,m}$ \autocite{Descoteaux2006} as

\begin{align}
  T(\theta, \phi) = \sum_{l=0}^{l_{max}} \sum_{m=-l}^l t_{l,m} Y_{l,m}(\theta, \phi) = \sum_j t_jY_j(\theta, \phi)
\end{align}

The individual \gls{tod} images at this stage still contain significant density bias, with exaggerated differences in magnitude between the core bundle portions and fanning extremities owing to tractography's tendency towards early termination outside of the densest collinear tract regions.\autocite{Rheault2020,Smith2013}
The purpose of the atlas is to capture only the likelihood of a tract's presence in any given voxel (spatial prior) and, in the case that it is present, its expected orientation (orientational prior).
If the spatial prior is to be determined by considering the spatial variation of the tract between subjects, then the only information needed for each individual subject is a binary visitation map for the bundle and orientational data.
Thus to remove the streamline density component, the \gls{tod} maps for each subject are normalised as follows.
The spherical integral of each \gls{sh} basis function $Y_{l,m}$ is

\begin{align}
  \int_{\Omega} Y^m_l(\theta, \phi) = \begin{cases}
   \sqrt{4\pi} & \text{ if } l=m=0\\
   0 & \text{ otherwise. }
  \end{cases}
\end{align}

Using the sum and constant rules of integration, the spherical integral of $T(\theta,\phi)$ is
\begin{align}
  \int_{\Omega} T(\theta,\phi) = t_0 \sqrt{4\pi}
\end{align}

where $t_0$ is the first \gls{sh} coefficient for $l=m=0$. Thus to remove density information the \gls{tod} map is normalised to unit integral as

\begin{align}
  \widetilde{T}(\theta, \phi) = \frac{T(\theta,\phi)}{\sqrt{4\pi} t_0}
\end{align}

After each individual \gls{tod} map has been normalised in MNI space, what remains contains only information about the tract's streamline orientations, and none about the number of streamlines passing through a given voxel in the original reconstruction.

Finally, the mean of all individual normalised \gls{tod} maps is computed to produce the final population tract \gls{tod} atlas.
Averaging all maps results in distributions that reflect all possible ranges of tract orientations in each voxel (Fig. \note{fig:mean}), while the first \gls{sh} coefficient of the atlas will reflect the proportion of training subjects in which the tract was present in a given voxel.
Outlier voxels visited by streamlines in only a single subject's reconstruction will contribute little weight to the final atlas.
This atlas can then be registered to a target subject for further processing.
Atlases have so far been created for the most commonly indicated pathways in neurosurgical planning and guidance, namely the \gls{cst}, \gls{or} and \gls{af} (Fig. \note{fig:atlases}), with the creation of further atlases to be the subject of future work.

\section{Discussion}

\note{Consider different elements of the atlas construction, e.g. what is the effect (smoothing) of using only non-linear registration and possibly put the discussion on number of training subjects here too?}
