\chapter{Tract Orientation Atlas}
\label{chapterlabel2}

\note{Somehow also include the ``mapping" part in the chapter title?}

\section{Purpose}

\note{A general of the rationale behind the atlas}

The tract orientation atlas captures the typical spatial and orientational distribution of a give tract across a sample of healthy subjects. Being derived from carefully curated streamline tractography reconstructions, it can be considered a store of anatomical prior expectations which would otherwise be utilised to draw appropriate ROIs for tractography in a target image.


An orientation distribution map is constructed from each subject streamline set using track orientation distribution mapping (\note{spelling? + cite}).\autocite{Dhollander2014}

In order to justify the atlas creation steps, it is worth first considering the information channels aimed to be contained within the atlas and their interpretation.

First is the orientation distribution, which will be determined from the TOD mapping of streamlines.
Second is the spatial distribution, i.e. the likelihood of the tract residing in a given voxel.
The latter concept is sometimes conflated with or loosely equated to streamline density derived from tractography, however this interpretation is highly flawed and biased.

Instead, it is more useful to consider the binary state of tract belonging in each individual subject, and then determine the proportion of subjects in which a voxel belonged to the tract.

\section{Construction}

\note{How each atlas is created in an artisanal Bloomsbury lab. Also general stuff like training data, preprocessing etc.}

\subsection{Streamline tractography}

\note{The streamline tractography and ROI strategies for each tract. Somewhere, will want to include brief lit reviews on the tracts studies justifying the chosen cortical terminations. Maybe here?. Also include filtering in DSI studio, OFPs etc.}

\subsection{TOD mapping}

\note{Registration to MNI space and TOD mapping. Includes the whole spiel about normalisation.}

\subsection{Discussion}

\note{Consider different elements of the atlas construction, e.g. what is the effect (smoothing) of using only non-linear registration and possibly put the discussion on number of training subjects here too?}
