\section{Tract mapping}
\label{chapterlabel4}

\note{This part is about the inner product, plus any other comparison between atlas and data, including e.g. KL divergence, fixel anaylsis...}

The methodological description has been, up to this point, concerned with capturing
prior information about tracts and bringing those priors into alignment with the target anatomy.
Finally, it remains to compare those priors with the information present in the target diffusion data itself.

\subsection{Inner Product}


The orientation atlas is registered from MNI to subject space using affine registration.
The tract atlas intentionally conveys a degree of spatial tolerance to account for individual variations in tract location, with the following step acting to refine the estimate according to observed local information in the target image.
The objective is to obtain a measure per voxel of how closely the predicted tract orientation distribution overlaps with the observed FOD, modelled from dMRI data using constrained spherical deconvolution (CSD). \autocite{Tournier2007}

This can be achieved by taking the inner product of the two functions, i.e. multiplying them and integrating the product over all spherical angles.
As with the TOD atlas, the FOD is represented in the modified spherical harmonic (SH) basis:

\begin{align}
  F(\theta, \phi) = \sum_{l=0}^{l_{max}} \sum_{m=-l}^l f_{l,m} Y_{l,m}(\theta, \phi) = \sum_j f_jY_j(\theta, \phi)
\end{align}


The spherical integral of the product of two spherical harmonic basis functions $Y_{l_1,m_1}$ and $Y_{l_2,m_2}$ is

\begin{align}
  \int_0^{\pi} \int_0^{2\pi} Y_{l_1,m_1}(\theta, \phi) Y_{l_2,m_2}(\theta, \phi) sin(\theta) d\theta d\phi \\
    = \delta_{m_1, m_2} \delta_{l_1, l_2}
\end{align}


Therefore, for two functions  $F(\theta, \phi)$ and $T(\theta, \phi)$ the integral of their product can be expressed as

\begin{align}
  \begin{split}
    & \int_0^{\pi} \int_0^{2\pi} F(\theta, \phi) T(\theta, \phi) sin(\theta) d\theta d\phi \\
    = & \int_0^{\pi} \int_0^{2\pi} (\sum_j f_jY_j(\theta, \phi)) (\sum_k t_kY_k(\theta, \phi)) sin(\theta) d\theta d\phi \\
    = & \sum_{k,j} f_j t_k \delta_{jk}
  \end{split}
\end{align}


Thus for two distributions represented by a vector containing their spherical harmonic coefficients, the integrated product can be obtained by taking the inner product of the two coefficient vectors.
The final result is this voxel-wise inner product of the registered atlas and subject FOD images. The resulting image is a pseudo-probability map of tract location, in arbitrary and dimensionless units. Typical values range from [0 - 0.5], with 0.05 empirically determined to be a suitable threshold for converting to binary segmentation.
We refer to this proposed segmentation approach, of registering a pre-constructed orientation atlas to a target image and computing the inner product as described as "tractfinder".


\subsubsection{Components of final value}

The tract atlases and FOD data contain both spatial and directional information, all of which contribute to the final inner product value.
It is worth investigating, however, to what extent those two factors influence the final result.
It is conceivable for example that the final map value is predominantly determined by the overall amplitude of one or both of the ODFs, with the non-zeroth SH coefficients having little additional effect.


\subsection{Fixel analysis}

The value of the inner product between an estimated orientation distribution representing a mixture of fibre populations and the expected orientation distribution for a specific isolated population will depend on a range of interacting components, making interpretation of the final value a complicated task.

\note{missing figure: fantasy voxels comparing different atlas/fod combo scenarios (same amp, matching, crossing etc.)}

For example, consider two voxels, both containing FOD lobes of equal amplitude perfectly aligned with the expected orientation represented by the atlas TOD.
\note{I actually can't describe these scenarios without running some simulations}

To disentangle the competing effects in crossing fibre voxels, we consider an extension of the inner product analysis using the concept of fixels.
A portmanteau of ``fibre" and ``voxel", a fixel describes a sub-voxel element representing a single fibre population.
A voxel can contain any number of fixels, and each can be analysed in isolation with fixel-level properties such as orientation, dispersion and fibre density.
By segmenting a full \gls{fod} into individual lobes and computing properties on those sub-distributions, we can separate the effects of different fibre populations within a single voxel.

The fixel framework has been adapted here to extend the inner product to consider individual \gls{fod} lobes in isolation.
\gls{fod} segmentation works by sampling the \gls{sh} distribution on a dense set of directions (>1000).
Each of these direction samples is binned into a fixel based on its neighbourhood, resulting in each fixel being represented by a collection of directional samples \note{and zero everywhere else. ? better explanation of FMLS algorithm}
These fixel samples are then converted to their \gls{sh} decomposition via a linear least-squares fit \note{to what? the basis functions or amplitudes?}
The result is a set of individual \gls{sh} distributions per voxel, each representing a different fixel.

The amplitude of the fixel \glspl{odf} will reflect that of the corresponding lobe in the original full multi-fibre distribution, and thus the integral of the fixel \gls{odf} will always be smaller or equal to that of the original \gls{fod}.
However, we are interested in whether the fibre population of interest is present in a voxel based on orientation alone, regardless of its relative weight to the other fibres occupying the same voxel.
The location probability, in other words, should be influenced only by the expected location probability \note{$t_0$} and the overall white matter signal integral \note{$FOD_0$}.
\note{need to figure out these notations!!}
To remove the relative weighting effect of crossing fibres, we renormalise each fixel \gls{odf} to the overall \gls{fod} amplitude:

\begin{align}
  \overline{I} = \frac{I f_0}{i_0}
\end{align}

where the fixel \gls{odf} is $I(\theta, \phi) = \sum_j i_j Y_j(\theta,\phi)$.
Then we compute the inner product of the atlas \gls{tod} and each fixel \gls{odf} $I_k$, taking the maximum value as the result:

\begin{align}
  p = \operatorname*{argmax}_k \int_\Omega I_k(\theta,\phi) T(\theta,\phi) \sin\theta d\theta d\phi
\end{align}

The result is improved tract sensitivity in areas of crossing fibres where there is otherwise sufficient evidence that the tract is present in the voxel.
During the \gls{fod} segmentation process, a minimum threshold can be placed on the lobe amplitude for a single fixel, reducing noise amplification.

\begin{figure}
  \centering
  \begin{subfigure}[b]{0.3\textwidth}
  \centering
  \includegraphics[width=\textwidth]{chapter_2/FCF_ground_truth_fibres.png}
  \caption{Ground truth fibres}
  \label{}
  \end{subfigure}%
  \begin{subfigure}[b]{0.3\textwidth}
  \includegraphics[width=\textwidth]{chapter_2/normal_ip_FCF.png}
  \caption{Simple inner product}
  \label{}
  \end{subfigure}%
  \begin{subfigure}[b]{0.3\textwidth}
  \includegraphics[width=\textwidth]{chapter_2/fixel_ip_FCF.png}
  \caption{Fixel inner product}
  \label{}
  \end{subfigure}
  \caption{Crossing fibres results in low signal for the brown tract in upper intersection region. In fixel IP this signal drop is eliminated resulting in smooth tract map.}\label{fig:fixip}
\end{figure}
