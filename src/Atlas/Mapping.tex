\section{Tract mapping}
\label{chapterlabel4}

\note{This part is about the inner product, plus any other comparison between atlas and data, including e.g. KL divergence, fixel anaylsis...}

The methodological description has been, up to this point, concerned with capturing
prior information about tracts and bringing those priors into alignment with the target anatomy.
Finally, it remains to compare those priors with the information present in the target diffusion data itself.

\subsection{Inner Product}


\subsubsection{Components of final value}

The tract atlases and FOD data contain both spatial and directional information, all of which contribute to the final inner product value.
It is worth investigating, however, to what extent those two factors influence the final result.
It is conceivable for example that the final map value is predominantly determined by the overall amplitude of one or both of the ODFs, with the non-zeroth SH coefficients having little additional effect.


\subsection{Fixel analysis}
