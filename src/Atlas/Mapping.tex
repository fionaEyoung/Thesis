\section{Tract mapping}
\label{chapterlabel4}

\note{This part is about the inner product, plus any other comparison between atlas and data, including e.g. KL divergence, fixel anaylsis...}

The methodological description has been, up to this point, concerned with capturing
prior information about tracts and bringing those priors into alignment with the target anatomy.
Finally, it remains to compare those priors with the information present in the target diffusion data itself.

\subsection{Inner Product}


The orientation atlas is registered from MNI to subject space using affine registration.
The tract atlas intentionally conveys a degree of spatial tolerance to account for individual variations in tract location, with the following step acting to refine the estimate according to observed local information in the target image.
The objective is to obtain a measure per voxel of how closely the predicted tract orientation distribution overlaps with the observed FOD, modelled from dMRI data using constrained spherical deconvolution (CSD). \autocite{Tournier2007}

This can be achieved by taking the inner product of the two functions, i.e. multiplying them and integrating the product over all spherical angles.
As with the TOD atlas, the FOD is represented in the modified spherical harmonic (SH) basis:

\begin{align}
  F(\theta, \phi) = \sum_{l=0}^{l_{max}} \sum_{m=-l}^l f_{l,m} Y_{l,m}(\theta, \phi) = \sum_j f_jY_j(\theta, \phi)
\end{align}


The spherical integral of the product of two spherical harmonic basis functions $Y_{l_1,m_1}$ and $Y_{l_2,m_2}$ is

\begin{align}
  \int_0^{\pi} \int_0^{2\pi} Y_{l_1,m_1}(\theta, \phi) Y_{l_2,m_2}(\theta, \phi) sin(\theta) d\theta d\phi \\
    = \delta_{m_1, m_2} \delta_{l_1, l_2}
\end{align}


Therefore, for two functions  $F(\theta, \phi)$ and $T(\theta, \phi)$ the integral of their product can be expressed as

\begin{align}
  \begin{split}
    & \int_0^{\pi} \int_0^{2\pi} F(\theta, \phi) T(\theta, \phi) sin(\theta) d\theta d\phi \\
    = & \int_0^{\pi} \int_0^{2\pi} (\sum_j f_jY_j(\theta, \phi)) (\sum_k t_kY_k(\theta, \phi)) sin(\theta) d\theta d\phi \\
    = & \sum_{k,j} f_j t_k \delta_{jk}
  \end{split}
\end{align}


Thus for two distributions represented by a vector containing their spherical harmonic coefficients, the integrated product can be obtained by taking the inner product of the two coefficient vectors.
The final result is this voxel-wise inner product of the registered atlas and subject FOD images. The resulting image is a pseudo-probability map of tract location, in arbitrary and dimensionless units. Typical values range from [0 - 0.5], with 0.05 empirically determined to be a suitable threshold for converting to binary segmentation.
We refer to this proposed segmentation approach, of registering a pre-constructed orientation atlas to a target image and computing the inner product as described as "tractfinder".


\subsubsection{Components of final value}

The tract atlases and FOD data contain both spatial and directional information, all of which contribute to the final inner product value.
It is worth investigating, however, to what extent those two factors influence the final result.
It is conceivable for example that the final map value is predominantly determined by the overall amplitude of one or both of the ODFs, with the non-zeroth SH coefficients having little additional effect.


\subsection{Fixel analysis}
