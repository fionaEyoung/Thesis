\chapter{Conclusions}
\label{chapterlabel6}

\note{Honestly no idea what will go here. I guess really big picture stuff, since the individual chapters will make their own points about results, pros and cons etc.}

\section{General discussion}

\note{Summary of main findings and outputs, pros and highlight specific achievements}

The reality of achieving reliable white matter neuronavigation with intraoperative diffusion MRI is not simply a matter of replacing tractography with a ``better" automated tract segmentation pipeline.
As we have seen, the problem of brain shift and tumour mass effect is substantial and will unlikely be solved with a single one-size-fits-all approach.
Simple radial tumour expansion modelling can go a long way towards expanding the applicability of atlas-based tract mapping to cases involving mass effect.
This works particularly well for encapsulated tumours and for tumours with a relatively simple shape and location.
Lesions growing in the cortex or subcortical white matter tend to displace tissue around them in a fairly predictable way which is captured in a radial model.
By contrast, tumours of the diencephalon or midbrain can produce unexpected deformations, owing to the complex arrangement of surrounding structures and their biomechanical relationships to each other: cranial nerves, peduncles, thalamic nuclei and subcortical grey matter. \note{point? clearer!}
In addition, a notable shortcoming of the radial deformation model is its ignoring of varying compressibilities of the brain's tissues, and in particular of the ventricles.
The ability of the fluid filled ventricles to absorb a sizeable amount of displacement force and prevent the mass effect from propagating further through the brain than it otherwise might in a medium of uniform elasticity (as assumed by the simple model presented here) is what most often \note{?} causes atlas registration inaccuracies in tumour cases.
This shortcoming should be the first to be addressed in any future modification and improvements of the tumour deformation model.
Another significant simplification is the requirement \note{?} that all healthy tissue is displaced to beyond the tumour boundary, effectively assuming a perfectly encapsulated lesion in which no healthy brain is either destroyed or infiltrated.
Indeed this assumption is necessary to ensure the formula for the displacement factor $k$ remains well-behaved and invertible.
Support infiltrating lesions is less of a priority, as rarely \note{ever?} would they be operated on with a goal of total or near-total resection.
As seen in the case of \note{ref: case study 2}, where resection of the infiltrating tumour portion was abandoned to protect the corticospinal tract.
And indeed, as seen in that case, tumour deformation modelling can still be effective in such cases, if only the solid component is segmented, or a scale factor $s<1$ is employed to reduce the effective tumour radius.
In this way, dMRI-based white matter mapping could play a role together with direct stimulation functional monitoring in subtotal resections and biopsies in high risk locations close to critical white matter.

The work described in this thesis came about during a time when increasingly sizeable contingents of the research and medical communities are focussing minds on the potential of machine learning techniques to disrupt and \note{revolutionise} many difficult to solve problems.
In the white matter imaging space, machine learning is gaining traction both for direct segmentation methods and as a means for finally overcoming the deep flaws in streamline tractography which have entangled researchers for years.
Against this backdrop of excitement for new possibilities, it's no surprise that the introduction of yet another atlas-based approach would be met with little enthusiasm.
Nevertheless, just as streamline tractography has long delived astonashing benefits while simultaneously remaining unable to break free from its sensitivity-specificity \note{gremlins}, already the difficulties of bringing deep learning solutions to real clinical translation are showing.\note{rewrite}
A stand-out snag is the enormous data requirement.
The need for large volumes of accurately annotated data may be easily fulfilled \note{when annotation is easy or you have lots of resources} or justified if the application is a well-defined problem unlikely to need revisiting.
Tract segmentation is none of those things: producing the ground truth reference annotated data is burdensome, and the likelihood of the effort may need duplicating as our understanding of white matter anatomy evolves is high.
In addition, clinical decision making must remain \note{traceable} and transparent, including when aided by computers.
In this light, a flexible atlas, which can be re-trained if needed using only a handful of exemplar datasets and which leads to intuitive and interpretable results \note{can} certainly remain relevant in the nascent era of AI.

\section{Dissemination}

\note{Chris wants me to put stuff in here but what exactly??}

Some people in the office have tried it out, yay!
Maybe one day it will be available to the public, that would be nice.
The research charity Children with Cancer UK have granted funding for a follow-on two year study to prospectively validate tractfinder in a clinical iMRI setup.

\section{Limitations}

\note{I'm sure there will be lots to put here}

\section{Future directions}

\note{Wouldn't it be nice ...}
