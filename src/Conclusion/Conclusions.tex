\chapter{Conclusions}
\label{chapterlabel6}

The growing availability of advanced \gls{mri} capabilities in health centres is bringing attention to the shortcomings of current image processing techniques in fully exploiting potential benefits for patients.
In a small internal survey of neuroradiologists and neurosurgeons at \gls{gosh}, all respondents expressed that they would find a tool for intraoperative imaging of white matter bundles after brain shift to be either slightly (2/5) or very (3/5) useful, while confirming reservations about the reliability or accuracy of tractography.
Against this background, this thesis has set out to explore the current capabilities of \gls{dmri} to map brain \gls{wm} for surgical planning and neuronavigation, and propose a novel technique to fulfil the requirements for rapid and robust \gls{wm} tract detection.
Guided by the objectives set out in Chapter \ref{chap:problem}, the proposed pipeline, called tractfinder, involves constructing tract-specific orientation atlases which are compared with a subject's \gls{fod} image, to achieve direct voxel-wise segmentation with incorporated \textit{a priori} anatomical knowledge.
Each atlas is constructed from meticulously filtered training streamlines based on available consensus neuroanatomical definitions which are then mapped to voxel-wise orientation distributions using \gls{tod} imaging and averaged over the training population (Chapter \ref{chap:atlas}).
Upon registration with the target image, the tract's location is estimated by comparing the voxel-wise prior tract orientation distribution with the \gls{fod} modelled
from \gls{dmri} data, which can be achieved by taking the inner product of the two spherical distributions.
If large tumour deformation effects are present in the target image, then the atlas is adjusted accordingly with an exponential radial deformation model prior to computing the tract map.
Through detailed evaluation against benchmark methods presented in Chapter \ref{chap:eval}, tractfinder has been shown to produce consistent and accurate segmentations at a standard level with streamline tractography and deep learning, issues with variable tract definitions and reference data quality notwithstanding.

Several design choices in the tractfinder pipeline are supported by the objective to keep point-of-application processing time and user-interaction to a minimum.
The anatomical priors, which would for tractography either be drawn by hand or automatically via cortical parcellation or deformable registration, are provided by the tract atlases which also account for a degree of inter-subject variability.
As we saw in Chapter \ref{chap:reg}, only linear registration is required to align atlas and subject data, which is faster and more robust than the non-linear or deformable registration that would be necessary for accurate segmentation using registration alone without comparison with subject diffusion data.
It is also applicable to challenging clinical data without need for manually adjusting registration parameters, where non-linear algorithms can fail to reach stable convergence.
Using the \gls{sh} basis for representing orientation distribution data allows for efficient inner product computation, and could flexibly support alternative comparison metrics.
Overall, the full pipeline can be run in under five minutes for a single tract including minimal preprocessing, with the only potential need for user input being the choice of tumour deformation model parameter $\lambda$, if the default adaptive value produces suboptimal results.

Given that tumour infiltration and oedema affect diffusion measurements and thus all downstream processing and modelling, we cannot claim that these effects do not impact tract mapping using tractfinder.
However, a voxel-wise approach is not susceptible to a compounding of errors in the same way streamline tractography is, where oedema in one part of the tract can derail tracking for the entire bundle, including parts not directly affected by infiltration.
Where disturbed diffusion does result in lower tractfinder values, they can be interpreted in the context of other imaging and even provide clinically useful information about \gls{wm} integrity.
By contrast, we have seen examples of the deep learning method TractSeg failing to recognise tracts disturbed by tumours without clear cause (Section \ref{sec:case}), displaying a lack of explainability that is unacceptable for effective clinical translation.
In this way, the requirement of a clinically applicable and robust method is achieved, although this has yet to be rigorously verified in a prospective study (see Section \ref{sec:future}).

A final stipulation was to keep reference data requirements for atlas creation to a minimum, to accommodate evolving neuroanatomical definitions and the difficulty in producing high quality reconstructions.
With only 16 training subjects used for the original atlases, and subsequent analysis presented in Section \ref{sec:ntrain} indicating that as few as ten may be sufficient, tractfinder's has a significant advantage over more data-intensive statistical and deep learning alternatives.

The work described in this thesis came about during a time when increasingly sizeable contingents of the research and medical communities are focussing minds on the potential of machine learning techniques to disrupt hitherto intractable problems.
In the white matter imaging space, machine learning is gaining traction both for direct segmentation methods and as a means for finally overcoming the fundamental roadblocks in streamline tractography which have entangled researchers for years.
Against this backdrop of excitement for new algorithmic and big-data possibilities, proposals for a new atlas-based technique have been met with some skepticism.
Nevertheless, just as streamline tractography has long delivered astonishing benefits while simultaneously remaining unable to expel its sensitivity-specificity trade-off gremlins, so too are the difficulties of bringing deep learning solutions to real clinical translation beginning to show.
Fulfilling the need for large volumes of accurately annotated data may be easy where the resources for producing said data are freely available, or at least justified if the application is a well-defined and static problem.
Tract segmentation is neither of those things:
Producing the ground truth reference annotated data is burdensome, and the likelihood that efforts may need duplicating as our understanding of \gls{wm} anatomy evolves is high.
We have seen this in Section \ref{sec:quant} with the \textit{TractoInferno} dataset, which represents a substantial contribution to the machine learning tractography research community, but which nonetheless contains a number of poor-quality samples undetected by rigorous quality control.
Work on single-shot and transfer learning as discussed in Section \ref{sec:ntrain} has underscored the need for methods which can easily be retrained or extended to support new tracts as and when they become relevant.
In addition, a key concern in the translation of complex, ``black box'' models is that clinical decision making must remain traceable and transparent, including when aided by computers.
In this light, a flexible atlas, which can be re-trained if needed using only a handful of exemplar datasets and which leads to intuitive and interpretable results can play a unique role alongside machine learning.

\section{Limitations and future directions}\label{sec:future}

The reality of achieving reliable white matter neuronavigation with intraoperative \gls{dmri} is not simply a matter of replacing tractography with a ``better'' automated tract segmentation pipeline.
As we have seen, the challenge of brain shift and tumour mass effect is substantial and will unlikely be solved with a single one-size-fits-all approach, given the degree of patient heterogeneity.

Exponential radial tumour deformation modelling can go a long way towards extending the applicability of atlas-based tract mapping to cases involving mass effect, however, the results must be inspected carefully as the effects of many tumours are not well captured by such a simplistic model.
It works particularly well for encapsulated tumours and for tumours with a relatively simple shape and location:
Lesions growing in the cortex or subcortical white matter tend to displace tissue around them in a fairly predictable way which is captured in a radial model.
By contrast, tumours of the diencephalon, midbrain, or hindbrain can produce deformations in non-radial directions, owing to the complex arrangement of surrounding structures and their biomechanical relationships to one another.
In addition, a notable shortcoming of the radial deformation model is the lack of awareness of different brain tissues' elasticities, especially that of the ventricles.
The ability of the fluid filled ventricles to absorb a sizeable amount of displacement force and prevent the mass effect from propagating further through the brain than it otherwise might in a medium of uniform elasticity, is what frequently causes atlas registration inaccuracies in tumour cases.
This shortcoming should be the first to be addressed in any future modification and improvements to the tumour deformation model.
A further significant simplification is the strict non-infiltration assumption, in which all healthy tissue is fully displaced to beyond the tumour boundary.
Indeed this assumption is necessary to ensure the formula for the displacement factor $k$ remains well-behaved and invertible.
It may be argued that explicitly modelling tumour infiltration is of lower priority, as tumours infiltrating eloquent tracts are not generally candidates for total resection, as we saw in the case of patient 8 in Section \ref{sec:case}, where resection of the infiltrating tumour portion was abandoned to protect the corticospinal tract.
We also saw that tumour deformation modelling can still be effective in such cases, if only the solid component is segmented, or a scale factor $s<1$ is employed to reduce the effective tumour radius.
In this way, \gls{dmri}-based white matter mapping could play a role together with direct stimulation functional monitoring in subtotal resections and biopsies in high risk locations close to critical white matter.

\Gls{imri} is a relatively new technique, and as additional time spent scanning under general anaesthetic and with an open craniotomy carries potential risks for the patient, acquiring supplementary sequences with no clear or confirmed clinical benefit may be unethical.
Particularly with regards to intraoperative \gls{dmri}, a standard of care or guidelines for its use based on large-cohort trials has yet to be established.
As a consequence, while the new \gls{imri} system at \gls{gosh} has been used extensively since installation, the acquisition of diffusion sequences has remained infrequent, leading to a lack of available modern data with which to rigorously validate tractfinder.
Two of the available \gls{gosh} datasets were discussed in detail in Section \ref{sec:case}, providing illustrative insights into the use of tractfinder in real-world scenarios.
There are also outstanding challenges to obtaining consistently high quality diffusion images intraoperatively within a short-enough scan time, as discussed in reference to \gls{epi} artefacts and accelerated imaging in Section \ref{sec:technical}.
A further aspect of clinical uptake which was not addressed in this thesis is acceptance of a volumetric, voxel-based intensity map where radiologists and neurosurgeons may prefer visualising tracts as the familiar three-dimensional streamline bundles.
Addressing this potential barrier to translation, for example through a combination of the two techniques or improved volumetric visualisation strategies, should be considered as part of future investigations.
These limitations are to be addressed in an upcoming follow-on prospective study, funded by a grant from Children with Cancer UK (Ref: CwC2022\textbackslash 100006), which will evaluate tractfinder against conventional tractography in a series of children undergoing brain tumour surgery with \gls{imri} at \gls{gosh} and assess its clinical applicability in a range of tumour histological types and locations.
It is hoped that the methodologies and technical considerations presented in this work can contribute to the wider exploitation and adoption of advanced \gls{dmri}-based \gls{wm} imaging in neurosurgical practice, bringing to bear the full potential of modern technological and neuroscientific developments to the benefit of patients.

\chapter*{List of publications and outputs}\addcontentsline{toc}{chapter}{List of publications and outputs}

\subparagraph*{Peer-reviewed journal articles}
\begin{itemize}
  \item[] \fullcite{Young2022}
  \item[] \fullcite{Young2024}
\end{itemize}
\subparagraph*{Articles in production}
\begin{itemize}
  \item[] \fullcite{Aylmore}
\end{itemize}
\subparagraph*{International conferences, accepted abstracts}
\begin{itemize}
  \item[] \fullcite{Young2022b}
  \item[] \fullcite{Young2022a}
  \item[] \fullcite{Young2023}
\end{itemize}

\subparagraph*{Software and data}
\begin{itemize}
  \item[] Tractfinder external MRtrix3 module, available at: \url{github.com/fionaEyoung/tractfinder}
  \item[] Tract orientation atlases and training streamlines, available at: \url{https://doi.org/10.5281/zenodo.10149873}
\end{itemize}

\noindent A follow-on prospective two-year study applying the techniques presented in this thesis has been granted funding by Children with Cancer UK (Grant Ref: CwC2022\textbackslash100006, Title:  A tailored image guidance approach for children undergoing surgery for brain tumours).
