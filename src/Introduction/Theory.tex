\section{Theory fundamentals}
\label{theory}

\note{This will cover the more theoretical background prerequisites, covering the fundamental concepts on the biological side (neuroanatomy and cellular physiology, maybe some oncology?) and physical side (MR physics)}

\subsection{Neuroanatomy: Micro to macro}

\subsubsection{Cells of the brain}

\note{neurons, glial cells, axons and basics of neural signalling.}

The building blocks of living organisms begin at the level of molecules and atoms, through macromolecules such as proteins and lipids, cell organelles, cells, tissues, organs and finally whole organisms.
For the purposes of this report, though, we can start at the cellular level, and focus on a single organ: the brain.
Brain tissues consist of numerous cell types.
The principle functional cells are neurons, which perform the computations underpinning all aspects of neural function. They are supported, and outnumbered ( by a \note{factor of?}), by a network of glial cells, each with specialised functions.

\subsubsection{Neuroanaomy}

\note{Structure of the brain, including different sections (hind, mid, forbrain etc.) and tissue types.
Discuss the lobes, cortical regions and functional divisions, then white matter tracts.}

\subsection{MRI Physics}

\subsubsection{Resonance and relaxation}

\note{Basics of resonance, magnetisation, spin-spin / spin-lattice relaxation}

MRI images are obtained by detecting the spin relaxation resonance stuff of principally hydrogen atoms (which are abundant in the body in H2O molecules).

The application of a strong linear magnetic field causes all spins to align themselves with the direction of the field. This is the bulk magnetisation, in its "relaxed" state (aligned with B0).

\subsubsection{Excitation and image acquisition}

\note{How MRI machines work: excitation pulses, effects of different relaxation weightings, examples of different common pulse sequences and their uses}

\subsubsection{Diffusion MRI}

\note{First describe diffusion in general but stick to tissue, i.e. restricted diffusion. Cover the different types of diffusion in different tissues in the brain, timescales etc.
Then look at the mechanism of diffusion MRI acquisition, pulse sequence and parameters e.g. b values.
Also mention common artefacts.}

In living tissue, water molecules are not free to diffusion for long distances in all directions. Some tissue environments are highly constrained, with diffusion occurring principally along a single direction \note{too simplistic}, while in others, fewer barriers allow free diffusion.
