\section{Theory fundamentals}
\label{theory}

\note{This will cover the more theoretical background prerequisites, covering the fundamental concepts on the biological side (neuroanatomy and cellular physiology, maybe some oncology?) and physical side (MR physics)}

\subsection{Neuroanatomy: Micro to macro}

\subsubsection{Cells of the brain}

\note{neurons, glial cells, axons and basics of neural signalling.}

The building blocks of living organisms begin at the level of molecules and atoms, through macromolecules such as proteins and lipids, cell organelles, cells, tissues, organs and finally whole organisms.
For the purposes of this report, though, we can start at the cellular level, and focus on a single organ: the brain.
Brain tissues consist of numerous cell types.
The principle functional cells are neurons, which perform the computations underpinning all aspects of neural function. They are supported, and outnumbered ( by a \note{factor of?}), by a network of glial cells, each with specialised functions.

\subsubsection{Neuroanaomy}

\note{Structure of the brain, including different sections (hind, mid, forbrain etc.) and tissue types.
Discuss the lobes, cortical regions and functional divisions, then white matter tracts.}

\subsection{MRI Physics}

\subsubsection{Resonance and relaxation}

\note{Basics of resonance, magnetisation, spin-spin / spin-lattice relaxation}

MRI images are obtained by detecting the spin relaxation resonance stuff of principally hydrogen atoms (which are abundant in the body in H2O molecules).

The application of a strong linear magnetic field causes all spins to align themselves with the direction of the field. This is the bulk magnetisation $M_z$, in its "relaxed" state (aligned with B0).
The ratio of spin magnetic moments aligned with the external field is increased, \note{as the field imparts energy or something causing magnetic moments to preferentially be aligned} resulting in a net magnetisation along the orientation of $B_0$, termed $M_z$.
Alignment of a magnetic moment with $B_0$ (spin up) is energetically favourable of the opposite spin down state to the tune of $\Delta E = \gamma h B_0$ \note{explain vars}.
Hence the stronger the external field, the larger the ratio of up to down spins and thus the a larger the magnitude of $M_z$.
As the magnitude of this bulk magnetisation vector underpins the strength of any subsequently sampled signals, stronger bore fields, as a general rule \note{are there exceptions?}, produce images with better \gls{snr}.
In thermal equlibrium, then, the magnetic moments within the sample are precessing at the larmor frequency and entirely out of phase with each other, resulting in no net transverse magnetisation, and with spins aligned preferentially along $B_0$, resulting in maximum longitudinal magnetisation $M_z$.

The manipulation of this net magnetisation vector away from, and subsequent relaxation back to this equilibrium state, is the basis of all signal production in \gls{mri}.
The application of a \gls{rf} pulse, that is, an oscilating magnetic field ($B_1$) precisely at the resonant larmor frequency and perpendicular to $B_0$, enacts two effects on this the bulk magnetisation.
The first is the impartion \note{??} of energy which overcomes $\Delta E$ \note{this is the bit I understand the least} to increase the number of spins in the energetically unfavourable spin-down orientation and thus eliminate the net magnetisation along $B_0$.
The second effect is to sycronise the spins' precession with the rotating \gls{rf} pulse, and by extension with each-other, resulting in a precessing net magnetisation in the transverse plane $M_{xy}$.
\note{but what about 180 degree pulses?}
As soon as the excitation field is switched off, the bulk magnetisation gradually relaxes to equilibrium as the induced effects on the individual moments revert:
interaction with other spins \note{local magnetic fields?} in the molecular environment cause gradual dephasing and consequent reduction in $M_{xy}$ (spin-spin relaxation), while the energy gained from the \gls{rf} pulse is gradually dissipated among the lattice field \note{??} as the spins revert to the stistically fabourable lower energy state in alignment with $B_0$ causing gradual recovery of $M_z$ (spin-lattice relaxation).

The recovery rate of $M_z$ and the rate of decay of $M_{xy}$ are described by the time constants $T1$ and $T2$ respectively, and each depend on the specific molecular environment and are thus tissue-dependent quantities.
It is the varying values of $T1$ and $T2$, as well as the overall differences in proton densities ($PD$) between tissues which lend contrast to MR images.

\subsubsection{Excitation and image acquisition}

\note{How MRI machines work: excitation pulses, effects of different relaxation weightings, examples of different common pulse sequences and their uses}

\paragraph{Pulse sequences}

Immediatly after application of the \gls{rf} excitation pulse, magnetisation will rapidly return to equilbrium in a spiral pattern \note{see image}.
This rotating magnetisation will induce an oscillating current within a detection coil placed near the sample in a process called \gls{fid}.
This is the most basic form of NMR signal and, since it decays very rapidly \note{isn't useful on it's own for image formation?}
In order to generate signal useful for spatial encoding and image acquisition \note{??} we make use of spin echos and/or gradient echos.

To produce a gradient echo, a readout gradient $G_x$ \note{necessarily?} is applied across the sample causing spins to dephase a different rates according to their local magnetic field $B_0 + G_xx$.
After a time $TE/2$ \note{how long?} the gradient polarity is reversed, causing 

\subsubsection{Diffusion MRI}

\note{First describe diffusion in general but stick to tissue, i.e. restricted diffusion. Cover the different types of diffusion in different tissues in the brain, timescales etc.
Then look at the mechanism of diffusion MRI acquisition, pulse sequence and parameters e.g. b values.
Also mention common artefacts.}

In living tissue, water molecules are not free to diffusion for long distances in all directions. Some tissue environments are highly constrained, with diffusion occurring principally along a single direction \note{too simplistic}, while in others, fewer barriers allow free diffusion.

\paragraph{Diffusion weighting}

Any MR pulse sequence can be modified to introduce additional sensitisation to brownian motion, or diffusion weighting, in addition to the existing T1 and T2 based constrasts.
This is achieved through the application of a pair of diffusion sensitisation graident pulses prior to echo generation and signal readout.

To illustrate the concept, we will consider the application of diffusion weighting along a single orthogonal direction, e.g. $G_x$.
After slice selection and \gls{rf} excitation, a gradient of magnitude $G_d$ is applied along $x$ for time period $\delta$ before it is reversed in polarity for subsequent $\delta$.
Consider a spin which is stationary along $x=x_1$ throught the application of these gradients.
Initially, it will gain a phase $\phi_1$ proportional to it's position along the gradient $x_1$ according to $\gamma G_d \delta x_1$.
After gradient reversal, having not changed position $x=x_2=x_1$ it will gain the opposite phase $\phi_2 = -\phi_1$, resulting in a net phase change after diffusion sensitisation of $0$.
Conversely, a spin which is net motion along $x$ will experience different gradient strengths across the two time points and will experience a net dephasing according to $\Delta\phi = \gamma G_d \delta (x_2 - x_1)$ \note{replace with integral form} and corresponding signal loss.
On subsequent signal sampling, those voxels in which diffusion was high will have have a high degree of diffusion weighting-induced dephasing and exhibit a corresponding signal dropout.
In those voxels with low diffusion, phase coherence will remain relatively intact after diffusion senistisation and signal loss will be correspondingly minimal.

\paragraph{Echo planar imaging}

Since diffusion imaging, in particular \gls{hardi}, involves the acquisition of numerous image volumes with different diffusion weightings, scan times can be particularly long.
Depending on the image resolution, number of diffusion weighted directions, and many other factors, scan times can run into the tens of minutes and even hours.
Aside from the cost and practicalities of longer scans, there is also the increased risk of motion artifacts.
So it is that the dominant \note{only?} type of pulse sequence used in \gls{dmri} is \gls{epi}.
In \gls{epi}, all phase encoding intervals are acquired after only a single \gls{rf} excitation.
\Gls{epi} can employ both gradient and spin echos and there are numerous different variations and associated contrast \note(??), but for the purpose of this summary we will focus on the typical \gls{epi} sequence employed for diffusion weighted imaging.
