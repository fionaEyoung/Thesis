\chapter*{Introduction}\addcontentsline{toc}{chapter}{Introduction}
\label{chapterlabel0}

Few objects of our scientific pursuits rival the human brain in mystery and complexity.
Understanding its structure and function has been a centuries-long undertaking, yet we are still far from forming a complete picture, all while the scientific and medical needs for such knowledge grow more pressing.
Much of the world is caught in a mental health crisis\autocite{Patel2018,Liu2020,Yang2021a}, in ageing populations the burden of neurodegenerative diseases is accelerating\autocite{Deuschl2020,Li2022}, after leukaemia, children's cancers arise most frequently in the brain and spinal cord\autocite{Ostrom2015}.

For many years, study of the brain was limited to postmortem dissections, or observing patients who had suffered life-altering brain injuries.
An entirely new medium for peering into the brain came with the development of \gls{mri} in the latter 20th century, which has since become a cornerstone technology of modern \textit{in vivo} neuroscience.
\Gls{mri} has also gained indispensable importance in clinical neuroimaging, wherever it is available\autocite{Geethanath2019}, with applications including the ability to map brain function and connectivity in pathology.

In \textit{A Manual of Diseases of the Nervous System}, published in 1888, pioneering neurologist Sir William Gowers wrote of brain tumours:
``The treatment of new growths [...] is always a sufficiently gloomy subject, and not least so when they are seated in an organ like the brain, in which they cause peculiar and varied suffering, and in which their development, even to a moderate degree, is rarely compatible with life.''\autocite{Gowers1888}
He goes on to note the sobering risks and almost unavoidable neural sequelae accompanying any attempt at surgical removal.
Today, surgical treatment of brain tumours is routine, brings great benefits to quality of life and prognosis, and can even be curative.
Even so, a brain tumour diagnosis remains a devastating prospect, and the outlook and survival rates for malignant brain tumours remain somber\autocite{Aldape2019}.
We wield countless weapons in the campaign against brain cancer, from temozolomide chemotherapy, gamma knife surgery, and 5-aminolevulinic acid (5-ALA) fluorescence guidance, to proton beam therapy, genomics, and focussed ultrasound, all while pre-clinical research continues searching for a better understanding of how brain cancers grow and how they can be stopped.
Among this ensemble cast, \gls{mri}-guided treatment may play a small but nonetheless significant role in improving brain cancer care globally.
Its clinical impact may be greater still, if technological limitations in the acquisition and processing of advanced high-dimensional \gls{mri} data can be overcome.

An \gls{mri} modality called \gls{dmri} is able to give insights into the structural organisation and connectivity of neural fibres within the brain.
The pioneering technique tractography, which uses \gls{dmri} data to trace fibre pathways, has opened up the study of \textit{in vivo} whole-brain network neuroscience and found use as a clinical tool for planning delicate neurosurgical procedures involving critical brain structures.
It does, however, suffer from multiple limitations and requires significant resources and expertise, which aren't as widely available as conventional \gls{mri} capabilities\autocite{GeorgeZakiGhali2020}.
In 2020, \gls{gosh} in London unveiled a new \gls{mri} system for intraoperative imaging, heralding new possibilities for more precise neurosurgery and better patient outcomes.
\gls{dmri} has the potential to play a part in advanced intraoperative imaging, if image processing techniques can be developed that are compatible with the stringent efficiency and technical requirements.

This research project was motivated by the need for a brain fibre mapping technique that could be applied to intraoperative \gls{dmri} images, or in any other environment where resources are limited.
Starting from the premise that we already have, in most cases, a lot of prior information about the shape and location of brain pathways, a technique will be described in the following pages for directly estimating the location of a fibre bundle in healthy and clinical \gls{dmri} scans.
The proposed technique's benefits and limitations are evaluated in the context of the current state-of-the art in neuroimaging and neurosurgery, providing a comprehensive overview and future outlook for the role of advanced \gls{dmri} in improving the outcomes of neurosurgical interventions.
