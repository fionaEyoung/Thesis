\chapter{Introductory Material}
\label{chapterlabel0}

\section{Opening}

Few objects of our scientific endeavours, from the depths of Earth's oceans to the Milky Way galaxy, rival the human brain in mystery and complexity.
\note{link}
Understanding the brain's structure and function has been a scientific \note{priority?endeavour} for centuries, and yet we are still far from forming a complete picture.
At the same time, the scientific and medical needs and for such knowledge are \note{growing/more urgent}.
Much of the world is caught in a mental health crisis, in our ageing population the devastation wrought by neurodegenerative diseases is \note{accelerating}, after leukaemia, paediatric cancers arise most frequently in the brain and spinal cord.

For many years, study of the brain was limited to postmortem dissections, or the study of patients who had suffered life-altering brain injuries.
\note{stuff about other in vivo imgaing techniques and how they're bad at brains}
With the development of \gls{mri} came an entirely new medium for peering into the brain.
It has since become the single most important \note{? flouroscopy, two photon imaging? check nr of citations or whatever} technology for conducting \textit{in vivo} neuroscience.
\Gls{mri} has also gained indispensible importance in clinical neuroimaging.
\note{link}
Among the clinical applications of \gls{mri} lies the ability to map the critical structures of the brain as they relate to a neurosurgical target, such as a tumour.
In clinics around the \note{rich??} world, presurgical planning is informed by the insights provided by \note{diffusion} \gls{mri}.
This is thanks, in part, to the techniques of \gls{dti} and streamline tractography, which exploit the directional coherence of white matter bundles \note{... give insight into white matter structure/organisation}

tractography great and all but it has its flaws, such as the amount of time and effort and computational expertise needed to get good results.
This is fine if you have loads of resources, but lots of places don't, so presurgical planning with tractograph is limited to centres with access to these resources.
strike one against tractography.
But there is another reason why the impracticalities of tractography are getting in the way.
Or are unable to step up to the table for the new potential untapped ground of intraoperative imaging.
Surgical planning is great and all, but it only gets you so far into surgery.
After a while, the effects of brain shift leave the preoperative images in the dust.
Intraoperative MRI can be used to scan again during surgery, throwing fresh light on the surgical field and surrounding brain structure.
It's mostly used to check for residual tumour and/or confirm resection.
But what if further surgery is necessary?
Wouldn't it be great if we could use the intraoperative MRI to provide updated white matter mapping and navigation?
Well this is where tractography falls short, as it's current implementations are really impractical for use intraoperatively.
All the more reason to develop a novel means of identifying white matter bundles from diffusion MRI images as an alternative to tractography.

Intraoperative dMRI has the potential to supplement existing imaging practices by offering a means of imaging fibre tracts after brain shift has invalidated preoperative imaging.\autocite{Nimsky2001}
Thus informed, surgeons would be better equipped to resect as much tumour as possible while leaving eloquent brain tissue intact.

This research project was motivated by the need for a white matter bundle mapping technique that could potentially be applied to intraoperative diffusion MRI images, or any other environment where resources are limited.
The basic premise is that we already have, in most cases, a lot of prior information about the shape and location of the target bundle.
These expectations on shape and location could be exploited to directly estimate the location of the bundle in a new diffusion MRI scan.
Of course, the reality is that inter-subject variability and in particular the complexly anonamlous anatomies of brains harbouring tumours or other space-occupying lesions present challenges to this relatively simple premise, as indeed to do to streamline tractography.
This report will describe how this concept nevertheless holds up in healthy brains and, with appropriate adaptation, in diseased brains as well.
\note{dramatic finishing sentence!}

\paragraph{Terminology}

\note{copypasta, but something like this should go here}
Throughout this paper, we adopt the terminology set out in \textcite{Cote2013}.
In particular we will mainly use the term \textit{tract} (synonymous with \textit{fibre bundle} in \textcite{Cote2013}) to refer to an anatomical structure of a group of fibres with a common functional or anatomical organisation.
Individual tracts are the targets of \gls{wm} segmentation.
The term \textit{bundle} (shortened from \textit{streamline bundle}) will be used to refer to a collection of tractography \textit{streamlines}, usually serving as a representation of, but not synonymous with, a corresponding anatomical tract.
In some contexts, to maintain consistency with prior publications, the term \textit{track} may be used in place of streamline.
