\chapter{Introductory Material}
\label{chapterlabel0}

\section{Opening}

\note{Here comes a brief overview of the background and context,
setting the stage for the more detailed literature review and later the problem statement and project aims.
For example:}

% -- DELETE AND REWRITE THIS; DEMO ONLY
Intraoperative dMRI has the potential to supplement existing imaging practices by offering a means of imaging fibre tracts after brain shift has invalidated preoperative imaging.\autocite{Nimsky2001}
Thus informed, surgeons would be better equipped to resect as much tumour as possible while leaving eloquent brain tissue intact. \note{(...)}

\paragraph{Terminology}

\note{copypasta, but something like this should go here}
Throughout this paper, we adopt the terminology set out in \textcite{Cote2013}.
In particular we will mainly use the term \textit{tract} (synonymous with \textit{fibre bundle} in \textcite{Cote2013}) to refer to an anatomical structure of a group of fibres with a common functional or anatomical organisation.
Individual tracts are the targets of \gls{wm} segmentation.
The term \textit{bundle} (shortened from \textit{streamline bundle}) will be used to refer to a collection of tractography \textit{streamlines}, usually serving as a representation of, but not synonymous with, a corresponding anatomical tract.
In some contexts, to maintain consistency with prior publications, the term \textit{track} may be used in place of streamline.
