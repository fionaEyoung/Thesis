
\chapter{Neurosurgery}\label{chap:neurosurgery}
%========================

% \note{This is for all the reasons for surgery, including tumour, epilepsy DBS.
% Also more detail into the types of tumours, and locations in the brain}

Many types of interventions fall under the remit of cranial and spinal neurosurgery, including inserting electrodes for \gls{dbs}, diagnostic biopsies, vascular procedures, and insertion of \gls{csf} shunts.
In all cases, precision is paramount and tools for accurate navigation form vital components of the surgical workflow.
The following review, parts of which first appeared in abbreviated form in \textcite{Young2024}, will focus on some of the most complex and invasive procedures, involving craniotomy and removal of tumours and epileptogenic brain tissue.
In England between 2013--2018, oncological procedures were the third most common of all neurosurgery subspecialties comprising approximately 9\% of total, while functional neurosurgeries (including for epilepsy and deep brain stimulation) made up 8\%\autocite{Wahba2022}.
At \gls[noindex=false]{gosh} in London, a leading paediatric centre, 10\% of neurosurgical procedures between 2018--2022 were tumour-related, while 12\% were for epilepsy\autocite{gosh2023}.

Invasive brain tumour operations are both highly complex and variable in their neurophysiology, microbiology, treatment plans, and prognosis.
Such diversity presents a significant barrier to the development of image processing methods intended for generalised use in tumour patients\autocite{Bauer2013}.
Neoplasms occur throughout the brain, with the location having unique impact on surrounding structures and associated function.
A tumour's natural history and histopathology also play a large role in determining its effects on its environment.
Malignant gliomas, a category of tumours arising from glial cells, often have complex structures, with infiltrating components and peritumoural oedema blurring the distinction between tumour and non-tumour tissue\autocite{Weller2021}.
On the other hand, many non-malignant tumours, including most meningiomas and low-grade astrocytomas, are encapsulated, with clear demarcation from neighbouring brain tissues, which are displaced rather than infiltrated\autocite{Lu2004,Gerard2017}.

This project is concerned with the visualisation of cerebral white matter tracts, and therefore this review will focus on those indications and interventions in which damage to and navigation around such structures is of particular concern.
This is typically not the case for posterior fossa and suprasellar lesions, although there is growing interest in the role of the cerebellum in wider cognition and the brain functional network, and the surgical community is paying increasing attention on the effects of posterior fossa surgery on cerebellar tracts\autocite{Toescu2021,Skye2023}.
Brainstem tumours are often not candidates for surgical removal due to their eloquent location, with limited access and excessive risk to vital brainstem function.
The discussions in this section therefore apply primarily to supratentorial lesions in the cerebral hemispheres and thalamus, candidates for biopsy or resection via craniotomy.
The most common intracranial tumours are meningiomas, arising from the protective membranes surrounding the \gls{cns}, of which the majority are benign with good overall survival rates and relatively low-risk surgical treatment options\autocite{Rogers2015,Spena2022}.
Far more complex and controversial are decisions surrounding the surgical treatment of gliomas, those tumours originating in glial cells including astrocytes and oligodendrocytes which are the most common malignant primary \gls{cns} neoplasms in both adults\autocite{Ostrom2015,Wanis2021} and children\autocite{Ostrom2015,Bauchet2009}.
While challenges regarding functional preservation and optimal surgical strategy are relevant to all intracranial surgeries, they are acutely highlighted within the context of glioma literature.
These tumours can embed themselves insidiously within the brain's functional architecture with devastating prognosis, challenging oncologists and surgeons with stark dilemmas in their bid to maximise both patient survival and quality of life.
Gliomas are classified into four \gls[noindex=false]{who} grades, commonly split into \gls[noindex=false]{lgg} (\gls{who} grades 1--2) and \gls[noindex=false]{hgg} (\gls{who} grades 3--4) to reflect differences is malignancy and prognosis.
There are many subtypes based on histological and genetic characteristics which are periodically updated\autocite{Louis2021}, but this overview will focus on the broad categories of \gls{hgg} and \gls{lgg}.

\section{Extent of resection}\label{sec:eor}

Complete removal of all pathological tissue, perhaps counterintuitively, is not always the surgical objective.
Though it may in many cases be the ideal outcome from an oncological perspective, this scenario would frequently be in conflict with other equally important outcome indicators, such as the preservation of surrounding brain structures and the patient's neurological wellbeing.
Successfully balancing these consequences is a central dilemma in neurosurgical practice, with the key measure being \gls[noindex=false]{eor}, the amount of tumour removed.
In theory, \gls{eor} is a straightforward concept, but in practice it is ill-defined and inconsistently reported, while remaining central to studies of surgical efficacy and outcomes.

Easily defined in oncology%
\footnote[2]{\gls{eor} is relevant to epilepsy surgery, although the terminology and calculations here are different. Epileptogenic centres cannot always be distinguished and measured on imaging, and functionally eloquent tissue may be the clear source of epileptic activity, and thus subject to removal.}
as either the absolute volume or relative percentage of tumour tissue removed, accurately and consistently determining \gls{eor} is very difficult.
It is often reported in terms of broad categories, the most common being biopsy, \gls[noindex=false]{str} or partial resection (PR), near total resection, \gls[noindex=false]{gtr} and supratotal resection\autocite{Wykes2021,Karschnia2021}.
There is also no general consensus on how these categories are defined, making comparison between studies even more difficult\autocite{Karschnia2021}.
Many studies simply give very rough percentage values as estimated visually by the operating surgeon or a radiologist based on whether or not tumour residue is visible in the resection cavity or on a postoperative scan, with limited accuracy\autocite{Sanai2008,Martino2013,Lau2018,Sezer2020}.
Over time, the definitions for \gls{eor} have evolved with the availability of techniques for measuring it, and the current accepted standard for quantifying \gls{eor} is with volumetric measurement on pre- and postoperative imaging\autocite{Rincon-Torroella2019}, but here too practices are inconsistent\autocite{Wykes2021}.
Full volumetric analysis requires accurately segmenting the entire lesion, though sometimes \gls{eor} is calculated by simply taking the diameter of the lesion on a single or multiple slices, or with approximate ellipsoid segmentation\autocite{Sanai2008,Albuquerque2021}.
% This also can't account for resected portions of the tumour being filled with fluid, or differing amounts of tissue compression caused by mass effect and postoperative brain shift.

Even manual delineation can be unreliable and inconsistent, especially for tumours with poorly defined borders and on postoperative imaging \autocite{Ertl-Wagner2009,Bo2017,Visser2019}.
Semi- or fully automatic segmentation improves reproducibility\autocite{Ertl-Wagner2009,Sezer2020} and modern algorithms are proving ever more accurate, although there are still challenges regarding computational performance and robust clinical translation\autocite{Angulakshmi2017,Wadhwa2019,Fawzi2021}.
Estimates of \gls{eor} can also be compromised by post-operative brain tissue shifting and obscuring the actual volume of resected tumour \autocite{Schucht2014a}, while microscopic tumour cell invasion means that complete resection, as viewed either on imaging or by intraoperative visual assessment, does not necessarily mean no tumour residue remains\autocite{Yordanova2017}.
Finally, while there is the most focus on reporting relative reductions in tumour volume as a percentage of original size, more recent studies have argued that absolute residual tumour volume is as, if not more relevant for determining postoperative outcomes\autocite{Ius2012,Rincon-Torroella2019,Smith2008,Karschnia2021}.

\section{Oncological and neurological outcomes: Necessarily in opposition?}

Inconsistent reporting of \gls{eor} is one factor complicating the study of its effects on clinical outcomes, even as there is widespread agreement on the importance of studying those effects\autocite{Rincon-Torroella2019,Wykes2021,Weller2021}.
% \note{defs: lgg = grades 1--2, hgg = grades 3--4, glioblastoma = grade 4 glioma}
% For many tumour types, particularly aggressive tumours such as \glspl{hgg}, subtotal resection is \note{never} curative even with adjuvant therapy.
Broadly speaking, \gls{gtr} has been shown to increase overall and progression free survival over \gls{str} across age groups in both high \autocite{Hatoum2022, Han2020, Adams2016, McCrea2015, Bloch2012, McGirt2009, Kramm2006} and low-grade \autocite{Keles2001, Pollack1995, Sanai2008} gliomas.
For \gls{lgg}, and especially in paediatric patients, \gls{gtr} has become the recommended standard of care, as complete resection leads to a lower rate of recurrence\autocite{Berger1994,Claus2005}.
In particular, maximal resection of \glspl{lgg} drastically reduces the risk of residual tumour evolving into \gls{hgg} (known as malignant transformation \autocite{Duffau2013,Hervey-Jumper2016,Rincon-Torroella2019}), though this is only a concern in adult patients, as malignant transformation in paediatric \glspl{lgg} is exceedingly rare\autocite{Collins2020}.
More recent voices have even argued for supratotal resection, beyond the margins of any abnormally enhancing areas on $T_1$-weighted and FLAIR $T_2$-weighted \gls{mri} scans, as reviewed in \textcite{deLeeuw2019}.
There is limited evidence, though controversial, to suggest that supratotal resection of \gls{who} grade 2 gliomas in adults is followed by fewer cases of malignant transformation and improved progression-free survival \autocite{Yordanova2011}.

However, due to a general lack of prospective randomisation and robust comparison with appropriately matched controls, drawing definitive conclusions from studies investigating the effects of \gls{eor} (or other surgical variables) on post-operative outcomes is contentious\autocite{deLeeuw2019,Keles2001}.
Results may be confounded by selection biases, for example, different tumour histological subtypes may lend themselves more or less easily to greater \gls{eor}, or arise more frequently in eloquent areas of the brain (which include cortex and subcortical \gls{wm} subserving language, motor, and sensory functions, as well as the thalamus, midline structures involved in memory processing, and the brain stem), where an aggressive surgical strategy is likely to be discounted\autocite{deLeeuw2019}.
Adult \glspl{lgg} tend to occur more frequently than \glspl{hgg} in highly eloquent cortical regions\autocite{Duffau2004}, indeed the control group for the supratotal \gls{lgg} study\autocite{Yordanova2011} mentioned above consisted of patients whose gliomas were located in eloquent brain areas, and who therefore underwent only \gls{gtr}.
One might therefore expect supratotal resection to be associated with worse postoperative neurological outcomes, and indeed \textcite{Rossi2019a} found higher probabilities of immediate postoperative deficits in supratotal versus total resection of \glspl{lgg}.
These were however significantly reversed at three month and one year follow-ups, and initial overall evidence suggests that neuropsychological outcomes are comparable between total and supratotal groups\autocite{Tabor2021}.

In adults with glioma, maximal safe resection, combined with adjuvant radio- and chemotherapy, has been the established standard of care for some time.
It has been less clear, however, whether the same should apply to children.
The most prevalent anatomical locations in which gliomas arise may differ between adults and children\autocite{Duffau2004}.
Thalamic gliomas, for example, are more frequent in children than in adults\autocite{Cinalli2018,Palmisciano2021,GomezVecchio2021}:
Adult gliomas are usually located in the hemispheres, mostly the frontal lobe, with only approximately 4--7\%\autocite{GomezVecchio2021,Larjavaara2007} situated in the thalamus, while as many as 19\% of paediatric \glspl{hgg} are thalamic\autocite{McCrea2015}.
There is also concern that oncological differences between adult and paediatric type gliomas preclude safe extrapolation of treatment plans from one patient group to the other\autocite{Jones2012,Greuter2021}.
% Adult LGG 31\% eloquent Jakola2012
% Adult mixed grade 2--3 65\% ''presumed eloquence'' GomezVecchio2021 ; 80\% eloquent adult LGG Greuter2021
In addition to paediatric tumours frequently arising in high-risk areas such as the thalamus and brain stem \autocite{Ostrom2015},  neurocognitive and functional preservation is an especially critical concern in children.
A meta-analysis published in 2022 analysed 37 articles to assess the association between \gls{eor} and survival in paediatric patients with \gls{hgg}\autocite{Hatoum2022}.
Notwithstanding the difficulties in consistently defining and reporting \gls{eor} as discussed above, the study found strong evidence for improved overall survival in \gls{gtr} over \gls{str} of gliomas located in the cerebral hemispheres, but no association between \gls{eor} and survival was observed in midline cases.
The authors emphasise that midline (thalamic and brain stem) gliomas are not often indicated for aggressive resection due to the elevated risk to critical neurological function, and the lack of observed association may stem from measurement biases, including lower sample size and the pooling of histologically distinct tumour types which may respond differently to treatment.
Moreover, no comparison was made for postoperative functional neurological outcomes, thus failing to capture the full picture of factors contributing to a decision to pursue radical surgery.

New postoperative neurological deficits occur in over one third of glioma surgeries \autocite{Zetterling2020a}, although most patients improve significantly over longer-term follow-up.
Unsurprisingly, higher chances of postoperative deficits were associated with higher \gls{eor} and with tumours situated in eloquent areas\autocite{Zetterling2020a}.
In \textcite{Gil-Robles2010}, authors argue for a more conservative resection margin in \gls{who} grade 2 gliomas to protect functional structures, yet current consensus recommends total resection in \gls{lgg} wherever possible\autocite{Rincon-Torroella2019,Albuquerque2021}. %but this is not the dominant opinion
For the most malignant tumour types, even maximal resection combined with adjuvant therapy is rarely curative, and may only lead to a survival advantage of just a few months\autocite{Rincon-Torroella2019,Karschnia2023}.
Given the overall poor survival outcomes associated with aggressive gliomas, oftentimes the risks to quality of life and postoperative neurological function associated with pursuing \gls{gtr} outweigh any potential oncological benefits\autocite{Rahman2016,Tabor2021}.
In \glspl{hgg}, the justification for \gls{gtr} or even supratotal resection is weaker than in \gls{lgg}, given that it cannot secure long-term survival for affected patients.
Where radical resection carries no likely oncological benefit and is contraindicated by a high functional risk to eloquent areas, the goal of surgery may be conservative debulking of the lesion to relieve pressure on the brain and reduce current neurological symptoms.
% In certain types of tumour, particularly when a combined treatment approach of surgery and radiotherapy is taken, little evidence has been found that \gls{gtr} offers greater tumour-related (that is to say, neurological condition affected by the presence of the tumour) outcomes than subtotal resection, while increasing risk to surrounding brain tissue.
% On the other hand, certain types of tumour, in particular lower grade and less aggressive types, show a lower risk of recurrence and malignant transformation when radically resected, offering a particularly good long-term outlook.
With the widespread evidence of an oncological advantage associated with more extensive resection, physicians have increasingly advocated for \gls{gtr} as the standard treatment for \gls{lgg} and maximal safe resection for \gls{hgg} \autocite{Rincon-Torroella2019}.
But this comes with the caveat that tumours of lower malignancy are also often those found to be more operable, muddying the causal link between overall survival and extent of resection\autocite{Weller2021}.
Furthermore, post-operative neurological deficits, due to cortical and subcortical injury, themselves have a negative impact on overall survival, independently of differences in pre-operative symptoms \autocite{Trinh2013,Rahman2016,Rincon-Torroella2019}.
Hence even if only overall survival is considered as the measure of treatment success, the evidence that both neurological injury and un-resected tumour negatively impact survival highlights the dilemma of surgically treating tumours in eloquent brain regions\autocite{Rincon-Torroella2019,Duffau2004,Rahman2016}.
The European Association of Neuro-Onocology's recommendation, as of 2021, is that prevention of new neurological deficits should be prioritised over maximal extent of resection in the surgical treatment of gliomas\autocite{Weller2021}.

A further consideration on the feasibility of \gls{gtr} or supratotal resection is neuroplasticity\autocite{Duffau2005}.
Slowly growing, low-grade, or recurring tumours may lead to functional reorganisation of surrounding brain tissue\autocite{Takahashi2012,Southwell2016,Das2019} or compensatory recruitment of equivalent contralateral regions\autocite{Mitolo2022}, enabling the safe removal of a greater margin of tissue than would otherwise be accepted for eloquent areas\autocite{Rossi2019a}.
Current understanding of neuroplasticity and brain tumours is limited to a small number of case studies, and more systematic research into the mechanisms and robust detection of functional reorganisation are required before these findings can be put into routine clinical practice\autocite{Duffau2005,Abel2015,Satoer2017}.
Taken together with the emergence of a hodological framework for neurosurgery discussed in Section \ref{sec:hodology}\autocite{Sala2019}, improved study of neuroplasticity could gradually lead to wider applicability of total or supratotal resection without compromising on neurological function and postoperative quality of life.

Early neurological principles of rigid functional localisation formed the basis for the concepts of eloquence and tumour operability guiding neurosurgeons throughout much of recent decades.
The recent move towards a more individualised view has only been made possible through developments in imaging and functional monitoring tools, allowing clinical teams to adapt the surgical strategy to each unique brain-tumour system, rather than relying on received assumptions about functional organisation\autocite{Boerger2023}.
The next section will explore some of those advanced technologies instrumental in the planning and execution of state-of-the-art neurosurgical practice.

% Overall view: moving goal posts, as changes in classifications of glioma and new understanding of mutations and associated risk, changing priorities from topological to hodological mapping and considering neuroplasticity, selection bias, ...

\section{Surgical planning and preoperative imaging}

Tumours can interact with their surroundings in a number of ways, depending on their nature and location.
Some tumours, including some \glspl{lgg} and meningiomas, are fully encapsulated and displace surrounding brain as they grow.
This strong demarcation between tumour and healthy tissue can facilitate surgical treatment and total removal of the tumour without undue risk to functioning neural tissue, but such lesions can cause neurological impairments as surrounding structures are stretched or compressed, leading to a recommendation for surgical removal of a tumour which poses an otherwise lower oncological threat.
Others cause almost no spatial displacement of brain tissue, with cancer cells instead invading the parenchyma and blurring the boundaries between disease and healthy brain.
Infiltrating tumours pose a particular surgical challenge due to the risk of surgical injury to eloquent tissue, and may only be conservatively debulked to relieve intracranial pressure and improve the effectiveness of adjuvant radiation or chemotherapy treatment.
In order to meet the goal of safely balancing maximal resection and functional preservation, the full tumour-brain interaction must be comprehensively mapped to determine the optimal resection margin.

Preoperatively, structural and functional non-invasive imaging are used for diagnosis and, if surgery is indicated, surgical planning.
At this stage the goal is to assess the spatial and functional relationships between involved and healthy tissues, map out a safe operative corridor to access the lesion, and determine the appropriate \gls{eor} under all considerations explored in the previous section.
Structural imaging with \gls{ct} and \gls{mri} provide critical anatomical information in high spatial detail.
Multi-contrast \gls{mri} examinations, including FLAIR, $T_1$-weighted and $T_2$-weighted imaging sequences, each provide unique contrasts for visualising different aspects of a tumour, such as necrotic and infiltrating regions, which can aid in determining tumour type, what \gls{eor} to aim for, or which region of the tumour to target for biopsy.
Angiography, detailed mapping of blood vessels with \gls{ct} and specialised \gls{mri} sequences, can also be employed for determining a tumour's vasculature and identifying major vessels involved\autocite{Kashimura2008,Kim2019}.
\Gls[noindex=false]{fmri} and navigated transcranial magnetic stimulation \autocite{WeissLucas2020} map out eloquent cortex lying in proximity to the lesion, including the motor, language, and sensory cortices, supplementing the purely structural data obtained from conventional \gls{mri} or \gls{ct}.
In epilepsy patients, \gls{eeg} may be employed to monitor activity in epileptogenic regions\autocite{Sarco2006}.

\Gls{dmri} is playing an increasingly important role for neurosurgical planning and navigation\autocite{Manan2022}, especially as focus moves from functional localisation towards viewing the brain as an interconnected network.
Tumour and \gls{wm} interactions are varied and can be difficult to distinguish on conventional contrast \gls{mri} alone.
Depending on the infiltrative nature of a lesion, \gls{wm} tracts may be displaced due to mass effect, invaded while remaining functionally intact, disrupted or destroyed, or experience a combination of effects\autocite{Essayed2017,DSouza2019,Manan2023}.
Diffusion \gls{dmri} can be instrumental in differentiating these circumstances and assessing tract integrity\autocite{Field2004,Manan2023}, but care must be taken to recognise how tumour effects may disturb diffusion patterns and impact the results.
Peritumoural oedema, caused by \gls{bbb} break down and abnormal brain parenchyma fluid regulation\autocite{Ohmura2023}, and invading cells can lead to drastically altered diffusion signal measurements and reduction of anisotropy, complicating their interpretation and inhibiting accurate streamline tracking\autocite{Bulakbas2009,Nimsky2010,Kuhnt2013}.
Nevertheless, \gls{dti} and streamline tractography have brought dramatic improvements to neurosurgical planning, unlocking detailed visualisations of \gls{wm} tracts and their spatial relationships to the surgical target (Fig. \ref{fig:nav}) and even acting as a predictor of postoperative deficits with implications for preoperative patient counselling\autocite{Manan2022}.
This potential was recognised almost immediately, with \gls{dti} and early tractography quickly making their way into clinical practice in the early 2000s \autocite{Lee2001,Mori2002a,Nimsky2005}.
% \paragraph*{White matter mapping}
%%%% --! The below may be picked up as plagiarism; heavily copied from HBM manuscript; check academic regulations !--
In the intervening years, research imaging has largely transitioned to the multi-fibre models and probabilistic algorithms described in Chapter \ref{chap:neuroimaging}, but in clinical practice tractography is frequently still based on \gls{dt} fibre orientation models \autocite{Toescu2020, Yang2021} and deterministic tracking algorithms.
As a notable example, popular neuronavigation platform provider Brainlab's iPlan\textregistered{} (single tensor)\autocite{Brainlab2012} and Elements (dual tensor)\autocite{Sollmann2020a} Fibre Tracking applications (Brainlab AG, Munich, Germany) use a version of the probabilistic \gls{fact} algorithm first proposed in \citeyear{Mori1999} \autocite{Mori1999}.

\begin{SCfigure}[][htb!]
  \includegraphics[width=0.5\textwidth]{chapter_2/neuronav.png}
  \caption[Streamline tractography for neurosurgical planning and navigation]{Idealised demonstration of tractography for neurosurgical planning and navigation in a paediatric patient with a left ependymoma (orange, outlined) involved with the \gls[noindex=false]{or} (green) and \gls{cst} (blue). (Illustrative only, not a depiction of real clinical tractography.)}
  \label{fig:nav}
\end{SCfigure}

Regardless of the particular combination of fibre model, algorithm and tracking criteria, streamline tractography is compromised by weaknesses that can lead to flawed results or interpretations if not accounted for\autocite{Rheault2020, Schilling2022, Schilling2019}.
The same techniques and associated limitations for reconstructing individual \gls{wm} bundles already described apply here too, and can even be exasperated by additional tumour-related effects.
\Gls{dt}-based tractography, already afflicted by low sensitivity in healthy applications, often encounters particular difficulties tracking through oedema and areas of infiltration even where intact and functioning fibres may persist\autocite{Leclercq2010}, leading to missed connections and dangerous blind spots in the very regions at risk during surgery, where accurate navigation is most critical \autocite{Kuhnt2013,Ashmore2020}.
Meanwhile, the high propensity for false positive streamlines typical of probabilistic algorithms can be even more difficult to manage when tumour deformations disturb normal fibre orientations and inhibit accurate placement of \glspl{roi}\autocite{Yang2021}.

Perhaps clinical translation of multi-fibre probabilistic tractography has also been muted on account of its lower ease of use and practicality.
\Gls{dt} acquisitions can have as few as six diffusion-weighted directions, resulting in much shorter scan times compared to full \gls[noindex=false]{hardi} scans.
Deterministic tracking itself is rapid, and the placement of \glspl{roi} need not be as strict as with probabilistic tractography owing to a lower sensitivity to false positives \autocite{ODonnell2017}.
A general lack of availability of the necessary expertise and time limits neurosurgical centres' access to state-of-the-art tractography \autocite{Toescu2020}.
Until recently, commercially available neurosurgical navigation platforms have exclusively supported \gls{dt} modelling and deterministic tractography (a recent exception is the Medtronic Stealth\texttrademark{} S8 Tractography application (Medtronic, USA), which implements \gls{csd}-based tractography\autocite{Pozzilli2023} as well as \gls{dt}).
This lack of readily available alternatives in the neurosurgeon's workflow and certification for safe clinical use is undoubtedly a major factor in the persisting preference for deterministic methods in clinical practice.
Nonetheless, there is growing consensus that (pending appropriate regulatory approval) the clinical community ought to adopt probabilistic, non-\gls{dt} tractography\autocite{Yang2021, Beare2022, Petersen2017}.
There is evidence that this shift is gradually underway, at least in the context of presurgical planning\autocite{Toescu2020}, driven probably by a combination and feedback loop of growing demand and better availability and integration of advanced techniques into the clinical workflow.

\section{Neuronavigation and brain shift}

During the surgical procedure itself, multimodal information streams continue to guide the safest possible resection.
Functional monitoring with \gls[noindex=false]{des} is a crucial component of neurosurgical workflows and widely considered the gold standard for localising neural function after craniotomy.
Electrical current is applied to the cortical surface at increasing strengths\autocite{Saito2015}, and where stimulation elicits a functional response or disruption, the corresponding region is deemed eloquent.
Additionally, stimulation of subcortical white matter behind the resection cavity wall can be used to indicate when surgery should be halted as underlying eloquent structures are approached\autocite{Sala2019}.
\Gls{des} can be utilised in awake or asleep paradigms.
In the former, patients are awakened after craniotomy, and perform structured cognitive tasks involving those cortical hubs that may be at risk while undergoing electrical stimulation.
It is commonly considered for \gls{lgg} treatment within the UK, and to a lesser extent for \gls{hgg}\autocite{WykesV.2017}.
Language function is perhaps the most common target for awake stimulation as well as high-level motor tasks (such as playing an instrument) and vision\autocite{Mazerand2017}.
Awake surgery is technically complex, psycho-cognitively and emotionally demanding of the patient, and not universally tolerated\autocite{Nossek2013a,Wang2019}.
In very young children, awake surgery is rarely possible except in the most cooperative and resilient patients, and with appropriate preparation\autocite{Zolotova2022}.
\Gls{des} may also be performed in asleep patients, albeit limited to assessing motor and somatosensory function, where stimulation elicited somatosensory or motor evoked potentials (SSEPs, MEPs) can be measured in the patient's sensory cortex or muscles\autocite{Stone2019} (although the effects of anaesthesia can limit the sensitivity and accuracy of this approach\autocite{Stone2019,WeissLucas2020}).
There is also a considerable risk of intraoperative seizures, particularly in younger patients, which can lead to failure of awake functional mapping and potential increased risk of postoperative deficits\autocite{Nossek2013,Wang2019,Rigolo2020a}.
The neurological and oncological benefits associated with greater \gls{eor} discussed in Section \ref{sec:eor} have been achieved in large part with the help of awake functional mapping, primarily in adult patients.
It is therefore vital to develop and improve alternatives to cortical mapping to achieve maximal safe resection in all populations and especially in those patients who cannot undergo awake surgery, including some children.

Imaging and functional data acquired in preparation for surgery is not only instrumental to surgical planning, it also serves to guide the surgeon throughout the procedure, providing real-time multidimensional navigational information to supplement their live view through the surgical microscope.
Image guidance can involve simply displaying preoperative imaging and mapping in the theatre, while more advanced systems can also indicate the positions of surgical tools, or overlay imaging information on the microscope view.
This is achieved through stereotactic image guided surgery, which arrived with the introduction of frame-based systems in the later half of the 20th century, later largely giving way to frameless setups for craniotomies\autocite{Sandeman1995}, which are considered more time and cost effective\autocite{Sattur2019}.
In the former case, the patient's head is fixed within a stereotactic frame which guides the positioning of surgical tools, while in modern frameless systems the tools are tracked remotely, most commonly by an infrared camera system\autocite{Sattur2019}.
Fiducial markers, affixed either to the frame or patient, are detected on imaging and registered to the operating room coordinate system, allowing the tools' and patient's positions to be mapped and displayed on imaging in real time.
Intraoperative navigation with preoperative \gls{fmri} and \gls{dti} or tractography can improve \gls{eor} and preservation of critical cortical and subcortical function\autocite{Wu2007,Bello2008,Bello2010d}, particularly when combined with \gls{des} or awake surgery\autocite{Aibar-Duran2020}.
Where awake surgery is contraindicated or abandoned, preoperative functional mapping remains the only guidance available for higher cognitive functions, playing a crucial role in improving surgical care for patients who would not qualify for awake surgery.
\textcite{Rigolo2020a} found that preoperative \gls{fmri} guidance enabled safe resection of tumours or epileptic foci to continue after failed or incomplete \gls{des} function mapping, with no significant difference in postoperative morbidity.

Maximising \gls{eor} has further been significantly improved with the introduction of 5-aminolevulinic acid (5-ALA) guidance.
This compound is administered orally prior to surgery and is converted within cells to protoporphyrin IX (PPIX), which fluoresces when excited by short wavelength light.
Uptake of 5-ALA is highest in tumour cells due to \gls{bbb} disruption, and the metabolic pathways producing PPIX are more active in tumour cells, allowing the surgeon to distinguish them from healthy tissue under the surgical microscope.
5-ALA guided surgery results in improved \gls{eor} and higher progression free survival, while maintaining preservation of functional tissue\autocite{Coburger2019,Golub2020}, and has seen widespread adoption in the surgical treatment of gliomas\autocite{Stummer2006} and inclusion in UK national care guidelines\autocite{NICE2021}.

\begin{SCfigure}[][hp!]
  \includesvg[height=\textheight,pretex=\small\sffamily]{chapter_2/brain_shift.svg}
  \caption[Intraoperative brain shift]{Illustration of intraoperative brain shift and its effect on neuronavigation.
  \textbf{\sffamily a.} Preoperative imaging of a right \gls{who} grade 1 epidermoid lesion patient. Image is a $T_1$ weighted structural scan overlaid with \gls{csd}-derived \gls{dec} map. White arrowhead indicates medial displacement of the \gls{cst} (coloured blue/purple).
  \textbf{\sffamily b.} Intraoperative imaging with partially resected lesion. Brain shift has caused the \gls{cst} to relax laterally towards the craniotomy (white arrowhead).
  \textbf{\sffamily c.} Streamline tractography reconstructions of the \gls{cst} from preoperative (red) and intraoperative (green) \gls{dmri}, with areas of overlap in yellow. Note how the red streamlines give the impression of a tract further from the resection margin.}
  \label{fig:shift}
\end{SCfigure}

The dynamic conditions of brain surgery result in the unpredictable and often substantial movement, compression and deformation of tissue referred to as brain shift.
A range of factors contribute to this phenomenon, including \gls{csf} drainage, sagging due to gravity, decompression of tissue surrounding the resection cavity, swelling, craniotomy herniation and the effects of surgical instruments\autocite{Gerard2017}.
These factors may act in competing directions and combine in complex ways, for example swelling and tumour debulking can cause brain shift towards the craniotomy, while gravity and \gls{csf} drainage may have the opposite effect\autocite{Roberts1998}.
With the magnitude and direction of brain shift being so unpredictable, ranging from 1~mm to as much as 50~mm\autocite{Gerard2017}, accounting for it with predictive modelling is very difficult\autocite{Bayer2017b}.

Brain shift can affect the neurosurgeon's perception of the shape and location of the target lesion and invalidate preoperative imaging used for navigation\autocite{Nimsky2000}.
Many neurosurgeons rely on intuition to update their mental map of the surgical site throughout the procedure.
On more advanced neuronavigational platforms which integrate preoperative imaging and intraoperative data such as \gls{des} stimulation sites, brain shift can lead to misleading and inaccurate depictions of the spatial relationships between tumour and surrounding structures.
In particular, accurate localisation on image-guided navigation systems of deep tumour margins and the functionally eloquent structures beyond is significantly compromised by brain deformation (Fig. \ref{fig:shift}) and cannot be as easily mentally compensated for by the neurosurgeon as visible surface movements\autocite{Nimsky2000}.
Numerous techniques have been proposed to address the problem of brain shift\autocite{Bayer2017b} by dynamically adjusting preoperative imaging with patient specific deformation modelling.
Some rely entirely on preoperative imaging in combination with predictive modelling to simulate deformations, others include sparse or alternative modality intraoperative data, including sparse tracking of cortical surface features\autocite{Luo2019}, optical imaging of the cortical surface\autocite{Skrinjar2002,Audette2005,Fan2017} and intraoperative ultrasound \autocite{Letteboer2005,Reinertsen2007,Bucki2012,Machado2019}, to estimate brain shift and update preoperative imaging accordingly.
Yet \textcite{Yang2017a} found that \gls{wm} tract shift direction was largely independent of cortical surface shift.
Ultimately, the most accurate 3D patient anatomy information can only be obtained with full 3D structural imaging after brain shift has occurred.

% Options without intraop imaging rely heavily on computationally modelling, with either limited accuracy or high computational time

\section{Intraoperative imaging}

To mitigate the effects of brain shift on neuronavigation accuracy, new structural and functional guidance information can be acquired intraoperatively.
Once more, different modalities offer different strengths and weaknesses.
Ultrasound imaging can probe into tissue beyond the surface, is safe, and can be used at the surgical table without needing to move the patient\autocite{Elmesallamy2019,Eljamel2016}.
Doppler ultrasound is particularly useful for detecting intra- and peritumoural vasculature\autocite{Steno2016}, although image quality is limited and can be difficult to compare with other imaging modalities such as \gls{mri}\autocite{Eljamel2016}.
Specialised \gls{ct} systems can also be utilised intraoperatively \autocite{Bayer2018}, but they cause additional patient exposure to ionising radiation which is to be avoided wherever possible.

\Gls[noindex=false]{imri} is becoming an increasingly common and valued addition to neurosurgical set-ups.
This includes low field ($<1$T) open bore systems which can be installed in the operating room, and allow for easy transfer of the patient into the scanner\autocite{Steinmeier1998,Senft2010}, as well as high-field (1.5--3T) systems which acquire far higher quality images with potentially more clinical utility\autocite{Makary2011} at the expense of practicality, as interrupting surgery for an extended scan session and safely transferring a patient from the operating table to inside the scanner bore in an adjoining room is a substantial logistical and medical undertaking\autocite{Senft2010,Giordano2016a,Sattur2019}.
Technical challenges notwithstanding, \gls{imri} is incredibly valuable for determining surgical margins and providing guidance after the effects of brain shift deformations have invalidated preoperative imaging.

Numerous works demonstrate the benefits of \gls{imri} for improving postsurgical outcomes in both tumour and epilepsy surgery, including greater \gls{eor}, fewer new postoperative deficits, greater postoperative seizure freedom, and reduced length of hospital stay in both adults and children
\autocite{Shah2012,Zhang2015a,Sacino2016,Rao2017c,Giordano2017,Lu2018a,Garzon-Muvdi2019,Leroy2019,Karsy2019,Golub2020,Hlavac2020,Englman2021}.
%Giordano2017: safe in children; positioning
Even when using advanced image and surgical guidance, postoperative \gls{mri} may reveal tumour nodules unintentionally left in the brain, concealed in corners of the resection cavity.
In some cases early repeat surgery is required to remove residual disease, resulting in significant additional clinical burden to the patient, longer hospital stays, heightened risk of complications including wound infection\autocite{Tenney1985,Chang2003}, delays in the planning of adjuvant treatment, and greater financial expense\autocite{Shah2012}.
Intraoperative imaging enables these remaining tumour margins to be detected and fully resected within the initial operation\autocite{Sattur2019,Hlavac2020}.
For low-grade lesions, the use of \gls{imri} could mean the difference between a potentially curative procedure and one leaving the patient at risk for recurrence and/or reoperation\autocite{Shah2012}.
Where \gls{imri} indicates no need for further resection, it can replace the need to subject patients to an additional scan immediately following surgery, with all the discomfort and possible sedation, or in children, general anaesthesia, that it would entail.

While there is still some debate over the overall cost to benefit ratio of high-field \gls{imri} systems\autocite{Eljamel2016,Giordano2016a,Giussani2022}, and indeed high initial capital expense remains one of the primary barriers to their installation\autocite{NICE2021}, cost-effectiveness analyses have indicated that upfront investments are recouped by lifetime savings associated with shorter hospital stays, and improved postoperative recovery and survival\autocite{Giordano2016a,Sacino2018}.
Cost-effectiveness can be further increased with a dual use set-up, in which an \gls{imri} system is used both for intraoperative and routine diagnostic scanning\autocite{Giordano2016a}, as is the case at \gls{gosh}.
Further challenges of implementing \gls{imri} include longer operating times, equipment compatibility, and patient positioning for both navigated surgery and scanning\autocite{Giordano2017}.
It is worth noting that the strength of evidence supporting all \gls{imri} use and cost-effectiveness is still under debate \autocite{Jenkinson2018,Garzon-Muvdi2019,Caras2020}, and interpretation of \gls{imri} studies is confounded by selection bias and a lack of randomised control trials\autocite{Kubben2011}.

% Next: skim open papers; discuss conventional contrast imaging, move to other acquisitions (dwi: infarct etc, then more advanced); compare with iop acquisition of dmri with coregistration (deformable) between preop advanced and intraop structural.
% end with technical considerations (in particular for children), costs etc.
% Note that more research needed to assess intraop diffusion mri.

The majority of \gls{imri} sequences are $T_1$-weighted (with or without injected contrast agent enhancement), $T_2$-weighted, and other conventional acquisitions with good tumour tissue contrast\autocite{Kubben2011,Coburger2019}.
In line with this finding, the majority of literature reviewed surrounding \gls{imri} focusses on evaluating \gls{eor}, and in this regard the technique has gradually established itself as a clearly beneficial and in some centres indispensable surgical aid\autocite{Garzon-Muvdi2019,Hlavac2020}.
What has been less extensively studied is the potential for \gls{imri} to provide updated advanced functional neuronavigation.
Broadly speaking, this can be approached in two ways.
The first is to use conventional \gls{imri} to dynamically adjust preoperative functional and connectivity information (including \gls{fmri} and tractography) using deformable image registration and/or biomechanical brain shift modelling, the second is to directly acquire new advanced \gls{mri} sequences intraoperatively.
While the former may seem like the preferred choice, as one avoids having to acquire and process additional lengthy scans and further prolonging surgery interruption, achieving robust deformable registration in a reasonable timeframe is by no means trivial.
It is especially complicated given the significantly disturbed anatomy typically observed following craniotomy and complex brain shift, as well as the effects of air in the resection cavity not present on preoperative imaging.
Intraoperative \gls{fmri} has been reported in a very limited number of studies in \gls{dbs}\autocite{Hiss2015,Knight2015}, tumour resection\autocite{Roder2016a,Qiu2017a}, and neurovascular\autocite{Muscas2019} surgeries, but it's unlikely to find widespread use on account of practicality and an established preference for \gls{des} for cortical mapping.
By contrast, intraoperative \gls{dmri} has several potential advantages.

Use of tractography for neuronavigation is valued by many, but its accuracy diminishes with increasing brain shift, with precision most compromised during later stages of an operation, at the same time as the resection is potentially approaching critical subcortical \gls{wm}\autocite{Yang2019} (Fig. \ref{fig:shift}).
Some account for this by registering preoperative tractography or \gls{dti} \gls{dec} maps to intraoperative structural \gls{mri}\autocite{Nimsky2006a,Tamura2022}, which depends on robust and accurate registration\autocite{Beare2016}.
Alternatively, intraoperative acquisition of \gls{dti} and even \gls{hardi} scans has been garnering interest.
Early work by \textcite{Nimsky2005}$^,$\autocite{Nimsky2005a} demonstrated the feasibility of intraoperative \gls{dmri} and tractography for illustrating \gls{wm} tract shifting after tumour resection.
Safe resection of tumours and epilepsy foci aided by intraoperative reconstruction of motor\autocite{Maesawa2010,Javadi2017}, language\autocite{DAndrea2016,Li2021} and visual\autocite{Daga2012,Cui2015} pathways has been confirmed in subsequent studies.
Overall, a systematic review by \citeauthor{Aylmore} (in production)\autocite{Aylmore} found 26 articles reporting intraoperative \gls{dti} or tractography and associated outcome measures, with moderate suggestion of benefit to surgical success.

There are considerable technical considerations associated with intraoperative \gls{dmri}, even more so than with routine imaging.
Duration is of course one, with minimum possible scan time a priority for interrupted procedures.
Secondly, intraoperative \gls{dmri} can suffer from degraded image quality\autocite{Roder2019}, and distortion artefacts common in \gls{dmri} sequences utilising \gls{epi} are exasperated by the tissue--air interface at the craniotomy site\autocite{Elliott2020}, which can significantly diminish the accuracy of \gls{wm} tract localisation\autocite{Yang2022}.
Early studies suggested that intraoperative tract reconstructions may not be reliable on their own, but useful in combination with adjuvant functional mapping such as \gls{des}\autocite{Ostry2013}.
Nearly all current implementations of intraoperative tractography are limited to commercially available \gls{dt}-based deterministic algorithms, which have known limitations particularly in application to pathology.
Nevertheless, concrete evidence of benefits to postoperative outcomes with intraoperative advanced \gls{dmri} is mounting\autocite{DAndrea2012,Cui2015,Maesawa2010}, and in combination with what we already know about conventional \gls{imri} and preoperative imaging for \gls{wm} neuronavigation, it stands to reason that intraoperative \gls{dmri} could bring significant improvements to brain surgery if some of the technical and image processing limitations can be overcome.

All neuronavigational tools are complementary, each providing distinct streams of information, and where possible, combining techniques can increase the chances of achieving an optimal outcome.
For example, maximal safe resection with combined 5-ALA and \gls{imri} guidance is recommended by UK national clinical guidelines for surgical treatment of gliomas, which further advise considering the use of \gls{dti}, if available, and awake craniotomy to optimise safety and effectiveness\autocite{NICE2021}.
Reviews have found similar benefits to \gls{eor} from both \gls{imri} and 5-ALA guidance separately\autocite{Golub2020} and even better outcomes when combined\autocite{Nickel2018,Coburger2019}.
5-ALA provides the surgeon with a direct visualisation of tumour cells in the surgical view, but residual nodules may remain out of sight behind cavity convexities and only show up on \gls{imri}\autocite{SueroMolina2019}.
Combining awake surgery with \gls{imri} is also viable and beneficial in some patients \autocite{Motomura2017,Tuleasca2021}.
In summary, advanced functional mapping, presurgical planning, and intraoperative monitoring, where the appropriate resources and expertise are available\autocite{GeorgeZakiGhali2020}, all increase the operability of highly eloquent gliomas
\autocite{Bello2008,Krieg2013,DellaPuppa2013b,Magill2018}.
