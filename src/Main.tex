% UCL Thesis LaTeX Template
%  (c) Ian Kirker, 2014
%
% This is a template/skeleton for PhD/MPhil/MRes theses.
%
% It uses a rather split-up file structure because this tends to
%  work well for large, complex documents.
% We suggest using one file per chapter, but you may wish to use more
%  or fewer separate files than that.
% We've also separated out various bits of configuration into their
%  own files, to keep everything neat.
% Note that the \input command just streams in whatever file you give
%  it, while the \include command adds a page break, and does some
%  extra organisation to make compilation faster. Note that you can't
%  use \include inside an \include-d file.
% We suggest using \input for settings and configuration files that
%  you always want to use, and \include for each section of content.
% If you do that, it also means you can use the \includeonly statement
%  to only compile up the section you're currently interested in.
% You might also want to put figures into their own files to be \input.

% For more information on \input and \include, see:
%  http://tex.stackexchange.com/questions/246/when-should-i-use-input-vs-include


% Formatting and binding rules for theses are here:
%  https://www.ucl.ac.uk/students/exams-and-assessments/research-assessments/format-bind-and-submit-your-thesis-general-guidance

% This package goes first and foremost, because it checks all
%  your syntax for mistakes and some old-fashioned LaTeX commands.
% Note that normally you should load your documentclass before
%  packages, because some packages change behaviour based on
%  your document settings.
% Also, for those confused by the RequirePackage here vs usepackage
%  elsewhere, usepackage cannot be used before the documentclass
%  command, while RequirePackage can. That's the only functional
%  difference as far as I'm aware.
\RequirePackage[l2tabu, orthodox]{nag}


% ------ Main document class specification ------
% The draft option here prevents images being inserted,
%  and adds chunky black bars to boxes that are exceeding
%  the page width (to show that they are).
% The oneside option can optionally be replaced by twoside if
%  you intend to print double-sided. Note that this is
%  *specifically permitted* by the UCL thesis formatting
%  guidelines.
\documentclass[12pt,phd,a4paper,oneside,draft]{ucl_thesis}

% Package configuration:
% -------- Packages --------

% This package means empty pages (pages with no text) won't get stuff
%  like chapter names at the top of the page. It's mostly cosmetic.
\usepackage{emptypage}

% The graphicx package adds the \includegraphics command,
%  which is your basic command for adding a picture.
\usepackage{graphicx}

% The float package improves LaTeX's handling of floats,
%  and also adds the option to *force* LaTeX to put the float
%  HERE, with the [H] option to the float environment.
\usepackage{float}

% The amsmath package enhances the various ways of including
%  maths, including adding the align environment for aligned
%  equations.
\usepackage{amsmath}
\usepackage{amssymb}


% Use these two packages together -- they define symbols
%  for e.g. units that you can use in both text and math mode.
\usepackage{gensymb}
\usepackage{textcomp}
% You may also want the units package for making little
%  fractions for unit specifications.
%\usepackage{units}


% The setspace package lets you use 1.5-sized or double line spacing.
\usepackage{setspace}
\setstretch{1.35}

% That just does body text -- if you want to expand *everything*,
%  including footnotes and tables, use this instead:
%\renewcommand{\baselinestretch}{1.5}


% PGFPlots is either a really clunky or really good way to add graphs
%  into your document, depending on your point of view.
% There's waaaaay too much information on using this to cover here,
%  so, you might want to start here:
%   http://pgfplots.sourceforge.net/
%  or here:
%   http://pgfplots.sourceforge.net/pgfplots.pdf
%\usepackage{pgfplots}
%\pgfplotsset{compat=1.3} % <- this fixed axis labels in the version I was using

% PGFPlotsTable can help you make tables a little more easily than
%  usual in LaTeX.
% If you're going to have to paste data in a lot, I'd suggest using it.
%  You might want to start with the manual, here:
%  http://pgfplots.sourceforge.net/pgfplotstable.pdf
%\usepackage{pgfplotstable}

% These settings are also recommended for using with pgfplotstable.
%\pgfplotstableset{
%	% these columns/<colname>/.style={<options>} things define a style
%	% which applies to <colname> only.
%	empty cells with={--}, % replace empty cells with '--'
%	every head row/.style={before row=\toprule,after row=\midrule},
%	every last row/.style={after row=\bottomrule}
%}


% Alternatively, you can use the ifdraft package to let you add
%  commands that will only be used in draft versions
\usepackage{ifdraft}
\ifdraft{
  % Draft mode geometry
  \usepackage[margin=1in]{geometry}
  \setstretch{1}
  \setlength \topmargin{10mm}
  \setlength \oddsidemargin {20mm} % Allow a mm for the bleed.
  \setlength \evensidemargin {20mm}
  % Line numbers
  \usepackage[modulo]{lineno}
  \linenumbers
}


% The multirow package adds the option to make cells span
%  rows in tables.
\usepackage{multirow}


% Subfig allows you to create figures within figures, to, for example,
%  make a single figure with 4 individually labeled and referenceable
%  sub-figures.
% It's quite fiddly to use, so check the documentation.
%\usepackage{subfig}

% The natbib package allows book-type citations commonly used in
%  longer works, and less commonly in science articles (IME).
% e.g. (Saucer et al., 1993) rather than [1]
% More details are here: http://merkel.zoneo.net/Latex/natbib.php
%\usepackage{natbib}

% The bibentry package (along with the \nobibliography* command)
%  allows putting full reference lines inline.
%  See:
%   http://tex.stackexchange.com/questions/2905/how-can-i-list-references-from-bibtex-file-in-line-with-commentary
\usepackage{bibentry}

% The isorot package allows you to put things sideways
%  (or indeed, at any angle) on a page.
% This can be useful for wide graphs or other figures.
%\usepackage{isorot}

% The caption package adds more options for caption formatting.
% This set-up makes hanging labels, makes the caption text smaller
%  than the body text, and makes the label bold.
% Highly recommended.
\usepackage[format=hang,font=small,labelfont=bf]{caption}
\usepackage{subcaption}
\usepackage{sidecap}
\sidecaptionvpos{figure}{t}
% If you're getting into defining your own commands, you might want
%  to check out the etoolbox package -- it defines a few commands
%  that can make it easier to make commands robust.
\usepackage{etoolbox}

% The microtype package adds `micro-typographic extensions' which
% generally makes text more readable and hyphenation less likely.
\usepackage{microtype}

% ---- Other Packages (PERSONALISED)

\usepackage{pdflscape}
\usepackage{rotating}


% For multicolumns
%\usepackage{multicol}
%\setlength{\columnsep}{.7cm}

% separate paragraphs with empty lines
\usepackage[parfill]{parskip}
\setlength{\parskip}{1\baselineskip plus 1pt minus 1pt}

% Create paragraph title format
\usepackage{titlesec}
\titleformat{\paragraph}[hang]{\normalfont\itshape\raggedright}{}{0pt}{\qquad}[]
\titlespacing*{\paragraph}{0pt}{0.5\baselineskip}{0.5\baselineskip}

% Section header spacing
\titlespacing*{\section}{0pt}{3\baselineskip}{\baselineskip}
\titlespacing*{\subsection}{0pt}{2\baselineskip}{\baselineskip}
\titlespacing*{\subsubsection}{0pt}{\baselineskip}{\baselineskip}

% Code listings
\usepackage{verbatim}
\usepackage{spverbatim} % not sure what this is for but oh well

% For formatting matlab code specifically, **MUST COPY mcode.sty FILE INTO FOLDER**
%\usepackage[framed,numbered,autolinebreaks,useliterate]{mcode}
\usepackage[final]{listings} % Use like this in place: \lstinputlisting{filname.ext}

% Enumeration, lists
\usepackage{enumerate, enumitem}
\usepackage{framed, color}

% Better tables
\usepackage{tabularx, bigstrut, multirow, booktabs, array}
\newcolumntype{+}{>{\global\let\currentrowstyle\relax}}
\newcolumntype{^}{>{\currentrowstyle}}
\newcommand{\rowstyle}[1]{\gdef\currentrowstyle{#1}%
    #1\ignorespaces
}% allows formatting of a whole row
\usepackage[table]{xcolor}

% Footnotes are symbols rather than numeric
\renewcommand{\thefootnote}{\fnsymbol{footnote}}


\raggedbottom
\widowpenalty10000
\clubpenalty10000
\interfootnotelinepenalty10000

% Sets up links within your document, for e.g. contents page entries
%  and references, and also PDF metadata.
% You should edit this!
\input{setup/LinksAndMetadata}

% And then some settings in separate files.
% These settings are partly from:
%  http://mintaka.sdsu.edu/GF/bibliog/latex/floats.html

% They give LaTeX more options on where to put your figures, and may
%  mean that fewer of your figures end up at the tops of pages far
%  away from the thing they're related to.

% Alters some LaTeX defaults for better treatment of figures:
% See p.105 of "TeX Unbound" for suggested values.
% See pp. 199-200 of Lamport's "LaTeX" book for details.

%   General parameters, for ALL pages:
\renewcommand{\topfraction}{0.9}	% max fraction of floats at top
\renewcommand{\bottomfraction}{0.8}	% max fraction of floats at bottom

%   Parameters for TEXT pages (not float pages):
\setcounter{topnumber}{2}
\setcounter{bottomnumber}{2}
\setcounter{totalnumber}{4}     % 2 may work better
\setcounter{dbltopnumber}{2}    % for 2-column pages
\renewcommand{\dbltopfraction}{0.9}	% fit big float above 2-col. text
\renewcommand{\textfraction}{0.2}	% page must be at least 20% text,
%                                  less than that and we get a floatpage

%   Parameters for FLOAT pages (not text pages):
\renewcommand{\floatpagefraction}{0.7}	% require fuller float pages
% N.B.: floatpagefraction MUST be less than topfraction !!
\renewcommand{\dblfloatpagefraction}{0.7}	% require fuller float pages

% remember to use [htp] or [htpb] for placement,
% e.g.
%  \begin{figure}[htp]
%   ...
%  \end{figure}

\usepackage[inkscapearea=page,
            inkscapeexe=/Applications/Inkscape.app/Contents/MacOS/inkscape,
            draft=false,
            inkscapelatex=false]{svg}

\ifdraft{%
  % Figure path
  \graphicspath{{draft_figs/}{figs/}}
  \svgpath{{draft_figs/}{figs/}}
  % Endfloats with chapter counters
  \usepackage[nolists, figuresonly]{endfloat}
}{%
  \graphicspath{{figs/}{draft_figs/}}
  \svgpath{{figs/}{draft_figs/}}
  \usepackage[disable]{endfloat}
}

\makeatletter%
  \@ifpackagewith{endfloat}{disable}{%
    \setkeys{Gin}{draft=false}}{%
    \renewcommand{\efloatseparator}{}
    \renewcommand\thepostfigure{\arabic{chapter}.\arabic{postfig}}
    \AddToHook{cmd/chapter/before}{%
      \processdelayedfloats
      \setcounter{postfig}{0}}}
\makeatother%
 % For things like figures and tables
%\bibliographystyle{unsrt}

% Bibliography and referencing settings

\usepackage[doi=true,isbn=true,url=false,natbib=true,style=nature,backend=bibtex]{biblatex} % Add citation styles as needed (see docs for biblatex)
\IfFileExists{bibliographies/Neuroimaging.bib}
          {\addbibresource{bibliographies/Neuroimaging.bib}}
          {\addbibresource{bibs/Neuroimaging.bib}}
\IfFileExists{bibliographies/Neurosurgery.bib}
          {\addbibresource{bibliographies/Neurosurgery.bib}}
          {\addbibresource{bibs/Neurosurgery.bib}}
\IfFileExists{bibliographies/Figures.bib}
          {\addbibresource{bibliographies/Figures.bib}}
          {\addbibresource{bibs/Figures.bib}}
% \addbibresource{bibliographies/Neuroimaging.bib}
% \addbibresource{./Neurosurgery.bib}


% Make DOIs look normal
\urlstyle{same}

 % Path to bib file (e.g. generated from Mendeley)
% \renewcommand*{\bibfont}{\small} % Make bibligraphy small

% Use the below in conjunction with nature style referencing to get proper behaviour of textcite command (numeric citation still superscript rather than bracket)
\usepackage{ifthen}
\renewcommand{\textcite}[2][]{
\ifthenelse { \equal {#1} {} }  %
    {\citeauthor{#2}\autocite{#2}}   % if #1 == blank
    {\citeauthor{#1}\autocite{#2}}}
   % For bibliographies
%%
%% Set up abbreviations and definitions
\usepackage[acronym,
            style=super,
            nogroupskip=true,
            nonumberlist,
            nopostdot]
            {glossaries}
\makeglossaries

%% MRI
\newacronym{mri}{MRI}{magnetic resonance imaging}
% Recursive acronym \protect solution from https://tex.stackexchange.com/a/502313
\newacronym{dmri}{dMRI}{diffusion \protect\ifglsused{mri}{MRI}{magnetic resonance imaging}}
\newacronym{csd}{CSD}{constrained spherical deconvolution}
\newacronym{odf}{ODF}{orientation distribution function}
\newacronym{fodf}{fODF}{fibre \protect\ifglused{odf}{ODF}{orientation distribution function}}
\newacronym{fod}{FOD}{fibre orientation distribution}
\newacronym{tod}{TOD}{track orientation distribution}
\newacronym{dti}{DTI}{diffusion tensor imaging}
\newacronym{dt}{DT}{diffusion tensor}
\newacronym{fa}{FA}{fractional anisotropy}
\newacronym{ssst}{SSST}{single-shell, single-tissue}
\newacronym{msmt}{MSMT}{multi-shell, multi-tissue}
%% Neuroanatomy
\newacronym{cst}{CST}{corticospinal tract}
\newacronym{or}{OR}{optic radiation}
\newacronym{af}{AF}{arcuate fasciculus}
\newacronym{ifof}{IFOF}{inferior fronto-occipital fasciculus}
\newacronym{uf}{UF}{uncinate fasciculus}
\newacronym{wm}{WM}{white matter}
\newacronym{gm}{GM}{grey matter}
%% Other
\newacronym{hcp}{HCP}{human connectome project}
\newacronyn{dice}{DSC}{Dice similarity coefficient}
\newacronym{gdice}{gDSC}{generalised Dice similarity coefficient}

%% Here we set up a new command to generate a list of abbreviations, e.g. for
%% float captions, formatted in a consistent way

% Define the separtors between abbreviation and definition, and between list items
% including any spaces!
\def\acrodefsep{ = }
\def\acrolistdelim{; }

% Process list of arbitratry length
\NewDocumentCommand{\acrolist}{>{\SplitList{,}}m}
    {\ProcessList{#1}{\acroformat}\firstitemtrue}
% Format the list using each item's glossary entry and the defined separators
\newif\iffirstitem
\firstitemtrue
\newcommand\acroformat[1]{%
  \iffirstitem
    \firstitemfalse
  \else
    \acrolistdelim %
  \fi
  \glsname{#1}\acrodefsep \glsdesc{#1}}
% ^ proper hangling of delimiter solution from https://tex.stackexchange.com/a/110906

%% Alternative solution from https://tex.stackexchange.com/a/110909 if the above causes issues
% \NewDocumentCommand{\acrolist}{>{\SplitList{,}}m}
%     {%
%     \def\acrodelim{\def\acrodelim{, }}%
%     \ProcessList{#1}{\acroformat} }
% \newcommand{\acroformat}[1]{\acrodelim\glsentryshort{#1} = \glsentrylong{#1}}

\newcommand\epigraph[2]{%
  \begin{flushright}
    \parbox{0.75\textwidth}{\raggedleft #1}
    \vskip 1.5\baselineskip
    --- #2
  \end{flushright}%
}

\newcommand\epipage[2]{%
  \clearpage\thispagestyle{empty}
  \vspace*{\fill}
  {\sffamily\epigraph{#1}{#2}}
  \vspace*{\fill}\pagebreak
}

\newcommand\epart[2][]{%
  \cleardoublepage          % Page break
  \thispagestyle{empty}
  \vspace*{.3\vsize}        % Vertical shift
  \refstepcounter{part}
  \addcontentsline{toc}{part}{#2}%
  % Part title, center, 1/3 down
  {\centering \textbf{\sffamily\Huge#2}\par}%
  \vspace*{\fill}
  {\sffamily#1}
  \vspace*{\fill}\pagebreak
}


% These control how many number sections your subsections will take
%    e.g. Section 2.3.1.5.6.3
%  and how many of those will get put into the contents pages.
\setcounter{secnumdepth}{3}
\setcounter{tocdepth}{3}

\newcommand\note[1]{\ifdraft{{\large\textcolor{red}{#1}}}{}}
%TC:macro \note [ignore]

\begin{document}

% \nobibliography*
% ^-- This is a dumb trick that works with the bibentry package to let
%  you put bibliography entries whereever you like.
% I used this to put references to papers a chapter's work was
%  published in at the end of that chapter.
% For more information, see: http://stefaanlippens.net/bibentry


\ifdraft{%
  \setcounter{tocdepth}{5}
  \tableofcontents
  \renewcommand{\part}[1]{}}{%
  % I may change the way this is done in a future version,
%  but given that some people needed it, if you need a different degree title
%  (e.g. Master of Science, Master in Science, Master of Arts, etc)
%  uncomment the following 3 lines and set as appropriate (this *has* to be before \maketitle)
% \makeatletter
% \renewcommand {\@degree@string} {Master of Things}
% \makeatother

\title{ Fibre tract imaging with intraoperative \gls{dmri} for neurosurgical navigation }
\author{ Fiona Young }
\department{ Department of Medical Physics and Biomedical Engineering \\ UCL Great Ormond Street Institute of Child Health }

\maketitle
\makedeclaration

\begin{abstract} % 300 word limit
Mapping and understanding the brain's structure and function is never more critical than when it suffers injury or illness.
Lifesaving neurosurgical procedures may put critical neural communication pathways called \gls{wm} tracts at risk, with grave consequences for the patient, so accurately depicting their location using \gls{dmri} is becoming a key component of modern neurosurgical practice.
More recently, obtaining new \gls{imri} partway through surgery has demonstrated potential to further improve outcomes by providing updated anatomical information after the dynamic effects of intraoperative brain shift have diminished the accuracy of preoperative imaging.
With the ability to sample directional water diffusivity in tissue, \gls{dmri} data is processed to produce millimetre-scale maps of \gls{wm} fibre orientations which are key to reconstructing individual tracts.
However, established image computational methods suffer from limitations in accuracy and practicality which restrict the wider clinical uptake of \gls{dmri} \gls{wm} imaging generally, and particularly for \gls{imri}.
After an in depth review of the state of the art in \gls{wm} imaging and image-guided neurosurgery, this thesis explores the development of a novel \gls{wm} tract mapping tool, named \textit{tractfinder}, which applies \textit{a priori} anatomical knowledge encoded within a statistical tract orientation and location atlas to achieve rapid tract segmentation in a patient \gls{dmri} scan.
The proposed pipeline includes explicit patient-specific modelling of tumour deformation effects, an element missing from many research-oriented tract reconstruction approaches.
Tractfinder's effectiveness in a range of applications is detailed through thorough quantitative evaluation, while clinical case studies demonstrate its key advantages over existing approaches.
In additional, the technical and practical challenges of intraoperative imaging are explored together with their implications for effective clinical translation of advanced \gls{dmri}-based \gls{wm} imaging.
\end{abstract}

\begin{impactstatement}

This thesis explores the use of \gls{dmri}'s ability to map brain fibre pathways (``tracts") in neurosurgical planning and guidance, in in particular the potential for more advanced \gls{dmri} analysis methods than those currently under routine clinical use to improve outcomes for neurological patients.

Though intraoperative \gls{mri} has gradually been recognised as a beneficial technology for improving neurosurgery, the use of high angular resolution \gls{dmri} specifically has not been widely adopted, partly because the associated potential benefits have yet to be determined.
The methods described in this thesis may make \gls{dmri} analysis more accessible to intraoperative \gls{mri} procedures, which would allow their advantages to be systematically studied.
Over the longer term, intraoperative \gls{dmri} may, if found to be beneficial, be incorporated into national healthcare guidelines, as preoperative \gls{dti} and general \gls{mri} currently are.
In addition, the proposal of an alternative to streamline tractography, the current standard for \gls{wm} tract imaging, which relies less on the availability of domain experts, time, and computational resources, could widen the availability of tract imaging for preoperative planning to more health centres, who may have the requisite equipment (\gls{mri} capabilities) but lack expertise in applying modern research methods.

The tract orientation atlas framework put forth here could inform to the study of \gls{wm} architecture and connectivity in individuals and across populations, an area in which we are still lacking a lot of understanding, and which will be critical in the future study and combatting of psychological and neurological disorders.
It also contributes to current discussions on how to adopt modern data science methods and innovation in a field, medical imaging, where appropriate data is often scarce and difficult to annotate.
Brain shift is an intraoperative phenomenon which presents a key challenge to image guided surgery techniques.
The tumour deformation modelling described in this thesis represents a novel solution to achieving acceptable \gls{wm} tract segmentations on both pre- and intraoperative imaging, in patients with deforming tumours, which may be applicable to other imaging modalities as well as to tumours outside of the brain.

Findings presented in this thesis have been shared with the wider research community at several international meetings, and have led to funding being granted for a two year follow-on prospective study to apply these methods in intraoperative cases at \gls{gosh}, London.
All image processing tools developed for this thesis, including data and code for tumour deformation modelling and tract segmentation, have been made available for researchers to apply in their own data.
In addition, a patent application covering much of the methodology has been submitted and is currently under examination.
If granted, the intellectual property could be licensed to national or international commercial producers of neuronavigational equipment.

\end{impactstatement}

\begin{acknowledgements}
THANKS!
\end{acknowledgements}

%	Self-plagiarism declaration form template for these typeset in LaTeX
%	Prepared by David Sheard 2022 and made available free of copyright

%	If you use the results of your own published, accepted or submitted data (text or figures) in your final
%	doctoral thesis, you have to give a clear indication of the previous work, stating the exact source of the
%	previous material, irrespective of whether copyright is owned by you or by a publisher. This indication
%	should take the form of
%		a) an appropriate citation of the original source in the relevant Chapter; and
%		b) completion of the UCL Research Paper Declaration form---this should be embedded after the
%		Acknowledgements page in the thesis.

%	For more information consult the following links:
%	\url{https://www.grad.ucl.ac.uk/essinfo/guidance-on-selfplagiarism/?utm_source=Students\%27+Union+UCL\&utm_campaign=2ed9e73ab7-\&utm_medium=email\&utm_term=0_fe8c0cbcf2-2ed9e73ab7-209240456\&mc_cid=2ed9e73ab7\&mc_eid=0496c22bfc}
%	\url{https://www.grad.ucl.ac.uk/essinfo/guidance-on-selfplagiarism/Declaration-form_published-work-in-thesis.docx}

%	I recommend using this template by simply copying everything between \begin{document} and \end{document}
%	into your thesis after the acknowledgements section. By changing the preamble appropriately this template
%	may also be altered to work using \input{...} or the subfiles package.

% \documentclass[12pt, twoside]{article}
%
% \usepackage[a4paper,inner=40mm,outer=20mm, top=30mm, bottom=30mm]{geometry}
% \usepackage{amssymb}
% \usepackage{array}
% \usepackage{setspace}
% 	\setstretch{1.5}
%
% \begin{document}
{\sffamily
\section*{UCL Research Paper Declaration Form: referencing the doctoral candidate’s own published work(s)}
% Uncomment the following line if you would like to add this declaration to your table of contents
% \addcontentsline{toc}{section}{UCL Research Paper Declaration Form}
%
% Please use this form to declare if parts of your thesis are already available in another format,
% e.g. if data, text, or figures:
% •	have been uploaded to a preprint server
% •	are in submission to a peer-reviewed publication
% •	have been published in a peer-reviewed publication, e.g. journal, textbook.
%
% This form should be completed as many times as necessary. For instance, if a student had seven
% thesis chapters, two of which having material which had been published, they would complete this form twice.

\begin{enumerate}[leftmargin=*,label={\bfseries\arabic*.}]\itemsep0em
	\item \textbf{For a research manuscript that has already been published} (if not yet published, please skip to section 2)\textbf{:}
	%
	\begin{enumerate}[label={\alph*)}]\itemsep0em
	%
	\item \textbf{What is the title of the manuscript?}

	\citetitle{Young2022}

	\item \textbf{Please include a link to or doi for the work:}

	\url{https://doi.org/10.1007/s11548-022-02617-z}

	\item \textbf{Where was the work published?}

	\citefield{Young2022}{journaltitle}% Answer here: e.g. journal name

	\item \textbf{Who published the work?}

	\citelist{Young2022}{publisher}% Answer here: e.g. Elsevier/Oxford University Press

	\item \textbf{When was the work published?}

	\citefield{Young2022}{year}% Answer here

	\item \textbf{List the manuscript's authors in the order they appear on the publication:}
	\citeauthor*{Young2022}% Answer here

	\item \textbf{Was the work peer reviewd?}

	Yes

	\item \textbf{Have you retained the copyright?}

	Yes, from the Licence to Publish agreement:
	\begin{quote}
		Ownership of copyright in the Article will be vested in the name of the Author.
	\end{quote}% Answer here

	\item \textbf{Was an earlier form of the manuscript uploaded to a preprint server (e.g. medRxiv)? If `Yes’, please give a link or doi}

	No% Answer here:
	\\
	If ‘No’, please seek permission from the relevant publisher and check the box next to the below statement:
	%
		\begin{itemize}\itemsep0em
		% To check this box, replace \Box with \boxtimes
		\item[$\boxtimes$] {\itshape I acknowledge permission of the publisher named under 1d to include in this thesis portions of the publication named as included in 1c.}
		\end{itemize}
	%
	\end{enumerate}
%
\item \textbf{For a research manuscript prepared for publication but that has not yet been published} (if already published, please skip to section 3)\textbf{:}
%
\begin{enumerate}[label={\alph*)}]\itemsep0em
	%
	\item \textbf{What is the current title of the manuscript?}
	% Answer here:
	\item \textbf{Has the manuscript been uploaded to a preprint server `e.g. medRxiv'?
	\\
	If `Yes', please please give a link or doi:}
	% Answer here:
	\item \textbf{Where is the work intended to be published?}
	% Answer here: e.g. journal name
	\item \textbf{List the manuscript's authors in the intended authorship order:}
	% Answer here
	\item \textbf{Stage of publication:}
	% answer here: e.g. in submission
	%
\end{enumerate}

\item \textbf{For multi-authored work, please give a statement of contribution covering all authors} (if single-author, please skip to section 4)\textbf{:}
\begin{description}[font=\sffamily]
	\item[Fiona Young] Methodology (conceptualisation and implementation), analysis, manuscript original draft and graphics
	\item[Kristian Aquilina] Supervision, manuscript review and editing
	\item[Chris A. Clark] Supervision, manuscript review and editing
	\item[Jonathan D. Clayden] Conceptualisation, supervision, manuscript review and editing
\end{description}
% Answer here
\item \textbf{In which chapter(s) of your thesis can this material be found?}
% Answer here

Chapters \ref{chap:reg}, \ref{chap:applications}

\end{enumerate}

\textbf{e-Signatures confirming that the information above is accurate}
(this form should be co-signed by the supervisor/ senior author unless this is not appropriate, e.g. if the paper was a single-author work)\textbf{:}\\
\textbf{}\\
\textbf{Candidate:}\\
\textbf{Date:}\\
% this form should be co-signed by the supervisor/ senior author unless this is not appropriate, e.g. if the paper was a single-author work):
\textbf{}\\
\textbf{Supervisor/Senior Author signature} (where appropriate)\textbf{:}\\
\textbf{Date:}
%

%%%%%%%%%%%%%%%%%%%%%%%%%%%%% --- HBM --- %%%%%%%%%%%%%%%%%%%%%%%%%%%%%%%%%%%%%%
\newpage
\begin{enumerate}[leftmargin=*,label={\bfseries\arabic*.}]\itemsep0em
%
\item \textbf{For a research manuscript that has already been published} (if not yet published, please skip to section 2)\textbf{:}
%
\begin{enumerate}[label={\alph*)}]\itemsep0em
	%
	\item \textbf{What is the title of the manuscript?}

	\citetitle{Young2024}

	\item \textbf{Please include a link to or doi for the work:}

	\citefield{Young2024}{doi}

	\item \textbf{Where was the work published?}

	\citefield{Young2024}{journaltitle}% Answer here: e.g. journal name

	\item \textbf{Who published the work?}

	\citelist{Young2024}{publisher}% Answer here: e.g. Elsevier/Oxford University Press

	\item \textbf{When was the work published?}

	\citefield{Young2024}{year}% Answer here

	\item \textbf{List the manuscript's authors in the order they appear on the publication:}

	\citeauthor*{Young2024}% Answer here

	\item \textbf{Was the work peer reviewd?}

	Yes

	\item \textbf{Have you retained the copyright?}

	Yes, from license agreement:
	\begin{quote}
		The Author and each Co-author or, if applicable, the Author’s or Co-author’s employer, retains all proprietary rights, such as copyright
	\end{quote}

	\item \textbf{Was an earlier form of the manuscript uploaded to a preprint server (e.g. medRxiv)? If `Yes’, please give a link or doi}

	No% Answer here:
	\\
	If ‘No’, please seek permission from the relevant publisher and check the box next to the below statement:
	%
	\begin{itemize}\itemsep0em
	% To check this box, replace \Box with \boxtimes
	\item[$\boxtimes$] {\itshape I acknowledge permission of the publisher named under 1d to include in this thesis portions of the publication named as included in 1c.}
	\end{itemize}
	%
\end{enumerate}
%
\item \textbf{For a research manuscript prepared for publication but that has not yet been published} (if already published, please skip to section 3)\textbf{:}
%
\begin{enumerate}[label={\alph*)}]\itemsep0em
	%
	\item \textbf{What is the current title of the manuscript?}
	% Answer here:
	\item \textbf{Has the manuscript been uploaded to a preprint server `e.g. medRxiv'?
	\\
	If `Yes', please please give a link or doi:}
	% Answer here:
	\item \textbf{Where is the work intended to be published?}
	% Answer here: e.g. journal name
	\item \textbf{List the manuscript's authors in the intended authorship order:}
	% Answer here
	\item \textbf{Stage of publication:}
	% answer here: e.g. in submission
	%
\end{enumerate}

\item \textbf{For multi-authored work, please give a statement of contribution covering all authors} (if single-author, please skip to section 4)\textbf{:}
\begin{description}[font=\sffamily]
	\item[Fiona Young] Methodology (conceptualisation and implementation), analysis, manuscript original draft and graphics
	\item[Kristian Aquilina] Supervision, conceptualisation, manuscript review and editing
	\item[Laura Mancini] Data contribution, manuscript review and editing
	\item[Kiran K. Seunarine] Data contribution and curation
	\item[Chris A. Clark] Supervision, manuscript review and editing
	\item[Jonathan D. Clayden] Conceptualisation, supervision, manuscript review and editing
\end{description}
% Answer here
\item \textbf{In which chapter(s) of your thesis can this material be found?}
% Answer here

Chapters \ref{chap:neuroimaging}, \ref{chap:atlas}, \ref{chap:reg} (Section \ref{sec:reg1}), \ref{chap:eval}, Appendix
\end{enumerate}

\textbf{e-Signatures confirming that the information above is accurate}
(this form should be co-signed by the supervisor/ senior author unless this is not appropriate, e.g. if the paper was a single-author work)\textbf{:}\\
\textbf{}\\
\textbf{Candidate:}\\
\textbf{Date:}\\
% this form should be co-signed by the supervisor/ senior author unless this is not appropriate, e.g. if the paper was a single-author work):
\textbf{}\\
\textbf{Supervisor/Senior Author signature} (where appropriate)\textbf{:}\\
\textbf{Date:}
%

%%%%%%%%%%%%%%%%%%%%%%%%%%%%% --- IEEE --- %%%%%%%%%%%%%%%%%%%%%%%%%%%%%%%%%%%%%
\newpage
\begin{enumerate}[leftmargin=*,label={\bfseries\arabic*.}]\itemsep0em
	%
	\item \textbf{For a research manuscript that has already been published} (if not yet published, please skip to section 2)\textbf{:}
	%
	\begin{enumerate}[label={\alph*)}]\itemsep0em
	%
	\item \textbf{What is the title of the manuscript?}

	\item \textbf{Please include a link to or doi for the work:}

	\item \textbf{Where was the work published?}

	\item \textbf{Who published the work?}

	\item \textbf{When was the work published?}

	\item \textbf{List the manuscript's authors in the order they appear on the publication:}

	\item \textbf{Was the work peer reviewd?}

	\item \textbf{Have you retained the copyright?}

	\item \textbf{Was an earlier form of the manuscript uploaded to a preprint server (e.g. medRxiv)? If ‘Yes’, please give a link or doi}
	\\
	If ‘No’, please seek permission from the relevant publisher and check the box next to the below statement:
	%
\begin{itemize}\itemsep0em
% To check this box, replace \Box with \boxtimes
\item[$\Box$] {\itshape I acknowledge permission of the publisher named under 1d to include in this thesis portions of the publication named as included in 1c.}
\end{itemize}
%
\end{enumerate}
%
\item \textbf{For a research manuscript prepared for publication but that has not yet been published} (if already published, please skip to section 3)\textbf{:}
%
\begin{enumerate}[label={\alph*)}]\itemsep0em
	%
	\item \textbf{What is the current title of the manuscript?}

	\citetitle{Young2023}
	\item \textbf{Has the manuscript been uploaded to a preprint server `e.g. medRxiv'?
	\\
	If `Yes', please please give a link or doi:}

	Yes, \url{https://www.researchgate.net/publication/375601134_Training_Data_Requirements_for_Atlas-Based_Brain_Fibre_Tract_Identification}
	% Answer here:
	\item \textbf{Where is the work intended to be published?}
	% Answer here: e.g. journal name

	Proceedings of the \citefield{Young2023}{eventtitle}
	\item \textbf{List the manuscript's authors in the intended authorship order:}
	% Answer here

	\citeauthor*{Young2023}
	\item \textbf{Stage of publication:}
	% answer here: e.g. in submission

	Work presented at conference, copyright transferred to IEEE, yet now indication of if/when proceedings may be published (copyright on submitted version retained).
\end{enumerate}

\item \textbf{For multi-authored work, please give a statement of contribution covering all authors} (if single-author, please skip to section 4)\textbf{:}
\begin{description}[font=\sffamily]
	\item[Fiona Young] Methodology (conceptualisation and implementation), analysis, manuscript original draft and graphics
	\item[Kristian Aquilina] Supervision
	\item[Chris A. Clark] Supervision
	\item[Jonathan D. Clayden] Conceptualisation, supervision, manuscript review and editing
\end{description}
% Answer here
\item \textbf{In which chapter(s) of your thesis can this material be found?}
% Answer here

Chapter \ref{chap:atlas}
\end{enumerate}

\textbf{e-Signatures confirming that the information above is accurate}
(this form should be co-signed by the supervisor/ senior author unless this is not appropriate, e.g. if the paper was a single-author work)\textbf{:}\\
\textbf{}\\
\textbf{Candidate:}\\
\textbf{Date:}\\
% this form should be co-signed by the supervisor/ senior author unless this is not appropriate, e.g. if the paper was a single-author work):
\textbf{}\\
\textbf{Supervisor/Senior Author signature} (where appropriate)\textbf{:}\\
\textbf{Date:}
%

%%%%%%%%%%%%%%%%%%%%%%%%% --- Diffusion ISMRM --- %%%%%%%%%%%%%%%%%%%%%%%%%%%%%
\newpage
\begin{enumerate}[leftmargin=*,label={\bfseries\arabic*.}]\itemsep0em
	%
	\item \textbf{For a research manuscript that has already been published} (if not yet published, please skip to section 2)\textbf{:}
	%
	\begin{enumerate}[label={\alph*)}]\itemsep0em
	%
	\item \textbf{What is the title of the manuscript?}

	\item \textbf{Please include a link to or doi for the work:}

	\item \textbf{Where was the work published?}

	\item \textbf{Who published the work?}

	\item \textbf{When was the work published?}

	\item \textbf{List the manuscript's authors in the order they appear on the publication:}

	\item \textbf{Was the work peer reviewd?}

	\item \textbf{Have you retained the copyright?}

	\item \textbf{Was an earlier form of the manuscript uploaded to a preprint server (e.g. medRxiv)? If ‘Yes’, please give a link or doi}
	\\
	If ‘No’, please seek permission from the relevant publisher and check the box next to the below statement:
	%
\begin{itemize}\itemsep0em
% To check this box, replace \Box with \boxtimes
\item[$\Box$] {\itshape I acknowledge permission of the publisher named under 1d to include in this thesis portions of the publication named as included in 1c.}
\end{itemize}
%
\end{enumerate}
%
\item \textbf{For a research manuscript prepared for publication but that has not yet been published} (if already published, please skip to section 3)\textbf{:}
%
\begin{enumerate}[label={\alph*)}]\itemsep0em
	%
	\item \textbf{What is the current title of the manuscript?}
	% Answer here:

	\citetitle{Young2022a}
	\item \textbf{Has the manuscript been uploaded to a preprint server `e.g. medRxiv'?
	\\
	If `Yes', please please give a link or doi:}

	Yes, \url{https://www.researchgate.net/publication/367116849_Stability_of_white_matter_tract_segmentation_methods_with_decreasing_data_quality}
	% Answer here:
	\item \textbf{Where is the work intended to be published?}
	% Answer here: e.g. journal name

	N/A
	\item \textbf{List the manuscript's authors in the intended authorship order:}
	% Answer here

	\citeauthor*{Young2022a}
	\item \textbf{Stage of publication:}
	% answer here: e.g. in submission

	Work presented at conference, no proceedings published.
\end{enumerate}

\item \textbf{For multi-authored work, please give a statement of contribution covering all authors} (if single-author, please skip to section 4)\textbf{:}
\begin{description}[font=\sffamily]
	\item[Fiona Young] Methodology (conceptualisation and implementation), analysis, manuscript original draft and graphics
	\item[Jonathan D. Clayden] Conceptualisation, supervision, manuscript review and editing
\end{description}
% Answer here
\item \textbf{In which chapter(s) of your thesis can this material be found?}
% Answer here

Chapter \ref{chap:applications}
\end{enumerate}

\textbf{e-Signatures confirming that the information above is accurate}
(this form should be co-signed by the supervisor/ senior author unless this is not appropriate, e.g. if the paper was a single-author work)\textbf{:}\\
\textbf{}\\
\textbf{Candidate:}\\
\textbf{Date:}\\
% this form should be co-signed by the supervisor/ senior author unless this is not appropriate, e.g. if the paper was a single-author work):
\textbf{}\\
\textbf{Supervisor/Senior Author signature} (where appropriate)\textbf{:}\\
\textbf{Date:}
%
}% End font switch


\setcounter{tocdepth}{2}
% Setting this higher means you get contents entries for
%  more minor section headers.

\tableofcontents
% \listoffigures
% \listoftables

  \printglossary[type=\acronymtype]
  \glsaddallunused[main]
  \printglossary
}

\epigraph{I'll tell you where the real road lies\\
Between your ears, behind your eyes}{Anais Mitchell, Hermes in \textit{Wait For Me (Reprise)}}
\chapter*{Introduction}\addcontentsline{toc}{chapter}{Introduction}
\label{chapterlabel0}

Few objects of our scientific endeavours, from the depths of Earth's oceans to the Milky Way galaxy, rival the human brain in mystery and complexity.
\note{link}
Understanding the brain's structure and function has been a scientific \note{priority?endeavour} for centuries, and yet we are still far from forming a complete picture.
At the same time, the scientific and medical needs and for such knowledge are \note{growing/more urgent}.
Much of the world is caught in a mental health crisis, in our ageing population the devastation wrought by neurodegenerative diseases is \note{accelerating}, after leukaemia, paediatric cancers arise most frequently in the brain and spinal cord.\autocite{Ostrom2015}

For many years, study of the brain was limited to postmortem dissections, or the study of patients who had suffered life-altering brain injuries.
\note{stuff about other in vivo imgaing techniques and how they're bad at brains}
With the development of \gls{mri} came an entirely new medium for peering into the brain.
It has since become the single most important \note{? flouroscopy, two photon imaging? check nr of citations or whatever} technology for conducting \textit{in vivo} neuroscience.
\Gls{mri} has also gained indispensible importance in clinical neuroimaging.
\note{link}
Among the clinical applications of \gls{mri} lies the ability to map the critical structures of the brain as they relate to a neurosurgical target, such as a tumour.
In clinics around the \note{rich??} world, presurgical planning is informed by the insights provided by \note{diffusion} \gls{mri}.
This is thanks, in part, to the techniques of \gls{dti} and streamline tractography, which exploit the directional coherence of white matter bundles \note{... give insight into white matter structure/organisation}

A brain tumour diagnosis once heralded, excepting a few rare cases, an unstoppable, painful descent into an end of suffering. \note{yikes}.
In \textit{A Manual of Diseases of the Nervous System}, published in 1888, pioneering neurologist Sir William Gowers wrote:
``The treatment of new growths [...] is always a sufficiently gloomy subject, and not least so when they are seated in an organ like the brain, in which they cause peculiar and varied suffering, and in which their development, even to a moderate degree, is rarely compatible with life."
He goes on to note the sobering risks and almost unavoidable neural sequelae accompanying any attempt at surgical removal.
Today, surgical removal of brain tumours is routine, brings great benefit to quality of life and prognosis, and can even be curative.
Even so, a brain tumour diagnosis remains a devastating prospect, one that is increasing, and the outlook and survival rates for malignant brain tumours remains somber.\autocite{Aldape2019}
We wield countless weapons, established and more experimental, in the campaign against brain cancer:
Temozolomide, gamma knife surgery, 5-aminolevulinic acid (5-ALA) fluorescence guided surgery, proton beam therapy, procarbazine, lomustine, and vincristine (PVC), genomics, focussed ultrasound are just some of the drugs and tools already in the armoury, while research into novel preclinical models, genetics and microbiology, blood-brain-barrier (BBB) disruption and drug development continues in the foundry, searching for better understanding of both how brain cancers grow and how they can be stopped.
The role of MRI-guided treatment, which still stands to benefit a relatively tiny fraction fortunate patients \note{elitist??}, when viewed in the context of this ensemble cast, may appear insignificant, but only thanks to our tendency to interface with such overwhelming \note{issues} via abstractified statistics.
To any one of those patients whose lives could be improved by advanced MRI techniques, their significance is unquestionable.

tractography great and all but it has its flaws, such as the amount of time and effort and computational expertise needed to get good results.
This is fine if you have loads of resources, but lots of places don't, so presurgical planning with tractograph is limited to centres with access to these resources.\autocite{GeorgeZakiGhali2020}
strike one against tractography.
But there is another reason why the impracticalities of tractography are getting in the way.
Or are unable to step up to the table for the new potential untapped ground of intraoperative imaging.
Surgical planning is great and all, but it only gets you so far into surgery.
After a while, the effects of brain shift leave the preoperative images in the dust.
Intraoperative MRI can be used to scan again during surgery, throwing fresh light on the surgical field and surrounding brain structure.
It's mostly used to check for residual tumour and/or confirm resection.
But what if further surgery is necessary?
Wouldn't it be great if we could use the intraoperative MRI to provide updated white matter mapping and navigation?
Well this is where tractography falls short, as it's current implementations are really impractical for use intraoperatively.
All the more reason to develop a novel means of identifying white matter bundles from diffusion MRI images as an alternative to tractography.

Intraoperative dMRI has the potential to supplement existing imaging practices by offering a means of imaging fibre tracts after brain shift has invalidated preoperative imaging.\autocite{Nimsky2001}
Thus informed, surgeons would be better equipped to resect as much tumour as possible while leaving eloquent brain tissue intact.

This research project was motivated by the need for a white matter bundle mapping technique that could potentially be applied to intraoperative diffusion MRI images, or any other environment where resources are limited.
The basic premise is that we already have, in most cases, a lot of prior information about the shape and location of the target bundle.
These expectations on shape and location could be exploited to directly estimate the location of the bundle in a new diffusion MRI scan.
Of course, the reality is that inter-subject variability and in particular the complexly anonamlous anatomies of brains harbouring tumours or other space-occupying lesions present challenges to this relatively simple premise, as indeed to do to streamline tractography.
This report will describe how this concept nevertheless holds up in healthy brains and, with appropriate adaptation, in diseased brains as well.
\note{dramatic finishing sentence!}


\part{Background}
\epigraph{Too much knowledge never makes for simple decisions.}{Frank Herbert, \textit{Children of Dune}}
\section{Theory fundamentals}
\label{theory}

\note{This will cover the more theoretical background prerequisites, covering the fundamental concepts on the biological side (neuroanatomy and cellular physiology, maybe some oncology?) and physical side (MR physics)}

\subsection{Neuroanatomy: Micro to macro}

\subsubsection{Cells of the brain}

\note{neurons, glial cells, axons and basics of neural signalling.}

The building blocks of living organisms begin at the level of molecules and atoms, through macromolecules such as proteins and lipids, cell organelles, cells, tissues, organs and finally whole organisms.
For the purposes of this report, though, we can start at the cellular level, and focus on a single organ: the brain.
Brain tissues consist of numerous cell types.
The principle functional cells are neurons, which perform the computations underpinning all aspects of neural function. They are supported, and outnumbered ( by a \note{factor of?}), by a network of glial cells, each with specialised functions.

\subsubsection{Neuroanaomy}

\note{Structure of the brain, including different sections (hind, mid, forbrain etc.) and tissue types.
Discuss the lobes, cortical regions and functional divisions, then white matter tracts.}

\subsection{MRI Physics}

\subsubsection{Resonance and relaxation}

\note{Basics of resonance, magnetisation, spin-spin / spin-lattice relaxation}

MRI images are obtained by detecting the spin relaxation resonance stuff of principally hydrogen atoms (which are abundant in the body in H2O molecules).

The application of a strong linear magnetic field causes all spins to align themselves with the direction of the field. This is the bulk magnetisation $M_z$, in its "relaxed" state (aligned with B0).
The ratio of spin magnetic moments aligned with the external field is increased, \note{as the field imparts energy or something causing magnetic moments to preferentially be aligned} resulting in a net magnetisation along the orientation of $B_0$, termed $M_z$.
Alignment of a magnetic moment with $B_0$ (spin up) is energetically favourable of the opposite spin down state to the tune of $\Delta E = \gamma h B_0$ \note{explain vars}.
Hence the stronger the external field, the larger the ratio of up to down spins and thus the a larger the magnitude of $M_z$.
As the magnitude of this bulk magnetisation vector underpins the strength of any subsequently sampled signals, stronger bore fields, as a general rule \note{are there exceptions?}, produce images with better \gls{snr}.
In thermal equlibrium, then, the magnetic moments within the sample are precessing at the larmor frequency and entirely out of phase with each other, resulting in no net transverse magnetisation, and with spins aligned preferentially along $B_0$, resulting in maximum longitudinal magnetisation $M_z$.

The manipulation of this net magnetisation vector away from, and subsequent relaxation back to this equilibrium state, is the basis of all signal production in \gls{mri}.
The application of a \gls{rf} pulse, that is, an oscilating magnetic field ($B_1$) precisely at the resonant larmor frequency and perpendicular to $B_0$, enacts two effects on this the bulk magnetisation.
The first is the impartion \note{??} of energy which overcomes $\Delta E$ \note{this is the bit I understand the least} to increase the number of spins in the energetically unfavourable spin-down orientation and thus eliminate the net magnetisation along $B_0$.
The second effect is to sycronise the spins' precession with the rotating \gls{rf} pulse, and by extension with each-other, resulting in a precessing net magnetisation in the transverse plane $M_{xy}$.
\note{but what about 180 degree pulses?}
As soon as the excitation field is switched off, the bulk magnetisation gradually relaxes to equilibrium as the induced effects on the individual moments revert:
interaction with other spins \note{local magnetic fields?} in the molecular environment cause gradual dephasing and consequent reduction in $M_{xy}$ (spin-spin relaxation), while the energy gained from the \gls{rf} pulse is gradually dissipated among the lattice field \note{??} as the spins revert to the stistically fabourable lower energy state in alignment with $B_0$ causing gradual recovery of $M_z$ (spin-lattice relaxation).

The recovery rate of $M_z$ and the rate of decay of $M_{xy}$ are described by the time constants $T1$ and $T2$ respectively, and each depend on the specific molecular environment and are thus tissue-dependent quantities.
It is the varying values of $T1$ and $T2$, as well as the overall differences in proton densities ($PD$) between tissues which lend contrast to MR images.

\subsubsection{Excitation and image acquisition}

\note{How MRI machines work: excitation pulses, effects of different relaxation weightings, examples of different common pulse sequences and their uses}

\paragraph{Pulse sequences}

Immediatly after application of the \gls{rf} excitation pulse, magnetisation will rapidly return to equilbrium in a spiral pattern \note{see image}.
This rotating magnetisation will induce an oscillating current within a detection coil placed near the sample in a process called \gls{fid}.
This is the most basic form of NMR signal and, since it decays very rapidly \note{isn't useful on it's own for image formation?}
In order to generate signal useful for spatial encoding and image acquisition \note{??} we make use of spin echos and/or gradient echos.

To produce a gradient echo, a readout gradient $G_x$ \note{necessarily?} is applied across the sample causing spins to dephase a different rates according to their local magnetic field $B_0 + G_xx$.
After a time $TE/2$ \note{how long?} the gradient polarity is reversed, causing 

\subsubsection{Diffusion MRI}

\note{First describe diffusion in general but stick to tissue, i.e. restricted diffusion. Cover the different types of diffusion in different tissues in the brain, timescales etc.
Then look at the mechanism of diffusion MRI acquisition, pulse sequence and parameters e.g. b values.
Also mention common artefacts.}

In living tissue, water molecules are not free to diffusion for long distances in all directions. Some tissue environments are highly constrained, with diffusion occurring principally along a single direction \note{too simplistic}, while in others, fewer barriers allow free diffusion.

\paragraph{Diffusion weighting}

Any MR pulse sequence can be modified to introduce additional sensitisation to brownian motion, or diffusion weighting, in addition to the existing T1 and T2 based constrasts.
This is achieved through the application of a pair of diffusion sensitisation graident pulses prior to echo generation and signal readout.

To illustrate the concept, we will consider the application of diffusion weighting along a single orthogonal direction, e.g. $G_x$.
After slice selection and \gls{rf} excitation, a gradient of magnitude $G_d$ is applied along $x$ for time period $\delta$ before it is reversed in polarity for subsequent $\delta$.
Consider a spin which is stationary along $x=x_1$ throught the application of these gradients.
Initially, it will gain a phase $\phi_1$ proportional to it's position along the gradient $x_1$ according to $\gamma G_d \delta x_1$.
After gradient reversal, having not changed position $x=x_2=x_1$ it will gain the opposite phase $\phi_2 = -\phi_1$, resulting in a net phase change after diffusion sensitisation of $0$.
Conversely, a spin which is net motion along $x$ will experience different gradient strengths across the two time points and will experience a net dephasing according to $\Delta\phi = \gamma G_d \delta (x_2 - x_1)$ \note{replace with integral form} and corresponding signal loss.
On subsequent signal sampling, those voxels in which diffusion was high will have have a high degree of diffusion weighting-induced dephasing and exhibit a corresponding signal dropout.
In those voxels with low diffusion, phase coherence will remain relatively intact after diffusion senistisation and signal loss will be correspondingly minimal.

\paragraph{Echo planar imaging}

Since diffusion imaging, in particular \gls{hardi}, involves the acquisition of numerous image volumes with different diffusion weightings, scan times can be particularly long.
Depending on the image resolution, number of diffusion weighted directions, and many other factors, scan times can run into the tens of minutes and even hours.
Aside from the cost and practicalities of longer scans, there is also the increased risk of motion artifacts.
So it is that the dominant \note{only?} type of pulse sequence used in \gls{dmri} is \gls{epi}.
In \gls{epi}, all phase encoding intervals are acquired after only a single \gls{rf} excitation.
\Gls{epi} can employ both gradient and spin echos and there are numerous different variations and associated contrast \note(??), but for the purpose of this summary we will focus on the typical \gls{epi} sequence employed for diffusion weighted imaging.

\chapter{Diffusion MRI analysis}\label{chap:neuroimaging}

Having given a brief introduction to the principles neuroanatomy and \gls{dmri} physics and image formation, we will turn now to the modern developments in applying \gls{dmri} to qualitative and quantitative analysis of \gls{wm} organisation.
This chapter will take us through early to state-of-the art uses of \gls{dmri} as a \gls{wm} imaging tool and outstanding challenges.

Though the tools for studying the structure and function of neural connectivity are numerous, \gls{dmri} remains the only available technique capable of investigating microscopic \gls{wm} structure of the entire human brain \textit{in vivo}.
% This is thanks to the \note{fibrous} organisation of \gls{wm} fibres, which group together in bundles forming a restrictive diffusion environment in which axon membranes and myelin sheaths form barriers to diffusing water molecules.
% The resulting anisotropic diffusion pattern, with preferential diffusion along the axon direction, forms the basis for \gls{dmri}-based \gls{wm} analysis.
Though the mechanism of diffusion-weighted signal is well understood (see section \ref{sec:dmri}), its correct interpretation hinges on an understanding of how diffusion is affected by the microstructural environment\autocite{LeBihan1995}, or even by just those features of the microstructural environment we are interested in measuring.
It is not obvious, for example, how much of diffusion within a voxel filled with myelinated axons can be best characterised as either hindered or restricted, and how properties such as axon density and diameter might contribute to the observed signal.\autocite{Panagiotaki2012}

\Gls{dmri} image \glspl{voxel} are on the order of cubic millimetres, and in such a volume are contained thousands of individual axons, with average diameters of around 1$\mu m$\autocite{Liewald2014,Lampinen2019}.
What's more, rarely are all axons within a voxel uniformly aligned with a single direction.
Throughout the brain, fibre tracts mingle and intersect, bend and fan out.\autocite{Jeurissen2013,Alexander2019}
The physical diffusion processes in such complex configurations and multicellular environments cannot be known, only approximated and modelled through a choice of assumptions and simplifications.
In this there are, of course, many contrasting approaches, developed for different applications and the directed study of specific quantities and microstructural features.
Such target properties of \gls{dmri} analysis include cellular composition of \gls{gm}, axon density and myelination, intra- and extracellular water content, and diffusivity perpendicular to axonal orientation.
Collectively, these scalar parameters are fit from a family of microstructural models known collectively as multi-compartment models, so named for their separation of the measured signal into isolated compartments representing, for example, intra-axonal or extracellular space, each with different diffusion patterns.\autocite{Panagiotaki2012,Alexander2019}
These parameters could capture virtual cytohistological information and even serve as biomarkers for disease\autocite{Alexander2008}.
For the purposes of studying macroscopic brain connectivity and the course and organisation of individual \gls{wm} fibre tracts, the key information of interest to be determined from raw \gls{dmri} is the distribution of axonal orientations within each voxel described by a \gls{fod}.

\section{Tissue microstructure  and fibre orientation modelling}


The first approach for modelling fibre orientations came in the form of \gls{dti}, which remains one of the dominant diffusion models in many applications today, particularly in clinical contexts.
The diffusion process within a \gls{wm} voxel is modelled as a three dimensional Gaussian distribution, whose covariance matrix is proportional to the diffusion \textit{diffusion tensor}\autocite{Basser1994,ODonnell2011} $\mathbf{D}$:
\begin{equation}
  \mathbf{D} = \begin{pmatrix}
                D_{xx} & D_{xy} & D_{xz}\\
                D_{yx} & D_{yy} & D_{yz} \\
                D_{zx} & D_{zy} & D_{zz}
                \end{pmatrix} \label{eq:dt}
\end{equation}
At least six unique directions are required to fully determine $\mathbf{D}$ (which is diagonally symmetric, i.e. $D_{ij} = D_{ji}$), although the diagonal components, sufficient for measuring \gls{adc} as previously described in (\ref{eq:trace}--\ref{eq:adc}), may be determined from only three orthogonal measurements.

By diagonalising $\mathbf{D}$, which is symmetric and positive-definite, three orthogonal eigenvectors and corresponding eigenvalues $\lambda_1 \geqslant \lambda_2 \geqslant \lambda_3$ can be computed.
Rotationally invariant indices including mean diffusivity and \gls{fa} provide scalar quantities which can be interpreted in terms of underlying tissue properties, the latter being given by
\begin{equation}
  FA = \sqrt{\frac{3}{2}}\frac{\sqrt{(\lambda_1 - \langle \lambda \rangle)^2 + (\lambda_2 - \langle \lambda \rangle)^2 + (\lambda_3 - \langle \lambda \rangle)^2}}{\sqrt{\lambda_1^2 + \lambda_2^2+ \lambda_3^2}}
\end{equation}
The principal eigenvector $\bm{\lambda}_1$ is usually interpreted as the direction of fastest, or least restricted, diffusivity, although this holds only in voxels with a single population of straight, parallel axons.
If the diffusion tensor, whose eigenvectors and eigenvalues can be represented in the form of an ellipsoid shaped surface of equal mean displacement, is used as a rudimentary approximation of the \gls{fod}, then $\bm{\lambda}_1$ is taken as the single peak orientation of axon fibres.
Depicting peak orientations is achieved through \gls{dec} mapping, in which directions are mapped to colours in RGB space, where the $x$-axis, or medial-lateral / left-right direction is assigned pure red, the $y$-axis, or posterior-anterior direction is assigned pure green, and the $z$-axis, or inferior-superior direction, is assigned pure blue.
\Gls{dti} images are often depicted as \gls{dec} \gls{fa} (or ``colour \gls{fa}) maps, where a voxel's colour is determined by $\lambda_1$ and its brightness by the \gls{fa} value.

In fact, it is rare for a voxel of \gls{wm} to contain only a single uniformly oriented bundle of axons.
In so-called crossing fibre voxels, where at least two distinctly oriented fibre populations reside, the diffusion tensor provides a woefully inaccurate or misleading picture:
A smaller population with a low signal contribution may be entirely unrepresented in the modelled \gls{odf} peak, or else the effect is one of averaging all contributing populations such that none of the actual underlying fibre directions are indicated by the \gls{dt} eigenvectors (Fig. \ref{fig:cross}).

\begin{figure}
  \centering
  \includesvg[width=0.6\textwidth,pretex=\small\sffamily]{chapter_2/fods.svg}
  \caption{Popular \gls{dmri} modelling techniques produce different estimates of the underlying fibre orientations. In a voxel with a single population of fibres (top row), both the diffusion tensor and \gls{csd} derived \gls{fod} capture the dominant fibre orientation. If two fibre bundles cross within a voxel (bottom row), the diffusion tensor does not accurately estimate any of the dominant fibre orientations, whereas the \gls{fod} is able to resolve both populations.}
  \label{fig:cross}
\end{figure}

The crossing fibres problem led to \gls{dti}, at least in research imaging, gradually giving way to higher-order fibre orientation modelling techniques which aim to account for more than one, or even an arbitrary number of distinctly oriented fibre populations\autocite{Alexander2005}.
These include to an extent the multi-compartment models which can include multiple directional intra-axonal compartments, such as the popular ``ball and sticks'' model\autocite{Behrens2003,Behrens2007}.
Alternatively, several approaches aim to retrieve an underlying orientation distribution with arbitrary number of peaks through transforming the raw signal \gls{odf} into a spherical distribution of diffusivity or correlates thereof.
Diffusion spectrum imaging (DSI)\autocite{Wedeen2008}, and the less data-demanding Q-ball imaging\autocite{Tuch2003,Tuch2004}, are examples of methods that reconstruct a distribution of water molecule displacement (\gls{dodf}), with peaks aligned with the (presumed) fibre orientation.
Deconvolution methods begin with the premise that a single population of parallel fibres will produce a characteristic \gls{dmri} signal profile.
Then, the observed signal $S(\theta,\phi)$ in a given voxel amounts to a convolution over the fibre orientation distribution function $F(\theta,\phi)$ with this single fibre ``response'' kernel $R(\theta)$:
\begin{align}
  S(\theta,\phi) = F(\theta,\phi) \otimes R(\theta)
\end{align}\label{eq:csd}
Solving (\ref{eq:csd}) for $F(\theta,\phi)$, represented in spherical harmonics basis as given by (\ref{eq:shfun}), involves computing the inverse operation:
A spherical \textit{de}convolution of the signal with the response function.
Early versions of this concept included \textcite{Anderson2005}, in which the response kernel was modelled as a diffusion tensor.
In another approach presented by \textcite{Tournier2004} $R(\theta)$ is estimated directly from the data, generally by averaging the signals from voxels with the highest diffusion anisotropy, and thus does not rely on a model of diffusion, although it does rely on the assumption that a single fibre population's response is uniform throughout the brain and for different fibre configurations.
To improve angular resolution and reduce noise sensitivity in the estimated \gls{fod}, a regularised implementation in which biophysically impossible negative $F(\theta,\phi)$ values are strongly penalised gave rise to the widely used \gls{csd} method (Fig. \ref{fig:cross}).
\Gls{csd} can reconstruct \glspl{fod} resolving multiple crossing fibre populations with high angular resolution in a matter of seconds, from \gls{dmri} datasets with acquisition parameters achievable in routine imaging practice\autocite{Tournier2013}.
In addition to providing a means for estimating the \gls{fod} to a high degree of angular resolution, \gls{csd} also spurred exploration of new quantities relating to tissue microstructure, notably the interpretation of the \gls{fod} amplitude as measure of intra-axonal volume fraction, or \gls{afd} \autocite{Raffelt2012a}.
\Gls{afd} can be defined as a directional quantity or as a single scalar per voxel, obtained by integrating $F$ over the sphere.

By design, \gls{csd} presumes that the entire signal in a voxel can be explained by contributions from a number of fibre populations, each forming a highly restrictive environment in which diffusion is anisotropic.
Within pure and highly organised \gls{wm}, this is a reasonable assumption, however limitations become apparent outside of these areas.
As we have seen in previous sections, brain tissue covers a spectrum of cytoarchitectures and a corresponding diversity in diffusion environments.
At the typical resolutions of \gls{dmri} data, voxels may contain, in addition to axon fibres, signal contributions from \gls{csf} or \gls{gm}, which are typically characterised by isotropic and freer diffusion.
The result of such partial volume effects are highly noisy \gls{fod} estimates with spurious peaks, overestimation of \gls{fod} peak amplitudes and overestimation of \gls{afd} in affected voxels.\autocite{Jeurissen2014}
In addition, the original \gls{csd} method was designed only for data acquired with a single diffusion weighting $b$ factor (``shell''), unable to take advantage of the additional information contained in more advanced and increasingly popular multi-shell acquisitions.
To address these limitations, an extension proposed in \textcite{Jeurissen2014} and referred to as \gls{msmt} \gls{csd} includes support for multiple $b$-value shells and separation of the signal into contributions from different tissue compartments, typically \gls{wm}, \gls{gm}, and \gls{csf}, each with their own characteristic response functions.
The resulting \gls{fod} estimates have higher angular precision than \gls{ssst} \gls{csd}, with fewer noisy spurious lobes to confound downstream processing and interpretation.
Furthermore, separation of the isotropic signal contributions greatly improves the interpretation of \gls{fod} amplitude as a measure of \gls{afd}, as the amplitudes of each tissue's deconvolved \gls{odf} closely correspond to their respective tissue volume fractions\autocite{Jeurissen2014}.

\begin{SCfigure}
  \includegraphics[width=0.5\textwidth]{chapter_1/coronalwm.png}
  \caption{Coronal section of the author's white matter, imaged with \gls{csd} \gls{dec} mapping and streamline tractography. The \glspl{cst} are visible radiating from the cortex to converge in the internal capsules before descending through the anterior pons, perpendicular to the middle cerebellar peduncle fibres. Either side of the pons, the trigeminal nerves (cranial nerve V) are visible as small green dots.}
  \label{fig:corwm}
\end{SCfigure}

\section{Streamline tractography}\label{sec:tractography}

% \note{Discuss as the mainstay of white matter bundle segmentation.
% Cover fundamentals (including what it isn't), det/prop, different fibre models, virtual dissection, the usual suspects}

Up to this point, we have only discussed the processing and analysis of \gls{dmri} data, and what it can or cannot reveal about \gls{wm} microstructure, at the level of individual image \glspl{voxel}.
However, the individual axons that form fibre tracts can traverse 100s of millimetres, and with the unique ability to measure fibre orientations in each voxel it wasn't long before this information was being exploited to reconstruct axonal connections in their entirety.
The basic principle is one of treating fibre orientations as a brain-wide vector field through which the paths of virtual neural fibres can be traced in a process called streamline tractography (or simply tractography).
It is vital to note here that the paths of individual \textit{in vivo} axons are entirely indiscernible from \gls{dmri}.
The individual streamlines of tractography are entirely abstract mathematical objects, each a collection of vertices, which aim to capture the \textit{potential} pathways of axons consistent with the observed \glspl{fod} representing an ensemble of thousands of axons.
Tractography is an immensely powerful and useful tool, and at the same time full of flaws due to this abstract and indirect nature.
All tractography algorithms consist at their core of the following steps:

\begin{lstlisting}[language=bash,label={lst:track},frame=single]
streamlines = []
while length(streamlines) < N do
  streamline = [get_seed_vertex()]
  STOP = false;
  while not STOP
    vertex = streamline[end]
    v = get_local_direction(vertex)
    new_vertex = vertex + step_size*v
    append(streamline, new_vertex)
    STOP = evaluate_stop(new_vertex)
  append(streamlines, streamline)
\end{lstlisting}

Once a predefined number of streamlines which fulfil all selection criteria have been generated, tracking is terminated, and the resulting streamlines, each consisting of a set of vertices in 3D space, maybe further analysed or visualised as required.
Volumetric images can also be generated by computing the number of streamlines in each voxel on a predetermined grid, a technique referred to as \gls{tdi}\autocite{Calamante2010}.

\begin{figure}
  \includegraphics[width=\textwidth]{chapter_2/streamlines.png}
  \caption{Streamline tracking involves tracing trajectories through a vector field of inferred dominant fibre orientations. Two example streamlines, from the \gls{cst} (blue) and \gls{cc} (blue) are depicted in a coronal slice of the human brain. Figure reproduced from \textcite{Jeurissen2019} licensed under \href{https://creativecommons.org/licenses/by-nc/4.0/}{CC-BY-NC 4.0}.}
  \label{fig:tracking}
\end{figure}

Within the simple algorithm outlined above are a plethora of parameters and decisions which which have transformative effects on the result.
They are apparent in the undefined functions such as \verb|get_local_direction()| or \verb|get_seed_vertex()| and the scalar parameters \verb|step_size| and \verb|N|.
The seed location, step length, conditions for terminating or entirely rejecting streamlines, interpolation of the surrounding vector field, number of streamlines to generate, are all choices to be made by the user (although in practice many parameters will be automatically determined or set to default values by the chosen algorithm).
There are two choices that most fundamentally affect the tractography process and which feature most heavily in discussions on its use.

First is the choice of method for representing orientation information from the underlying data.
The earliest tractography algorithms were developed almost concurrently with \gls{dti}\autocite{Mori1998,Mori1999}, with the orientation vector field constructed from the principal eigenvectors of fitted diffusion tensors.
In the \gls{fact} algorithm proposed by \textcite{Mori1999}, the local direction for each vertex is assigned from $\bm{\lambda}_1$ of the current voxel, and the stop criterion is a measure of neighbourhood fibre collinearity falling below a predefined threshold.
With only a single possible propagation direction at each location, which, as discussed above, may throughout much of the brain have little to do with any true axon orientations at that point, diffusion tensor-based tractography can only track fibre pathways with rather limited accuracy.
Streamlines may continue happily along a physiologically plausible path until encountering a region of intersecting tracts, at which point it may be prematurely terminated or diverted onto the trajectory of this intersecting tract if a continuation of the current path is entirely unsupported by the principal eigenvector field.

It becomes clear when considering what we learned about neural connections in section \ref{sec:hodology}, about compact fibre bundles diverging to distributed cortical targets, that a single dominant fibre direction at every point is incompatible with the dynamic organisation of \gls{wm} tracts, and the result is a tendency to reconstruct narrow and incomplete fibre bundles\autocite{Farquharson2013}.
A well cited example of this limitation can be seen in reconstructions of the \gls{cst}, which arises from the entire motor cortex from the apex down to the Sylvian fissure, but which is rendered by \gls{dt}-based algorithms only as a vertical pathway without any lateral projections.
Against this background, tractography based on higher-order fibre orientation models represents a vast improvement in the ability to contend with tracking in the complex configurations of \gls{wm}.
Now when two perpendicular tracts occupy the same voxel, the possible tracking direction is not limited to either that of the dominant bundle such that tracking the smaller one is impossible, or of an average of the two such that neither is properly represented.
However, with the flexibility of multi-peak distributions comes ambiguity, as when there are multiple distinct possible directions in which to propagate the streamline at any position, the decision of which direction to take becomes far more complex.
While tensor-based tractography may be particularly prone to false negatives, or neglecting certain pathways, multi-peak tractography can easily produce false-positives by hopping onto the paths of intersecting tracts.

The second significant distinction is between deterministic and probabilistic tracking approaches.
In deterministic tractography, there is only one single direction in which a streamline can be propagated from any given point, and two streamlines seeded in exactly the same location will be identical.
But the certainty implied by this deterministic approach is at significant odds with the reality that tractography operates in a domain and resolution far removed from that of individual axons.
Probabilistic tractography algorithms are here to acknowledge the uncertainty inherent in the tracking process.
Numerous probabilistic tracking algorithms have been developed, and while the end effect is essentially the same, whereby the next step direction is sampled from a probability distribution instead of deterministically selected, and seeding in the exact same location will not give rise to identical streamlines, there are two subtly different schools in what sort of uncertainty is being considered\autocite{Jeurissen2019}.
One considers the \textit{measurement} uncertainty of the calculated orientations.
Under this approach, the general direction to take is not under question, but the accuracy of that direction is.
It holds that due to noise and inherent limitations in our measurement equipment and signal modelling, the fibre orientations can only be calculated with limited accuracy, and the tracking directions are sampled from a distribution reflecting this measurement uncertainty.
The probabilistic algorithm probtrackx\autocite{Behrens2007}, based on the ball-and-sticks fibre orientation model\autocite{Behrens2003}, is a notable example of this approach.

A second school takes the view that uncertainty in the choice of streamline step direction stems from the obscurity of the underlying physiological reality, and sampling that direction from the fibre \gls{odf} reflects that microstructural complexity.
Crucially though, tracking is not proceeding under any guidance relating to real biophysical connections, and though a streamline's direction in a given voxel may well be in accordance with real axons, whether that direction is appropriate in the context of the preceding steps of the same streamline is unresolvable.
In other words, \gls{fod}-based probabilistic tractography, of which first or second-order integration over \glspl{fod}  (iFOD1/iFOD2)\autocite{Tournier2012,Tournier2010} are notable examples, can capture the local dispersion of fibres in high detail, but that doesn't necessarily translate to long-range accuracy.
It is possible to constrain tractography according to broad heuristics about fibre tract geometries, but inevitably such simplifications will not be globally applicable.
For example, strategies to prevent streamlines from ``hopping'' onto intersecting, but not physically connected, pathways can include placing upper limits on the angle between successive steps under the expectation that most tracts will carry more or less straight on, but there are plenty of tracts in the brain with regions of high curvature, which become much harder to accurately reconstruct if the ``straight ahead'' constraint is too strict.

Due to streamline tractography's locally oriented and step-by-step nature, errors and missteps accumulate rapidly with little to no opportunity to correct them, resulting in some wildly implausible streamlines.
Attempts to address this blindness to biophysical reality are at the focus of much of modern tractography research\autocite{Bastiani2017,Rheault2019,Aydogan2021}, as the consequences of these ongoing challenges to connectivity research and neurology are substantial\autocite{Schilling2019, Yang2021, Grisot2021}.

\section{Segmenting white matter tracts with tractography}

The functional division of white matter into distinct tracts is of great consequence to neuroscience, psychology and neurology in their efforts to analyse brain structure and function, and as we will see later, identifying tracts is also of vital importance in neurosurgery.
It follows that the spatial delineation of individual tracts is a key step in many \gls{dmri} analysis pipelines.

Streamline tractography was the first, and remains the dominant answer to this task.
Though the field is wide and the specific approaches numerous, we will outline the two main frameworks through which individual tract segmentations are derived using streamline tractography.
The first, sometimes dubbed ``virtual fibre dissection'', involves generating a large number (on the order of 10s of millions) of streamlines, usually covering the entire brain, followed by a selection process whereby streamlines are assigned to a tract of interest or discarded.
Streamline tracking proceeds virtually uninhibited, terminating only if a maximum length is reached or when leaving the white matter (as indicated by tissue segmentations\autocite{Smith2012} or a \gls{fod} amplitude threshold).
After tracking, one approach for selecting streamlines belonging to the target bundle is to use logical \glspl{roi}, specifying inclusion volumes which must be visited and exclusion volumes to filter unwanted tracks.
These selection and exclusion \glspl{roi} encapsulates our \textit{a priori} neuroanatomical knowledge, and the resulting bundle, comprising only those streamlines fulfilling the criteria, represents the segmented tract (Fig. \ref{fig:tg_rois}).
The streamlines may be viewed as three-dimensional objects, or further processed into volumetric streamline density maps\autocite{Calamante2010} and thresholded binary segmentations.

Alternatives to \gls{roi}-based selection are clustering methods, which classify streamlines according to their proximity or similarity to each other, or other geometrical properties.
RecoBundles\autocite{Garyfallidis2018}, White Matter Analysis \autocite{ODonnell2017, ODonnell2007}, atlas based adaptive clustering \autocite{Tunc2014}, and example-based automatic tract labelling \autocite{Yoo2015} are all examples of data driven, group-wise streamline clustering and matching approaches.
They typically rely on registration of example data or streamline atlases based on which similar streamlines are recognised in the target data and labelled accordingly.
Streamline clustering methods have been shown to generate more consistent and reproducible results across subjects compared to \gls{roi}-based segmentation\autocite{Sydnor2018}.
Another approach, named Classifyber, uses a learned linear classification of streamline features to label streamlines belonging to the target bundle in a new subject \autocite{Berto2021}.
In all clustering approaches, the necessary generation of whole brain tractograms in test subjects and the additional construction of example or reference tractography data present barriers to application, as well as, in some cases, long processing times and high memory requirements\autocite{Wasserthal2018}.

The whole brain approach is computationally extremely wasteful, as the vast number of streamlines generated will not even represent \textit{any} anatomically valid pathway through the brain, let alone one belonging to the tract of interest.
Furthermore, if streamlines are randomly seeded throughout the brain, then longer tracts covering a larger volume are more likely to be sampled, the tendency to continue straight along a ``path of least resistance'' at diverging or crossing fibres results in inordinate overrepresentation of certain pathways\autocite{Smith2013}.
All this means that, after perhaps hours of tracking and billions of streamlines created, only a handful may be included in a final bundle reconstruction.

\begin{SCfigure}
  \includegraphics[width=0.5\textwidth]{chapter_2/tg_rois_glass.png}
  \caption{\Gls{wm} tracts are virtually dissected with streamline tractography and anatomically informed \glspl{roi}. In this toy example, streamlines for the \gls{cst} are seeded in the cerebral peduncles (blue ring) and selected with an inclusion \gls{roi} in the posterior limb of the internal capsule (green ring). Streamlines following the paths of the \gls{cc} or cerebellar peduncles are excluded (red rings).}
  \label{fig:tg_rois}
\end{SCfigure}

The second approach may be called ``targeted tractography'', and involves only seeding streamlines in a tract-specific \gls{roi} and retaining those that fulfil tract-specific selection criteria, provided as additional inclusion and exclusion regions (Fig. \ref{fig:tg_rois}), until a target number of streamlines have been selected.
A seed region can but does not necessarily have to be placed at one of the actual anatomical ends of the tract, and in some cases it makes more sense to seed from the middle of the tract and propagate bidirectionally, placing additional include regions at the ends to ensure complete coverage.
This approach does not mean that no streamlines are discarded (seeded fibres may be terminated before fulfilling all inclusion criteria, or stray into exclusion regions) but targeted seeding and selection certainly leads to a higher number of admissible streamlines being generated in far less computational time than in the whole brain approach, while discarding unwanted streamlines on the fly reduces storage requirements.
Targeted tractography is the more common approach particularly in applications where only a few or even just a single tract are relevant, such as in neurosurgery\autocite{Yang2021}.

Manual placement of \glspl{roi} in both whole-brain and targeted pipelines represents a significant intellectual burden, relies on expert anatomical knowledge and can be extremely time consuming, so it is often automated by registering structural atlases and defining tracts in terms of logical relations to atlas structures, as in TractQuerier\autocite{Wassermann2016} or a similar proposed method using fuzzy logic\autocite{Delmonte2019}, and Tracula \autocite{Yendiki2011}, or pre-defined \glspl{roi}, as in XTRACT \autocite{Warrington2020}.
The former two are examples of methods that rely on comprehensive cortical parcellations, typically obtained with a software tool such as FreeSurfer\autocite{Desikan2006,FischlSalat2002} which can take many hours to run.
Seed placement may also be optimised using reference tracts, as in probabilistic neighbourhood tractography \autocite{Clayden2006,Clayden2009}.
In many scenarios, manual \gls{roi} placement remains the default method, particularly in clinical contexts where automatic \gls{roi} registration or segmentation may fail due to pathology.
Here, not only is a good understanding of the anatomy of a tract is vital to produce high quality reconstructions, but the user will also need to understand the biases and pitfalls of their chosen \gls{fod} model and tractography algorithm to ensure proper interpretation and qualification of the results\autocite{Rheault2020,Rheault2022}.
Even to an experienced user, producing quality bundles is often time-consuming and tedious.
While modern research applications and increasingly more clinical applications almost exclusively favour probabilistic and multi-fibre \gls{odf} algorithms thanks to higher sensitivity to complex fibre configurations\autocite{Yang2021}, an inevitable trade-off is a high prevalence of false positive streamlines, representing either irrelevant or unphysical connections.
Filtering out these unwanted streamlines remains a considerable challenge\autocite{Jorgens2021}.
Attempts to reduce their creation in the first instance include injecting more anatomical priors into the tracking process, such as by modifying \glspl{fod} to favour the directions associated with the target tract\autocite{Rheault2019}, using directional \glspl{roi} for particularly tricky geometries \autocite{Chamberland2017}, or designing alternatives to the piece-wise linear tracking paradigm that aim to generate streamlines with more anatomical plausibility \autocite{Schomburg2017,Aydogan2021}.
Finally, due to a combination of the different computational methods available, and a general lack of consensus on the precise anatomical extents of many commonly reconstructed pathways, tractography suffers from notoriously low reproducibility\autocite{Schilling2021a}.

In view of these limitations, some in the field are continuing efforts to improve streamline tractography with novel tracking algorithms, finding new ways to incorporate anatomical priors and developing more powerful streamline filtering, clustering and selection strategies.
Others are looking towards white matter segmentation solutions that do not rely on tractography at the point of application, but instead produce voxel-wise tract segmentations directly from \gls{dmri} or \gls{fod} data.

\section{Streamline-free white matter imaging}

There have been numerous works addressing the \gls{wm} tract identification task as a classic voxel-wise segmentation problem, utilising techniques including multi-label supervised clustering \autocite{Ratnarajah2014}, level-set and front propagation\textcite{Nazem-Zadeh2011, Hao2014}, and deep learning for direct segmentation from fibre orientation representations \autocite{Wasserthal2018,Li2020}.
Typically, direct methods require some number of samples with which to train a classifier, atlas, Bayesian model or neural network.

In \textcite{Hagler2009}, a fibre location and orientation atlas is created by averaging the \gls{dt} and tractography-derived information from multiple subjects and subsequently used to estimate the voxel-wise \textit{a posteriori} tract probability in a test subject.
As orientation information was encoded by averaging \gls{dt} principal eigenvectors across subjects this approach is not optimised for crossing fibre configurations.
The spatial probability was given by the averaged, normalised track density values from individual deterministic streamline tractography, although tracking biases discussed above mean that equating streamline density with likelihood of tract location is problematic\autocite{Rheault2019,Smith2013}.
\textcite{Bazin2011} also proposed a direct approach based on diffusion tensor-derived priors (``Diffusion-Oriented Tract Segmentation'', or DOTS) also based on \gls{dt} modelling.
Here the atlas orientation prior consisted of a single principal direction per voxel, and comparisons with the test subject data are made using Markov random field models and neighbouring tensor connectivity.

More recent developments have made use of advances in data science techniques including deep learning segmentation models, of which TractSeg\autocite{Wasserthal2018} and Neuro4Neuro\autocite{Li2020} are notable examples, using \gls{fod} peaks and diffusion tensors as inputs, respectively.
Deep learning-based approaches have the advantage of producing highly reproducible results in short processing time, without the need for template or atlas registration.
However, drawbacks of direct, deep learning-based methods which produce binary segmentations include a lack of explainability, and a dependence on large volumes of annotated training data which are labour-intensive to produce.
This limits their flexibility:
If a user requires a tract segmentation which is either anatomically different or not covered by an existing pre-trained model, then the necessary production of new training data and subsequent model training represents a high logistical and computational barrier.

Inference models trained on large volumes of healthy data may not be entirely robust to pathologies, particularly those causing significant topological changes.
In addition to healthy data, Neuro4Neuro\autocite{Li2020} was validated only in a dementia dataset, and TractSeg\autocite{Wasserthal2018} was qualitatively validated in schizophrenia and autism datasets in the original work.
TractSeg has also been qualitatively validated in a tumour dataset with mostly successful results, with more complete segmentations in cases with minimally deforming tumours.\autocite{Richards2021}
In \textcite{Moshe2022}, the authors trained their own TractSeg model, on approximately 500 datasets, to segment the \gls{cst} in brain tumour patients.
The results were more reproducible than for the compared manual method, and obtained an average \gls{dice} of 0.64, almost 25\% worse than the performance in healthy data reported in the original TractSeg study (for the same tract).
The authors cite a lack of reliable and sufficient labelled training data as a reason for limiting their study to a single tract, despite the importance of other tracts in preoperative fibre mapping.


\chapter{Neurosurgery}\label{chap:neurosurgery}
%========================

% \note{This is for all the reasons for surgery, including tumour, epilepsy DBS.
% Also more detail into the types of tumours, and locations in the brain}

Many types of interventions fall under the remit of cranial and spinal neurosurgery, including inserting electrodes for \gls{dbs}, diagnostic biopsies, vascular procedures, and insertion of \gls{csf} shunts.
In all cases, precision is paramount and tools for accurate navigation form vital components of the surgical workflow.
The following review will focus on some of the most complex and invasive procedures, involving the removal of tumours and epileptogenic brain tissue.
In England between 2013--2018, oncological procedures were the third most common of all neurosurgery subspecialties comprising approximately 9\% of total, while functional neurosurgeries (including for epilepsy and deep brain stimulation) made up 8\%\autocite{Wahba2022}.
At \gls{gosh} in London, a leading paediatric centre, 10\% of neurosurgical procedures between 2018--2022 were tumour-related, while 12\% were for epilepsy\autocite{gosh2023}.

Invasive brain tumour operations are both highly complex and variable in their neurophysiology, microbiology, treatment plans, and prognosis.
Such diversity presents a significant barrier to the development of image processing methods intended for generalised use in tumour patients\autocite{Bauer2013}.
Neoplasms occur throughout the brain, with the location having unique impact on surrounding structures and associated function.
A tumour's natural history and histopathology also play a large role in determining its effects on its environment.
Malignant gliomas, a category of tumours arising from glial cells, often have complex structures, with infiltrating components and peritumoural oedema blurring the distinction between tumour and non-tumour tissue\autocite{Weller2021}.
On the other hand, many non-malignant tumours including most meningiomas and low-grade astrocytomas, are encapsulated, with clear demarcation from neighbouring brain tissues, which are displaced rather than infiltrated\autocite{Lu2004,Gerard2017}.

This project is concerned with the visualisation of cerebral white matter tracts, and therefore this review will focus on those indications and interventions in which damage to and navigation around such structures is of particular concern.
This is typically not the case for posterior fossa and suprasellar lesions, although there is growing interest in the role of the cerebellum in wider cognition and the brain functional network, and the surgical community is paying increasing attention on the effects of posterior fossa surgery on cerebellar tracts\autocite{Toescu2021,Skye2023}.
Brainstem tumours are often not candidates for surgical removal due to their eloquent location, with limited access and excessive risk to vital brainstem function.
The discussions in this section therefore apply primarily to supratentorial lesions in the cerebral hemispheres and thalamus, candidates for biopsy or resection via craniotomy.
The most common intracranial tumours are meningiomas, arising from the protective membranes surrounding the \gls{cns}, of which the majority are benign with good overall survival rates and relatively low-risk surgical treatment \autocite{Rogers2015,Spena2022}.
Far more complex and controversial are decisions surrounding the surgical treatment of gliomas, those tumours originating in glial cells including astrocytes and oligodendrocytes which are the most common malignant primary \gls{cns} neoplasms in both adults\autocite{Ostrom2015,Wanis2021} and children\autocite{Ostrom2015,Bauchet2009}.
While challenges regarding functional preservation and optimal surgical strategy are relevant to all intracranial surgeries, they are acutely highlighted within the context of glioma literature.
These tumours can embed themselves insidiously within the brain's functional architecture with devastating prognosis, facing challenging oncologists and surgeons with stark dilemmas in their bid to maximise both patient survival and quality of life.
Gliomas are classified into four \gls{who} grades, commonly split into \gls{lgg} (\gls{who} grades 1--2) and \gls{hgg} (\gls{who} grades 3--4) to reflect differences is malignancy and prognosis.
There are many subtypes based on histological and genetic characteristics which are periodically updated\autocite{Louis2021}, but this overview will focus on the broad categories of \gls{hgg} and \gls{lgg}.

\section{Extent of resection}\label{sec:eor}

Complete removal of all pathological tissue, perhaps counterintuitively, is not always the surgical objective.
Though it may in many cases be the ideal outcome from an oncological perspective, this scenario would frequently be in conflict with other equally important outcome indicators, such as the preservation of surrounding brain structures and the patient's neurological wellbeing.
Successfully balancing these consequences is a central dilemma in neurosurgical practice, with the key measure being \gls{eor}, the amount of tumour removed.
In theory, \gls{eor} is a straightforward concept, but in practice it is ill-defined and inconsistently reported, while remaining central to studies of surgical efficacy and outcomes.

Easily defined in oncology%
\footnote[2]{\gls{eor} is relevant to epilepsy surgery, although the terminology and calculations here are different. Epileptogenic centres cannot always be distinguished and measured on imaging, and functionally eloquent tissue may be the clear source of epileptic activity, and thus subject to removal.}
as either the absolute volume or relative percentage of tumour tissue removed, accurately and consistently determining \gls{eor} is very difficult.
It is often reported in terms of broad categories, the most common being biopsy, \gls{str} or partial resection (PR), near total resection, \gls{gtr} and supratotal resection\autocite{Wykes2021,Karschnia2021}.
There is also no general consensus on how these categories are defined, making comparison between studies even more difficult\autocite{Karschnia2021}.
Many studies simply give very rough percentage values as estimated visually by the operating surgeon or a radiologist based on whether or not tumour residue is visible in the resection cavity or on a postoperative scan, with limited accuracy\autocite{Sanai2008,Martino2013,Lau2018,Sezer2020}.
Over time, the definitions for \gls{eor} have evolved with the availability of techniques for measuring it, and the current accepted standard for quantifying \gls{eor} is with volumetric measurement on pre- and postoperative imaging\autocite{Rincon-Torroella2019}, but here too practices are inconsistent\autocite{Wykes2021}.
Full volumetric analysis requires accurately segmenting the entire lesion, though sometimes \gls{eor} is calculated by simply taking the diameter of the lesion on a single or multiple slices, or with approximate ellipsoid segmentation\autocite{Sanai2008,Albuquerque2021}.
% This also can't account for resected portions of the tumour being filled with fluid, or differing amounts of tissue compression caused by mass effect and postoperative brain shift.

Even manual delineation can be unreliable and inconsistent, especially for tumours with poorly defined borders and on postoperative imaging \autocite{Ertl-Wagner2009,Bo2017,Visser2019}.
Semi- or fully automatic segmentation improves reproducibility\autocite{Ertl-Wagner2009,Sezer2020} and modern algorithms are proving ever more accurate, although there are still challenges regarding computational performance and robust clinical translation\autocite{Angulakshmi2017,Wadhwa2019,Fawzi2021}.
Estimates of \gls{eor} can also be compromised by post-operative brain tissue shifting and obscuring the actual volume of resected tumour \autocite{Schucht2014a}, while microscopic tumour cell invasion means that complete resection, as viewed either on imaging or by intraoperative visual assessment, does not necessarily mean no tumour residue remains\autocite{Yordanova2017}.
Finally, while there is the most focus on reporting relative reductions in tumour volume as a percentage of original size, more recent studies have argued that absolute residual tumour volume is as, if not more relevant for determining postoperative outcomes\autocite{Ius2012,Rincon-Torroella2019,Smith2008,Karschnia2021}.

\section{Oncological and neurological outcomes: Necessarily in opposition?}

Inconsistent reporting of \gls{eor} is one factor complicating the study of its effects on clinical outcomes, even as there is widespread agreement on the importance of studying those effects\autocite{Rincon-Torroella2019,Wykes2021,Weller2021}.
% \note{defs: lgg = grades 1--2, hgg = grades 3--4, glioblastoma = grade 4 glioma}
% For many tumour types, particularly aggressive tumours such as \glspl{hgg}, subtotal resection is \note{never} curative even with adjuvant therapy.
Broadly speaking, \gls{gtr} has been shown to increase overall and progression free survival over \gls{str} across age groups in both high \autocite{Hatoum2022, Han2020, Adams2016, McCrea2015, Bloch2012, McGirt2009, Kramm2006} and low-grade \autocite{Keles2001, Pollack1995, Sanai2008} gliomas.
For \gls{lgg}, and especially in paediatric patients, \gls{gtr} has become the recommended standard of care, as complete resection leads to a lower rate of recurrence\autocite{Berger1994,Claus2005}.
In particular, maximal resection of \glspl{lgg} drastically reduces the risk of residual tumour evolving into \gls{hgg} (known as malignant transformation \autocite{Duffau2013,Hervey-Jumper2016,Rincon-Torroella2019}, though this is only a concern in adult patients, as malignant transformation in paediatric \glspl{lgg} is exceedingly rare\autocite{Collins2020}.
More recent voices have even argued for supratotal resection, beyond the margins of any abnormally enhancing areas on $T_1$-weighted and FLAIR $T_2$-weighted \gls{mri} scans, as reviewed in \textcite{deLeeuw2019}.
There is limited evidence, though controversial, to suggest that supratotal resection of \gls{who} grade 2 gliomas in adults is followed by fewer cases of malignant transformation and improved progression-free survival \autocite{Yordanova2011}.

However, due to a general lack of prospective randomisation and robust comparison with appropriately matched controls, drawing definitive conclusions from studies investigating the effects of \gls{eor} (or other surgical variables) on post-operative outcomes is contentious\autocite{deLeeuw2019,Keles2001}.
Results may be confounded by selection biases, for example, different tumour histological subtypes may lend themselves more or less easily to greater \gls{eor}, or arise more frequently in eloquent areas of the brain (which include cortex and subcortical \gls{wm} subserving language, motor, and sensory functions, as well as the thalamus, midline structures involved in memory processing, and the brain stem), where an aggressive surgical strategy is likely to be discounted\autocite{deLeeuw2019}.
Adult \glspl{lgg} tend to occur more frequently than \glspl{hgg} in highly eloquent cortical regions\autocite{Duffau2004}, indeed the control group for the supratotal \gls{lgg} study\autocite{Yordanova2011} mentioned above consisted of patients whose gliomas were located in eloquent brain areas, and who therefore underwent only \gls{gtr}.
One might therefore expect supratotal resection to be associated with worse postoperative neurological outcomes, and indeed \textcite{Rossi2019a} found higher probabilities of immediate postoperative deficits in supratotal versus total resection of \glspl{lgg}.
These were however significantly reversed at three month and one year follow-ups, and initial overall evidence suggests that neuropsychological outcomes are comparable between total and supratotal groups\autocite{Tabor2021}.

In adults with glioma, maximal safe resection, combined with adjuvant radio- and chemotherapy, has been the established standard of care for some time.
It has been less clear, however, whether the same should apply to children.
The most prevalent anatomical locations in which gliomas arise may differ between adults and children\autocite{Duffau2004}.
Thalamic gliomas, for example, are more frequent in children than in adults\autocite{Cinalli2018,Palmisciano2021,GomezVecchio2021}:
Adult gliomas are located most frequently in the hemispheres, mostly the frontal lobe, with only approximately 4--7\%\autocite{GomezVecchio2021,Larjavaara2007} situated in the thalamus, while as many as 19\% of paediatric \glspl{hgg} are thalamic\autocite{McCrea2015}.
There is also concern that oncological differences between adult and paediatric type gliomas preclude safe extrapolation of treatment plans from one patient group to the other\autocite{Jones2012,Greuter2021}.
% Adult LGG 31\% eloquent Jakola2012
% Adult mixed grade 2--3 65\% ''presumed eloquence'' GomezVecchio2021 ; 80\% eloquent adult LGG Greuter2021
In addition to paediatric tumours frequently arising in high-risk areas such as the thalamus and brain stem \autocite{Ostrom2015},  neurocognitive and functional preservation is an especially critical concern in children.
A meta-analysis published in 2022 analysed 37 articles to assess the association between \gls{eor} and survival in paediatric patients with \gls{hgg}\autocite{Hatoum2022}.
Notwithstanding the difficulties in consistently defining and reporting \gls{eor} as discussed above, the study found strong evidence for improved overall survival in \gls{gtr} over \gls{str} of gliomas located in the cerebral hemispheres, but no association between \gls{eor} and survival was observed in midline cases.
The authors emphasise that midline (thalamic and brain stem) gliomas are not often indicated for aggressive resection due to the elevated risk to critical neurological function, and the lack of observed association may stem from measurement biases, including lower sample size and the pooling of histologically distinct tumour types which may respond differently to treatment.
Moreover, no comparison was made for postoperative functional neurological outcomes, thus failing to capture the full picture of factors contributing to a decision to pursue radical surgery.

New postoperative neurological deficits occur in over one third of glioma surgeries \autocite{Zetterling2020a}, although most patients improve significantly over longer-term followup.
Unsurprisingly, higher chances of postoperative deficits were associated with higher \gls{eor} and with tumours situated in eloquent areas\autocite{Zetterling2020a}.
In \textcite{Gil-Robles2010}, authors argue for a more conservative resection margin in \gls{who} grade 2 gliomas (low-grade) to protect functional structures, although current consensus recommends total resection in \gls{lgg} wherever possible\autocite{Rincon-Torroella2019,Albuquerque2021}. %but this is not the dominant opinion
For the most malignant tumour types, even maximal resection combined with adjuvant therapy is rarely curative, and may only lead to a survival advantage of just a few months\autocite{Rincon-Torroella2019,Karschnia2023}.
Given the overall poor survival outcomes associated with aggressive gliomas, oftentimes the risks to quality of life and postoperative neurological function associated with pursuing \gls{gtr} outweigh any potential oncological benefits\autocite{Rahman2016,Tabor2021}.
In \glspl{hgg}, the justification for \gls{gtr} or even supratotal resection is weaker than in \gls{lgg}, given that it cannot secure long-term survival for affected patients.
Where radical resection carries no likely oncological benefit and is contraindicated by a high functional risk to eloquent areas, the goal of surgery may be conservative debulking of the lesion to relieve pressure on the brain and reduce current neurological symptoms.
% In certain types of tumour, particularly when a combined treatment approach of surgery and radiotherapy is taken, little evidence has been found that \gls{gtr} offers greater tumour-related (that is to say, neurological condition affected by the presence of the tumour) outcomes than subtotal resection, while increasing risk to surrounding brain tissue.
% On the other hand, certain types of tumour, in particular lower grade and less aggressive types, show a lower risk of recurrence and malignant transformation when radically resected, offering a particularly good long-term outlook.
With the widespread evidence of an oncological advantage associated with more extensive resection, physicians have increasingly advocated for \gls{gtr} as the standard treatment for \gls{lgg} and maximal safe resection for \gls{hgg} \autocite{Rincon-Torroella2019}.
But this comes with the caveat that tumours of lower malignancy are also often those found to be more operable, muddying the causal link between overall survival and extent of resection\autocite{Weller2021}.
Furthermore, post-operative neurological deficits, due to cortical and subcortical injury, themselves have a negative impact on overall survival, independently of differences in pre-operative symptoms \autocite{Trinh2013,Rahman2016,Rincon-Torroella2019}.
Hence even if only overall survival is considered as the measure of treatment success, the evidence that both neurological injury and un-resected tumour negatively impact survival highlights the dilemma of surgically treating tumours in eloquent brain regions\autocite{Rincon-Torroella2019,Duffau2004,Rahman2016}.
The European Association of Neuro-Onocology's recommendation, as of 2021, is that prevention of new neurological deficits should be prioritised over maximal extent of resection in the surgical treatment of gliomas\autocite{Weller2021}.

A further consideration on the feasibility of \gls{gtr} or supratotal resection is neuroplasticity\autocite{Duffau2005}.
Slowly growing, low-grade, or recurring tumours may lead to functional reorganisation of surrounding brain tissue\autocite{Takahashi2012,Southwell2016,Das2019} or compensatory recruitment of equivalent contralateral regions\autocite{Mitolo2022}, enabling the safe removal of a greater margin of tissue than would otherwise be accepted for eloquent areas\autocite{Rossi2019a}.
Current understanding of neuroplasticity and brain tumours is limited to a small number of case studies, and more systematic research into the mechanisms and robust detection of functional reorganisation are required before these findings can be put into routine clinical practice\autocite{Duffau2005,Abel2015,Satoer2017}.
Taken together with the emergence of a hodological framework for neurosurgery discussed in Section \ref{sec:hodology}\autocite{Sala2019}, improved study of neuroplasticity could gradually lead to wider applicability of total or supratotal resection without compromising on neurological function and postoperative quality of life.

Early neurological concepts of rigid functional localisation formed the basis for the concepts of eloquence and operability of tumours guiding neurosurgeons throughout much of recent decades.
The recent move towards a more individualised view has only been made possible through developments in imaging and functional monitoring tools, allowing clinical teams to adapt the surgical strategy to each unique brain-tumour system, rather than relying on received assumptions about functional organisation\autocite{Boerger2023}.
The next section will explore some of those advanced technologies instrumental in the planning and execution of state-of-the-art neurosurgical practice.

% Overall view: moving goal posts, as changes in classifications of glioma and new understanding of mutations and associated risk, changing priorities from topological to hodological mapping and considering neuroplasticity, selection bias, ...

\section{Surgical planning and preoperative imaging}

Tumours can interact with their surroundings in a number of ways, depending on their nature and location.
Some tumours, including some \glspl{lgg} and meningiomas, are fully encapsulated and displace surrounding brain as they grow.
This strong demarcation between tumour and healthy tissue can facilitate surgical treatment and total removal of the tumour without undue risk to functioning neural tissue, but such lesions can cause neurological impairments as surrounding structures are stretched or compressed, leading to a recommendation for surgical removal of a tumour which poses an otherwise lower oncological threat.
Others cause almost no spatial displacement of brain tissue, with cancer cells instead invading the parenchyma and blurring the boundaries between disease and healthy brain.
Infiltrating tumours pose a particular surgical challenge due to the risk of surgical injury to eloquent tissue, and may only be conservatively debulked to relieve intracranial pressure and improve the effectiveness of adjuvant radiation or chemotherapy treatment.
In order to meet the goal of safely balancing maximal resection and functional preservation, the full tumour-brain interaction must be comprehensively mapped to determine the optimal resection margin.

Preoperatively, structural and functional non-invasive imaging are used for diagnosis and, if surgery is indicated, surgical planning.
At this stage the goal is to assess the spatial and functional relationships between involved and healthy tissues, map out a safe operative corridor to access the lesion, and determine the appropriate \gls{eor} under all considerations explored in the previous section.
Structural imaging with \gls{ct} and \gls{mri} provide critical anatomical information in high spatial detail.
Multi-contrast \gls{mri} examinations, including FLAIR, $T_1$-weighted and $T_2$-weighted imaging sequences, each provide unique contrasts for visualising different aspects of a tumour, such as necrotic and infiltrating regions, which can aid in determining tumour type, what \gls{eor} to aim for, or which region of the tumour to target for biopsy.
Angiography, detailed mapping of blood vessels with \gls{ct} and specialised \gls{mri} sequences, can also be employed for determining a tumour's vasculature and identifying major vessels involved\autocite{Kashimura2008,Kim2019}.
\Gls{fmri} and navigated transcranial magnetic stimulation \autocite{WeissLucas2020} map out eloquent cortex lying in proximity to the lesion, including the motor, language, and sensory cortices, supplementing the purely structural data obtained from conventional \gls{mri} or \gls{ct}.
In epilepsy patients, \gls{eeg} may be employed to monitor epileptogenic regions\autocite{Sarco2006}.

\Gls{dmri} is playing an ever-increasingly important role for neurosurgical planning and navigation\autocite{Manan2022}, especially as focus moves from functional localisation towards viewing the brain as an interconnected network.
Tumour and \gls{wm} interactions are varied and can be difficult to distinguish on conventional contrast \gls{mri} alone.
Depending on the infiltrative nature of a lesion, \gls{wm} tracts may be displaced due to mass effect, invaded but remain functionally intact, disrupted or destroyed, or experience a combination of effects\autocite{Essayed2017,DSouza2019,Manan2023}.
\Gls{dmri} can be instrumental in differentiating these circumstances and assessing tract integrity\autocite{Field2004,Manan2023}, but care must be taken to recognise how tumour effects may disturb diffusion patterns and impact the results.
Peritumoural oedema and invading cells can lead to drastically altered diffusion signal measurements and reduction of anisotropy, complicating their interpretation and inhibiting accurate streamline tracking\autocite{Bulakbas2009,Nimsky2010,Kuhnt2013}.
Nevertheless, \gls{dti} and streamline tractography have brought dramatic improvements to neurosurgical planning, unlocking detailed visualisations of \gls{wm} tracts and their spatial relationships to the surgical target and even acting as a predictor of postoperative deficits with implications for preoperative patient counselling\autocite{Manan2022}.
This potential was recognised almost immediately, with \gls{dti} and early tractography quickly making their way into clinical practice \autocite{Lee2001,Mori2002a,Nimsky2005}.
% \paragraph*{White matter mapping}
%%%% --! The below may be picked up as plagiarism; heavily copied from HBM manuscript; check academic regulations !--
In the intervening years, research imaging has largely transitioned to the multi-fibre models and probabilistic algorithms described in Chapter \ref{chap:neuroimaging}, but in clinical practice tractography is frequently still based on \gls{dt} fibre orientation models \autocite{Toescu2020, Yang2021} and deterministic tracking algorithms.
As a notable example, popular neuronavigation platform provider Brainlab's iPlan\textregistered{} (single tensor)\autocite{Brainlab2012} and Elements (dual tensor)\autocite{Sollmann2020a} Fibre Tracking applications (Brainlab AG, Munich, Germany) use a version of the probabilistic \gls{fact} algorithm \autocite{Mori1999}.

Regardless of the particular combination of fibre model, algorithm and tracking criteria, streamline tractography is compromised by weaknesses that can lead to flawed results or interpretations if not accounted for\autocite{Rheault2020, Schilling2022, Schilling2019}.
The same techniques and associated limitations for reconstructing individual \gls{wm} bundles already described apply here too, and can even be exasperated by additional tumour-related effects.
\Gls{dt}-based tractography, already afflicted by low sensitivity in healthy applications, often encounters particular difficulties tracking through oedema and areas of infiltration even where intact and functioning fibres may persist\autocite{Leclercq2010}, leading to missed connections and dangerous blind spots in the very regions at risk during surgery, where accurate navigation is most critical \autocite{Kuhnt2013,Ashmore2020}.
Meanwhile, the high propensity for false positive streamlines typical of probabilistic algorithms can be even more difficult to manage when tumour deformations disturb normal fibre orientations and inhibit accurate placement of \glspl{roi}\autocite{Yang2021}.

Perhaps clinical translation of multi-fibre probabilistic tractography has also been muted on account of its lower ease of use and practicality.
\Gls{dt} acquisitions can have as few as six diffusion-weighted directions, resulting in much shorter scan times compared to full \gls{hardi} scans.
Deterministic tracking itself is rapid, and the placement of \glspl{roi} need not be as strict as with probabilistic tractography owing to a lower sensitivity to false positives \autocite{ODonnell2017}.
A general lack of availability of the necessary expertise and time limits neurosurgical centres' access to state-of-the-art tractography \autocite{Toescu2020}.
Until recently, commercially available neurosurgical navigation platforms have exclusively supported \gls{dt} modelling and deterministic tractography (a recent exception is the Medtronic Stealth\texttrademark{} S8 Tractography application (Medtronic, USA), which implements \gls{csd}-based tractography\autocite{Pozzilli2023} as well as \gls{dt}).
This lack of readily available alternatives in the neurosurgeon's workflow and certification for safe clinical use is undoubtedly a major factor in the persisting preference for deterministic methods in clinical practice.
Nonetheless, there is growing consensus that (pending appropriate regulatory approval) the clinical community ought to adopt probabilistic, non-\gls{dt} tractography\autocite{Yang2021, Beare2022, Petersen2017}.
There is evidence that this shift is gradually underway, at least in the context of presurgical planning\autocite{Toescu2020}, driven probably by a combination and feedback loop of growing demand and better availability and integration of advanced techniques into the clinical workflow.

\begin{SCfigure}[][htb!]
  \includegraphics[width=0.5\textwidth]{chapter_2/neuronav.png}
  \caption[Streamline tractography for neurosurgical planning and navigation]{Idealised demonstration of tractography for neurosurgical planning and navigation in a paediatric patient with a left ependymoma (orange, outlined) involved with the \gls{or} (green) and \gls{cst} (blue). (Illustrative only, not a depiction of real clinical tractography.)}
  \label{fig:nav}
\end{SCfigure}

\section{Neuronavigation and brain shift}

During the surgical procedure itself, multimodal information streams continue to guide the safest possible resection.
Functional monitoring with \gls{des} is a crucial component of neurosurgical workflows and widely considered the gold standard for localising neural function after craniotomy.
Electrical current is applied to the cortical surface at increasing strengths\autocite{Saito2015}, and where stimulation elicits a functional response or disruption, the corresponding region is deemed eloquent.
Additionally, stimulation of subcortical white matter behind the resection cavity wall can be used to indicate when surgery should be halted as underlying eloquent structures are approached\autocite{Sala2019}.
\Gls{des} can be utilised in awake or asleep paradigms.
In the former, patients are awakened after craniotomy, and perform structured cognitive tasks involving those cortical hubs that may be at risk while undergoing electrical stimulation.
It is commonly considered for \gls{lgg} treatment within the UK, and to a lesser extent for \gls{hgg}\autocite{WykesV.2017}.
Language function is perhaps the most common target for awake stimulation as well as high-level motor tasks (such as playing an instrument) and vision\autocite{Mazerand2017}.
Awake surgery is technically complex, psycho-cognitively and emotionally demanding of the patient, and not universally tolerated\autocite{Nossek2013a,Wang2019}.
In very young children, awake surgery is rarely possible except in the most cooperative and resilient patients, and with appropriate preparation\autocite{Zolotova2022}.
\Gls{des} may also be performed in asleep patients, albeit limited to assessing motor and somatosensory function, where stimulation elicited somatosensory or motor evoked potentials (SSEPs, MEPs) can be measured in the patient's sensory cortex or muscles\autocite{Stone2019} (although the effects of anaesthesia can limit the sensitivity and accuracy of this approach\autocite{Stone2019,WeissLucas2020}).
There is also a considerable risk of intraoperative seizures, particularly in younger patients, which can lead to failure of awake functional mapping and potential increased risk of postoperative deficits\autocite{Nossek2013,Wang2019,Rigolo2020a}.
The neurological and oncological benefits associated with greater \gls{eor} discussed in Section \ref{sec:eor} have been achieved in large part with the help of awake functional mapping, primarily in adult patients.
It is therefore vital to develop and improve alternatives to cortical mapping to achieve maximal safe resection in all populations and especially in those patients who cannot undergo awake surgery, including some children.

Imaging and functional data acquired in preparation for surgery is not only instrumental to surgical planning, it also serves to guide the surgeon throughout the procedure, providing real-time multidimensional navigational information to supplement their live view through the surgical microscope.
Image guidance can involve simply displaying preoperative imaging and mapping in the theatre, while more advanced systems can also indicate the positions of surgical tools, or overlay imaging information on the microscope view.
This is achieved through stereotactic image guided surgery, which arrived with the introduction of frame-based systems in the later half of the 20th century, later largely giving way to frameless setups for craniotomies\autocite{Sandeman1995}, which are considered more time and cost effective\autocite{Sattur2019}.
In the former case, the patient's head is fixed within a stereotactic frame which guides the positioning of surgical tools, while in modern frameless systems the tools are tracked remotely, most commonly by an infrared camera system\autocite{Sattur2019}.
Fiducial markers, affixed either to the frame or patient, are detected on imaging and registered to the operating room coordinate system, allowing the tools' and patient's positions to be mapped and displayed on imaging in real time.
Intraoperative navigation with preoperative \gls{fmri} and \gls{dti} or tractography can improve \gls{eor} and preservation of critical cortical and subcortical function\autocite{Wu2007,Bello2008,Bello2010d}, particularly when combined with \gls{des} or awake surgery\autocite{Aibar-Duran2020}.
Where awake surgery is contraindicated or abandoned, preoperative functional mapping remains the only guidance available for higher cognitive functions, playing a crucial role in improving surgical care for patients who would not qualify for awake surgery.
\textcite{Rigolo2020a} found that preoperative \gls{fmri} guidance enabled safe resection of tumours or epileptic foci to continue after failed or incomplete \gls{des} function mapping, with no significant difference in postoperative morbidity.

Maximising \gls{eor} has further been significantly improved with the introduction of 5-aminolevulinic acid (5-ALA) guidance.
This compound is administered orally prior to surgery and is converted within cells to protoporphyrin IX (PPIX), which fluoresces when excited by short wavelength light.
Uptake of 5-ALA is highest in tumour cells due to \gls{bbb} disruption, and the metabolic pathways producing PPIX are more active in tumour cells, allowing the surgeon to distinguish them from healthy tissue under the surgical microscope.
5-ALA guided surgery results in improved \gls{eor} and higher progression free survival, while maintaining preservation of functional tissue\autocite{Coburger2019,Golub2020}, has seen widespread adoption in the surgical treatment of gliomas\autocite{Stummer2006}, and is included in UK national care guidelines\autocite{NICE2021}.

\begin{SCfigure}[][h!]
  \includesvg[height=\textheight,pretex=\small\sffamily]{chapter_2/brain_shift.svg}
  \caption[Intraoperative brain shift]{Illustration of intraoperative brain shift and its effect on neuronavigation. \textbf{a.} Preoperative imaging of a right \gls{who} grade 1 epidermoid lesion patient. Image is a $T_1$ weighted structural scan overlaid with \gls{csd}-derived \gls{dec} map. White arrowhead indicates medial displacement of the \gls{cst} (coloured blue/purple). \textbf{b.} Intraoperative imaging with partially resected lesion. Brain shift has caused the \gls{cst} to relax laterally towards the craniotomy (white arrowhead). \textbf{c.} Streamline tractography reconstructions of the \gls{cst} from preoperative (red) and intraoperative (green) \gls{dmri}, with areas of overlap in yellow. Note how the red streamlines give the impression of a tract further from the resection margin.}
  \label{fig:shift}
\end{SCfigure}

The dynamic conditions of brain surgery result in the unpredictable and often substantial movement, compression and deformation of tissue referred to as brain shift.
A range of factors contribute to this phenomenon, including \gls{csf} drainage, sagging due to gravity, decompression of tissue surrounding the resection cavity, swelling, craniotomy herniation and the effects of surgical instruments\autocite{Gerard2017}.
These factors may act in competing directions and combine in complex ways, for example swelling and tumour debulking can cause brain shift towards the craniotomy, while gravity and \gls{csf} drainage may have the opposite effect\autocite{Roberts1998}.
With the magnitude and direction of brain shift being so unpredictable, ranging from 1~mm to as much as 50~mm\autocite{Gerard2017}, accounting for it with predictive modelling is very difficult\autocite{Bayer2017b}.

Brain shift can affect the neurosurgeon's perception of the shape and location of the target lesion and invalidate preoperative imaging used for navigation\autocite{Nimsky2000}.
Many neurosurgeons rely on intuition to update their mental map of the surgical site throughout the procedure.
On more advanced neuronavigational platforms which integrate preoperative imaging and intraoperative data such as \gls{des} stimulation sites, brain shift can lead to misleading and inaccurate depictions of the spatial relationships between tumour and surrounding structures.
In particular, accurate localisation on image-guided navigation systems of deep tumour margins and the functionally eloquent structures beyond is significantly compromised by brain deformation and cannot be as easily mentally compensated for by the neurosurgeon as visible surface movements\autocite{Nimsky2000}.
Numerous techniques have been proposed to address the problem of brain shift\autocite{Bayer2017b} by dynamically adjusting preoperative imaging with patient specific deformation modelling.
Some rely entirely on preoperative imaging in combination with predictive modelling to simulate deformations, others include sparse or alternative modality intraoperative data, including sparse tracking of cortical surface features\autocite{Luo2019}, optical imaging of the cortical surface\autocite{Skrinjar2002,Audette2005,Fan2017} and intraoperative \gls{us} \autocite{Letteboer2005,Reinertsen2007,Bucki2012,Machado2019}, to estimate brain shift and update preoperative imaging accordingly.
Yet \textcite{Yang2017a} found that \gls{wm} tract shift direction was independent of cortical surface shift.
Ultimately, the most accurate 3D patient anatomy information can only be obtained with full 3D structural imaging after brain shift has occurred.

% Options without intraop imaging rely heavily on computationally modelling, with either limited accuracy or high computational time

\section{Intraoperative imaging}

To mitigate the effects of brain shift on neuronavigation accuracy, new structural and functional guidance information can be acquired intraoperatively.
Once more, different modalities offer different strengths and weaknesses.
Ultrasound imaging can probe into tissue beyond the surface, is safe, and can be operated at the surgical table without needing to move the patient\autocite{Elmesallamy2019,Eljamel2016}.
Doppler ultrasound is particularly useful for detecting intra- and peritumoural vasculature\autocite{Steno2016}, although image quality is limited and can be difficult to compare with other imaging modalities such as \gls{mri}\autocite{Eljamel2016}.
Specialised \gls{ct} systems can also be utilised intraoperatively \autocite{Bayer2018}, but they cause additional patient exposure to ionising radiation which is to be avoided wherever possible.

\Gls{imri} is becoming an increasingly common and valued addition to neurosurgical set-ups.
This includes low field ($<1$T) open bore systems which can be installed in the operating room, and allow for easy transfer of the patient into the scanner\autocite{Steinmeier1998,Senft2010}, as well as full high-field ($1.5-3$T) systems which acquire far higher quality images with potentially more clinical utility\autocite{Makary2011} at the expense of practicality, as interrupting surgery for an extended scan session and safely transferring a patient from the operating table to inside the scanner bore in an adjoining room is a substantial logistical and medical undertaking\autocite{Senft2010,Giordano2016a,Sattur2019}.
Technical challenges notwithstanding, \gls{imri} is incredibly valuable for determining surgical margins and providing guidance after the effects of brain shift deformations have invalidated preoperative imaging.

Numerous works demonstrate the benefits of \gls{imri} for improving postsurgical outcomes in both tumour and epilepsy surgery, including greater \gls{eor}, fewer new postoperative deficits, greater postoperative seizure freedom, and reduced length of hospital stay in both adults and children
\autocite{Shah2012,Zhang2015a,Sacino2016,Rao2017c,Giordano2017,Lu2018a,Garzon-Muvdi2019,Leroy2019,Karsy2019,Golub2020,Hlavac2020,Englman2021}.
%Giordano2017: safe in children; positioning
Even with advanced image and surgical guidance, postoperative \gls{mri} may show tumour nodules unintentionally left in the brain, concealed in corners of the resection cavity.
In some cases early repeat surgery is required to remove residual disease, resulting in significant additional clinical burden to the patient, longer hospital stays, heightened risk of complications including wound infection\autocite{Tenney1985,Chang2003}, delays in the planning of adjuvant treatment, and greater financial expense\autocite{Shah2012}.
Intraoperative imaging enables these remaining tumour margins to be detected and fully resected within the same surgery\autocite{Sattur2019,Hlavac2020}.
For low-grade lesions, the use of \gls{imri} could mean the difference between a potentially curative procedure and one leaving the patient at risk for recurrence and/or reoperation\autocite{Shah2012}.
Where \gls{imri} indicates no need for further resection, it can replace the need to subject patients to an additional scan immediately following surgery with all the discomfort and possible sedation, or in children, general anaesthesia, that it would entail.

While there is still some debate over the overall cost to benefit ratio of high-field \gls{imri} systems\autocite{Eljamel2016,Giordano2016a,Giussani2022}, and indeed high initial capital expense remains one of the primary barriers to their installation, cost-effectiveness analyses have indicated that upfront investments are recouped by lifetime savings associated with shorter hospital stays, and improved postoperative recovery and survival\autocite{Giordano2016a,Sacino2018}.
Cost-effectiveness can be further increased with a dual use set-up, in which an \gls{imri} system is used both for intraoperative and routine diagnostic scanning\autocite{Giordano2016a}, as is the case at \gls{gosh}.
Further challenges of implementing \gls{imri} include longer operating times, equipment compatibility, and patient positioning for both navigated surgery and scanning\autocite{Giordano2017}.
It is worth noting that the strength of evidence supporting all \gls{imri} use and cost-effectiveness is still under debate \autocite{Jenkinson2018,Garzon-Muvdi2019,Caras2020}, and interpretation of \gls{imri} studies is confounded by selection bias and a lack of randomised control trials\autocite{Kubben2011}.

% Next: skim open papers; discuss conventional contrast imaging, move to other acquisitions (dwi: infarct etc, then more advanced); compare with iop acquisition of dmri with coregistration (deformable) between preop advanced and intraop structural.
% end with technical considerations (in particular for children), costs etc.
% Note that more research needed to assess intraop diffusion mri.

The majority of \gls{imri} sequences are $T_1$-weighted (with or without injected contrast agent enhancement), $T_2$-weighted, and other conventional acquisitions with good tumour tissue contrast\autocite{Kubben2011,Coburger2019}.
In line with this finding, the majority of literature reviewed surrounding \gls{imri} focusses on evaluating \gls{eor}, and in this regard the technique has gradually established itself as a clearly beneficial and in some centres indispensable surgical aid\autocite{Garzon-Muvdi2019,Hlavac2020}.
What has been less extensively studied is the potential for \gls{imri} to provide updated advanced functional neuronavigation.
Broadly speaking, this can be approached in two ways.
The first is to use conventional \gls{imri} to dynamically adjust preoperative functional information (including \gls{fmri} and tractography) using deformable image registration and/or biomechanical brain shift modelling, the second is to directly acquire new advanced \gls{mri} sequences intraoperatively.
While the former may seem like the preferred choice, as one avoids having to acquire and process additional lengthy scans and further prolonging surgery interruption, achieving robust deformable registration in a reasonable timeframe is by no means trivial.
It is especially complicated given the significantly disturbed anatomy following craniotomy and complex brain shift, as well as the effects of air in the resection cavity not present on preoperative imaging.
Intraoperative \gls{fmri} has been reported in a very limited number of studies in \gls{dbs}\autocite{Hiss2015,Knight2015}, tumour resection\autocite{Roder2016a,Qiu2017a}, and neurovascular\autocite{Muscas2019} surgeries, but it's unlikely to find widespread use on account of practicality and an established preference for \gls{des} for cortical mapping.
By contrast, intraoperative \gls{dmri} has several potential advantages.

Use of tractography for neuronavigation is valued by many, but its accuracy diminishes with increasing brain shift, with precision most compromised during later stages of a resection, at the same time as it is potentially approaching critical subcortical \gls{wm}\autocite{Yang2019}.
Some account for this by registering preoperative tractography or \gls{dti} \gls{dec} maps to intraoperative structural \gls{mri}\autocite{Nimsky2006a,Tamura2022}, which depends on robust and accurate registration\autocite{Beare2016}.
Alternatively, intraoperative acquisition of \gls{dti} and even \gls{hardi} scans has been garnering interest.
Early work by \textcite{Nimsky2005}$^,$\autocite{Nimsky2005a} demonstrated the feasibility of intraoperative \gls{dmri} and tractography, illustrating \gls{wm} tract shifting after tumour resection.
Safe resection of tumours and epilepsy foci aided by intraoperative reconstruction of motor\autocite{Maesawa2010,Javadi2017}, language\autocite{DAndrea2016,Li2021} and visual\autocite{Daga2012,Cui2015} pathways has been confirmed in subsequent studies.
Overall, a systematic review by \citeauthor{Aylmore} (in production) found 26 articles reporting intraoperative \gls{dti} or tractography and associated outcome measures, with moderate suggestion of benefit to surgical success\autocite{Aylmore}.

There are considerable technical considerations associated with intraoperative \gls{dmri}, even more so than with routine imaging.
Duration is of course one, with minimum possible scan time a priority for interrupted surgeries.
Secondly, intraoperative \gls{dmri} can suffer from degraded image quality\autocite{Roder2019}, and distortion artefacts common in \gls{dmri} sequences utilising \gls{epi} are exasperated by the tissue--air interface at the craniotomy site\autocite{Elliott2020}, which can significantly diminish the accuracy of \gls{wm} tract localisation\autocite{Yang2022}.
Early studies suggested that intraoperative tract reconstructions may not be not reliable on their own, but useful in combination with adjuvant functional mapping such as \gls{des}\autocite{Ostry2013}.
Nearly all current implementations of intraoperative tractography are limited to commercially available \gls{dt}-based deterministic algorithms, which have known limitations particularly in application to pathology.
Nevertheless, concrete evidence of benefits to postoperative outcomes with intraoperative advanced \gls{dmri} is mounting\autocite{DAndrea2012,Cui2015,Maesawa2010}, and in combination with what we already know about conventional \gls{imri} and preoperative imaging for \gls{wm} neuronavigation, it stands to reason that intraoperative \gls{dmri} could bring significant improvements to brain surgery if some of the technical and image processing limitations can be overcome.

All neuronavigational tools are complementary, each providing distinct streams of information, and where possible, combination of techniques can increase the chances of achieving an optimal outcome.
For example, maximal safe resection with combined 5-ALA and \gls{imri} guidance is recommended by UK national clinical guidelines for surgical treatment of gliomas, which further advise considering the use of \gls{dti} and awake craniotomy to optimise safety and effectiveness\autocite{NICE2021}.
Reviews have found similar benefits to \gls{eor} from both \gls{imri} and 5-ALA guidance separately\autocite{Golub2020} and even better outcomes when combined\autocite{Nickel2018,Coburger2019}.
5-ALA provides the surgeon with a direct visualisation of tumour cells in the surgical view, but residual nodules may remain out of sight behind cavity convexities and only show up on \gls{imri}\autocite{SueroMolina2019}.
Combining awake surgery with \gls{imri} is also viable and beneficial in some patients \autocite{Motomura2017,Tuleasca2021}.
Advanced functional mapping, presurgical planning, and intraoperative monitoring, where the appropriate resources and expertise are available\autocite{GeorgeZakiGhali2020}, all increase the operability of highly eloquent gliomas
\autocite{Bello2008,Krieg2013,DellaPuppa2013b,Magill2018}.

\section{Research topic}
\label{research}

\note{This section needs a better title.
Basically drawing on the conclusions of preceding literature review to lay out the problem statement and project aims.}


\part{Tractfinder}
\epigraph{I've spent my life trying to make things simpler. \\ Because I find ultimately that complicated doesn't reach the heart.}{Hans Zimmer}
\chapter{Tract Orientation Atlas}
\label{chapterlabel2}

\note{Somehow also include the ``mapping" part in the chapter title?}

\section{Purpose}

\note{A general of the rationale behind the atlas}

The tract orientation atlas captures the typical spatial and orientational distribution of a give tract across a sample of healthy subjects. Being derived from carefully curated streamline tractography reconstructions, it can be considered a store of anatomical prior expectations which would otherwise be utilised to draw appropriate ROIs for tractography in a target image.


An orientation distribution map is constructed from each subject streamline set using track orientation distribution mapping (\note{spelling? + cite}).\autocite{Dhollander2014}

In order to justify the atlas creation steps, it is worth first considering the information channels aimed to be contained within the atlas and their interpretation.

First is the orientation distribution, which will be determined from the TOD mapping of streamlines.
Second is the spatial distribution, i.e. the likelihood of the tract residing in a given voxel.
The latter concept is sometimes conflated with or loosely equated to streamline density derived from tractography, however this interpretation is highly flawed and biased.

Instead, it is more useful to consider the binary state of tract belonging in each individual subject, and then determine the proportion of subjects in which a voxel belonged to the tract.

\textit{From HBM submission:}

The key component of the proposed tract mapping method is the use of a tract-specific atlas of fibre bundle orientation and location. The purpose of the tract atlas is to capture and store prior anatomical knowledge of a given tract, including its typical location and orientation across subjects.
While this is hereinafter referred to simply as a tract orientation atlas, and this section will focus on the orientation component, each tract atlas incorporates both orientational and spatial information.

\section{Construction}

\note{How each atlas is created in an artisanal Bloomsbury lab. Also general stuff like training data, preprocessing etc.}


The objective is to create a map in a template space capturing, at each location, the range of possible orientations the tract can take on as a single spherical distribution.
A narrow distribution may be found where the tract's orientation is highly consistent across all subjects, whereas a more spread-out distribution would reflect a wider range of possible orientations, which may be seen in regions of fanning or sharp turning (Fig. \note{fig:mean}).

\subsection{Streamline tractography and filtering}

\note{The streamline tractography and ROI strategies for each tract. Somewhere, will want to include brief lit reviews on the tracts studies justifying the chosen cortical terminations. Maybe here?. Also include filtering in DSI studio, OFPs etc.}


To obtain such a mapping, a combination of streamline tractography and TOD mapping \autocite{Dhollander2014} is used.
While tractography has significant limitations as discussed above, it remains the standard way of segmenting white matter bundles from \textit{in vivo} dMRI data, and biases and errors can, with appropriate post-processing steps, be at least partially corrected for.
In addition, tractography uniquely enables the extraction of orientation information specific to the reconstructed bundle, which would not be possible from a binary voxel-wise segmentation.

A dataset of 16 healthy adult HARDI acquisitions (``EEG, fMRI and NODDI dataset",\autocite{Clayden2020} available online at osf.io/94c5t) was used in the creation of reference bundles for developing the atlas.
In each subject, the bundle of interest was reconstructed in both hemispheres using probabilistic streamline tractography with iFOD2 \autocite{Tournier2010} and a consistent ROI strategy based on anatomical landmarks broadly agreed upon in prior works. (See \note{sec:tractography} for full details on tractography parameters and ROI strategies used for each tract.)

After streamline generation, each streamline bundle was transformed into MNI space using affine registration implemented in FMRIB's Linear Image Registration Tool\autocite{Jenkinson2002} between the subject's T1-weighted image and the MNI152 T1 template.\autocite{Fonov2011}
Affine registration rather than non-linear, was used for this step to capture individual anatomical variation and minimise unrealistic warping of streamlines from local registration errors or overfitting.
With all subject streamlines aggregated in MNI space, manual filtering of streamlines was performed (Fig. \note{fig:atlases}) to remove not only ``volumetric false positives", which depart from the accepted volume of the tract, but also ``orientational false positives" (OFPs), which remain entirely within the tract volume but are at least in part aligned with a different, intersecting bundle.
An example of such OFPs are depicted in supplementary Figure \note{fig:ofp}. Such streamlines have little effect on any volumetric applications of the reconstruction, e.g. via a track density depiction.
However, their removal is vital for the construction of the orientation atlas, which summarises the orientational distribution of streamlines on a voxel-wise basis.
Filtering was performed in DSI studio (v2021\_04, \url{https://dsi-studio.labsolver.org/})\autocite{Yeh2021a}, which enables the filtering of streamlines based on angle of intersection with a cutting plane.
The percentage of streamlines filtered for each tract and summarised reasons for removal are presented in Table \ref{tab:filt}.

%%%%%%%%%%%%%%%%%%%%%%%%%%%%%%%%%%%%%%%%%%%%%%%%%%%%%%%%%%%%%%%%%%%%%%%%%%%%%%%%
\begin{table*}[t]
  \caption{Streamline filtering statistics. Abbreviations: AF, arcuate fasciculus; CrP, cerebellar peduncles; CC, corpus callosum; CST, corticospinal tract; EC, external capsule; SLF, superior longitudinal fasciculus; SFOF, superior fronto-occipital fasciculus }
  \label{tab:filt}
  \begin{tabularx}{\textwidth}{llllll X}
   &  & Original & Filtered & Difference & Reduction & Reasons for discarding \\
   \hline
  CST & left & 148833 & 145300 & 3533 & 2.37\% & \multirow{4}{20em}{Contamination from: AF / SLF, SFOF, CC, CrP} \\
   & right & 144759 & 139019 & 5740 & 3.97\% &  \\
   & total & 293592 & 284319 & 9273 & 3.16\% &  \\
   \\
  AF & left & 61922 & 49778 & 12144 & 19.61\% & \multirow{4}{20em}{Contamination from:  EC, CST, CC\\ Overextension into: Motor cortex, anterior temporal cortex, superior frontal cortex} \\
   & right & 61834 & 43027 & 18807 & 30.42\% &  \\
   & total & 123756 & 92805 & 30951 & 25.01\% &  \\
   \\
  OR & left & 123842 & 99984 & 23858 & 19.26\% & \multirow{4}{20em}{Contamination from: Tapetum of CC, SLF} \\
   & right & 122534 & 109265 & 13269 & 10.83\% &  \\
   & total & 246376 & 209249 & 37127 & 15.07\% & \\
   \\
 \end{tabularx}
\end{table*}
%%%%%%%%%%%%%%%%%%%%%%%%%%%%%%%%%%%%%%%%%%%%%%%%%%%%%%%%%%%%%%%%%%%%%%%%%%%%%%%%

\subsection{TOD mapping}

\note{Registration to MNI space and TOD mapping. Includes the whole spiel about normalisation.}


After aggregate filtering, the retained streamlines were re-separated into individual subject bundles and the TOD was computed from the individual bundles as described in \textcite{Dhollander2014} and implemented in MRtrix3. \autocite{Tournier2019}
TOD mapping is the generalisation of track density imaging into the angular domain, creating a 5D spatio-angular representation of streamline tracks on a voxel-wise basis.
The TOD image is represented in modified spherical harmonic basis \autocite{Descoteaux2006} using only even orders up to a maximum order $l_{max}=8$, meaning each image consists of 45 coefficients, denoted $t_j$, per voxel.
The distribution is described by those coefficients and the modified spherical harmonic basis functions $Y_{l,m}$ \autocite{Descoteaux2006} as

\begin{align}
  T(\theta, \phi) = \sum_{l=0}^{l_{max}} \sum_{m=-l}^l t_{l,m} Y_{l,m}(\theta, \phi) = \sum_j t_jY_j(\theta, \phi)
\end{align}

The individual TOD images at this stage still contain significant density bias, with exaggerated differences in magnitude between the core bundle portions and fanning extremities owing to tractography's tendency towards early termination outside of the densest collinear tract regions.\autocite{Rheault2020,Smith2013}
The purpose of the atlas is to capture only the likelihood of a tract's presence in any given voxel (spatial prior) and, in the case that it is present, its expected orientation (orientational prior).
If the spatial prior is to be determined by considering the spatial variation of the tract between subjects, then the only information needed for each individual subject is a binary visitation map for the bundle and orientational data.
Thus to remove the streamline density component, the TOD maps for each subject are normalised as follows.
The spherical integral of each SH basis function $Y_{l,m}$ is

\begin{align}
  \int_{\Omega} Y^m_l(\theta, \phi) = \begin{cases}
   \sqrt{4\pi} & \text{ if } l=m=0\\
   0 & \text{ otherwise. }
  \end{cases}
\end{align}

Using the sum and constant rules of integration, the spherical integral of $T(\theta,\phi)$ is
\begin{align}
  \int_{\Omega} T(\theta,\phi) = t_0 \sqrt{4\pi}
\end{align}

where $t_0$ is the first SH coefficient for $l=m=0$. Thus to remove density information the TOD map is normalised to unit integral as

\begin{align}
  \widetilde{T}(\theta, \phi) = \frac{T(\theta,\phi)}{\sqrt{4\pi} t_0}
\end{align}

After each individual TOD map has been normalised in MNI space, what remains contains only information about the tract's streamline orientations, and none about the number of streamlines passing through a given voxel in the original reconstruction.

Finally, the mean of all individual normalised TOD maps is computed to produce the final population tract TOD atlas.
Averaging all maps results in distributions that reflect all possible ranges of tract orientations in each voxel (Fig. \note{fig:mean}), while the first SH coefficient of the atlas will reflect the proportion of training subjects in which the tract was present in a given voxel.
Outlier voxels visited by streamlines in only a single subject's reconstruction will contribute little weight to the final atlas.
This atlas can then be registered to a target subject for further processing.
Atlases have so far been created for the most commonly indicated pathways in neurosurgical planning and guidance, namely the corticospinal tract (CST), optic radiations (OR) and arcuate fasciculus (AF; Fig. \note{fig:atlases}), with the creation of further atlases to be the subject of future work.

\subsection{Discussion}

\note{Consider different elements of the atlas construction, e.g. what is the effect (smoothing) of using only non-linear registration and possibly put the discussion on number of training subjects here too?}

\section{Tract mapping}
\label{chapterlabel4}

\note{This part is about the inner product, plus any other comparison between atlas and data, including e.g. KL divergence, fixel anaylsis...}

The methodological description has been, up to this point, concerned with capturing
prior information about tracts and bringing those priors into alignment with the target anatomy.
Finally, it remains to compare those priors with the information present in the target diffusion data itself.

\subsection{Inner Product}

\subsection{Fixel analysis}

\section{Atlas data requirements}\label{sec:ntrain}

In segmentation methods which ``learn" patterns from seen data to apply to unseen data, the volume and range of training data influences the prediction accuracy and generalisability.
For complex deep learning models, which have many thousands of network parameters to learn, the amount of training data required to achieve accurate and stable performance can be immense, posing a particular barrier to the use of such models in applications where suitably annotated data is scarce.
In the case of the deep learning tool TractSeg\autocite{Wasserthal2018}, for example, 105 subjects in total were used for cross-validation training, with each fully trained model having seen randomly sampled data slices from 63 unique subjects.
%https://github.com/MIC-DKFZ/TractSeg/issues/240
In tractfinder, the number of subjects used to construct each atlas influences the amount of inter-subject anatomical variation reflected in the spatial and orientational components.
It is to be expected that, save for extreme outliers, the additional information gained from adding more training subjects would reach a point of saturation.

To investigate this, an experiment was conducted whereby the number of subjects included in atlas construction was varied, and the effect on segmentation accuracy compared.
For this purpose the TractSeg \gls{hcp} bundles were used in order to enable objective evaluation against reference segmentations defined in the same manner as the training data, and direct comparison with TractSeg itself.
Using the same train - test data split as described in \textcite{Wasserthal2018b}, subsets of 1, 3, 5, 10, 15 and 30, as well as the full 63 training subjects were randomly selected, from which separate \gls{tod} atlases where constructed.
Tractfinder maps were then generated in the 42 test subjects using each of the different subset atlases and compared with the reference segmentations using the \gls{dice} and density correlation metrics.

\begin{figure}[htb!]
    \centering
    \makebox[\linewidth][r]{%
    \includegraphics{chapter_3/ntrain.pdf}}
    \caption{Comparison of segmentation performance using different numbers of atlas training subjects. Results are grouped by tract, colour represents number of training subjects. The IFO and \gls{or} are in places indistinguishable. \acrolist{af,cst,or,ifof}}
    \label{fig:ntrain}
\end{figure}

When using only a single subject's normalised \gls{tod} map as an ``atlas", mean \glspl{dice} ranged from 0.65 to 0.71 for the \gls{ifof} and \gls{cst} respectively (Fig. \ref{fig:ntrain}).
% These figures are from using the script compare_atlas_size.py committed at sha 33777217
The maximum increase in mean \gls{dice} between the 15 and 63 subjects atlases was 0.00835, for the \gls{cst}, representing only a 1\% increase from the lower score of 0.759.
Across all tracts and both comparison metrics, differences in performance between the different atlases were consistently negligible.
These results indicate that additional atlas subjects beyond a minimum number of around 10 to 15 do little or nothing to improve tractfinder results.
This can be interpreted as the extra training subjects offering minimal additional information on inter-subject variability, as a lot of this variability is already smoothed out due to affine (instead of diffeomorphic) co-registration of training subjects into template space.

The effects of additional training data may present differently if the atlases are constructed with non-linear co-registration of training subjects. %\note{this might belong somewhere else.}
There are two sources of inter-subject variability wrapped up in the atlas:
The first is the global anatomical variability including skull shape and differences in cortical surface geometry, the second is variability in position and shape of the tract itself.
Theoretically, diffeomorphic registration of training subjects would eliminate the first of these effects (global variability), leaving only the tract specific variation.
However, such an atlas would necessitate subsequent applications in new target subjects to also utilise diffeomorphic registration between subject and template space, as the atlas would contain no ``allowance" for global variability, expecting perfect alignment with a target image.
Requiring diffeomorphic registration at the point of application would greatly inhibit the robustness and speed of tractfinder, and is therefore not the preferred approach.

% Query link: https://www.webofscience.com/wos/woscc/summary/28ab4b31-92c1-4d9e-b117-d5cd310c9927-a4fe70c6/relevance/1
The issue of deep learning's data requirements has been attracting increased attention.
A search on the online publication database Web of Science for \spverb|(few OR single OR one) AND shot AND learning|  (excluding results about single-shot echo planar imaging) revealed a sharply increasing trend in publication volume within the field of medical imaging (Fig. \ref{fig:pubs}).
A recent example, by \textcite{Liu2023a} and building on previous work in \textcite{Lu2021}, looked at using single-shot learning to train a deep neural network, an extension of the TractSeg architecture, for white matter tract segmentation.
They trained a network on 60 of the 72 total TractSeg white matter tracts, and applied data-augmentation and transfer learning techniques to adapt the model to segment the remaining 12 tracts using only a single exemplar training dataset for those novel tracts, which included both the \gls{cst} and \gls{or}.

Considering only their best results (which differed depending on how many novel tracts the network was extended to) for these two tracts, they achieved a mean \gls{dice} of 0.719 for the \gls{cst} and 0.624 for the \gls{or}, equally good or worse than for tractfinder ``trained" even on only a single dataset (Fig. \ref{fig:ntrain}).
While the computational time spent on network training is not disclosed in either \textcite{Liu2023a} or \textcite{Wasserthal2018}, \textcite{Berto2021} re-trained their own TractSeg model as a benchmark comparison for the streamline clustering method Classifyber and reported GPU-accelerated training times of 3--7 hours.
Even so, one-shot training of a deep neural network appears to offer no improvement over simply registering a segmentation from one subject to another.

\begin{SCfigure}[][h!]
  \centering
  \includegraphics{chapter_3/pubs.pdf}
  \caption{Publication records by year including the term ``single/one/few shot learning" (or similar) on the database Web of Science.}
  \label{fig:pubs}
\end{SCfigure}


\chapter{Aligning atlas and data}
\label{chap:reg}

Each tract atlas is constructed in a standard template space, an averaged, idealised brain image in standardised reference coordinates, and before being used to estimate the location of a tract in a new subject it must first be aligned with that individual's anatomy.
In order to achieve this, some sort of registration strategy is needed, with the most appropriate registration approach depending on the type of application.
In healthy subjects, it is sufficient to use affine registration to transform the atlas from template into subject space, before the inner product (or whichever comparison is used) is calculated.
However, in some patient data, more advanced processing may be necessary to ensure satisfactory alignment with target anatomy.
The following sections, some elements of which have been previously published in \textcite{Young2022}, will describe the registration methods used in tractfinder for different scenarios, including the use of tumour deformation modelling to account for space-occupying lesions.

\section{Registration from standard space}
\label{sec:reg1}

There are two main categories of image registration.
In global registration, a single transform applies to the entire image, preserving its topology.
Different degrees of freedom define some commonly used terms:
Rigid body registration comprises just translation and rotation, while affine registration further includes scaling and shearing (up to 12 degrees of freedom on $\mathbb{R}^3$).
Following purely algebraic definition, a linear transform does not include translation, however in practice (at least in the medical imaging sphere) ``linear'' and ``affine'' are often used interchangeably.
Non-linear, deformable, or diffeomorphic registrations compute local deformations on a voxel-by-voxel basis, and therefore have orders of magnitude more degrees of freedom.
They are useful for fine-grained alignment of local structures, but many algorithms are unstable and prone to converging on local minima, making them difficult to successfully integrate into automated pipelines.

As we saw in Chapter \ref{chap:atlas}, a certain degree of inter-subject variability, including both tract-specific and global anatomical differences, is a feature of the atlas due to the use of affine registration to combine the training data in template space.
We mirror this relationship by allowing the atlas to be registered to a new target dataset using affine registration alone, with the subsequent comparison with native \gls{dmri} data acting to correct slight misalignments due to anatomical variability.
Computing a non-linear registration between atlas and subject at the point of application is undesirable principally due the lack of robust, stable and generalisable algorithms that are open source.
Maintaining practicality is a key requisite of tractfinder, and various non-linear registration tools were found during prototyping to either be too unstable (requiring manual adjustment of parameters in difficult cases such as some clinical scans) or having too long a computation time\autocite{Visser2020}.

As we shall see in the range of evaluations presented in Chapters \ref{chap:eval}--\ref{chap:applications}, the use of affine registration does not result in significant segmentation inaccuracies or errors, thus there is no credible incentive to favour the use of non-linear registration at the cost of increased processing time and potential instabilities.
Indeed, \textcite{Visser2020} showed that subject--to--standard registration accuracy of the tumoural region in low- and high-grade gliomas is not significantly improved using non-linear registration across a range of different packages, concluding there is little to justify the additional time cost and lack of robust automation.
Nevertheless, the trade-offs associated with relying on affine registration must be acknowledged.
Particularly in brains with modest deformations or other lesions, the accuracy of anatomical alignment can be impacted, and there follows the risk of suboptimal spatial and/or orientational alignment, resulting in missed areas (false negatives) or erroneously included regions (false positives).

The risk of misalignment is especially high for small structures, such as the fornix or anterior commissure, both narrow bundles which are reliably difficult to segment.
For these cases, the construction of the atlas with spatial inter-subject smoothing is advantageous, as even with slight misalignment there is still a good chance that enough of the atlas will overlap with the structure in the target image to achieve detection.
However, this same feature may also lead to false positives in some cases, such as when two distinct tracts run in parallel, oriented the same way in relative spatial proximity.
An example of this scenario can be seen at the temporoparietal fibre intersection area, where at least seven different identified bundles converge\autocite{Martino2013}.
Here the vertical portion of the arcuate fasciculus lies lateral to the sagittal stratum containing antero-posteriorly oriented association fibres (including the \gls{or}), which in turn lies lateral to the tapetum of the corpus callosum.
The similar orientations of the the \gls{af} and tapetum in this region, together with their close proximity, could lead to fibres of one being wrongly attributed to the other if the atlas is too broad (Fig. \ref{fig:tpfia}).

\begin{SCfigure}[][h!]
  \includesvg[width=0.6\textwidth,pretex=\ttfamily\small]{chapter_4/tpfia.svg}
  \caption[Atlas misalignment with linear registration]{Example of potential for atlas misalignment. The \gls{af} (A) and tapetum (T) are proximal and parallel at the temporoparietal fibre intersection area. Linearly registered right \gls{af} atlas \glspl{tod} may overlap with the tapetum (arrowhead).}
  \label{fig:tpfia}
\end{SCfigure}

Fortunately, such misalignment effects are small and unlikely to impact the overall segmentation quality, in part because only the margins of the atlas are likely to ``spill'' into neighbouring tracts, and the low spatial probability will result in very low or sub-threshold mapped values.
To confirm this assessment, Figure \ref{fig:nrr} shows tractfinder segmentation similarity scores compared with probabilistic targeted tractography when using either affine (FMRIB's Linear Image Registration Tool\autocite{Jenkinson2002}) or non-linear (ANTs registration package Symmetric Normalisation algorithm\autocite{Tustison2013,Avants2011}) atlas registration in 71 healthy subjects (\textit{TractoInferno} dataset, see Section \ref{sec:data} for more details).
Note that the atlases themselves are unchanged from those previously described and used in subsequent analyses, meaning the generation of the atlases still involves only affine registration between training subjects.
There is no discernible difference in the scores, indicating that non-linear registration does not improve the final output.
This comparison originally appeared as supplementary material in \textcite{Young2024}.

\begin{figure}
  \makebox[\linewidth][r]{%
  \includegraphics{chapter_4/registration_2.pdf}}
  \caption[Comparing linear and non-linear atlas registration]{Difference in tractfinder performance when using either affine or non-linear atlas registration, compared with targeted \gls{roi} tractography, in the \textit{TractoInferno} dataset of healthy subjects. For the \gls{dice} a threshold of 0.05 applies.}\label{fig:nrr}
\end{figure}

In conclusion, affine registration between template and subject space is preferred in most applications, including for healthy subjects and clinical subjects without large space-occupying lesions.
The next sections will address the scenarios where patient anatomical topology differs significantly from atlas space:
In the presence of deforming tumours, or intraoperative brain shift.

\section{Tumour deformation modelling}
\label{chapterlabel3}

\note{this has been dumped from IJCARS paper}

The tract orientation atlas represents the expected orientation and location of the tract for a typical healthy subject.
In order to correct for displacement of white matter tracts due to space occupying lesions, a simple radial tumour expansion model is employed.

The model has been adapted from the one described by Nowinski and Belov \citep{Nowinski2005}.
The model inputs are the segmentations of the tumour and brain volumes.
Tumour segmentations were drawn manually for this study, while brain masks are readily computed by standard MRI analysis software.
We define the direction $\mathbf{\hat{e}}$, which is the unit vector along the line connecting a point $P(x,y,z)$ to the tumour centre of mass, $S$.
Along $\mathbf{\hat{e}}$ we also define $D_p$ as the distance  $\|\overrightarrow{SP}\|$, $D_b$ as the distance from $S$ to the brain surface and $D_t$ as the distance from $S$ to the tumour surface (Fig. \note{fig:virtue}).

Then for a point in the original image $P = (x,y,z)$ the transformed location in the deformed image $P' = (x',y',z')$ is

\begin{align}\label{eq:forwardP}
  P' = f(P) = P + \mathbf{\hat{e}}kD_ts.
\end{align}

An exponentially decaying function is used to model the displacement of each voxel (Fig. \ref{fig:virtue-demo}).
This choice was made in contrast to the linear relationship used in \citep{Nowinski2005} as it provides a better approximation to typically observed tumour displacement patterns, while remaining an easily computable, closed-form and invertible function.
The amount of displacement depends exponentially on the relative distance to the tumour and brain surfaces via the following relationship:

\begin{align}\label{eq:forwardk}
  k(P) = (1-c)e^{-\lambda \frac{D_p}{D_b}} +c
\end{align}

where the normalisation constant $ c = \frac{e^{-\lambda}}{e^{-\lambda}-1} $ ensures that $k = 1$ when $D_P = 0$ and $k = 0$ when $D_p = D_b$. The appropriate value for the decay parameter $\lambda$ will depend on the specific lesion being modelled. For example, smaller lesions (20-30mm diameter) typically displace tissue only in their immediate surroundings, with distant tissue remaining virtually unmoved. In such cases, a higher value of $\lambda$ ($\geq 3$), indicating stronger decay of deformation, would be appropriate (Figure \ref{fig:virtue}).

Equations (\ref{eq:forwardP}) and (\ref{eq:forwardk}) describe the deformation field in forward warp convention. To deform an image using reverse warp (``pull-back") convention, the inverse mapping $P' = f^{-1}(P)$ is needed, which is obtained by solving equation (\ref{eq:forwardP}) for $P$:

\begin{align}
  P = P' - \mathbf{\hat{e}}(D_t c - \frac{D_b}{\lambda}\mathcal{W}_0(\frac{-\lambda D_t (1-c) e^{-\lambda(D_p'-D_tc)/D_b}}{D_b}))
\end{align}

where $\mathcal{W}_0(y)$ is the principal branch of the lambert $\mathcal{W}$ function, defined as the inverse function of $ y(x) = xe^x $ for $x,y \in \mathbb{R}$.

If the lesion is not invading the surrounding tissue but instead fully displacing it (non-infiltrative), then under the simplified assumption that no original, healthy tissue is destroyed, $\lambda$ should be set to a value that ensures that every point $P$ within the lesion boundary is displaced to a new position $P'$ that is strictly outside the boundary. In other words,

\begin{equation}\label{eq:lambdabound}
  k(P) = (1-c)e^{-\lambda \frac{D_P}{D_b}} +c \geq 1 - \frac{D_P}{D_t}
\end{equation}

must hold for all $P$.

Given that the gradient of $k$ is strictly decreasing and $g(D_P) = 1 - \frac{D_P}{D_t}$ is linear, it is sufficient to set
\begin{align}
  \frac{d}{dP}\bigg\rvert_{D_P=0}k(D_P) = \frac{d}{dP}\bigg\rvert_{D_P=0}g(D_P)
\end{align}.

Differentiating both functions at $D_P=0$ and solving for $\lambda$, we have $\lambda_{max} = \frac{D_b}{D_t (1-c)}$.
Thus for strictly non-infiltrating lesions, we set $\lambda \leq \lambda_{max}$ to satisfy equation (\ref{eq:lambdabound}), where $\lambda_{max}$ is used as the default value if none is specified. Note that $\lambda_{max}$ varies throughout the brain, as it depends on the relative distances to brain and tumour surfaces for each specific $P$.

The tumour deformation model is implemented in Python, and full execution takes on average 1 min for a 208 x 256 x 256 voxel image.
If lookup tables for $ D_t$ and $D_b$ are precomputed and saved, then subsequent executions of the model (e.g. with different values for $\lambda$ and $s$, as appropriate for a given tumour) take less than 10 seconds, as long as the tumour and brain segmentations remain unchanged.

\note{Maybe also some speculative stuff about infiltration}

\section{Intra-patient registration and brain shift}

Given the target application of intraoperative \gls{wm} imaging, it would be entirely remiss to not discuss the additional registration issues brought about by brain shift.
Predictably, the sheer variability in direction, magnitude and extent of brain shift means that a universally elegant and robust solution cannot be found.
Instead, we will discuss some common intraoperative scenarios and how tractfinder would be applied to them.

One major source of brain shift is tumour debulking.
As tumour is removed through the craniotomy, the surrounding tissue may collapse and the associated mass effect be correspondingly reduced.
For this situation, tumour deformation can be utilised using a preoperative segmentation of the tumour and adjustment of the tumour scale parameter $s$, which if set to a value below one will virtually decrease the tumour radius, mimicking the debulking effect.
A case study demonstrating this is presented in Section \ref{sec:case}.
The necessary steps in such a scenario would be affine registration between pre- and intraoperative subject space, re-computation of tumour deformation field using preoperative tumour segmentation and $s<1$, affine atlas transformation to intraoperative space, atlas deformation, and finally tract mapping with intraoperative \gls{fod} image.
The additional steps of pre- to intraoperative registration and tumour deformation would add less than five minutes to the total processing time.

Of course, tumour debulking does not always result in intraoperative brain shift that looks like a simple reduction in mass effect.
Often, a certain degree of tissue inertia means that surrounding structures don't instantly relax back into a more ``normal'' position, and the effects of gravity, \gls{csf} drainage, herniation, or intracranial pressure changes will have a far more drastic influence on overall brain shift than the removal of tumour tissue.
In these cases, tumour deformation modelling cannot be leveraged to account for brain shift, and different registration strategies will be required.
Affine registration with anisotropic scaling may be sufficient for brain shift principally characterised by sagging or compression due to gravity.

Finally, there are those cases in which topological differences between pre- and intraoperative scans are so great that neither affine registration nor adjusted tumour deformation can sufficiently align the data.
In these subjects, advanced deformable registration is necessary.
The task of intraoperative registration to account for brain shift is a well studied one, although in most cases the approach is to deform preoperative data (such as streamlines) to continually provide accurate neuronavigation after brain shift.\autocite{Clatz2005,Archip2007,Wittek2007,Archip2008}
This requires highly accurate registration, as there is no intraoperative \gls{dmri} acquisition to inform tract identification.
For tractfinder, it would be preferable to impose stricter regularisation to increase registration stability and minimise the creation of unphysical distortions (e.g. surrounding the site of resection), since any consequent minor misalignments can be handled by the atlas smoothness and comparison with local diffusion data.
In individual case studies, non-linear registration has successfully been leveraged in this way (Fig. \ref{fig:nrrex}).
However, this is still only possible in an experimental setting and involves significant trial-and-error in determining the appropriate registration input parameters, and it remains the subject of future work to achieve robust generalisation and automation.

\begin{figure}[hb!]
  \centering
  \includegraphics[width=\textwidth]{chapter_3/nrr_gosh_2.png}
  \caption{Example of non-linear registration applied to a \gls{gosh} \gls{imri} patient. Left: Intraoperative scan with colour \gls{fa} map overlaid to enhance \gls{wm} contrast, registered to preoperative scan (rigid registration). Outline of white matter segmentation of preoperative scan is overlaid, demonstrating substantial brain shift away from the craniotomy (*). Arrow highlights shifting of the ipsilateral external capsule. Dotted line indicates position of coronal plane in right image. Right: Tractfinder map of bilateral \gls{cst} after non-linear registration between pre- and intraoperative scans, using the Fast Free-Form Deformation algorithm\autocite{Modat2010} from the NiftyReg package (\url{http://cmictig.cs.ucl.ac.uk/wiki/index.php/NiftyReg}).}
  \label{fig:nrrex}
\end{figure}

\section{Summary}

All atlas-based image segmentation or analysis techniques have to contend with the problem of aligning said atlas with the target image, and tractfinder is no exception.
In this chapter we have considered the available solutions to this problem, their respective advantages and drawbacks, and applicability to various practical scenarios.

By designing a fuzzy atlas which allows for inter-subject variability to provide an initial estimate for a tract's expected features in the target image, and relying on additional subject-specific information to refine that estimate, we can avoid the need for voxel-perfect atlas alignment, as long as the registered atlas fully covers the domain of the target tract.
In healthy and structurally normal data, affine registration with 12 degrees of freedom is fully sufficient to achieve this coverage.
We saw in section \ref{sec:reg1} that non-linear registration is unlikely to provide any significant improvement in performance, and only decrease speed and practicality.

Where space occupying lesions are concerned, however, simple affine registration can leave the registered tract atlas entirely misaligned with the corresponding subject anatomy.
In such cases, more advanced atlas deformation is necessary to account for the effects of tumours, while still keeping the compute complexity to an acceptable minimum for clinical implementation.
A simple radial deformation algorithm is proposed to specifically account for mass effect after initial affine registration, which successfully models gross tract displacements.

Finally, we saw how the stringent assumptions of the tumour expansion model do not adequately support the modelling of infiltrating tumours, and how the complexities of brain shift may call for more involved registration tactics, including non-linear algorithms and others developed specifically for intraoperative registration.
In the next chapters we will put the methodological components together and assess the proposed techniques' performance in a series of quantitative evaluations and practical applications.


\part{Results and applications}
\epigraph{Unfortunately, nature seems unaware of our intellectual need for convenience and unity, and very often takes delight in complication and diversity.}{Santiago Ramón y Cajal, 1906 Nobel Lecture}
% \epigraph{What could be more natural than complexity?}{Hank Green}
% \epigraph{The most natural thing in the world is complexity}{Hank Green}
\chapter{Experiments}
\label{}

\note{All of the validation in clinical and control data. Also other random stuff like the diffusion workshop data quality stuff?}

\section{Datasets}
\label{sec:data}

\note{A description of the various datasets used, maybe a big overview table with names etc. Maybe also preprocessing? Although that may be specific to each experiment...}


\subsection{HCP}

We accessed 49 scans from the WU-Minn \gls{hcp} Young Adult S1200 data release (\url{https://www.humanconnectome.org/study/hcp-young-adult/document/1200-subjects-data-release}) \autocite{VanEssen2013}.
These images have been preprocessed as documented in \textcite{Glasser2013}.
We additionally downsampled them to 2.5 mm isotropic voxels and extracted a subset of 60 directions at $b=1000 mm/s^2$.

\subsection{TractoInferno}

The recently released TractoInferno database (v1.1.1, available at \url{https://openneuro.org/datasets/ds003900/versions/1.1.1}),\autocite{Poulin2022} created for the training of machine learning tractography approaches, contains diffusion and T1-weighted MRI scans for 284 subjects pooled from several studies, accompanied by reference streamline tractography reconstructions.
Of the 284 subjects included in the full TractoInferno database, we selected the 144 subjects with tractography of the \gls{cst}, \gls{or} and \gls{af} for our study.
9 subjects were excluded from the final analysis due to inadequate non-linear registration performance resulting in failed in-house tractography, leaving a final 135 subjects.
Diffusion acquisition parameters and preprocessing steps are described in \textcite{Poulin2022}, and we additionally resampled all data to 2.3 mm isotropic voxels, the lowest resolution present in the dataset and one in line with clinical acquisitions.

\subsection{Clinical (GOSH \& NHNN)}

Tract segmentation comparisons are presented for 15 individual scans from eight different subjects from two different institutions.
They include four adult glioma subjects acquired in 2009 at the National Hospital for Neurology and Neurosurgery, London (NHNN) (cases 4 and 5 from \textcite{Mancini2022}, others unpublished data),
three paediatric subjects from Great Ormond Street Hospital, London (GOSH) (each with one preoperative and one intraoperative scan),
a mock “intraoperative” scan on a healthy adult volunteer acquired with the GOSH intraoperative \gls{dti} protocol and using simulated intraoperative setup (flex-coils wrapped around the head instead of a head coil, head significantly displaced from scanner isocenter etc),
and a partridge in a pear tree.
For acquisition details see Table \ref{tab:datasets}.
All clinical scans involved non-deforming tumours, in the sense that any lesions did not appreciably displace white matter structures from their typical positions.

This study and the use of GOSH clinical data was approved by UCL REC (ID2780/003) and the UCL Institute of Child Health/GOSH joint R\&D office (reference 19NI12).
Use of NHNN data was approved under retrospective research ethics by the NHNN (University College London Hospitals NHS Foundation Trust) and UCL Institute of Neurology Joint Research Ethics Committee (REC 18/NW/0395, IRAS No: 213920).
In addition, the acquisition and use of some NHNN MRI data was also approved by the NHNN (University College London Hospitals NHS Foundation Trust) and UCL Institute of Neurology Joint Research Ethics Committee (REC 12/LO/1977).
All clinical data was acquired within the course of routine clinical care, and as no identifying information of any subject is present, there is no need for informed consent.
To protect patient confidentiality, clinical data will not be made openly available.

Each \gls{dmri} scan was minimally preprocessed with Marchenko-Pastur principal component analysis denoising\autocite{Veraart2016, Cordero-Grande2019} Gibbs-ringing correction\autocite{Kellner2016} and bias field correction,\autocite{Zhang2001, Smith2004} as implemented in MRtrix3 \autocite{Tournier2019}.
Preoperative scans additionally had eddy current and motion distortion correction\autocite{Andersson2016a, Smith2004} (MRtrix3 v3.0.3 and FMRIB Software Library (FSL, \url{https://fsl.fmrib.ox.ac.uk}) v6.0) applied, while this step was omitted for intraoperative scans to maintain a clinically realistic timeline.
No EPI distortion correction was performed, as it is frequently omitted from clinical pipelines due to lack of requisite reverse phase encoding or field map information and long processing times.\autocite{Yang2022}

%%%%%%%%%%%%%%%%%%%%%%%%%%%%%%%%%%%%%%%%%%%%%%%%%%%%%%%%%%%%%%%%%%%%%%%%%%%%%%%%
\begin{sidewaystable*}[t]
  \caption{Overview of acquisition parameters for the datasets included. \dag Resampled from original, see text for details.}
  \label{tab:datasets}
  \small
  \begin{tabularx}{\textwidth}{l l l l l l l l l}
             & \multicolumn{2}{c}{GOSH} & \multicolumn{2}{c}{NHNN} & \multicolumn{2}{c}{BTC\autocite{Aerts2018, Aerts2020a}} & HCP\autocite{Sotiropoulos2013, Glasser2013} & TractoInferno\autocite{Poulin2022} \\
             & pre-op   & intra-op      & pre-op & intra-op        & pre-op & post-op       & & \\
  \hline%
  %           GOSH pre    GOSH intra          NHNN pre      NHNN intra                  BTC pre  BTC post                   HCP           TractoInferno
  n subjects & 3          & 4                & 4      & 4                               & 10    & 9                         & 49         & 135     \\[1em]
  age        & paediatric & paediatric (n=3) & \multicolumn{2}{c}{adult}                & adult & adult                     & adult      & adult   \\
             &            & adult (n=1)      &        &                                 &       &                           &            &         \\[1em]
  indication  & tumour    & tumour (n=3)   & \multicolumn{2}{c}{oligodendroglioma (n=2)}& \multicolumn{2}{c}{meningioma (n=4)} & healthy & healthy \\
              &           & healthy (n=1)  & \multicolumn{2}{c}{other tumour (n=2)}     & \multicolumn{2}{c}{glioma (n=5)}  &            & \\[1em]
  b values  & 800 (n=1)   & 1000           & 1000     & 1000                       & \multicolumn{2}{c}{0, 700, 1200, 2800} & 1000       & 1000 (n=128) \\
  (s/mm$^2$) & 1000, 2200 (n=2) &          &          &                            &            &                           &            & 700 (n=7) \\[1em]
  n dirs   & 15 (n=1)     & 30             & 64 (n=3) & 30 (n=3)                   & \multicolumn{2}{c}{8, 16, 30, 50}      & 60\dag     & 21-128 \\
           & 60, 60 (n=2) &                & 61 (n=1) & 3 x 12 (n=1)               &            &                           &            & \\[2em]
  voxel size & 1.75\textsf{x}1.75\textsf{x}2.5 (n=1) & 2.5 (n=1) & 2.5 & 2.5\textsf{x}2.5\textsf{x}2.7 & \multicolumn{2}{c}{2.5} & 2.5\dag    & 2.3\dag \\
  (mm)       & 2\textsf{x}2\textsf{x}2.2 (n=2)       & 2.3 (n=3) & & & & & & \\[1em]
  scanner & Philips Ingenia & Siemens Vida 3T & Siemens & Siemens & \multicolumn{2}{c}{Siemens} & Siemens 3T & variable\\
          & 1.5T (n=1)    &                & Trio 3T  & Espree 1.5T                & \multicolumn{2}{c}{Trio 3T}           & ``Connectome  & \\
          & Siemens Prima  &               &          &                            &            &                          & Skyra”        & \\
          & 3T (n=2)  & & & & & & &

  \end{tabularx}
\end{sidewaystable*}
%%%%%%%%%%%%%%%%%%%%%%%%%%%%%%%%%%%%%%%%%%%%%%%%%%%%%%%%%%%%%%%%%%%%%%%%%%%%%%%%

\subsection{BTCD}

The \gls{btc} dataset comprises a series of pre- and postoperative scans of glioma and meningioma patients undergoing craniotomy for tumour resection.
Of the total 25 patients in the original dataset, 10 with visible, non-infiltrating \note{exact criteria?} were selected for the validation of tractfinder in tumour patients.
The data consists of high quality HARDI acquisitions, including reverse phase encoding for susceptibility distortion, and structural T1w images.
In addition, tumour segmentation masks for each preoperative scan are available in the original dataset, which can be used for tumour deformation modelling in those patients with strong tumour deformation.
The 10 selected subjects include 4 with tumours large enough (3 meningioma and 1 anaplastic astrocytoma) to warrant the use of tumour deformation modelling to improve atlas alignment.

dMRI data was preprocessed with denoising, bias field correction and EPI distortion correction (\note{eddy cuda version}).
\gls{msmt}-\gls{csd}, tractfinder, TractSeg and manual tractography of the \gls{af}, \gls{cst}, \gls{ifof} and \gls{or} were performed as previously described.

\subsection{Fibercup phantom}

\note{synthetic phantom data description}

While testing and validating on \textit{in vivo} human data is irreplaceable, the complexities of such data, as well as the lack of definitive ground truth for fibre bundle reconstructions in unknown anatomies, can make results difficult to interpret.
This is where synthetic phantom data, with known parameters inputs \note{``anatomy" is known, better way of saying ground truth?}, are valuable for protoyping and validating methods.
Within the white matter imaging space there is a synthetic phantom known as the Fibre Cup phantom after a tractography challenge held for the 12th international conference on Medical Image Computing and Computer-Assisted Intervention (MICCAI) in 2009.
The phantom features seven bundles arranged in a way that aims to mimic the range of configurations found in the brain, including sharp curves, multi-way crossings, kissing and fanning fibres.
While the original data is not available \note{??}, a digital recreation was constructed in \note{??} and is openly available, along with a set of ground truth streamline bundles representing each tract. 

\note{this was useful for prototyping but there's not much of interest to say about it now; it's too trivial for validation of tractfinder.}

\section{Validation}

\note{How the method was validated. This is all the Tractoinferno / HCP data stuff, plus maybe qualitative results?}
\note{Also discussion about lack of ground truth data and different validation metrics.}

\subsection{Comparison with benchmark methods}

\note{Largescale validation in HCP and tractoinferno datasets, comparing to reference tractography and TractSeg}

While a ground truth for white matter tract segmentation is not obtainable \textit{in vivo}, we compare our technique with two other widely adopted methods for a quantitative estimation of reliability and accuracy.
Results are presented for three different datasets: Two large healthy datasets and one smaller dataset of clinical neurosurgical acquisitions, together covering a range of acquisition protocols and scanner specifications.
In segmentation tasks, it is common to present a single numeric score of similarity with a ground truth by way of establishing accuracy.
However, in the absence of a ground truth for this particular task, we aim to present as rounded a picture as possible of the differences and characteristic features of each method through a range of different volumetric distance-based similarity metrics.
The purpose of this validation is therefore not to determine which method is best, as indeed cannot be determined without a reliable reference point, but to highlight the ways in which they are similar, and their characteristic tendencies.

\subsubsection{Data}

We considered three different datasets with which to compare the proposed method against alternative tract segmentation methods.
Each dataset and any dataset-specific preprocessing is described below. \note{In section \ref{sec:data}}
In addition, for all subjects we performed brain masking,\autocite{Tournier2019} linear registration\autocite{Jenkinson2001,Jenkinson2002} (FSL v.6.0 Linear Image Registration Tool) between subject space and MNI152\autocite{Fonov2011} space and two versions of constrained spherical deconvolution (CSD): single-shell, single-tissue (SSST) CSD (``original flavour")\autocite{Tournier2007,Tournier2019} and multi-shell, multi-tissue CSD\autocite{Jeurissen2014} restricted to white matter and grey matter tissue compartments.
In both cases, response functions were obtained using the Dhollander unsupervised 3-tissue response function estimation algorithm.\autocite{Dhollander2016,Dhollander2019}
All CSD processing was conducted using the MRtrix3 image processing software package v3.0.2-3.0.3 (\url{https://www.mrtrix.org/}).\autocite{Tournier2019}
Processing and analysis pipelines for the two openly available datasets are available at \url{https://github.com/fionaEyoung/pipelines}

\subsubsection{Compared methods}

We considered three alternative segmentation approaches and compared each with the proposed method: Probabilistic streamline tractography, representing the current standard, the deep learning direct segmentation technique TractSeg, and a ``naive" atlas registration.

\paragraph{Streamline tractography}

We ran targeted probabilistic streamline tractography (iFOD2 algorithm\autocite{Tournier2010}, from MRtrix3\autocite{Tournier2019} v3.0.3) in each scan using an in-house ROI strategy (see \ref{sec:rois} for ROI details), with tractography input FODs derived from multi-shell, multi-tissue CSD \autocite{Jeurissen2014} with white matter and grey matter tissue compartments.
In the clinical dataset, ROIs were placed manually for each subject.
For 193 HCP and TractoInferno subjects, manual ROI placement was infeasible.
Instead the same ROIs were drawn in MNI152 space aided by the FSL HCP-1065 DTI template\autocite{FSLATLAS} and transformed to subject space using non-linear registration (HCP data includes MNI transformation warps, while warps were created for the TractoInferno data using the ANTs registration package v2.4.2 (http://stnava.github.io/ANTs/).\autocite{Tustison2013,Avants2011}).
This in-house tractography is subsequently abbreviated to ``TG", while the reference TractoInferno bundles are referred to as ``TGR".

\paragraph{TractSeg}

TractSeg \autocite{Wasserthal2018} is a deep learning tract segmentation model which produces volumetric segmentations for 72 tracts directly from fibre orientation distribution peak directions (TractSeg v2.3-2.6, available at \url{https://github.com/MIC-DKFZ/TractSeg}).
There are two models available: one (``DKFZ") trained on modified streamline reconstructions using TractQuerier \autocite{Wassermann2016} as described in \textcite{Wasserthal2018}, and a second (``XTRACT") trained on streamline density maps output by FSL's XTRACT application. \autocite{Warrington2020}
We compared with both versions, as they feature significant differences in anatomical tract definition.
Input peaks were derived from the SSST CSD FODs.

\paragraph{Atlas registration}

As well as the full tractfinder method, we compared our results with a ``naive" tract atlas approach, taking the density component (first SH coefficient) of the linearly registered tract atlases.
This amounts to a segmentation based on prior expectation only, without taking into account the diffusion data.

\subsubsection{Comparison metrics}

We compared each technique against the others, rather than designating any single technique as ``ground truth".
Several comparison metrics were computed, to capture different kinds of agreement between segmentations.
Dice-Soerensen similarity coefficient (DSC) \autocite{Dice1945} is a popular, symmetric measure of segmentation similarity given by

\begin{align}
  DSC &= \frac{2 |A \cap B|}{|A| + |B|} \\
\end{align}

for two binary voxel sets $A$ and $B$.
Since DSC is a measure for binary segmentations, it requires the thresholding of continuous-valued maps such as track density maps and the pseudo-probability maps produced by tractfinder.
Firstly, the conversion from continuous-valued to binary representation introduces a high degree of ambiguity over the appropriate choice of threshold value.
While the simplest approach may be to include all voxels with value $>0$ in the segmentation, this makes little sense in practice.
In the case of tractography, a small number of rogue false positive streamlines can massively increase the extent of the binary segmentation, and in the case of TractSeg, very few voxels actually are assigned a probability of 0.
The following thresholds were used throughout, wherever binary segmentations are concerned, are given in Table \ref{tab:thresh}.

\captionof{table}{\label{tab:thresh}}
\begin{tabularx}{0.5\textwidth}{X c}
  Method    & Threshold value \\
  \hline
  Tractfinder   & 0.05 \\
  Tractography (streamline density)  & 10 \\
  Reference tractography (TractoInferno only) & 0 \\
  Atlas         & 0.1 \\
  TractSeg      & 0.5 \\
  \hline
\end{tabularx}

Secondly, the binary nature of DSC discounts the additional confidence information present in density and probability maps.
In addition to the binary DSC measure, therefore, we consider a generalisation of the DSC (gDSC) for continuous valued segmentations.

\begin{align}
  gDSC &= \frac{2 \sum_i \sqrt{a_ib_i} }{\sum_ia_i + \sum_ib_i}
   =  \frac{2 \sum_i \sqrt{a_ib_i} }{||\mathbf{a}||_1 + ||\mathbf{b}||_1}
\end{align}

The density correlation metric provides an alternative measure of agreement between two continuous valued segmentations with different scales:
it is simply the Pearson correlation coefficient between the two sets of voxel values.

In addition to the volumetric overlap and density metrics DSC, gDSC and density correlation, we measured the volumetric bundle adjacency as defined in \textcite{Schilling2021a}.
However, to avoid confusion with the streamline-based bundle adjacency\autocite{Radwan2022, Garyfallidis2012, Rheault2022} metric previously defined in \textcite{Garyfallidis2012},
and to give more intuitive meaning to the obtained values, we will refer to it as bundle distance $BD$.
It is computed by taking the mean of minimum distances from every non-overlapping voxel, in each segmentation, to the closest voxel in the other segmentation (Fig. \ref{fig:BD}).
Finally, to give a sense of whether the boundary of one segmentation is within or outside that of a second segmentation, we also measured the \textit{signed} bundle distance $BD_s$.
This metric is asymmetric, with $BD_s (A,B) = -BD_s(B,A)$.
Thus $BD$ and $BD_s$ are defined as

\begin{align}
  BD(A,B) &= \frac{\sum_{i \in A\setminus B} d_i(B) + \sum_{i \in B\setminus A} d_i(A)}{|A\Delta B|} \label{eq:bd} \\
  BD_s(A,B) &= \frac{\sum_{i \in A\setminus B} - d_i(B) + \sum_{i \in B\setminus A} d_i(A)}{|A\Delta B|} \label{eq:bds}
\end{align}

where $| \cdot |$ denotes set cardinality and $d_i(X)$ is the Euclidean distance transform (defined relative to the foreground of segmentation $X$, i.e. $d_i(X) = 0$ when $i \in X $ and $d_i(X) = |\overrightarrow{ij}|$ when $i \not\in X$ and where $j \in X$ is the voxel in $X$ closest to voxel $i$)  of segmentation $X$ at voxel $i$.

\begin{figure}
  \centering
  \includegraphics[width=0.3\textwidth]{segmentations_distance_euc.pdf}
  \caption{Illustration of regions involved in calculating bundle distance metric. Light grey is $A\setminus B$, dark grey area is $B\setminus A$. To compute bundle distance $BD(A,B)$ (Eq. \ref{eq:bd}), the mean minimum absolute distance to the intersection (solid black) is taken across all voxels in the two grey areas $BD(A,B) = (14+4\sqrt{2}+3\sqrt{5})/17 = 1.55$. To compute the signed bundle distance $BD_s(A,B)$ (Eq. \ref{eq:bds}), distance values in A are negated. $BD_s(A,B) = (2-2\sqrt{2}-\sqrt{5})/17 = -0.18$. The Dice score for these two segmentations would be $DSC = 2*4/(13+12) = 0.32$}
  \label{fig:BD}
\end{figure}

\subsection{Benchmark comparison results}

\subsubsection{Processing times}

Atlas transformation and inner product computation time per subject for all three tracts and both hemispheres was $18\pm5 s$, plus 1-2 minutes for MSMT-CSD and 20 seconds for MNI registration.
For TractSeg (DKFZ or XTRACT), mean processing time (for all tracts, 72 for DKFZ and 23 for XTRACT, both hemispheres) was 4:00$\pm$1:00 $min$, plus $15-20 s$ for SSST-CSD.
For a full processing time breakdown see Table \ref{tab:time}.

For manual streamline tractography, processing time was not explicitly measured, due to the high variability that comes with manual ROI drawing (between 10--25 minutes for all tracts in a single subject, although this varies significantly between operators).
HCP and TractoInferno tractography was run on a high performance computing cluster, taking approximately 10s per tract (single hemisphere), using 36 CPU cores, and additionally up to 2 minutes for non-linear ROI registration (Table \ref{tab:time}).
However, since the time taken depends greatly on several factors, including number of streamlines to select and streamline acceptance rate (often low in pathological brains due to oedema, deformation etc.), a detailed time analysis for manual tractography is not provided here.

%%%%%%%%%%%%%%%%%%%%%%%%%%%%%%%%%%%%%%%%%%%%%%%%%%%%%%%%%%%%%%%%%%%%%%%%%%%%%%%%
\begin{table*}[t]
  \caption{Measured processing times mean and standard deviation for TractoInferno dataset. Individual steps shown and total average for the four different pipelines. Note that the tractography pipeline was partially run on a high performance computing cluster, so the reported total time is not representative of a typical setup. Further note that for the present study, tractography ROIs were drawn once for the whole dataset, whereas for clinical datasets manual ROI delineation will have to be repeated for each subject. \dag Desktop Mac with 4 GHz Quad-Core Intel Core i7 \ddag High performance computing cluster, 1 node per subject, 36 Intel(R) Xeon(R) Gold 6240 CPU @ 2.60GHz cores per node.}
  \label{tab:time}
  \small
  \begin{tabularx}{\textwidth}{+>{\raggedright}X ^c ^>{\sffamily}c ^>{\sffamily}c ^>{\sffamily}c ^>{\sffamily}c}
  \rowstyle{\rmfamily}
  Step & Processing time (per subject) & tractfinder & TractSeg & Atlas & tractography \\
  \hline
  \dag Brain masking & 3 $\pm$ 2 s & x & x & x & x\\
  \dag Affine MNI registration & 20 $\pm$ 4s & x &  & x &  \\
  \dag Response function & 5 $\pm$ 3 s & x & x & x & x\\
  \dag MSMT CSD & 01:50 min $\pm$ 55 s & x &  & x & x\\
  \dag SSST CSD + peaks estimation & 18 $\pm$ 8 s &  & x &  &  \\
  \dag Atlas transformation + inner product (3 tracts) & 18 $\pm$ 5 s & x &  & (x) &  \\
  \dag TractSeg (72 / 23 tracts) & 04:00 $\pm$ 01:00 min &  & x & & \\
  \dag Manual ROI delineation (once for whole dataset) & 20:00 min & & & & x \\
  \ddag Non-linear ROI registration + tractography (3 tracts, 2 hemispheres) & 4:05 $\pm$ 2:08 min & & & & x \\
  \rowstyle{\bfseries\rmfamily}
  Total &  & 2:36 min & 4:25 min & \textless2:36min & $\gtrsim$26:03 min
  \end{tabularx}
\end{table*}
%%%%%%%%%%%%%%%%%%%%%%%%%%%%%%%%%%%%%%%%%%%%%%%%%%%%%%%%%%%%%%%%%%%%%%%%%%%%%%%%


\subsubsection{Qualitative results}

Qualitative results can be seen in Figures \note{fig:SURFACE}, \note{fig:lb.cst}-\note{fig:lb.clin}.
The raw tract maps typically have values ranging from 0 to 0.5 (in arbitrary units, derived from the magnitudes of FOD and atlas distribution functions).
Due to the combined effects of ODF amplitude and orientation information, a low tract map value can have several causes: a) the FOD amplitude is low, indicating low evidence for white matter tissue in the voxel in question; b) the atlas amplitude is low, indicating low prior likelihood of the tract being present in that location; c) the peak orientations between the FOD and atlas are poorly aligned.

Thus combining information from the atlas and data-derived FODs improves the tract map estimation over the ``raw" registered atlas in both the spatial and orientational domain. For example, the TOD atlases have poor definition of gyri and sulci, due to the effect of averaging over many subjects and linear registration. The reduced overall FOD amplitude in grey matter corrects this non-specificity. And in regions where different white matter structures lie in close proximity, where the atlas can erroneously predict the likely presence of the tract, and FOD amplitude is high, the lack of orientational agreement discounts the presence of the tract of interest in that location.

\subsubsection{Quantitative results in healthy data}

Volumetric and agreement metrics indicate consistent, if not always high, levels of agreement between tractfinder and compared techniques, TractSeg and tractography.
Visual assessment reveals that differences in the spatial extent of the segmented tracts accounts for a large part of the discrepancy between methods.
This is most apparent in the arcuate fasciculus, where anatomical definitions differ widely (Fig. \note{fig:lb.afr}, \note{fig:lb.afl}).
For example, TractSeg (DKFZ) includes extensive coverage of the frontal and temporal lobe in its AF segmentations, including parts of the primary motor cortex.
Conversely in the corticospinal tract, which has a relatively well agreed-upon domain, segmentation results have much higher volumetric agreement between methods.

The signed bundle distance gives an indication of the nature of disagreement between two techniques where other metrics show little difference.
For example, in the HCP dataset and for the arcuate fasciculus, mean bundle distance between the naive atlas and tractography was $5.45 mm$ and mean bundle distance between TractSeg (DKFZ) and tractography was very similar at $5.41 mm$ (Tab. \note{tab:DATAHCP}).
However, the signed bundle distances for those same two comparisons were $+2.57 mm$ and $-2.68 mm$ respectively.
This indicates that, while if only considering the bundle distance metric, both TractSeg and the atlas appear to agree to a similar degree with tractography, TractSeg actually systematically over-segments the AF (relative to tractography), while the naive atlas segmentation tends towards under-segmentation.

Density correlation and gDSC help illustrate the cases where the choice of threshold may have a disproportionate influence on subsequent binary comparisons.
For example, in the HCP dataset and for the corticospinal tract, mean binary DSC was $0.69$ between tractfinder and tractography and $0.51$ between TractSeg (XTRACT) and tractography (a difference of $0.18$).
For the same two comparisons, the density correlations differed only by $0.04$ ($0.63$ and $0.59$) respectively, indicating strong agreement between areas of high confidence (``density'').

\paragraph{HCP data}

Volume surface visualisations obtained with tractfinder are shown for the three tracts in a single HCP subject in Figure \note{fig:SURFACE}.
Figure \note{fig:HCPDICE} gives an indication of how the five different segmentation methods compare, across all HCP dataset subjects.
There is considerable variance between tracts, however some observations remain consistent.

Comparisons with tractography exhibit very low gDSC values.
Binary DSCs are low across the board for the arcuate fasciculus, owing to the dramatically different spatial extents of the segmentations.
For the corticospinal tracts, tractfinder agrees well with each of the other methods, both using binary and generalised comparisons.
Agreement is similarly high in the optic radiations, with slightly lower DSCs compared to the two TractSeg methods, which tend to include more thalamus and a lesser extent of Meyer's loop.

Tractfinder segmentations are highly consistent, with comparison metrics with alternative methods varying by little across subjects (Tab. \note{tab:DATAHCP}).

\paragraph{TractoInferno}

Qualitative results for a representative subject (identified as the only subject within the top 30 smallest deviations from the mean scores for all three of bundle distance, binary DSC and density correlation) are shown in Figures \note{fig:lb.afr}-\note{fig:lb.or}.

Figure \note{fig:METRICSBOXPLOTS} compares each studied method against the reference streamline bundles in the TractoInferno dataset.
Noticeably, the differences in scores within a single method, between different tracts, are in places greater than the differences between methods within a tract.
For example, the binary DSC scores for the CST are similar for tractfinder and TractSeg (DKFZ) ($0.48$ and $0.45$ on average respectively), however the binary DSCs of TractSeg (DKFZ) are markedly different between the CST and OR ($0.45$ and $0.59$ on average respectively).
These differences highlight the difficulty in assessing the ``accuracy" of white matter segmentation methods given limited consensus on the precise anatomical definitions of different pathways.
DSC, gDSC and density correlation values for tractfinder were on par with TractSeg (XTRACT) in all three tracts, with the exception of density correlation in AF, while gDSC and density correlation were higher than TractSeg (DKFZ) in all tracts.
Binary DSC scores were highest for TractSeg (DKFZ) in the CST and AF, and equal between tractography, tractfinder and TractSeg (DKFZ)  for the optic radiation.
The results in Figure \note{fig:METRICSBOXPLOTS} are consistent with the comparisons between TractSeg and RecoBundles published in \textcite{Wasserthal2018}.
There, a mean DSC of between 0.58 and 0.67 across all tracts was reported.
Our measured DSCs between TractSeg (DKFZ) and reference tractography (which is based on RecoBundlesX\autocite{Garyfallidis2018}) range between 0.45 and 0.59 across the three tracts studied (Tab. \note{tab:DATATI}).

From Figure \note{fig:METRICSBOXPLOTS}, it is apparent that comparisons with TractoInferno reference streamlines yield a large number of outliers and low scores.
Further investigation into these outliers revealed numerous subjects with incomplete or highly asymmetric bundles.
For example, in several cases, optic radiation streamlines only reach the superior portion of the occipital lobe (Fig. \note{fig:DUDSOR}).
In others, the right arcuate fasciculus is significantly smaller than the left (Fig. \note{fig:DUDSAF}).
There is lower variability in the pairwise comparisons between the other four methods: the results for tractfinder and TractSeg remain in more consistent agreement with each other across the TractoInferno dataset (Fig. \note{fig:TIDICE}).

\subsubsection{Results in clinical data}

For the present analysis we included clinical scans with non-deforming lesions, meaning the orientation atlas could be registered to the target image using only affine registration without the need for tumour deformation modelling.
For qualitative results in clinical scans featuring deforming lesions, see \textcite{Young2022}

In the clinical dataset, mean values of agreement with other segmentation methods were consistent with those in the ideal, healthy datasets, while variability between subjects was higher (Fig. \note{fig:CLINICALDICE}).
Again, comparisons between the segmentation methods vary significantly between tracts.
The size of this dataset is far smaller than the other two, but even so the results are consistent with those seen for the larger, healthy subject datasets.

Two example clinical subjects, one adult and one paediatric, are displayed in Figure \note{fig:lb.clin}.
In Figure \note{fig:lb.nh}, a sagittal view displays the interaction between the surgical resection cavity and the CST.
Here our proposed method maps the CST in relatively close proximity to the resection site, where the TractSeg segmentations are far more conservative, potentially missing CST locations influenced by oedema or other tumour effects.
In Figure \note{fig:lb.gosh}, the extent of Meyer's loop depicted by tractography is similarly included in the proposed segmentation, but absent from the TractSeg results.

When the results for the clinical dataset were split on hospital / age group (paediatric or adult), no appreciable difference in results was observed (data not shown).
Equally, no systematic difference was observed between intraoperative and preoperative datasets.
The mean score results for all tracts and comparisons are given in Supplementary Table \note{tab:DATACL}.


\subsection{TractSeg atlas}

\note{Direct comparison with TractSeg using their training data to create atlases}

\chapter{Practical application}

\section{Data requirements}

\note{Basically the findings from ISMRM diffusion workshop with different data inputs.}


It is important to assess the applicability of image processing techniques developed with research quality data in acquisitions more typical of a clinical setting.
It is common for advanced \gls{dmri} processing techniques to \note{require} a certain set of constraints on the input data, such as recommending or requiring a minimum number of diffusion weighted volumes (angular resolution), $b$-values or spatial resolution.

If you can demonstrate that an image processing pipeline produces excellent results in a high quality research dataset, which can be interpreted with confidence, then the question follows: where one to acquire a lower quality dataset of the same subject, and perform the same data processing, how comparable would the resulting segmentation be to that of the high quality scan?

It is fairly straightforward to explore this question, given a high quality sample dataset.
Such data can be downsampled, in both the spatial and angular domains, to produce a simulated lower quality scan of the same subject.
Here we set out to determine the minimum data requirements to obtain successful segmentation, comparing different numbers of direction samples, $b$-values and post-processing strategies, as well as the affects of decreasing data quality on segmentation stability.
The effects on the perfomance of TractSeg and tractography are compared as well.

\subsection{Data and methods}

49 Preprocessed dMRI datasets from the HCP 1200 data release were spatially resampled to 2.5mm isotropic voxel size (from the original of 1mm isotropic), a resolution far more common in both clinical and research settings.
Then, for each subject, the following 5 subsampled diffusion schemes were extracted from the full dataset (see Tab. \ref{tab:subschemes}):
60 directions each at $b=1000s/mm^2$ and $b=2000s/mm^2$ (120 directions total, ``DWI-1”), 30 directions each at $b=1000s/mm^2$ and $b=2000s/mm^2$ (60 directions total, ``DWI-2”), 60 directions at $b=1000s/mm^2$ (``DWI-3”), 30 directions at $b=1000s/mm^2$ (``DWI-4”) and 12 directions at $b=1000s/mm^2$ (``DWI-5”).

\begin{table}
  \centering
  \begin{tabular}{c c c}
    Identifier & $b = 1000$ & $b=2000$ \\
    \hline
    DWI-1 & 60 & 60 \\
    DWI-2 & 30 & 30 \\
    DWI-3 & 60 &    \\
    DWI-4 & 30 &    \\
    DWI-5 & 12 &    \\
  \end{tabular}
  \caption{Subsampled diffusion schemes}
  \label{tab:subschemes}
\end{table}

For each dataset four different tract segmentations are produced using three different approaches: Tractfinder, targeted ROI-based, probabilistic streamline tractography (iFOD2), and TractSeg.
TractSeg, a deep learning-based direct segmentation technique, ships with two different pre-trained models:
one trained on tract segmentations obtained as described in Wasserthal et al. (2018) (``TractSeg - DKFZ”), and another trained on XTRACT7 outputs (``TractSeg - XTRACT”).
Two different TractSeg results were obtained by running both models.
Segmentations were compared for the three tracts most commonly reconstructed in neurosurgical applications: the corticospinal tract, arcuate fasciculus and optic radiations.
All tract segmentation methods are predicated on fibre orientation distributions (FOD) modelled from diffusion data, and although tractography and TractSeg could both use other FOD modelling outputs such as BedpostX, the current analysis is kept to constrained spherical deconvolution (CSD) only.
For the two-shelled datasets, multi-shell multi-tissue (MSMT) CSD10 can be run to separate the signal contributions from white matter (WM), grey matter (GM) and cerebrospinal fluid (CSF) and thus obtain an optimal WM FOD image free of noisy extraneous signal in non-WM regions.
For the single-shelled datasets, full, direct MSMT with all three tissue types is not possible.
Instead, the following approaches are compared: 1) Modelling two tissue types with MSMT.
MSMT CSD is performed twice, once modelling WM and GM compartments, and once modelling WM and CSF compartments; 2) Standard single shell CSD ``original flavour".

To investigate the effects of different data quality and modelling on a given method, the Dice similarity coefficient (DSC) is computed between each segmentation in datasets DWI 2-5 and the corresponding segmentation of the same method in DWI 1 using MSMT (considered the baseline segmentation).
This ``self-similarity" approach is intended to answer the question put forward above: how close to the results produced from an ``ideal" dataset be achieved from a lower quality dataset.
This comparison was made for each combination of data quality and CSD approach.

\subsection{Findings}

\begin{figure}
  \includegraphics{chapter_4/self_dice.png}
  \caption{Self dice scores \note{caption}}
  \label{fig:self_dice}
\end{figure}

DSC results are plotted in Figure \ref{fig:self_dice}.
Each datapoint corresponds to the segmentation produced from a particular dataset, method and pipeline \textit{compared against} the result obtained from the highest quality dataset.

The consistency in segmentation results differed significantly with both data quality and CSD approach.
The high sensitivity to CSD approach in the tractography and Tractfinder results, particularly the large differences between MSMT WM+CSF / SSST and MSMT WM+GM, can be attributed to the amount of spurious grey matter signal included in the WM FOD reconstructions.
In the former two approaches, the GM compartment is not as aggressively suppressed from the WM reconstruction, resulting in WM ``signal” in grey matter areas (which is then incorporated into the segmentation: in tractography via the further propagation of streamlines into GM, and in Tractfinder directly due to higher WM signal amplitude in GM).
TractSeg, while not immune, appears less sensitive to CSD reconstruction technique.
When considering only the best results (MSMT in DWI-2 and, in the single-shelled datasets, MSMT WM+GM for tractography and Tractfinder and SSST for TractSeg; Fig. 2), tractography displays the greatest instability with decreasing data quality, with the similarity score between DWI-5 MSMT WM+GM and DWI-1 MSMT falling as low as 0.78 (subject and cerebral hemisphere mean) for the arcuate fasciculus.
TractSeg is trained on SSST CSD in single-shelled data as well as multi-shelled data, explaining why it does best with either SSST or MSMT WM+CSF in our single-shell results.

The comparatively high sensitivity of probabilistic tractography to acquisition and processing pipelines is consistent with the well-established reproducibility and noise sensitivity problems associated with tractography.
Meanwhile, voxel-wise segmentation methods are not susceptible to error propagation along the tract and are more robust to lower angular resolution.
Understanding the behaviour of different tract segmentation techniques when applied to varying qualities of dMRI acquisition and post-processing approaches is important if segmentation methods developed in research settings are to be consistently and reproducibly applied to clinical quality acquisitions.
It is useful to know how comparable segmentation results in a single-shelled, 30 direction dataset are to those one might have obtained with a HARDI dataset and state-of-the-art post-processing.
This is particularly relevant for longitudinal studies, or when comparing different acquisitions of the same subject in a clinical context (e.g., for monitoring disease progressions, post-operative changes etc.).


\note{qualitative demos, clinical cases with intraop deformation }

\note{More qualitative validation in as many sample datasets as possible, including intraoperative data, tumour deformation and debulking cases. Show specific cases where differs from e.g. tractography or TractSeg in specific areas like peritumoural oedema. Also analyse failure cases.}

\note{Would be nice to also have e.g. epilepsy cases in here?}

\section{Intraoperative brain shift}

\note{dump from IJCARS manuscript}

The target application is in intraoperative imaging.
The main difference therein is the need to account for brain shift, which is unpredictable: differing effects stem from drainage of fluid, pressure changes, tumour debulking and gravitational sag.
Nevertheless, we aim to achieve intraoperative tract segmentation while avoiding the need to perform additional tumour and / or resection cavity segmentation intraoperatively.

As the atlas is designed to be spatially inclusive, with the inner product acting to correct small spatial inaccuracies, it is possible in some cases where brain shift is minimal to reuse the properative tumour deformation field.
In cases of significant tumour debulking, the deformation field can be recomputed from the preoperative tumour segmentation by adjusting the value of $s$ to simulate a reduction in tumour volume.

This scenario is demonstrated in Figure \ref{fig:shrink}, showing the resection of a large temporal epidermoid cyst. There is significant reduction in cyst volume and the adjacent corticospinal tract has shifted accordingly, however by reusing the preoperative lesion segmentation and setting $s=0.8$, the resulting deformation field is able to capture the rough location of the shifted tract. By only adjusting the value of $s$ and reusing preoperatively computed values of $D_t$ and $D_b$, we can avoid time and resource-intensive intraoperative lesion segmentation, brain shift modelling or non-linear registration.

\begin{figure*}[h!]
  \centering
  \includegraphics[width=0.8\textwidth]{chapter_4/figure_4.png}
  \caption{Example results in intraoperative image using scaled preoperative tumour segmentation. Blue outline: Tumour segmentation. Green outline: effective tumour boundary with $s=0.8$ used for intraoperative segmentation}
  \label{fig:shrink}
\end{figure*}

\begin{figure*}[h!]
  \centering
  \includegraphics[width=\textwidth]{chapter_4/figure_5_revised.png}
  \caption{Sample results in 4 different clinical subjects. First column: linearly registered tract atlas (spatial component only). Second column: atlas after tumour deformation. Third column: Final tract map. Fourth column: Track density image from streamline tractography, where intensity corresponds to streamline count per $(2.5mm)^3$ voxel (thresholded at 10 streamlines).}
  \label{fig:res}
\end{figure*}

In the \gls{btc} dataset, tumour deformation modelling successfully improved tract and subject alignment and consequently tract mapping in all preoperative cases where it was necessary.
Examples in \note{figure ??} demonstrate how tractfinder including tumour deformation produces tract margins closer to those indicated by tractography, in contrast to the deep learning method TractSeg which appears to still rely on a learned healthy anatomical configuration without generalising to displaced tracts.

\clearpage
\section{iMRI case studies}

From \note{early 2022 to mid 2023}, 8 \note{verify!} tumour resections where carried out at Great Ormond Street Hospital for Children with intraoperative MRI guidance including multi-directional diffusion weighted imaging.
Throughout this period, neurosurgical and radiological staff were continually learning how best to integrate the iMRI facility into their practice.

The subjects included two posterior fossa tumour, two midline (thalamic) glioma and four hemispheric tumour patients.
Of these, thalamic tumours are the indications with potentially the most to gain from advanced intraoperative guidance, given their complex functional environment and difficult access.
Indeed, such tumours have widely been considered inoperable and have only recently seen surgical treatment thanks in part to changing attitudes\autocite{Souweidane1996,Puget2007} advances in neuronavigation.\autocite{Sunderland2021}
Paediatric patients in particular\autocite{Ferroli2023} experience better overall survival rates when thalamic tumours are more radically resected compared to subtotal resection or biopsy.\autocite{Cinalli2018a}
The Alder Hey Children's Hospital in Liverpool, UK experienced an increase in substantial resection \note{define} rates from 37\% to 94\% with the introduction of intraoperative MRI navigation, without any associated increase in postoperative morbidity.\autocite{Sunderland2021}
Similarly, a review of 38 thalamic tumour patients (paediatric and adult) treated at the Chinese PLA General Hospital in Beijing, China found that the use of iMRI increased GTR rates from 42\% to 68\%.\autocite{Zheng2016}
Motor deficits are the most common functional symptoms of thalamic tumours,\autocite{Puget2007, Zheng2016, Palmisciano2021} and preservation of the posterior limb of the internal capsule, containing the corticospinal tract, is a major concern when resecting them.
\note{to cite: Grewal2019, Wong2016, Kis2014, Dorfer2021, Steinbok2016, Celtikci2017}

The following will look at two of the GOSH diffusion iMRI cases in detail, considering the imaging features, tumours and peri-operative clinical presentation and neurolgoy.
Hopefully this will provide a unique an balance perspective into the challenges and potential of intraoperative \gls{dmri} for such cases.

\subsection{Patient 5: Low grade pilocytic astrocytoma}

\note{add some sort of disclaimer about how this is all hearsay based on MDT / scanner room notes?}

The first of the two patients harboured a low-grade pilocytic astrocytoma in the left thalamus and presented at 23 months old with progressive right side weakness.
The unusual radiological presentation showed a sliver of tissue within the tumour, presumed to be a part of the CST, while the rest of the tract was displaced (Fig. \ref{fig:5p}).

\begin{figure}
  \centering
  \includegraphics[width=0.6\textwidth]{case_studies/5_preop}
  \caption{Preoperative radiological presentation of patient 5, with a low-grade astrocytoma of the left thalamus. Top: T1, bottom: DTI FA. Arrow heads indicate the CST, partially travelling through the tumour.}
  \label{fig:5p}
\end{figure}

Confirming the location of the CST, divided as it appeared to be into an intratumoural and a displaced portion, was particularly difficult in this case.
A \gls{dmri} sequence with colour FA visualisation was able to confirm that the strip of tissue inside the tumour indeed constituted part of the CST.
Streamline tractography also reconstructed the CST as both within and displaced posteriorly around the tumour.
The part of the pathway involved is at the cerebral peduncle level, which in healthy cases is a relatively narrow bundle.
Consequently, the tractfinder atlas was unable to account for both the tumour-engulfed section, which remained spatially relatively unmoved, and the posteriorly displaced portion simultaneously.
Basic tract mapping reconstructed only the first part, while additional tumour deformation modelling enabled detection of the second part (Fig. \ref{fig:5p_cst}).
Of course, such a hacked-together reconstruction would be entirely impractical and difficult to interpret, compared to tractography which is better at exploring the available pathways regardless of anatomical prior expectations.
As such, this is a case for which tractfinder is entirely unsuited.

\begin{figure}
  \centering
  \includegraphics[width=0.6\textwidth]{case_studies/5_preop_tf}
  \caption{Reconstruction of the corticospinal tract on preop imaging. Top: track density imaging, bottom: tractfinder (combined with and without deformation modelling), with tractography outlined in white for comparison.}
  \label{fig:5p_cst}
\end{figure}

Due to this complicated involvement of the internal capsule a debulking resection under iMRI guidance was indicated to relieve pressure on the \gls{cst}.
After a large part of the tumour was resected, the patient was brought through to \gls{imri} (Fig \ref{fig:5i}).
Here, again, reconstruction of the \gls{cst} proved difficult.

\begin{figure}
  \centering
  \includegraphics[width=0.5\textwidth]{case_studies/5_iop}
  \caption{Intraoperative T1w (top) and diffusion colour FA (bottom) for patient 5, showing left craniotomy and surgical corridor through the temporal lobe}
  \label{fig:5i}
\end{figure}

\begin{figure}
  \centering
  \includegraphics[width=0.6\textwidth]{case_studies/5_iop_tf}
  \caption{Reconstruction of the CST on intraoperative imaging using tractography (top) and tractfinder (bottom) with standard affine registration.}
  \label{fig:5i_cst}
\end{figure}

This time, significant brain shift away from the craniotomy prevented accurate atlas registration, again resulting in a mismatch in anatomical alignement between atlas and target image (Fig \ref{fig:5i_cst}).
Non-linear registration, with harsh penalties on excessive local deformation  \note{why?}, could mitigate this, but would be impractical for regular use.

A partial resection was completed, and as the intraoperative frozen section \note{??} was reported as appearing high grade, the decision was made not to pursue extensive resection, given the risk of functional injury and the unlikelihood of achieving curative resection for a high grade lesion \note{????}.
There was concern of an area of infarction involving the CST, however the patient, after experiencing worsened motor deficit immediately postoperatively, approved to preoperative levels of the following days.
The patient went on to recieve ajuvant chemotherapy and showed further improvement to their hemiparesis compared to their preoperative condition.

\subsection{Patient 8: High grade glioma}

The second thalamic tumour case study involves a seven year old with a diffuse midline astrocytoma (WHO grade IV) also in the left thalamus.
A high degree of mass effect called for tumour deformation modelling, so the lesion was segmented on the preoperative structural scan (Fig. \ref{fig:8p}).

\begin{figure}
  \centering
  \includegraphics[width=0.5\textwidth]{case_studies/8_preop}
  \caption{Preoperative T1 weighted scan (top) of patient 8 with high grade left thalamic glioma. Bottom row shows tumour deformation modelling applied to a coregistered MNI template image.}
  \label{fig:8p}
\end{figure}

The diffuse and infiltrative nature of this tumour made identifying the posterior capsule, thalamus and tumour margins exceedingly difficult.
Intraoperative DTI was requested by the radiologist after extensive discussion of the already acquired conventional contrast scans, from which the extent of PLIC infiltration was indiscernible (Fig. \ref{fig:8i}).

\begin{figure}
  \centering
  \includegraphics[width=0.5\textwidth]{case_studies/8_iop}
  \caption{Intraoperative imaging showing transcollosal approach to thalamic glioma. Top: T1w, bottom: DTI colour FA. White arrowhead on DTI indicates partially infiltrated and displaced internal capsule.}
  \label{fig:8i}
\end{figure}

Colour FA showed a largely intact and slightly displaced internal capsule with extensive tumour infiltration, so the decision was made not to attempt further resection of that part of the tumour.
In this situation, identification of the internal capsule and assessment of its condition could have been significantly assisted with more advanced \gls{dmri} analysis.
The smearing of colours on the directional FA map make it difficult to discern what affect is leading to the change from the expected indigo colour of a healthy internal capsule.
What is unclear is the state of the neuronal fibres in the infiltrated portion: are they intact but surrounded with odoema/tumour tissue, or simply in a rotated orientation, or destroyed?
Visualising the fibre \glspl{odf} using \gls{csd} gets us closer to the answer (Fig. \ref{fig:8i_fod}).

\begin{figure}
  \begin{subfigure}{0.45\textwidth}
    \centering
    \includegraphics[width=\textwidth]{case_studies/8_iop_csd}
    \caption{CSD FOD}
    \label{fig:8i_csd}
  \end{subfigure}%
  \begin{subfigure}{0.45\textwidth}
    \centering
    \includegraphics[width=\textwidth]{case_studies/8_iop_dt}
    \caption{Diffusion Tensor}
    \label{fig:8i_dt}
  \end{subfigure}
  \caption{Partial view of partially resected tumour and internal capsule on coronal slice and two different fibre orientation models: diffusion tensor and constrained spherical deconvolution. The multi-fibre FODs make it easier to identify where the descending fibres a likely to remain in tact.}
  \label{fig:8i_fod}
\end{figure}

Using the tumour deformation field modelled from the preoperative scan, tractfinder reconstructs the corticospinal tract in very close accordance with probabilistic streamline tractography.

\begin{figure}
  \includegraphics[width=\textwidth]{case_studies/8_iop_tf}
  \caption{Coronal slices in 5mm increments from intraoperative imaging for patient 8. Tractfinder CST map is overlaid along with tractography track density image outline in white for comparison (\note{100 streamline isosurface})}
  \label{fig:8i_tf}
\end{figure}

In the end, roughly 50\% of the lesion was resected, predominantly in the ventral and medial parts.
Postoperatively the patient's right side weakness was slightly improved and the patient received adjuvant radiotherapy \note{did they? only have this down as planned} after confirmation of high-grade histology.
\note{complete with imaging analysis for patient 8: imaging features preop and intraop, and tractography / tractfinder reconstructions}


\epigraph{It is good to have an end to journey toward, but it is the journey that matters in the end.}{Ursula K. Le Guin}
\chapter{Conclusions}
\label{chapterlabel6}

\note{Honestly no idea what will go here. I guess really big picture stuff, since the individual chapters will make their own points about results, pros and cons etc.}

\section{General discussion}

\note{Summary of main findings and outputs, pros and highlight specific achievements}

\section{Limitations}

\note{I'm sure there will be lots to put here}

\section{Future directions}

\note{Wouldn't it be nice ...}


\phantomsection
\addcontentsline{toc}{chapter}{Appendices}

% The \appendix command resets the chapter counter, and changes the chapter numbering scheme to capital letters.
%\chapter{Appendices}
\appendix


\chapter{Tractography parameters and ROI protocols}
\label{app:rois}

The following tracking parameters were used for all probabilistic streamline tractography experiments unless otherwise indicated:


%%%%%%%%%%%%%%%%%%%%%%%%%%%%%%%%%%%%%%%%%%%%%%%%%%%%%%%%%%%%%%%%%%%%%%%%%%%%%%%%
\begin{center}
\begin{tabular}{ l l }
  Parameter & Value \\
 \hline
 Algorithm      &   iFOD2\autocite{Tournier2010} \\
 Number of streamlines selected &   5000 \\
 Maximum angle  &   45\degree  \\
 Step size & $0.5 \, \mathsf{x}$ voxel size \\
 FOD amplitude threshold & 0.1 \\
\end{tabular}
\end{center}
%%%%%%%%%%%%%%%%%%%%%%%%%%%%%%%%%%%%%%%%%%%%%%%%%%%%%%%%%%%%%%%%%%%%%%%%%%%%%%%%

For all pathways, streamlines were generated separately for each hemisphere, before concatenation for further processing.


%%%%%%%%%%%%%%%%%%%%%%%%%%%%%%%%%%%%%%%%%%%%%%%%%%%%%%%%%%%%%%%%%%%%%%%%%%%%%%%%
\begin{figure}[htbp]
  \centering
  \includegraphics[width=\textwidth]{Appendix/OR.png}
  \caption{Optic radiation ROIs. Top: Coronal slice of saggital stratum. Middle: Lateral geniculate nucleus. Bottom: Exclusion slices.}\label{fig:orrois}
\end{figure}
%%%%%%%%%%%%%%%%%%%%%%%%%%%%%%%%%%%%%%%%%%%%%%%%%%%%%%%%%%%%%%%%%%%%%%%%%%%%%%%%


\section{Optic radiation}

Streamlines for the optic radiation were generated according to the following protocol (Fig. \ref{fig:orrois}):
Streamlines were seeded randomly and unidirectionally within a region of interest (ROI) placed on the lateral geniculate nucleus.
An inclusion ROI was placed on two consecutive coronal slices covering the saggital stratum.
Exclusion ROI slices were placed at the level of the superior boundary of the lateral ventricles, two axial slices inferior to the lowest extent the anterior horn of the lateral ventricles, two coronal slices anterior to the anterior horn of the lateral ventricles, along the saggital midline, coronally through the fornix and sagitally through the thalamus.
Additional exclusion slices were used in some cases to exclude streamlines from the corpus callosum and brainstem.



\section{Corticospinal Tract}

Streamlines for the corticospinal tract were generated according to the following protocol (Fig. \ref{fig:cstrois}):
Streamlines were seeded randomly and bidirectionally within an ROI placed on three consecutive axial slices of the posterior limb of the internal capsule.
Inclusion ROIs were placed on the cerebral peduncles (two consecutive axial slices) and at the level of the pons.
Exclusion ROIs were drawn for the cerebellar peduncles (single coronal slice), medial lemniscus (single axial slice), superior fronto-occipital fasciculus (consecutive axial slices), superior longitudinal fasciculus (consecutive axial slices) and saggital midline.



%%%%%%%%%%%%%%%%%%%%%%%%%%%%%%%%%%%%%%%%%%%%%%%%%%%%%%%%%%%%%%%%%%%%%%%%%%%%%%%%
\begin{figure}[htbp]
  \centering
  \includegraphics[width=\textwidth]{Appendix/CST.png}
  \caption{ROIs used in corticospinal tract tractography. Clockwise from bottom right: Posterior limb of internal capsule, CST at level of pons, cerebral peduncles, cerebellar peduncles, medial lemniscus, superior fronto-occipital fasciculus and anterior limb of internal capsule, superior longitudinal fasciculus}\label{fig:cstrois}
\end{figure}
%%%%%%%%%%%%%%%%%%%%%%%%%%%%%%%%%%%%%%%%%%%%%%%%%%%%%%%%%%%%%%%%%%%%%%%%%%%%%%%%




\chapter{Colophon}
\label{appendixlabel3}
\textit{This is a description of the tools you used to make your thesis. It helps people make future documents, reminds you, and looks good.}

\textit{(example)} This document was set in the Times Roman typeface using \LaTeX\ and Bib\TeX , composed with a text editor.
 % description of document, e.g. type faces, TeX used, TeXmaker, packages and things used for figures. Like a computational details section.
% e.g. http://tex.stackexchange.com/questions/63468/what-is-best-way-to-mention-that-a-document-has-been-typeset-with-tex#63503

% Side note:
%http://tex.stackexchange.com/questions/1319/showcase-of-beautiful-typography-done-in-tex-friends


\addcontentsline{toc}{chapter}{Bibliography}
\printbibliography

\chapter*{Colophon}

This document was typeset in \sfdefault{} with \LaTeX\ and Bib\TeX, using the UCL \LaTeX\ \href{https://github.com/UCL/ucl-latex-thesis-templates}{thesis template} created by Ian Kirker.
Original vector graphics were produced in \href{https://inkscape.org/}{Inkscape}, Keynote, in Python using the \href{https://matplotlib.org/}{Matplotlib} package, and in MATLAB, and typeset using the \verb|svg| package.
Original raster graphics were produced with Inkscape and \verb|mrview| from the \href{https://www.mrtrix.org/}{MRtrix3} software package, and edited in \href{https://www.gimp.org/}{GIMP} 2.1 and with the \href{https://imagemagick.org/index.php}{ImageMagick} software suite.

% All done. \o/
\end{document}
