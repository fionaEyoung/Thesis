% UCL Thesis LaTeX Template
%  (c) Ian Kirker, 2014
%
% This is a template/skeleton for PhD/MPhil/MRes theses.
%
% It uses a rather split-up file structure because this tends to
%  work well for large, complex documents.
% We suggest using one file per chapter, but you may wish to use more
%  or fewer separate files than that.
% We've also separated out various bits of configuration into their
%  own files, to keep everything neat.
% Note that the \input command just streams in whatever file you give
%  it, while the \include command adds a page break, and does some
%  extra organisation to make compilation faster. Note that you can't
%  use \include inside an \include-d file.
% We suggest using \input for settings and configuration files that
%  you always want to use, and \include for each section of content.
% If you do that, it also means you can use the \includeonly statement
%  to only compile up the section you're currently interested in.
% You might also want to put figures into their own files to be \input.

% For more information on \input and \include, see:
%  http://tex.stackexchange.com/questions/246/when-should-i-use-input-vs-include


% Formatting and binding rules for theses are here:
%  https://www.ucl.ac.uk/students/exams-and-assessments/research-assessments/format-bind-and-submit-your-thesis-general-guidance

% This package goes first and foremost, because it checks all
%  your syntax for mistakes and some old-fashioned LaTeX commands.
% Note that normally you should load your documentclass before
%  packages, because some packages change behaviour based on
%  your document settings.
% Also, for those confused by the RequirePackage here vs usepackage
%  elsewhere, usepackage cannot be used before the documentclass
%  command, while RequirePackage can. That's the only functional
%  difference as far as I'm aware.
\RequirePackage[l2tabu, orthodox]{nag}


% ------ Main document class specification ------
% The draft option here prevents images being inserted,
%  and adds chunky black bars to boxes that are exceeding
%  the page width (to show that they are).
% The oneside option can optionally be replaced by twoside if
%  you intend to print double-sided. Note that this is
%  *specifically permitted* by the UCL thesis formatting
%  guidelines.
\documentclass[12pt,phd,twoside,executivepaper]{ucl_thesis_executive} % https://www.filepicker.io/api/file/X2O1aJM1Rca68mmYm4PT
% \usepackage[paperheight=8in,
%   paperwidth=5.25in,]{geometry}
\usepackage{showframe}
% Package configuration:
% -------- Packages --------

% This package means empty pages (pages with no text) won't get stuff
%  like chapter names at the top of the page. It's mostly cosmetic.
\usepackage{emptypage}

% The graphicx package adds the \includegraphics command,
%  which is your basic command for adding a picture.
\usepackage{graphicx}

% The float package improves LaTeX's handling of floats,
%  and also adds the option to *force* LaTeX to put the float
%  HERE, with the [H] option to the float environment.
\usepackage{float}

% The amsmath package enhances the various ways of including
%  maths, including adding the align environment for aligned
%  equations.
\usepackage{amsmath}
\usepackage{amssymb}


% Use these two packages together -- they define symbols
%  for e.g. units that you can use in both text and math mode.
\usepackage{gensymb}
\usepackage{textcomp}
% You may also want the units package for making little
%  fractions for unit specifications.
%\usepackage{units}


% The setspace package lets you use 1.5-sized or double line spacing.
\usepackage{setspace}
\setstretch{1.35}

% That just does body text -- if you want to expand *everything*,
%  including footnotes and tables, use this instead:
%\renewcommand{\baselinestretch}{1.5}


% PGFPlots is either a really clunky or really good way to add graphs
%  into your document, depending on your point of view.
% There's waaaaay too much information on using this to cover here,
%  so, you might want to start here:
%   http://pgfplots.sourceforge.net/
%  or here:
%   http://pgfplots.sourceforge.net/pgfplots.pdf
%\usepackage{pgfplots}
%\pgfplotsset{compat=1.3} % <- this fixed axis labels in the version I was using

% PGFPlotsTable can help you make tables a little more easily than
%  usual in LaTeX.
% If you're going to have to paste data in a lot, I'd suggest using it.
%  You might want to start with the manual, here:
%  http://pgfplots.sourceforge.net/pgfplotstable.pdf
%\usepackage{pgfplotstable}

% These settings are also recommended for using with pgfplotstable.
%\pgfplotstableset{
%	% these columns/<colname>/.style={<options>} things define a style
%	% which applies to <colname> only.
%	empty cells with={--}, % replace empty cells with '--'
%	every head row/.style={before row=\toprule,after row=\midrule},
%	every last row/.style={after row=\bottomrule}
%}


% Alternatively, you can use the ifdraft package to let you add
%  commands that will only be used in draft versions
\usepackage{ifdraft}
\ifdraft{
  % Draft mode geometry
  \usepackage[margin=1in]{geometry}
  \setstretch{1}
  \setlength \topmargin{10mm}
  \setlength \oddsidemargin {20mm} % Allow a mm for the bleed.
  \setlength \evensidemargin {20mm}
  % Line numbers
  \usepackage[modulo]{lineno}
  \linenumbers
}


% The multirow package adds the option to make cells span
%  rows in tables.
\usepackage{multirow}


% Subfig allows you to create figures within figures, to, for example,
%  make a single figure with 4 individually labeled and referenceable
%  sub-figures.
% It's quite fiddly to use, so check the documentation.
%\usepackage{subfig}

% The natbib package allows book-type citations commonly used in
%  longer works, and less commonly in science articles (IME).
% e.g. (Saucer et al., 1993) rather than [1]
% More details are here: http://merkel.zoneo.net/Latex/natbib.php
%\usepackage{natbib}

% The bibentry package (along with the \nobibliography* command)
%  allows putting full reference lines inline.
%  See:
%   http://tex.stackexchange.com/questions/2905/how-can-i-list-references-from-bibtex-file-in-line-with-commentary
\usepackage{bibentry}

% The isorot package allows you to put things sideways
%  (or indeed, at any angle) on a page.
% This can be useful for wide graphs or other figures.
%\usepackage{isorot}

% The caption package adds more options for caption formatting.
% This set-up makes hanging labels, makes the caption text smaller
%  than the body text, and makes the label bold.
% Highly recommended.
\usepackage[format=hang,font=small,labelfont=bf]{caption}
\usepackage{subcaption}

% If you're getting into defining your own commands, you might want
%  to check out the etoolbox package -- it defines a few commands
%  that can make it easier to make commands robust.
\usepackage{etoolbox}

% The microtype package adds `micro-typographic extensions' which
% generally makes text more readable and hyphenation less likely.
\usepackage{microtype}

% ---- Other Packages (PERSONALISED)

\usepackage{pdflscape}
\usepackage{rotating}


% For multicolumns
%\usepackage{multicol}
%\setlength{\columnsep}{.7cm}

% separate paragraphs with empty lines
\usepackage[parfill]{parskip}

% Create paragraph title format
\usepackage{titlesec}
\titleformat{\paragraph}[hang]{\normalfont\it\raggedright}{}{0pt}{\qquad}[]
\titlespacing*{\paragraph}{0pt}{0pt}{0pt}

% Section header spacing
\titlespacing*{\section}{0pt}{3\baselineskip}{\baselineskip}
\titlespacing*{\subsection}{0pt}{2\baselineskip}{\baselineskip}
\titlespacing*{\subsubsection}{0pt}{\baselineskip}{\baselineskip}

% Code listings
\usepackage{verbatim}
\usepackage{spverbatim} % not sure what this is for but oh well

% For formatting matlab code specifically, **MUST COPY mcode.sty FILE INTO FOLDER**
%\usepackage[framed,numbered,autolinebreaks,useliterate]{mcode}
\usepackage{listings} % Use like this in place: \lstinputlisting{filname.ext}

% Enumeration, lists
\usepackage{enumerate, enumitem}
\usepackage{framed, color}


\raggedbottom
\widowpenalty10000
\clubpenalty10000
\interfootnotelinepenalty10000

% Sets up links within your document, for e.g. contents page entries
%  and references, and also PDF metadata.
% You should edit this!
%%
%% This file uses the hyperref package to make your thesis have metadata embedded in the PDF, 
%%  and also adds links to be able to click on references and contents page entries to go to 
%%  the pages.
%%

% Some hacks are necessary to make bibentry and hyperref play nicely.
% See: http://tex.stackexchange.com/questions/65348/clash-between-bibentry-and-hyperref-with-bibstyle-elsart-harv
\usepackage{bibentry}
\makeatletter\let\saved@bibitem\@bibitem\makeatother
\usepackage[pdftex,hidelinks]{hyperref}
\makeatletter\let\@bibitem\saved@bibitem\makeatother
\makeatletter
\AtBeginDocument{
    \hypersetup{
        pdfsubject={Thesis Subject},
        pdfkeywords={Thesis Keywords},
        pdfauthor={Author},
        pdftitle={Title},
    }
}
\makeatother
    


% And then some settings in separate files.
% These settings are partly from:
%  http://mintaka.sdsu.edu/GF/bibliog/latex/floats.html

% They give LaTeX more options on where to put your figures, and may
%  mean that fewer of your figures end up at the tops of pages far
%  away from the thing they're related to.

% Alters some LaTeX defaults for better treatment of figures:
% See p.105 of "TeX Unbound" for suggested values.
% See pp. 199-200 of Lamport's "LaTeX" book for details.

%   General parameters, for ALL pages:
\renewcommand{\topfraction}{0.9}	% max fraction of floats at top
\renewcommand{\bottomfraction}{0.8}	% max fraction of floats at bottom

%   Parameters for TEXT pages (not float pages):
\setcounter{topnumber}{2}
\setcounter{bottomnumber}{2}
\setcounter{totalnumber}{4}     % 2 may work better
\setcounter{dbltopnumber}{2}    % for 2-column pages
\renewcommand{\dbltopfraction}{0.9}	% fit big float above 2-col. text
\renewcommand{\textfraction}{0.2}	% page must be at least 20% text,
%                                  less than that and we get a floatpage

%   Parameters for FLOAT pages (not text pages):
\renewcommand{\floatpagefraction}{0.7}	% require fuller float pages
% N.B.: floatpagefraction MUST be less than topfraction !!
\renewcommand{\dblfloatpagefraction}{0.7}	% require fuller float pages

% remember to use [htp] or [htpb] for placement,
% e.g.
%  \begin{figure}[htp]
%   ...
%  \end{figure}

\usepackage[inkscapearea=page,
            inkscapeexe=/Applications/Inkscape.app/Contents/MacOS/inkscape,
            draft=false,
            inkscapelatex=false]{svg}

\ifdraft{%
  % Figure path
  \graphicspath{{draft_figs/}{figs/}}
  \svgpath{{draft_figs/}{figs/}}
  \setkeys{Gin}{draft=false}
}{%
  \graphicspath{{figs/}{draft_figs/}}
  \svgpath{{figs/}{draft_figs/}}
}

% "Times" symbol for resolutions etc.
\newcommand{\x}{\nobreak\hspace{.1em minus .045em}$\times$\nobreak\hspace{.1em minus .045em}}
 % For things like figures and tables
%\bibliographystyle{unsrt}

% Bibliography and referencing settings

\usepackage[doi=true,
            isbn=true,
            url=false,
            natbib=true,
            maxcitenames=1,
            maxbibnames=50,
            style=nature,
            backend=bibtex]{biblatex} % Add citation styles as needed (see docs for biblatex)
\IfFileExists{bibliographies/Neuroimaging.bib}
          {\addbibresource{bibliographies/Neuroimaging.bib}}
          {\addbibresource{bibs/Neuroimaging.bib}}
\IfFileExists{bibliographies/Neurosurgery.bib}
          {\addbibresource{bibliographies/Neurosurgery.bib}}
          {\addbibresource{bibs/Neurosurgery.bib}}
\IfFileExists{bibliographies/Figures.bib}
          {\addbibresource{bibliographies/Figures.bib}}
          {\addbibresource{bibs/Figures.bib}}
\IfFileExists{bibliographies/Oncology.bib}
          {\addbibresource{bibliographies/Oncology.bib}}
          {\addbibresource{bibs/Oncology.bib}}
\addbibresource{bibliographies/Maths.bib}
\addbibresource{bibliographies/Neuroscience.bib}
\addbibresource{bibliographies/Thesis.bib}


% Make DOIs look normal
\urlstyle{same}

 % Path to bib file (e.g. generated from Mendeley)
% \renewcommand*{\bibfont}{\small} % Make bibligraphy small

% Use the below in conjunction with nature style referencing to get proper behaviour of textcite command (numeric citation still superscript rather than bracket)
\usepackage{ifthen}
\renewcommand{\textcite}[2][]{
\ifthenelse { \equal {#1} {} }  %
    {\citeauthor{#2}\autocite{#2}}   % if #1 == blank
    {\citeauthor{#1}\autocite{#2}}}

\DeclareCiteCommand{\fullcite}
  {\usebibmacro{prenote}}
  {\usedriver
     {\defcounter{maxnames}{50}}
     {\thefield{entrytype}}.}
  {\multicitedelim}
  {\usebibmacro{postnote}}
   % For bibliographies
%%
%% Set up abbreviations and definitions
\usepackage[acronym,
            toc=false,
            style=super,
            seeautonumberlist,
            nogroupskip=true]
            {glossaries-extra}
\setabbreviationstyle[acronym]{long-short}
\GlsXtrSetDefaultGlsOpts{noindex}
\renewcommand*\glspostdescription{\quad}
\preto\chapter{\glsresetall}
\makeglossaries
%% MRI
\newacronym{mri}{MRI}{magnetic resonance imaging}
\newacronym{nmr}{NMR}{nuclear magnetic resonance}
% Recursive acronym \protect solution from https://tex.stackexchange.com/a/502313
\newacronym{dmri}{dMRI}{diffusion-weighted \protect\ifglsused{mri}{MRI}{magnetic resonance imaging}}
\newacronym{imri}{iMRI}{intraoperative \protect\ifglsused{mri}{MRI}{magnetic resonance imaging}}
\newacronym{fmri}{fMRI}{functional \protect\ifglsused{mri}{MRI}{magnetic resonance imaging}}
\newacronym{hardi}{HARDI}{high angular resolution diffusion imaging}
\newacronym{epi}{EPI}{echo planar imaging}
\newacronym{rsepi}{RS-EPI}{readout-segmented \protect\ifglsused{epi}{EPI}{echo planar imaging}}
\newacronym{ssepi}{SS-EPI}{single-shot \protect\ifglsused{epi}{EPI}{echo planar imaging}}
\newacronym{rf}{RF}{radio frequency}
\newacronym{snr}{SNR}{signal to noise ratio}
\newacronym{fid}{FID}{free induction decay}
\newacronym{2dft}{2DFT}{two dimensional \protect\ifglsused{ft}{FT}{Fourier transform}}
\newacronym{ft}{FT}{Fourier transform}
\newacronym{fe}{FE}{frequency encoding}
\newacronym{pe}{PE}{phase encoding}
\newacronym[longplural={proton densities}]{pd}{PD}{proton density}
\newacronym{ir}{IR}{inversion recovery}
\newacronym{mprage}{MPRAGE}{magnetization prepared rapid gradient echo}
\newacronym{dce}{DCE}{dynamic contrast enhanced}
% Image computing
\newacronym{csd}{CSD}{constrained spherical deconvolution}
\newacronym{odf}{ODF}{orientation distribution function}
\newacronym{dodf}{dODF}{diffusion orientation distribution function}
\newacronym{fodf}{fODF}{fibre \protect\ifglsused{odf}{ODF}{orientation distribution function}}
\newacronym{fod}{FOD}{fibre orientation distribution}
\newacronym{tod}{TOD}{track orientation distribution}
\newacronym{dti}{DTI}{diffusion tensor imaging}
\newacronym{dt}{DT}{diffusion tensor}
\newacronym{fa}{FA}{fractional anisotropy}
\newacronym{dec}{DEC}{directionally encoded colour}
\newacronym{ssst}{SSST}{single-shell, single-tissue}
\newacronym{msmt}{MSMT}{multi-shell, multi-tissue}
\newacronym[longplural={regions of interest}]{roi}{ROI}{region of interest}
\newacronym{afd}{AFD}{apparent fibre density}
\newacronym{adc}{ADC}{apparent diffusion coefficient}
\newacronym{fact}{FACT}{fibre assignment by continuous tracking}
\newacronym{tdi}{TDI}{track density imaging}
\newacronym{mppca}{MPPCA}{Marchenko-Pastur principal component analysis}
%% Neuroanatomy
\newacronym{cns}{CNS}{central nervous system}
\newacronym{pns}{PNS}{peripheral nervous system}
\newacronym{bbb}{BBB}{blood-brain barrier}
\newacronym{cst}{CST}{corticospinal tract}
\newacronym{or}{OR}{optic radiation}
\newacronym[longplural={lateral geniculate nuclei}]{lgn}{LGN}{lateral geniculate nucleus}
\newacronym{ml}{ML}{Meyer's loop}
\newacronym[longplural={arcuate fasciculi}]{af}{AF}{arcuate fasciculus}
\newacronym[longplural={inferior fronto-occipital fasciculi}]{ifof}{IFOF}{inferior fronto-occipital fasciculus}
\newacronym[longplural={uncinate fasciculi}]{uf}{UF}{uncinate fasciculus}
\newacronym{wm}{WM}{white matter}
\newacronym{gm}{GM}{grey matter}
\newacronym{csf}{CSF}{cerebrospinal fluid}
\newacronym{crp}{CrP}{cerebellar peduncle}
\newacronym{cp}{CP}{cerebral peduncle}
\newacronym{cc}{CC}{corpus callosum}
\newacronym{ec}{EC}{external capsule}
\newacronym[longplural={superior longitudinal fasciculi}]{slf}{SLF}{superior longitudinal fasciculus}
\newacronym[longplural={superior fronto-occipital fasciculi}]{sfof}{SFOF}{superior fronto-occipital fasciculus}
\newacronym[longplural={vertical occipital fasciculi}]{vof}{VOF}{vertical occipital fasciculus}
\newacronym{ss}{SS}{sagittal stratum}
%% Neurosurgery & Neurology
\newacronym{dbs}{DBS}{deep brain stimulation}
\newacronym[longplural={extents of resection}]{eor}{EOR}{extent of resection}
\newacronym{gtr}{GTR}{gross total resection}
\newacronym{str}{STR}{subtotal resection}
\newacronym{hgg}{HGG}{high-grade glioma}
\newacronym{lgg}{LGG}{low-grade glioma}
\newacronym{eeg}{EEG}{electroencephalography}
\newacronym{ct}{CT}{computed tomography}
\newacronym{des}{DES}{direct electrical stimulation}
\newacronym{who}{WHO}{World Health Organisation}
%% Tractography et al.
\newacronym{tg}{TG}{tractography}
\newacronym{tgr}{TGR}{reference tractography}
\newacronym{tf}{TF}{tractfinder}
\newacronym{tsd}{TSD}{TractSeg DKFZ}
\newacronym{tsx}{TSX}{TractSeg XTRACT}
\newacronym{at}{AT}{atlas registration}
%% Other
\newacronym{hcp}{HCP}{human connectome project}
\newacronym{btc}{BTC}{Brain Tumour Connectomics}
\newacronym{dice}{DSC}{Dice-Soerensen similarity coefficient}
\newacronym{sh}{SH}{spherical harmonic}
\newacronym{fsl}{FSL}{FMRIB software library}
\newacronym{gosh}{GOSH}{Great Ormond Street Hospital}
\newacronym{nhnn}{NHNN}{National Hospital for Neurology and Neurosurgery}

%% Here we set up a new command to generate a list of abbreviations, e.g. for
%% float captions, formatted in a consistent way

% Define the separtors between abbreviation and definition, and between list items
% including any spaces!
\def\acrodefsep{ = }
\def\acrolistdelim{; }

% Process list of arbitratry length
\NewDocumentCommand{\acrolist}{>{\SplitList{,}}m}
    {\ProcessList{#1}{\acroformat}\firstitemtrue}
% Format the list using each item's glossary entry and the defined separators
\newif\iffirstitem
\firstitemtrue
\newcommand\acroformat[1]{%
  \iffirstitem
    \firstitemfalse
  \else
    \acrolistdelim %
  \fi
  \glsname{#1}\acrodefsep \glsdesc{#1}}
% ^ proper hangling of delimiter solution from https://tex.stackexchange.com/a/110906

%% Alternative solution from https://tex.stackexchange.com/a/110909 if the above causes issues
% \NewDocumentCommand{\acrolist}{>{\SplitList{,}}m}
%     {%
%     \def\acrodelim{\def\acrodelim{, }}%
%     \ProcessList{#1}{\acroformat} }
% \newcommand{\acroformat}[1]{\acrodelim\glsentryshort{#1} = \glsentrylong{#1}}

\newcommand\epigraph[2]{%
  \begin{flushright}
    \parbox{0.75\textwidth}{\raggedleft #1}
    \vskip 1.5\baselineskip
    --- #2
  \end{flushright}%
}

\newcommand\epipage[2]{%
  \clearpage\thispagestyle{empty}
  \vspace*{\fill}
  {\sffamily\epigraph{#1}{#2}}
  \vspace*{\fill}\pagebreak
}

\newcommand\epart[2][]{%
  \cleardoublepage          % Page break
  \thispagestyle{empty}
  \vspace*{.3\vsize}        % Vertical shift
  \refstepcounter{part}
  \addcontentsline{toc}{part}{#2}%
  % Part title, center, 1/3 down
  {\centering \textbf{\sffamily\Huge#2}\par}%
  \vspace*{\fill}
  {\sffamily#1}
  \vspace*{\fill}\pagebreak
}


% These control how many number sections your subsections will take
%    e.g. Section 2.3.1.5.6.3
%  and how many of those will get put into the contents pages.
\setcounter{secnumdepth}{3}
\setcounter{tocdepth}{2}

\newcommand\note[1]{\ifdraft{{\large\textcolor{red}{#1}}}{}}
%TC:macro \note [ignore]
\includeonly{Frontback/Preamble,Introduction/Introduction,Introduction/Theory}
\begin{document}

% \nobibliography*
% ^-- This is a dumb trick that works with the bibentry package to let
%  you put bibliography entries whereever you like.
% I used this to put references to papers a chapter's work was
%  published in at the end of that chapter.
% For more information, see: http://stefaanlippens.net/bibentry

% I may change the way this is done in a future version,
%  but given that some people needed it, if you need a different degree title
%  (e.g. Master of Science, Master in Science, Master of Arts, etc)
%  uncomment the following 3 lines and set as appropriate (this *has* to be before \maketitle)
% \makeatletter
% \renewcommand {\@degree@string} {Master of Things}
% \makeatother

\title{ Fibre tract imaging with intraoperative diffusion MRI for neurosurgical navigation }
\author{ Fiona Young }
\department{ Department of Medical Physics and Biomedical Engineering \\ UCL GOS Institute of Child Health }

\maketitle
\makedeclaration

\begin{abstract} % 300 word limit
My research is about stuff.

It begins with a study of some stuff, and then some other stuff and things.

There is a 300-word limit on your abstract.
\end{abstract}

\begin{impactstatement}

	UCL theses now have to include an impact statement. \textit{(I think for REF reasons?)} The following text is the description from the guide linked from the formatting and submission website of what that involves. (Link to the guide: {\scriptsize \url{http://www.grad.ucl.ac.uk/essinfo/docs/Impact-Statement-Guidance-Notes-for-Research-Students-and-Supervisors.pdf}})

\begin{quote}
The statement should describe, in no more than 500 words, how the expertise, knowledge, analysis,
discovery or insight presented in your thesis could be put to a beneficial use. Consider benefits both
inside and outside academia and the ways in which these benefits could be brought about.

The benefits inside academia could be to the discipline and future scholarship, research methods or
methodology, the curriculum; they might be within your research area and potentially within other
research areas.

The benefits outside academia could occur to commercial activity, social enterprise, professional
practice, clinical use, public health, public policy design, public service delivery, laws, public
discourse, culture, the quality of the environment or quality of life.

The impact could occur locally, regionally, nationally or internationally, to individuals, communities or
organisations and could be immediate or occur incrementally, in the context of a broader field of
research, over many years, decades or longer.

Impact could be brought about through disseminating outputs (either in scholarly journals or
elsewhere such as specialist or mainstream media), education, public engagement, translational
research, commercial and social enterprise activity, engaging with public policy makers and public
service delivery practitioners, influencing ministers, collaborating with academics and non-academics
etc.

Further information including a searchable list of hundreds of examples of UCL impact outside of
academia please see \url{https://www.ucl.ac.uk/impact/}. For thousands more examples, please see
\url{http://results.ref.ac.uk/Results/SelectUoa}.
\end{quote}
\end{impactstatement}

\begin{acknowledgements}
THANKS!
\end{acknowledgements}

%	Self-plagiarism declaration form template for these typeset in LaTeX
%	Prepared by David Sheard 2022 and made available free of copyright

%	If you use the results of your own published, accepted or submitted data (text or figures) in your final
%	doctoral thesis, you have to give a clear indication of the previous work, stating the exact source of the
%	previous material, irrespective of whether copyright is owned by you or by a publisher. This indication
%	should take the form of
%		a) an appropriate citation of the original source in the relevant Chapter; and
%		b) completion of the UCL Research Paper Declaration form---this should be embedded after the
%		Acknowledgements page in the thesis.

%	For more information consult the following links:
%	\url{https://www.grad.ucl.ac.uk/essinfo/guidance-on-selfplagiarism/?utm_source=Students\%27+Union+UCL\&utm_campaign=2ed9e73ab7-\&utm_medium=email\&utm_term=0_fe8c0cbcf2-2ed9e73ab7-209240456\&mc_cid=2ed9e73ab7\&mc_eid=0496c22bfc}
%	\url{https://www.grad.ucl.ac.uk/essinfo/guidance-on-selfplagiarism/Declaration-form_published-work-in-thesis.docx}

%	I recommend using this template by simply copying everything between \begin{document} and \end{document}
%	into your thesis after the acknowledgements section. By changing the preamble appropriately this template
%	may also be altered to work using \input{...} or the subfiles package.

% \documentclass[12pt, twoside]{article}
%
% \usepackage[a4paper,inner=40mm,outer=20mm, top=30mm, bottom=30mm]{geometry}
% \usepackage{amssymb}
% \usepackage{array}
% \usepackage{setspace}
% 	\setstretch{1.5}
%
% \begin{document}
{\sffamily
\section*{UCL Research Paper Declaration Form: referencing the doctoral candidate’s own published work(s)}
% Uncomment the following line if you would like to add this declaration to your table of contents
% \addcontentsline{toc}{section}{UCL Research Paper Declaration Form}
%
% Please use this form to declare if parts of your thesis are already available in another format,
% e.g. if data, text, or figures:
% •	have been uploaded to a preprint server
% •	are in submission to a peer-reviewed publication
% •	have been published in a peer-reviewed publication, e.g. journal, textbook.
%
% This form should be completed as many times as necessary. For instance, if a student had seven
% thesis chapters, two of which having material which had been published, they would complete this form twice.

\begin{enumerate}[leftmargin=*,label={\bfseries\arabic*.}]\itemsep0em
	\item \textbf{For a research manuscript that has already been published} (if not yet published, please skip to section 2)\textbf{:}
	%
	\begin{enumerate}[label={\alph*)}]\itemsep0em
	%
	\item \textbf{What is the title of the manuscript?}

	\citetitle{Young2022}

	\item \textbf{Please include a link to or doi for the work:}

	\url{https://doi.org/10.1007/s11548-022-02617-z}

	\item \textbf{Where was the work published?}

	\citefield{Young2022}{journaltitle}% Answer here: e.g. journal name

	\item \textbf{Who published the work?}

	\citelist{Young2022}{publisher}% Answer here: e.g. Elsevier/Oxford University Press

	\item \textbf{When was the work published?}

	\citefield{Young2022}{year}% Answer here

	\item \textbf{List the manuscript's authors in the order they appear on the publication:}
	\citeauthor*{Young2022}% Answer here

	\item \textbf{Was the work peer reviewd?}

	Yes

	\item \textbf{Have you retained the copyright?}

	Yes, from the Licence to Publish agreement:
	\begin{quote}
		Ownership of copyright in the Article will be vested in the name of the Author.
	\end{quote}% Answer here

	\item \textbf{Was an earlier form of the manuscript uploaded to a preprint server (e.g. medRxiv)? If `Yes’, please give a link or doi}

	No% Answer here:
	\\
	If ‘No’, please seek permission from the relevant publisher and check the box next to the below statement:
	%
		\begin{itemize}\itemsep0em
		% To check this box, replace \Box with \boxtimes
		\item[$\boxtimes$] {\itshape I acknowledge permission of the publisher named under 1d to include in this thesis portions of the publication named as included in 1c.}
		\end{itemize}
	%
	\end{enumerate}
%
\item \textbf{For a research manuscript prepared for publication but that has not yet been published} (if already published, please skip to section 3)\textbf{:}
%
\begin{enumerate}[label={\alph*)}]\itemsep0em
	%
	\item \textbf{What is the current title of the manuscript?}
	% Answer here:
	\item \textbf{Has the manuscript been uploaded to a preprint server `e.g. medRxiv'?
	\\
	If `Yes', please please give a link or doi:}
	% Answer here:
	\item \textbf{Where is the work intended to be published?}
	% Answer here: e.g. journal name
	\item \textbf{List the manuscript's authors in the intended authorship order:}
	% Answer here
	\item \textbf{Stage of publication:}
	% answer here: e.g. in submission
	%
\end{enumerate}

\item \textbf{For multi-authored work, please give a statement of contribution covering all authors} (if single-author, please skip to section 4)\textbf{:}
\begin{description}[font=\sffamily]
	\item[Fiona Young] Methodology (conceptualisation and implementation), analysis, manuscript original draft and graphics
	\item[Kristian Aquilina] Supervision, manuscript review and editing
	\item[Chris A. Clark] Supervision, manuscript review and editing
	\item[Jonathan D. Clayden] Conceptualisation, supervision, manuscript review and editing
\end{description}
% Answer here
\item \textbf{In which chapter(s) of your thesis can this material be found?}
% Answer here

Chapters \ref{chap:reg}, \ref{chap:applications}

\end{enumerate}

\textbf{e-Signatures confirming that the information above is accurate}
(this form should be co-signed by the supervisor/ senior author unless this is not appropriate, e.g. if the paper was a single-author work)\textbf{:}\\
\textbf{}\\
\textbf{Candidate:}\\
\textbf{Date:}\\
% this form should be co-signed by the supervisor/ senior author unless this is not appropriate, e.g. if the paper was a single-author work):
\textbf{}\\
\textbf{Supervisor/Senior Author signature} (where appropriate)\textbf{:}\\
\textbf{Date:}
%

%%%%%%%%%%%%%%%%%%%%%%%%%%%%% --- HBM --- %%%%%%%%%%%%%%%%%%%%%%%%%%%%%%%%%%%%%%
\newpage
\begin{enumerate}[leftmargin=*,label={\bfseries\arabic*.}]\itemsep0em
%
\item \textbf{For a research manuscript that has already been published} (if not yet published, please skip to section 2)\textbf{:}
%
\begin{enumerate}[label={\alph*)}]\itemsep0em
	%
	\item \textbf{What is the title of the manuscript?}

	\citetitle{Young2024}

	\item \textbf{Please include a link to or doi for the work:}

	\citefield{Young2024}{doi}

	\item \textbf{Where was the work published?}

	\citefield{Young2024}{journaltitle}% Answer here: e.g. journal name

	\item \textbf{Who published the work?}

	\citelist{Young2024}{publisher}% Answer here: e.g. Elsevier/Oxford University Press

	\item \textbf{When was the work published?}

	\citefield{Young2024}{year}% Answer here

	\item \textbf{List the manuscript's authors in the order they appear on the publication:}

	\citeauthor*{Young2024}% Answer here

	\item \textbf{Was the work peer reviewd?}

	Yes

	\item \textbf{Have you retained the copyright?}

	Yes, from license agreement:
	\begin{quote}
		The Author and each Co-author or, if applicable, the Author’s or Co-author’s employer, retains all proprietary rights, such as copyright
	\end{quote}

	\item \textbf{Was an earlier form of the manuscript uploaded to a preprint server (e.g. medRxiv)? If `Yes’, please give a link or doi}

	No% Answer here:
	\\
	If ‘No’, please seek permission from the relevant publisher and check the box next to the below statement:
	%
	\begin{itemize}\itemsep0em
	% To check this box, replace \Box with \boxtimes
	\item[$\boxtimes$] {\itshape I acknowledge permission of the publisher named under 1d to include in this thesis portions of the publication named as included in 1c.}
	\end{itemize}
	%
\end{enumerate}
%
\item \textbf{For a research manuscript prepared for publication but that has not yet been published} (if already published, please skip to section 3)\textbf{:}
%
\begin{enumerate}[label={\alph*)}]\itemsep0em
	%
	\item \textbf{What is the current title of the manuscript?}
	% Answer here:
	\item \textbf{Has the manuscript been uploaded to a preprint server `e.g. medRxiv'?
	\\
	If `Yes', please please give a link or doi:}
	% Answer here:
	\item \textbf{Where is the work intended to be published?}
	% Answer here: e.g. journal name
	\item \textbf{List the manuscript's authors in the intended authorship order:}
	% Answer here
	\item \textbf{Stage of publication:}
	% answer here: e.g. in submission
	%
\end{enumerate}

\item \textbf{For multi-authored work, please give a statement of contribution covering all authors} (if single-author, please skip to section 4)\textbf{:}
\begin{description}[font=\sffamily]
	\item[Fiona Young] Methodology (conceptualisation and implementation), analysis, manuscript original draft and graphics
	\item[Kristian Aquilina] Supervision, conceptualisation, manuscript review and editing
	\item[Laura Mancini] Data contribution, manuscript review and editing
	\item[Kiran K. Seunarine] Data contribution and curation
	\item[Chris A. Clark] Supervision, manuscript review and editing
	\item[Jonathan D. Clayden] Conceptualisation, supervision, manuscript review and editing
\end{description}
% Answer here
\item \textbf{In which chapter(s) of your thesis can this material be found?}
% Answer here

Chapters \ref{chap:review}, \ref{chap:atlas}, \ref{chap:reg} (Section \ref{sec:reg1}), \ref{chap:eval}, Appendix
\end{enumerate}

\textbf{e-Signatures confirming that the information above is accurate}
(this form should be co-signed by the supervisor/ senior author unless this is not appropriate, e.g. if the paper was a single-author work)\textbf{:}\\
\textbf{}\\
\textbf{Candidate:}\\
\textbf{Date:}\\
% this form should be co-signed by the supervisor/ senior author unless this is not appropriate, e.g. if the paper was a single-author work):
\textbf{}\\
\textbf{Supervisor/Senior Author signature} (where appropriate)\textbf{:}\\
\textbf{Date:}
%

%%%%%%%%%%%%%%%%%%%%%%%%%%%%% --- IEEE --- %%%%%%%%%%%%%%%%%%%%%%%%%%%%%%%%%%%%%
\newpage
\begin{enumerate}[leftmargin=*,label={\bfseries\arabic*.}]\itemsep0em
	%
	\item \textbf{For a research manuscript that has already been published} (if not yet published, please skip to section 2)\textbf{:}
	%
	\begin{enumerate}[label={\alph*)}]\itemsep0em
	%
	\item \textbf{What is the title of the manuscript?}

	\item \textbf{Please include a link to or doi for the work:}

	\item \textbf{Where was the work published?}

	\item \textbf{Who published the work?}

	\item \textbf{When was the work published?}

	\item \textbf{List the manuscript's authors in the order they appear on the publication:}

	\item \textbf{Was the work peer reviewd?}

	\item \textbf{Have you retained the copyright?}

	\item \textbf{Was an earlier form of the manuscript uploaded to a preprint server (e.g. medRxiv)? If ‘Yes’, please give a link or doi}
	\\
	If ‘No’, please seek permission from the relevant publisher and check the box next to the below statement:
	%
\begin{itemize}\itemsep0em
% To check this box, replace \Box with \boxtimes
\item[$\Box$] {\itshape I acknowledge permission of the publisher named under 1d to include in this thesis portions of the publication named as included in 1c.}
\end{itemize}
%
\end{enumerate}
%
\item \textbf{For a research manuscript prepared for publication but that has not yet been published} (if already published, please skip to section 3)\textbf{:}
%
\begin{enumerate}[label={\alph*)}]\itemsep0em
	%
	\item \textbf{What is the current title of the manuscript?}

	\citetitle{Young2023}
	\item \textbf{Has the manuscript been uploaded to a preprint server `e.g. medRxiv'?
	\\
	If `Yes', please please give a link or doi:}

	Yes, \url{https://www.researchgate.net/publication/375601134_Training_Data_Requirements_for_Atlas-Based_Brain_Fibre_Tract_Identification}
	% Answer here:
	\item \textbf{Where is the work intended to be published?}
	% Answer here: e.g. journal name

	Proceedings of the \citefield{Young2023}{eventtitle}
	\item \textbf{List the manuscript's authors in the intended authorship order:}
	% Answer here

	\citeauthor*{Young2023}
	\item \textbf{Stage of publication:}
	% answer here: e.g. in submission

	Work presented at conference, copyright transferred to IEEE, yet now indication of if/when proceedings may be published (copyright on submitted version retained).
\end{enumerate}

\item \textbf{For multi-authored work, please give a statement of contribution covering all authors} (if single-author, please skip to section 4)\textbf{:}
\begin{description}[font=\sffamily]
	\item[Fiona Young] Methodology (conceptualisation and implementation), analysis, manuscript original draft and graphics
	\item[Kristian Aquilina] Supervision
	\item[Chris A. Clark] Supervision
	\item[Jonathan D. Clayden] Conceptualisation, supervision, manuscript review and editing
\end{description}
% Answer here
\item \textbf{In which chapter(s) of your thesis can this material be found?}
% Answer here

Chapter \ref{chap:atlas}
\end{enumerate}

\textbf{e-Signatures confirming that the information above is accurate}
(this form should be co-signed by the supervisor/ senior author unless this is not appropriate, e.g. if the paper was a single-author work)\textbf{:}\\
\textbf{}\\
\textbf{Candidate:}\\
\textbf{Date:}\\
% this form should be co-signed by the supervisor/ senior author unless this is not appropriate, e.g. if the paper was a single-author work):
\textbf{}\\
\textbf{Supervisor/Senior Author signature} (where appropriate)\textbf{:}\\
\textbf{Date:}
%

%%%%%%%%%%%%%%%%%%%%%%%%% --- Diffusion ISMRM --- %%%%%%%%%%%%%%%%%%%%%%%%%%%%%
\newpage
\begin{enumerate}[leftmargin=*,label={\bfseries\arabic*.}]\itemsep0em
	%
	\item \textbf{For a research manuscript that has already been published} (if not yet published, please skip to section 2)\textbf{:}
	%
	\begin{enumerate}[label={\alph*)}]\itemsep0em
	%
	\item \textbf{What is the title of the manuscript?}

	\item \textbf{Please include a link to or doi for the work:}

	\item \textbf{Where was the work published?}

	\item \textbf{Who published the work?}

	\item \textbf{When was the work published?}

	\item \textbf{List the manuscript's authors in the order they appear on the publication:}

	\item \textbf{Was the work peer reviewd?}

	\item \textbf{Have you retained the copyright?}

	\item \textbf{Was an earlier form of the manuscript uploaded to a preprint server (e.g. medRxiv)? If ‘Yes’, please give a link or doi}
	\\
	If ‘No’, please seek permission from the relevant publisher and check the box next to the below statement:
	%
\begin{itemize}\itemsep0em
% To check this box, replace \Box with \boxtimes
\item[$\Box$] {\itshape I acknowledge permission of the publisher named under 1d to include in this thesis portions of the publication named as included in 1c.}
\end{itemize}
%
\end{enumerate}
%
\item \textbf{For a research manuscript prepared for publication but that has not yet been published} (if already published, please skip to section 3)\textbf{:}
%
\begin{enumerate}[label={\alph*)}]\itemsep0em
	%
	\item \textbf{What is the current title of the manuscript?}
	% Answer here:

	\citetitle{Young2022a}
	\item \textbf{Has the manuscript been uploaded to a preprint server `e.g. medRxiv'?
	\\
	If `Yes', please please give a link or doi:}

	Yes, \url{https://www.researchgate.net/publication/367116849_Stability_of_white_matter_tract_segmentation_methods_with_decreasing_data_quality}
	% Answer here:
	\item \textbf{Where is the work intended to be published?}
	% Answer here: e.g. journal name

	N/A
	\item \textbf{List the manuscript's authors in the intended authorship order:}
	% Answer here

	\citeauthor*{Young2022a}
	\item \textbf{Stage of publication:}
	% answer here: e.g. in submission

	Work presented at conference, no proceedings published.
\end{enumerate}

\item \textbf{For multi-authored work, please give a statement of contribution covering all authors} (if single-author, please skip to section 4)\textbf{:}
\begin{description}[font=\sffamily]
	\item[Fiona Young] Methodology (conceptualisation and implementation), analysis, manuscript original draft and graphics
	\item[Jonathan D. Clayden] Conceptualisation, supervision, manuscript review and editing
\end{description}
% Answer here
\item \textbf{In which chapter(s) of your thesis can this material be found?}
% Answer here

Chapter \ref{chap:applications}
\end{enumerate}

\textbf{e-Signatures confirming that the information above is accurate}
(this form should be co-signed by the supervisor/ senior author unless this is not appropriate, e.g. if the paper was a single-author work)\textbf{:}\\
\textbf{}\\
\textbf{Candidate:}\\
\textbf{Date:}\\
% this form should be co-signed by the supervisor/ senior author unless this is not appropriate, e.g. if the paper was a single-author work):
\textbf{}\\
\textbf{Supervisor/Senior Author signature} (where appropriate)\textbf{:}\\
\textbf{Date:}
%
}% End font switch


\setcounter{tocdepth}{2}
% Setting this higher means you get contents entries for
%  more minor section headers.

\tableofcontents
% \listoffigures
% \listoftables


\epipage{I'll tell you where the real road lies\\
Between your ears, behind your eyes}{Anais Mitchell, \textit{Wait For Me (Reprise)}}
\chapter{Introductory Material}
\label{chapterlabel1}

% -- DELETE AND REWRITE THIS; DEMO ONLY
Intraoperative dMRI has the potential to supplement existing imaging practices by offering a means of imaging fibre tracts after brain shift has invalidated preoperative imaging.\autocite{Nimsky2001}
Thus informed, surgeons would be better equipped to resect as much tumour as possible while leaving eloquent brain tissue intact.



\epart[\epigraph{Too much knowledge never makes for simple decisions.}{Frank Herbert, \textit{Children of Dune}}]{Background}
% \part*{Background}
% \chapter{Theory and Background}
% \label{chapterlabel1}

\chapter{Theory fundamentals}
\label{theory}

\note{This will cover the more theoretical background prerequisites, covering the fundamental concepts on the biological side (neuroanatomy and cellular physiology, maybe some oncology?) and physical side (MR physics)}

\section{Neuroanatomy: Micro to macro}

\subsection{Cells of the brain}

\note{neurons, glial cells, axons and basics of neural signalling.}

The building blocks of living complex organisms begin at the level of molecules and atoms, through macromolecules such as proteins and lipids, organelles, cells, tissues, organs and finally whole organisms.
For the purposes of this report, though, we can start at the cellular level, and focus on a single organ: the brain.
Brain tissues consist of numerous cell types.
The principle functional cells of the human nervous system are neurons, which perform the integration and transmission of signals underpinning all aspects of neural function.
They are supported, and outnumbered ( by a \note{factor of?}), by a network of glial cells, each with specialised functions.
There are countless types of neurons covering the spectrum of specialised function, however they all have the following basic components \note{true?}:
The cell body, or soma, houses the cell nucleus, metabolic activity and organelles responsible for the day-to-day upkeep and maintenance of the entire cell.
Neurons recieve signal inputs from other neurons via a web of protrusions called dendrites. \note{anything else to say about dendrites?}
Signals are sent via a single, specialised protrusion called an axon.
Axon terminals interface with the dentrites of other neurons or, in the peripheral nervous system, with muscles and other innervated tissues.
The \note{nanoscopic?} gap between the axon terminal and the receving cell is called a synapse.

Glial cells are the other major class of nervous system residents, previously thought to outnumber neurons \note{10:1, although this figure is now disputed.}
Astrocytes are supportive glial cells provide a range of services including phagocytosis, scar tissue formation and blood-brain barrier formation.
Oligodendrocytes are the \gls{cns}'s myelinating glia.
They form processes which wrap around axons as myelin sheaths, with each oligdendrocyte able to mylenate multiple axons simultaneously.
In the \gls{pns}, myelination is performed by Schwann cells, which each form a single myelin sheath.
\note{add microglia}

\subsection{Neural signalling}

Neural impulses begin their life at the axon hillock, the transition between the soma and the axon.
At resting state, an electrical potential exists across the axonal membrane, between the extra- and intraaxonal spaces.
This membrane potential is maintained by active ion pumps \note{wiki says NaK pumps have low impact on membrane potential?} which continuously exchange two potassium for three sodium ions across the membrane.
The consequences of this unequal exchange are three-fold: two opposing chemical potential gradients are established, where the extracellular sodium and intracellular potassium concentrations, respecitively, are high, while the three-for-two exchange ratio establishes a net extracellular positive charge.
When the membrane potential at the axon hillock is depolarised to reach a threshold \note{?}, a wave of depolarisation is triggered which propagates down the axon as an action potential. \note{add more deets on axon potential: phases, repolarisation, channel mechanisms etc}
In myelinated axons, propagation of the axon potential is greatly accelerated by the insulating effects of the myelin sheaths, separate only by the narrow myelin-free nodes of Ranvier.
Instead of the axon membrane depolarising in a continuous wave, depolarisation cannot occur at the electrically insulated sections, and instead the signal appears to ``jump" from node to node, where depolarisation replenishes the signal preventing degradatio. \note{big ol simplification}
In myelinated axons, action potentials travel at speeds of \note{??} mm/s, compared to \note{??} in unmyelinated axons.

Upon reaching the axon terminal and synapse, the signal is converted from electrical to chemical.
The arrival of the action potential triggers the release of signalling molecules called neurotransmitters, which diffuse across the synaptic cleft to ligase with receptors in the receiving cell to influence the mebrane potential on the other side of the cleft.
The effect on the target neuron depends on the neurotransmitter: it may induce hyperpolarisation, in effect an inhibitory signal which decreases the likelihood of the target cell firing itself, or depolarisation, which brings the target cell closer to firing threshold.
A neuron may recieve synaptic inputs from \note{millions?} of neurons across its dedrites.
Each impulse cases a slight de- or hyperpolarisation of the membrane, which diffuses towards the soma and axon hillock.
It's the cumulative effects of all those impulses that determine whether a neuron fires, transmitting a new action potential down its axon.

\note{details to potentially add: different neurotransmitters and effects; more biochem details about ion channels and resting membrane/action potentials; nuances of signalling like firing patterns (lots of potentials usually fired in rapid succession)}

\subsection{Neuroanatomy}

\note{Structure of the brain, including different sections (hind, mid, forbrain etc.) and tissue types.
Discuss the lobes, cortical regions and functional divisions, then white matter tracts.}

On the macro scale, the human brain can be divided into \note{several} different divisions, starting with the at the most macro scale the hind-, mid- and forebrain.
The hind brain comprises the caudal brainstem (pons and medulla) and cerebellum.
The former forms a continuum between the forebrain to the spinal cord and houses the bodies vital and involuntary functions and control of the cranial nerves.

The brainstem is further divided into, moving inferior to superior, the medulla, pons and midbrain.
Posterior to the brainstem, and seated within the posterior fossa, sits the cerebellum, or "little brain",
a highly specialised and complex structure whose full significance in cerebral function is yet to be fully described, but which is conventionally known to play an integral role in automated motor and speech functions.
The cerebellum connects with the brain via three white matter bundles, the inferior, middle and superior cerebellar peduncles, of which the middle is the largest and forms the bulk of the pons. \note{?}

The forebrain consists of the diencephalon (thalamus and hypothalamus) and the telecephalon (cerebral hemispheres).
Evolutionarily speaking, the cerebral hemispheres are the youngest developments, responsible for much of the complex neural functionality we associated with the general intelligence of modern humans, as well as personality and consciousness \note{??}.

The surface of the cerebral hemispheres, or cortex, is the site of integration of neural signals and complex neural computation.
The cortical tissue is a form of grey matter, meaning it consists of dense layers of neural somae, dendrites, interneurons \note{and short range / local axonal connections (like inhibitory stuff?)}.
Since antiquity \note{??} scientists suspected that various cognitive functions, rather than being all equally distributed throughout the brain, arise from distinct locations.
While early incarnations of this idea may have featured wildly unfounded and \note{wrong} conclusions regarding both the functions that can be ascribed to specific localities, the underlying principle of functional localisation has been subsequently borne out and continuously refined.
\note{mention Brodmann?}
Historically, much of the understanding of the functions seated in a given cortical region was predicated on case studies involving injury (e.g. through trauma or stroke) to that region.
This resulted in inevitably imperfect understanding of the affected areas and functions, given that such random injuries invariably span multiple functional regions. \note{hmm}
Modern non-invasive imaging techniques \note{does this include stimulation like ntms?} have vastly accelerated our study of the brain, however, clinical case studies remain significant contributors to the field. \note{weak!}
Some cortical regions have been studied and mapped extensively, particularly those linked with easy to measure quantities like external stimuli and motor outputs.
Consequenty, the organisation of the visual, motor, and to lesser extent \note{based on what?} auditory, somatosensory and \note{primitive, like basic syntax?} language cortices are well described.
Conversely, the precise localisation of high-level functions such as advanced language, planning, personality, not least because such complex multifaceted phenomena are unlikely to be associated with just a single structual-functional unit.
Indeed, modern neuroscience understands that even seemingly primitive functions, such as visual object identification, involve the integration of information and processing across multiple, often distant cortical loci, and thus there can be no understanding of any cognitive function by considering only \note{artificially} demarcated cortical areas in isolation: the brain is a network, and the unit of connection between nodes is the neural axon.

Medium (between neighbouring gyri) and long range connecting axons project from the cortex to enter the dense network of axons that lies beneath.
Myelin, the fatty substance providing electrical insulation to axons, gives this tissue consisting of densely packed axons a pale appearence, hence its name "\glsdesc{wm}".
White matter axons, rather than existing as an incoherent \note{spaghetti salad}, form \note{coherent} bundles, or tracts, comprising fibres connecting similar brain regions.
There are three categories of white matter tracts.
Projection tracts connect the cortex with other, non-cortical brain structures and the spinal cord.
They include both efferent (motor) and afferent (sensory) connections.
Association tracts connect different cortical regions within the same cerebral hemisphere, while commisural tracts form interhemispheric connections.
While an understanding of white matter tracts introduces a certain level of \note{coherence/clarity/?} to the intricate brain network, their anatomical and functional definitions are far from clear-cut.
Mirroring what we saw in the problem of cortical functional localisation, the projection tracts, which have clear physiological correlates \note{?} are relatively easy to study and have been correspondingly well defined, arguably none more so than the corticospinal tract \note{getting too detailed?}.
Meanwhile, the precise spatial delineation of and \note{ascribing of functional correlates} to the many association tracts is a monumental and woefully incomplete task.
Fibres from different tracts cross and mingle throughout the white matter, and while bundles may be tight and strongly coherent for a portion of their length, they diverge and fan out to terminate in broad swathes of cortex.
There is no perfect method for observing the full extent of a tract from one end to another.
Blunt dissection remains the gold standard for anatomical identification, but it's delicate work and disentangling the full course of one bundel of fibres from all other surrounding ones is impossible.
So it is that anatomical dissections \note{have given} rise to the identification and description of tracts which turned out to be entirely artefactual. \note{round off this soapbox about tract anatomy. Copied from notes on IFOF:}
\textit{With association tracts, one can observe a backwards and forwards between deducing possible corical terminations based on observed function, e.g. particular functional deficits induced through stimulation of the core of the tract, and theorising on possible functions based on observed cortical terminations, e.g. through blunt postmortem dissections.}

\section{MRI Physics}

Having remained in our exploration of neuroanatomy at the microscopic scale or above, we now need to \note{journey} down to the atomic, even subatomic scale to cover the mechanisms of \gls{nmr}.
\note{describe atoms and quantum fun stuff}

\subsection{Nuclei and spin}

The nucleus of a hydrogen atom, in its most abundant isotope (Protium), is a single, positively charged proton.
\note{something about x nuclei, but we will only consider single protons}
While magnetic resonance can be \note{activated} in any nucleus with non-zero spin,

Atomic nuclei are made up of protons and neutrons (collectively termed nucleons) held together, dispite the electrostatic forces repelling the positively charged protons, by the nuclear force, or residual strong force.
The number of protons, or atomic number, in a nucleus determines the element: Every Hydrogen atom has exactly one proton, every Oxygen atom exactly eight, and so on.
A given element can have a variable number of neutrons, with the total number of nucleons determining the \textit{isotope} of that atom.
The particular combination of nucleons determines that nucleus' spin and quantum magnetic moment, proterties which in turn determine the effects on that nucleus brought by an external magnetic field.

Spin (or spin angular momentum) $S$ is an intrinsic property of elemental particles, and of their composites, including nuclei.
Each particle has a fixed, dimensionless spin quantum number $s$, which is $\frac{1}{2}$ for single electrons, protons and neutrons.
Nuclei, too, possess a unique spin $I$, arising non-trivially from the interactions of the individual spins of its constituent nucleons.
Spin is a vector quantity.
It's magnitude is given by $S = \hslash \sqrt{s(s+1)}$, however, according to the uncertainty principle, the total spin angular momentum vector (magnitude \textit{and} direction) cannot be measured.
Instead, the associated observable quantity is the component of $S$ along the observed direction, typically chosen to be the $z$-axis, which is determined by the spin magnetic quantum number $m_s$: $S_z = \hslash m_s$.

$m_s$ is a quantised property, meaning it can inhabit a discrete number $2s+1$ of values ranging from $-s$ to $+s$.
Any nucleus with non-zero spin can be studied under \gls{nmr} theory and experiments, however it suffices for our purposes to consider only the simplest species:
The nucleus of a hydrogen atom, in its most abundant isotope (Protium), which is a single proton.
As a spin-$\frac{1}{2}$ particle, the proton has two possible spin states, commonly called ``spin-up" and ``spin-down".

The existence of an intrinsinc angular momentum of a particle with electrical charge (i.e. a single electron, proton or nucleus) gives rise directly to another vector quantity, a magnetic moment $\mu$.
Following directly from the spin picture, the observable component $\mu_z$ is related to the spin angular momentum: $\mu_z = \gamma S_z = \gamma m_s \hslash$.
where $\gamma$ is the \textit{gyromagnetic ratio}, a species-specific quantity derived from its charge, mass and intrinsic $g$-factor.

Under quantum mechanics, the temporally and spatially evolving wave function $\Psi(x,t)$ of a particle system is described by the famous Schrödinger equation

\begin{equation}
  \frac{\hslash^2}{2m}\frac{\delta^2 \Psi(x,t)}{\delta x^2} + V(x,t)\Psi(x,t) = i\hslash \frac{\delta \Psi(x,t)}{\delta t}
\end{equation}.

Solutions to SE can take the form of eigenfunctions described by the quantum numbers $n,l,m_l$ and $m_s$, the latter of which we are now familiar with.

Where the eigenvalues are the possible energy states of the system.

Under a magnetic field, the eignestates $m_s = \pm \frac{1}{2}$ are no longer degnerate (of equal energy), but split into different energy levels, a phenomenon called the Zeemann effect.
The ratio of states in spin up vs spin down state is a tiny fraction, but big enough to be measured.
It's given by the Bolzmann distribution, which gives the expected value of the energy states over all possible quantum superpositions:
$\frac{N_{\uparrow}}{N_{\downarrow}} = e^{\Delta E / k T}$, where $\Delta E$ is the difference in engery between the two states.
A spin could be excited from the lower energy to higher energy state if absorbing a photon of a specfic frequency: the Larmor frequency $\omega = \gamma B_0$.

\note{boldify vectors}

\note{Round off explanation and polish wave energy function bit.}

\subsection{Resonance and relaxation}

\note{Basics of resonance, magnetisation, spin-spin / spin-lattice relaxation}

MRI images are obtained by detecting the spin relaxation resonance stuff of principally hydrogen atoms (which are abundant in the body in H$_2$O molecules).

The application of a strong linear magnetic field causes all spins to align themselves with the direction of the field. This is the bulk magnetisation $M_z$, in its "relaxed" state (aligned with B0).
The ratio of spin magnetic moments aligned with the external field is increased, \note{as the field imparts energy or something causing magnetic moments to preferentially be aligned} resulting in a net magnetisation along the orientation of $B_0$, termed $M_z$.
Alignment of a magnetic moment with $B_0$ (spin up) is energetically favourable of the opposite spin down state to the tune of $\Delta E = \gamma h B_0$ \note{explain vars}.
Hence the stronger the external field, the larger the ratio of up to down spins and thus the a larger the magnitude of $M_z$.
As the magnitude of this bulk magnetisation vector underpins the strength of any subsequently sampled signals, stronger bore fields, as a general rule \note{are there exceptions?}, produce images with better \gls{snr}.
In thermal equlibrium, then, the magnetic moments within the sample are precessing at the larmor frequency and entirely out of phase with each other, resulting in no net transverse magnetisation, and with spins aligned preferentially along $B_0$, resulting in maximum longitudinal magnetisation $M_z$.

The manipulation of this net magnetisation vector away from, and subsequent relaxation back to this equilibrium state, is the basis of all signal production in \gls{mri}.
The application of a \gls{rf} pulse, that is, an oscilating magnetic field ($B_1$) precisely at the resonant larmor frequency and perpendicular to $B_0$, enacts two effects on this the bulk magnetisation.
The first is the impartion \note{??} of energy which overcomes $\Delta E$ \note{this is the bit I understand the least} to increase the number of spins in the energetically unfavourable spin-down orientation and thus eliminate the net magnetisation along $B_0$.
The second effect is to sycronise the spins' precession with the rotating \gls{rf} pulse, and by extension with each-other, resulting in a precessing net magnetisation in the transverse plane $M_{xy}$.
\note{but what about 180 degree pulses?}
As soon as the excitation field is switched off, the bulk magnetisation gradually relaxes to equilibrium as the induced effects on the individual moments revert:
interaction with other spins \note{local magnetic fields?} in the molecular environment cause gradual dephasing and consequent reduction in $M_{xy}$ (spin-spin relaxation), while the energy gained from the \gls{rf} pulse is gradually dissipated among the lattice field \note{??} as the spins revert to the stistically fabourable lower energy state in alignment with $B_0$ causing gradual recovery of $M_z$ (spin-lattice relaxation).

The recovery rate of $M_z$ and the rate of decay of $M_{xy}$ are described by the time constants $T1$ and $T2$ respectively, and each depend on the specific molecular environment and are thus tissue-dependent quantities.
It is the varying values of $T1$ and $T2$, as well as the overall differences in proton densities ($PD$) between tissues which lend contrast to MR images.

\subsection{Excitation and image acquisition}

\note{How MRI machines work: excitation pulses, effects of different relaxation weightings, examples of different common pulse sequences and their uses}

\paragraph*{Generating echos}

Immediatly after application of the \gls{rf} excitation pulse, magnetisation will rapidly return to equilbrium in a spiral pattern \note{see image}.
This rotating magnetisation will induce an oscillating current within a detection coil placed near the sample in a process called \gls{fid}.
This is the most basic form of NMR signal and, since it decays very rapidly \note{isn't useful on it's own for image formation?}
In order to generate signal useful for spatial encoding and image acquisition \note{??} we make use of spin echos and/or gradient echos.

To produce a gradient echo, a readout gradient $G_x$ \note{necessarily?} is applied across the sample causing spins to dephase at different rates according to their local magnetic field $B_0 + G_xx$.
After a time $TE/2$ \note{how long?} the gradient polarity is reversed, and with it the rates of dephasing across the sample.
The effect is that after experiencing equal amounts of time of gradient polarities, the spins find themselves momentarily realigned, forming a peak in bulk transverse magnetisation and associated signal (``echo") at the time period since excitation designated $TE$ (echo time).

Since the reversal of gradient polarity effects only the gradient coil generated fields, the dephasing caused by local inhomogeneities in the $B_0$ field cannot be reversed, and the resulting signal is T2\* weighted. \note{explain T2*}
The FID signal decay is accelerated by the effects of field inhomogeneities, but the signal is in fact not entirely destroyed, it merely becomes disorganised in a manner that can be symmetrically reversed, using the spine echo phenomenon.

Spin echos are generated not through the manipulation of graident polarity, but through a complete inversion of spin orientation. \note{?}
At time $TE/2$, an additional ``refocussing" \gls{rf} pulse introduces a 180$\deg$ rotation, effectively flipping the system in the $xy$ plane \note{how to describe this more mathematically?}.
Now, those spins that are, due to slightly higher local field strength, precessing fastest, have accumulated a certain amount of phase ``ahead" of the average.
After being flipped, they are precessing at the same speed but no lag behind the average by the same phase difference.
Thus after an additional symmetrical time span of $TE/2$, the uneven effects of field inhomogeneities have been cancelled out and the spins find themselves briefly refocussed, producing a spin echo signal peak at $TE$.

Spin echo sequences benefit from the overcoming of T2* decay to produce signals with pure T2 weighting. \note{why is this useful?}
On the other hand, the need for an additional \gls{rf} pulse introduces instrumentation delay times and corresponding loss of signal magnitude.
Gradient echo sequences can work with shorter TEs \note{true?} and \note{...thus?}

\note{this section is pretty ropey but will do for now as a scaffold}

\paragraph*{Spatial encoding}

The signal induced in a reciever coil is a combination of contributions from spins located throughout the excited sample.
The attribution of signal contributions to positions in 3D space is achieved with slice selection, phase encoding and frequency encoding.
In theory, these three encoding dimensions can apply to any permutation of the three spatial axes, but in practice, and for the purposes of this description, it's easiest and conventional to select slices along the $z$-axis, parallel with the main magnetic field. \note{the other two are often interchanged in practice}

\note{the following description is specifically 2DFT method, mention?}

Slice selection ensures that the \gls{rf} pulse only excites a specific slab within the sample, and is achieved by overlaying a magnetic field gradient along $z$, altering the larmor frequencies of the spins along $z$ according to $\omega_0 + \gamma G_zz$.
A \gls{rf} pulse with (central) frequency $\frac{1}{2\pi}\omega_0 + \gamma G_zz_c$ \note{the 2 pi thing is about converting from angular frequency?} will excite spins located only at position $z_c$.
All other spins in the sample will be unaffected by subsequent manipulation of the magnetisation vector throughout the pulse sequence.

Within the selected plain, the manipulation of two spin quantities determines 2D localisation.
The first is spin precession frequency, which can be modulated along an axis in the same manner as is used in slice selection -- a magnetic field gradient $G_{fe}$ (applied, for the purposes of this description, along $x$) imparts spatially varying frequencies according to $\omega_0 + \gamma G_{fe}x$.
The frequency encoding gradient is applied throughout signal readout. \note{explain more?}
The reciever coil digitally samples the echo signal a discreet points, with the number of samples corresponding to the image matrix size along the $x$ axis.

The second is spin phase, or a spin's instantaneous angle in the $xy$ plane relative to \note{what?}.
Before signal readout, a phase encoding gradient is applied along $y$ for a time $\Delta t_{pe}$, after which spins precessing faster have accummulated a larger phase difference $\phi_{pe}$ over those experiencing lower total field strength, to the tune of $\phi_{pe}=\gamma G_{pe}y \Delta t_{pe}$.
Since the phase encoding gradient is switched off before signal readout, those phase differences become locked in and serve to localise spins along the phase encoding direction.
A single application of the phase encoding gradient provides insufficient information for determining $y$-coordinates of recieved signal components.
The excitation--phase-encoding--echo-readout sequence is repeated with varying phase-encoding magnitudes.
The number of phase encoding steps corresponds to the image matrix size along the $y$ axis.

The raw signal read by the reciever encodes the image in dimensions of frequency and phase.
These can be represented in the 2-dimensional frame known as $k$-space, an intermediate space defined as the \gls{2dft} of the measured image.
Pulse sequence parameters encode the trajectory and resolution in which datapoints in $k$-space are measured.
At the end of the sequence, when $k$-space has been filled, an inverse \gls{2dft} is computed to reconstruct the image in the spatial domain.
\Gls{2dft} imaging, in which full 3D image volumes are acquired as a series of individually reconstructed 2D slices \note{is this true?}, remains the most common \note{and efficient?} mode of image formation in MR imaging.

\paragraph*{Pulse sequences}

MR imaging is a stunningly diverse modality.
Beyond the basics of spatial encoding and echo signal generation, there lies a whole host of techniques for generating specific contrasts, neutralising unwanted signals and artefacts and accelerating acquisition times which are outside of the scope of this introduction.
The unique combination of these \note{which} parameters which produces a specific image is described by its pulse sequence.
A pulse sequence refers to the specific timeline of \gls{rf} excitations, gradient field manipulations and signal readouts programmed to form an image.
The development of novel pulse sequences and contrasts is a perennial and active field of research which will continue to expand and improve the capabilities of \gls{mri} for years to come.
\note{this paragraph feels a bit wet. and what about image reconstruction?}

\subsection{Diffusion MRI}

\note{First describe diffusion in general but stick to tissue, i.e. restricted diffusion. Cover the different types of diffusion in different tissues in the brain, timescales etc.
Then look at the mechanism of diffusion MRI acquisition, pulse sequence and parameters e.g. b values.
Also mention common artefacts.}

Molecules travel through their environment in a stochastic pattern of thermal motion called diffusion, the basis for the contrast formed in \gls{dmri}. \note{lofty}
In living tissue, water molecules are not as free to diffuse for long distances in all directions as they would be in a pure, endless medium.
The notable exception for neuroimaging is \gls{csf}, in which, over the timescales relevant to \gls{dmri}, water diffusion is free and unrestricted. \note{true?}
Elsewhere in the brain, the presence of highly organised biochemical structures, cell membranes, and cytoskeletons, places varying degrees of restriction \note{hindering?} on the diffusion of water molecules.
In \gls{gm}, the cell membranes of somae, glial cells, dendrites and \note{short, unmyelinated axons} as well as subcellular structures and organelles all present obstacles to diffusing molecules, resulting in a hindered diffusion pattern.
However, since these barriers are arranged in a random fashion, and are to be encountered equally likely in any direction, there is no preferential direction in which diffusion is freer than in others, thus the diffusivity is isotropic.
In \gls{wm}, conversely, the highly coherent arrangement of axons poses very strong restrictions on diffusion occuring perpendicular to those axons, while diffusivity is relatively higher along the direction parallel to the bundled axon fibres.
In such an environment, diffusion is anisotropic, biased towards certain directions over others.

\paragraph*{Diffusion weighting}

Any MR pulse sequence can be modified to introduce additional sensitisation to brownian motion, or diffusion weighting, in addition to the existing T1 and T2 based constrasts.
This is achieved through the application of a pair of diffusion sensitisation graident pulses prior to echo generation and signal readout.

To illustrate the concept, we will consider the application of diffusion weighting along a single orthogonal direction, e.g. $G_x$.
After slice selection and \gls{rf} excitation, a gradient of magnitude $G_d$ is applied along $x$ for time period $\delta$ before it is reversed in polarity for subsequent $\delta$.
Consider a spin which is stationary along $x=x_1$ throught the application of these gradients.
Initially, it will gain a phase $\phi_1$ proportional to it's position along the gradient $x_1$ according to $\gamma G_d \delta x_1$.
After gradient reversal, having not changed position $x=x_2=x_1$ it will gain the opposite phase $\phi_2 = -\phi_1$, resulting in a net phase change after diffusion sensitisation of $0$.
Conversely, a spin which is net motion along $x$ will experience different gradient strengths across the two time points and will experience a net dephasing according to $\Delta\phi = \gamma G_d \delta (x_2 - x_1)$ \note{replace with integral form} and corresponding signal loss.
On subsequent signal sampling, those voxels in which diffusion was high will have have a high degree of diffusion weighting-induced dephasing and exhibit a corresponding signal dropout.
In those voxels with low diffusion, phase coherence will remain relatively intact after diffusion sensitisation and signal loss will be correspondingly minimal.

There are variations on diffusion encoding pulse sequences, with the most influencial, described by Stejskal and Tanner \note{cite} in 1965, forming the basis for most modern \gls{dmri} sequence designs.
The Stejskal-Tanner sequence is a spin-echo method, featuring two gradient pulses separated by a 180\textdegree refocussing pulse.
The spin-echo signal magnitude is then determined by

\begin{align}
    S = S_0e^{-bD}
\end{align}\label{eq:S}

where $S_0$ is the baseline signal value without diffusion weighting, $D$ is the \note{effective??} diffusion coefficient (a quantity that depends on the molecular environment) and $b$ is the gradient factor, first described in \autocite{LeBihan1986} and since commonly termed the $b$-value.
In the Stejskal-Tanner sequence, $b = \gamma^2 G_d^2 \delta^2 (\Delta-\delta/3)$, where $\Delta$ is the time separating onset of the two gradient pulses.
The choice of $b$-value is highly influential on the final image, and typically values between $600--3000 s/mm^2$ are used.
Higher $b$-values reflect stronger diffusion sensitisation, but also produce far noisier images and require more advanced scanner hardware and \note{higher field strengths?}.
Values up to 1000 are more common in routine and clinical settings, while higher values are employed in more advanced research methods.

A single, full diffusion weighted MR image only employs a diffusion weighted gradient along a single direction, with all signal samples pertaining to the diffusivity along that specific orientation.
As described above, in much of the brain, diffusion is not isotropic, but directionally biased.
In medical MR imaging, therefore, diffusion weighting is applied along multiple (at least three) directions, with each direction producing a whole image volume.
The series of measurments from these source images is then combined into the desired parametric map or further processed in advanced techniques.
The simplest map, used for diagnostic images, is computed from the geometric mean of three orthogonal diffusion gradient directions, commonly referred to as a trace image.
In modern acquisitions, anywhere from 12 to 100s of directions are measured, and the number of scans may be additionally multiplied by signal averages \note{repeats?} to boost \gls{snr}.
Consequently, \Gls{dmri} scan times can be exceedingly long, from several minutes for clinical quality sequences into hours for advanced research studies, prompting efforts to reduce scan time by means of accererated and parallel imaging techniques.

\paragraph*{Echo planar imaging}

Since diffusion imaging, in particular \gls{hardi}, involves the acquisition of numerous image volumes with different diffusion weightings, scan times can be particularly long.
Depending on the image resolution, number of diffusion weighted directions, and many other factors, scan times can run into the tens of minutes and even hours.
Aside from the cost and practicalities of longer scans, there is also the increased risk of motion artifacts.
So it is that the dominant \note{only?} type of pulse sequence used in \gls{dmri} is \gls{epi}.
In \gls{epi}, all phase encoding intervals are acquired after only a single \gls{rf} excitation.
\Gls{epi} can employ both gradient and spin echos and there are numerous different variations and associated contrast \note{??}, but for the purpose of this summary we will focus on the typical \gls{epi} sequence employed for diffusion weighted imaging.

\paragraph*{Artefacts}

\note{missing section: epi artefacts, T2 shine etc.}

\section{Representing spherical functions}
\label{sec:sh}
\note{move this to fibre orientation modelling section? but also relevant for tod etc. Appendix?}

The study of macroscopic \gls{wm} structure involves descriptions of orientations in 3D space.
This often requires the description of in terms of spherical functions, that is to say, mathematical representations of some quantity and its distribution on the sphere.

One common such representation, and the one employed throught this thesis are the \gls{sh} basis functions.
The origin of this set of basis function lies in quantum mechanics \note{does it?} but their use extends to the general representation of any spherical function.
The \gls{sh} functions form a complete orthonormal basis such that any \note{well behaved} function can be expressed as a linear combination of them.

\note{to complete: functions expression, (modified) real functions (used in dmri),..??}

\chapter{Review of Literature}
\label{review}

Having given a brief introduction to the principles neuroanatomy and \gls{dmri} physics and image formation, we will turn now to the modern developments in applying \gls{dmri} to qualitative and quantitative analysis of \gls{wm} organisation.
This chapter will take us through early to state-of-the art uses of \gls{dmri} as a \gls{wm} imaging tool and outstanding challenges, followed by a review of key relevant concepts in neurosurgery.

\section{Diffusion MRI analysis}\label{sec:wms}
%==================================

Though the tools for studying the structure and function of neural connectivity are numerous, \gls{dmri} remains the only available technique capable of investigating microscopic \gls{wm} structure of the entire human brain \textit{in vivo}.
% This is thanks to the \note{fibrous} organisation of \gls{wm} fibres, which group together in bundles forming a restrictive diffusion environment in which axon membranes and myelin sheaths form barriers to diffusing water molecules.
% The resulting anisotropic diffusion pattern, with preferential diffusion along the axon direction, forms the basis for \gls{dmri}-based \gls{wm} analysis.
Though the mechanism of diffusion weighted signal is well understood (see section \ref{sec:dmri}), its correct interpretation hinges on an understanding of how diffusion is affected by the microstructural environment\autocite{LeBihan1995}, or even by just those features of the microstructural environment we are interested in measuring.
It is not obvious, for example, how much of diffusion within a voxel filled with myelinated axons can be best characterised as either hindered or restricted, and how properties such as axon density and diameter might contribute to the observed signal.\autocite{Panagiotaki2012}

\gls{dmri} image \glspl{voxel} are on the order of cubic millimetres, and in such a volume are contained thousands of individual axons, with average diameters of around 1$\mu m$\autocite{Liewald2014,Lampinen2019}.
What's more, rarely are all axons within a voxel uniformly aligned with a single direction.
Throughout the brain, fibre tracts mingle and intersect, bend and fan out.\autocite{Jeurissen2013,Alexander2019}
The physical diffusion processes in such complex configurations and multicellular environments cannot be known, only approximated and modelled through a choice of assumptions and simplifications.
In this there are, of course, many contrasting approaches, developed for different applications and the directed study of specific quantities and microstructural features.
Such target properties of \gls{dmri} analysis include cellular composition of \gls{gm}, axon density and myelination, intra- and extracellular water content, and diffusivity perpendicular to axonal orientation.
Collectively, these scalar parameters are fit from a family of microstructural models known collectively as multi-compartment models, so named for their separation of the measured signal into isolated compartments representing, for example, intra-axonal or extracellular space, each with different diffusion patterns.\autocite{Panagiotaki2012,Alexander2019}
These parameters could capture virtual cytohistological information and even serve as biomarkers for disease\autocite{Alexander2008}.
For the purposes of studying macroscopic brain connectivity and the course and organisation of individual \gls{wm} fibre tracts, the key information of interest to be determined from raw \gls{dmri} is the distribution of axonal orientations within each voxel described by a \gls{fod}.

\subsection{Tissue microstructure  and fibre orientation modelling}

\note{move DTI stuff from 1.2 down to here}

The first approach for modelling fibre orientations came in the form of \gls{dti}, which remains one of the dominant diffusion models in many applications today, particularly in clinical contexts.
The diffusion process within a \gls{wm} voxel is modelled as a three dimensional Gaussian distribution, whose covariance matrix is proportional to the diffusion tensor\autocite{Basser1994,ODonnell2011} in equation (\ref{eq:dt}).
By diagonalising this matrix, which is symmetric and positive-definite, three orthogonal eigenvectors and corresponding eigenvalues $\lambda_1 \geqslant \lambda_2 \geqslant \lambda_3$ can be computed.
Rotationally invariant indices including mean diffusivity and \gls{fa} provide scalar quantities which can be interpreted in terms of underlying tissue properties, the latter being given by
\begin{equation}
  FA = \sqrt{\frac{3}{2}}\frac{\sqrt{(\lambda_1 - \langle \lambda \rangle)^2 + (\lambda_2 - \langle \lambda \rangle)^2 + (\lambda_3 - \langle \lambda \rangle)^2}}{\sqrt{\lambda_1^2 + \lambda_2^2+ \lambda_3^2}}
\end{equation}
The principal eigenvector $\bm{\lambda}_1$ is usually interpreted as the direction of fastest, or least restricted, diffusivity, although this holds only in voxels with a single population of straight, parallel axons.
If the diffusion tensor, whose eigenvectors and eigenvalues can be represented in the form of a zeppelin shaped surface of equal mean displacement, is used as a rudimentary approximation of the \gls{fod}, then $\bm{\lambda}_1$ is taken as the single peak orientation of axon fibres.
In fact, it is rare for a voxel of \gls{wm} to contain only a single uniformly oriented bundle of axons.
In so-called crossing fibre voxels, where at least two distinctly oriented fibre populations reside, the diffusion tensor provides a woefully inaccurate or misleading picture: a smaller population with a low signal contribution may be entirely unrepresented in the modelled \gls{odf} peak, or else the effect is one of averaging all contributing populations such that none of the actual underlying fibre directions are indicated by the \gls{dt} eigenvectors. \note{mention colour FA maps}

The crossing fibres problem led to \gls{dti}, at least in research imaging, gradually giving way to higher-order fibre orientation modelling techniques which aim to account for more than one, or even an arbitrary number of distinctly oriented fibre populations\autocite{Alexander2005}.
These include to an extent the multi-compartment models which can include multiple directional intra-axonal compartments, such as the popular ``ball and sticks" model\autocite{Behrens2003,Behrens2007}.
Alternatively, several approaches aim to retrieve an underlying orientation distribution with arbitrary number of peaks through transforming the raw signal \gls{odf} into a spherical distribution of diffusivity or correlates thereof.
Diffusion spectrum imaging (DSI)\autocite{Wedeen2008}, and the less data-demanding Q-ball imaging\autocite{Tuch2003,Tuch2004}, are examples of methods that reconstruct a distribution of water molecule displacement (\gls{dodf}), with peaks aligned with the (presumed) fibre orientation.
Deconvolution methods begin with the premise that a single population of parallel fibres will produce a characteristic \gls{dmri} signal profile.
Then, the observed signal $S(\theta,\phi)$ in a given voxel amounts to a convolution over the fibre orientation distribution function $F(\theta,\phi)$ with this single fibre ``response" kernel $R(\theta)$:
\begin{align}
  S(\theta,\phi) = F(\theta,\phi) \otimes R(\theta)
\end{align}\label{eq:csd}
Solving (\ref{eq:csd}) for $F(\theta,\phi)$, represented in spherical harmonics basis as given by (\ref{eq:shfun}), involves computing the inverse operation: A spherical \textit{de}convolution of the signal with the response function.
Early versions of this concept included \textcite{Anderson2005}, in which the response kernel was modelled as a diffusion tensor.
In another approach presented by \textcite{Tournier2004} $R(\theta)$ is estimated directly from the data, generally by averaging the signals from voxels with the highest diffusion anisotropy, and thus does not rely on a model of diffusion, although it does rely on the assumption that a single fibre population's response is uniform throughout the brain and for different fibre configurations.
To improve angular resolution and reduce noise sensitivity in the estimated \gls{fod}, a regularised implementation in which biophysically impossible negative $F(\theta,\phi)$ values are strongly penalised gave rise to the widely used \gls{csd} method.
\Gls{csd} can reconstruct \glspl{fod} resolving multiple crossing fibre populations with high angular resolution in a matter of seconds, from \gls{dmri} datasets with acquisition parameters achievable in routine imaging practice\autocite{Tournier2013}.
In addition to providing a means for estimating the \gls{fod} to a high degree of angular resolution, \gls{csd} also spurred exploration of new quantities relating to tissue microstructure, notably the interpretation of the \gls{fod} amplitude as measure of intra-axonal volume fraction, or \gls{afd} \autocite{Raffelt2012a}.
\Gls{afd} can be defined as a directional quantity or as a single scalar per voxel, obtained by integrating $F$ over the sphere.

By design, \gls{csd} presumes that the entire signal in a voxel can be explained by contributions from a number of fibre populations, each forming a highly restrictive environment in which diffusion is anisotropic.
Within pure and highly organised \gls{wm}, this is a reasonable assumption, however limitations become apparent outside of these areas.
As we have seen in previous sections, brain tissue covers a spectrum of cytoarchitectures and a corresponding diversity in diffusion environments.
At the typical resolutions of \gls{dmri} data, voxels may contain, in addition to axon fibres, signal contributions from \gls{csf} or \gls{gm}, which are typically characterised by isotropic and freer diffusion.
The result of such partial volume effects are highly noisy \gls{fod} estimates with spurious peaks, overestimation of \gls{fod} peak amplitudes and overestimation of \gls{afd} in affected voxels.\autocite{Jeurissen2014}
In addition, the original \gls{csd} method was designed only for data acquired with a single diffusion weighting $b$ factor (``shell"), unable to take advantage of the additional information contained in more advanced and increasingly popular multi-shell acquisitions.
To address these limitations, an extension proposed in \textcite{Jeurissen2014} and referred to as \gls{msmt} \gls{csd} includes support for multiple $b$-value shells and separation of the signal into contributions from different tissue compartments, typically \gls{wm}, \gls{gm}, and \gls{csf}, each with their own characteristic response functions.
The resulting \gls{fod} estimates have higher angular precision than \gls{ssst} \gls{csd}, with fewer noisy spurious lobes to confound downstream processing and interpretation.
Furthermore, separation of the isotropic signal contributions greatly improves the interpretation of \gls{fod} amplitude as a measure of \gls{afd}, as the amplitudes of each tissue's deconvolved \gls{odf} closely correspond to their respective tissue volume fractions\autocite{Jeurissen2014}.

\begin{SCfigure}
  \includegraphics[width=0.5\textwidth]{chapter_1/coronalwm.png}
  \caption{Coronal section of the author's white matter, imaged with \gls{csd} \gls{dec} mapping and streamline tractography. The \glspl{cst} are visible radiating from the cortex to converge in the internal capsules before descending through the anterior pons, perpendicular to the middle cerebellar peduncle fibres. Either side of the pons, the trigeminal nerves (cranial nerve V) are visible.}
  \label{fig:corwm}
\end{SCfigure}

\subsection{Streamline tractography}\label{sec:tractography}

% \note{Discuss as the mainstay of white matter bundle segmentation.
% Cover fundamentals (including what it isn't), det/prop, different fibre models, virtual dissection, the usual suspects}

Up to this point, we have only discussed the processing and analysis of \gls{dmri} data, and what it can or cannot reveal about \gls{wm} microstructure, at the level of individual image \glspl{voxel}.
However, the individual axons that form fibre tracts can traverse 100s of millimetres, and with the unique ability to measure fibre orientations in each voxel it wasn't long before this information was being exploited to reconstruct axonal connections in their entirety.
The basic principle is one of treating fibre orientations as a brain-wide vector field through which the paths of virtual neural fibres can be traced in a process called streamline tractography.
It is vital to note here that the paths of individual \textit{in vivo} axons are entirely indiscernible from \gls{dmri}.
The individual streamlines of tractography are entirely abstract mathematical objects, each a collection of vertices, which aim to capture the \textit{potential} pathways of axons consistent with the observed \glspl{fod} representing an ensemble of thousands of axons.
Tractography is an immensely powerful and useful tool, and at the same time full of flaws due to this abstract and indirect nature.
All tractography algorithms consist at their core of the following steps:

\begin{lstlisting}[language=bash,label={lst:track},frame=single]
streamlines = []
while length(streamlines) < N do
  streamline = [get_seed_vertex()]
  STOP = false;
  while not STOP
    vertex = streamline[end]
    v = get_local_direction(vertex)
    new_vertex = vertex + step_size*v
    append(streamline, new_vertex)
    STOP = evaluate_stop(new_vertex)
  append(streamlines, streamline)
\end{lstlisting}

Once a predefined number of streamlines which fulfil all selection criteria have been generated, tracking is terminated, and the resulting streamlines, each consisting of a set of vertices in 3D space, maybe further analysed or visualised as required.
Volumetric images can also be generated by computing the number of streamlines in each voxel on a predetermined grid, a technique referred to as \gls{tdi}\autocite{Calamante2010}.

\begin{figure}
  \includegraphics[width=\textwidth]{chapter_2/streamlines.png}
  \caption{Streamline tracking involves tracing trajectories through a vector field of inferred dominant fibre orientations. Two example streamlines, from the \gls{cst} (blue) and \gls{cc} (blue) are depicted in a coronal slice of the human brain. Figure reproduced from \textcite{Jeurissen2019} used under \href{https://creativecommons.org/licenses/by-nc/4.0/}{Creative Commons Attribution-NonCommercial License}.}
  \label{fig:tracking}
\end{figure}

Within the simple algorithm outlined above are a plethora of parameters and decisions which which have transformative effects on the result.
They are apparent in the undefined functions such as \verb|get_local_direction()| or \verb|get_seed_vertex()| and the scalar parameters \verb|step_size| and \verb|N|.
The seed location, step length, conditions for terminating or entirely rejecting streamlines, interpolation of the surrounding vector field, number of streamlines to generate, are all choices to be made by the user (although in practice many parameters will be automatically determined or set to default values by the chosen algorithm).
There are two choices that most fundamentally affect the tractography process and which feature most heavily in discussions on its use.

First is the choice of method for representing orientation information from the underlying data.
The earliest tractography algorithms were developed almost concurrently with \gls{dti}\autocite{Mori1998,Mori1999}, with the orientation vector field constructed from the principal eigenvectors of fitted diffusion tensors.
In the \gls{fact} algorithm proposed by \textcite{Mori1999}, the local direction for each vertex is assigned from $\bm{\lambda}_1$ of the current voxel, and the stop criterion is a measure of neighbourhood fibre collinearity falling below a predefined threshold.
With only a single possible propagation direction at each location, which, as discussed above, may throughout much of the brain have little to do with any true axon orientations at that point, diffusion tensor-based tractography can only track fibre pathways with rather limited accuracy.
Streamlines may continue happily along a physiologically plausible path until encountering a region of intersecting tracts, at which point it may be prematurely terminated or diverted onto the trajectory of this intersecting tract if a continuation of the current path is entirely unsupported by the principal eigenvector field.

It becomes clear when considering what we learned about neural connections in section \ref{sec:hodology}, about compact fibre bundles diverging to distributed cortical targets, that a single dominant fibre direction at every point is incompatible with the dynamic organisation of \gls{wm} tracts, and the result is a tendency to reconstruct narrow and incomplete fibre bundles\autocite{Farquharson2013}.
A well cited example of this limitation can be seen in reconstructions of the \gls{cst}, which arises from the entire motor cortex from the apex down to the Sylvian fissure, but which is rendered by \gls{dt}-based algorithms only as a vertical pathway without any lateral projections.
Against this background, tractography based on higher-order fibre orientation models represents a vast improvement in the ability to contend with tracking in the complex configurations of \gls{wm}.
Now when two perpendicular tracts occupy the same voxel, the possible tracking direction is not limited to either that of the dominant bundle such that tracking the smaller one is impossible, or of an average of the two such that neither is properly represented.
However, with the flexibility of multi-peak distributions comes ambiguity, as when there are multiple distinct possible directions in which to propagate the streamline at any position, the decision of which direction to take becomes far more complex.
While tensor-based tractography may be particularly prone to false negatives, or neglecting certain pathways, multi-peak tractography can easily produce false-positives by hopping onto the paths of intersecting tracts.

The second significant distinction is between deterministic and probabilistic tracking approaches.
In deterministic tractography, there is only one single direction in which a streamline can be propagated from any given point, and two streamlines seeded in exactly the same location will be identical.
But the certainty implied by this deterministic approach is at significant odds with the reality that tractography operates in a domain and resolution far removed from that of individual axons.
Probabilistic tractography algorithms are here to acknowledge the uncertainty inherent in the tracking process.
Numerous probabilistic tracking algorithms have been developed, and while the end effect is essentially the same, whereby the next step direction is sampled from a probability distribution instead of deterministically selected, and seeding in the exact same location will not give rise to identical streamlines, there are two subtly different schools in what sort of uncertainty is being considered\autocite{Jeurissen2019}.
One considers the \textit{measurement} uncertainty of the calculated orientations.
Under this approach, the general direction to take is not under question, but the accuracy of that direction is.
It holds that due to noise and inherent limitations in our measurement equipment and signal modelling, the fibre orientations can only be calculated with limited accuracy, and the tracking directions are sampled from a distribution reflecting this measurement uncertainty.
The probabilistic algorithm probtrackx\autocite{Behrens2007}, based on the ball-and-sticks fibre orientation model\autocite{Behrens2003}, is a notable example of this approach.

A second school takes the view that uncertainty in the choice of streamline step direction stems from the obscurity of the underlying physiological reality, and sampling that direction from the fibre \gls{odf} reflects that microstructural complexity.
Crucially though, tracking is not proceeding under any guidance relating to real biophysical connections, and though a streamline's direction in a given voxel may well be in accordance with real axons, whether that direction is appropriate in the context of the preceding steps of the same streamline is unresolvable.
In other words, \gls{fod}-based probabilistic tractography, of which first or second-order integration over FODs (iFOD1/iFOD2)\autocite{Tournier2012,Tournier2010} are notable examples, can capture the local dispersion of fibres in high detail, but that doesn't necessarily translate to long-range accuracy.
It is possible to constrain tractography according to broad heuristics about fibre tract geometries, but inevitably such simplifications will not be globally applicable.
For example, strategies to prevent streamlines from ``hopping" onto intersecting, but not physically connected, pathways can include placing upper limits on the angle between successive steps under the expectation that most tracts will carry more or less straight on, but there are plenty of tracts in the brain with regions of high curvature, which become much harder to accurately reconstruct if the ``straight ahead" constraint is too strict.

Due to streamline tractography's locally oriented and step-by-step nature, errors and missteps accumulate rapidly with little to no opportunity to correct them, resulting in some wildly implausible streamlines.
Attempts to address this blindness to biophysical reality are at the focus of much of modern tractography research\autocite{Bastiani2017,Rheault2019,Aydogan2021}, as the consequences of these ongoing challenges to connectivity research and neurology are substantial\autocite{Schilling2019, Yang2021, Grisot2021}.

\subsection{White matter imaging part 1: Tractography}

The functional division of white matter into distinct tracts is of great consequence to neuroscience, psychology and neurology in their efforts to analyse brain structure and function, and as we will see later, identifying tracts is also of vital importance in neurosurgery.
It follows that the spatial delineation of individual tracts is a key step in many \gls{dmri} analysis pipelines.

Streamline tractography was the first, and remains the dominant answer to this task.
Though the field is wide and the specific approaches numerous, we will outline the two main frameworks through which individual tract segmentations are derived using streamline tractography.
The first, sometimes dubbed ``virtual fibre dissection", involves generating a large number (on the order of 10s of millions) of streamlines, usually covering the entire brain, followed by a selection process whereby streamlines are assigned to a tract of interest or discarded.
Streamline tracking proceeds virtually uninhibited, terminating only if a maximum length is reached or when leaving the white matter (as indicated by tissue segmentations\autocite{Smith2012} or a \gls{fod} amplitude threshold).
After tracking, one approach for selecting streamlines belonging to the target bundle is to use logical \glspl{roi}, specifying inclusion volumes which must be visited and exclusion volumes to filter unwanted tracks.
These selection and exclusion \glspl{roi} encapsulates our \textit{a priori} neuroanatomical knowledge, and the resulting bundle, comprising only those streamlines fulfilling the criteria, represents the segmented tract (Fig. \ref{fig:tg_rois}).
The streamlines may be viewed as three-dimensional objects, or further processed into volumetric streamline density maps\autocite{Calamante2010} and thresholded binary segmentations.

Alternatives to \gls{roi}-based selection are clustering methods, which classify streamlines according to their proximity or similarity to each other, or other geometrical properties.
RecoBundles\autocite{Garyfallidis2018}, White Matter Analysis \autocite{ODonnell2017, ODonnell2007}, atlas based adaptive clustering \autocite{Tunc2014}, and example-based automatic tract labelling \autocite{Yoo2015} are all examples of data driven, group-wise streamline clustering and matching approaches.
They typically rely on registration of example data or streamline atlases based on which similar streamlines are recognised in the target data and labelled accordingly.
Reference tracts can also be used to optimise seed placement according to streamline similarity metrics \autocite{Clayden2006,Clayden2009}.
Streamline clustering methods have been shown to generate more consistent and reproducible results across subjects compared to \gls{roi}-based segmentation\autocite{Sydnor2018}.
Another approach, named Classifyber, uses a learned linear classification of streamline features to label streamlines belonging to the target bundle in a new subject \autocite{Berto2021}.
In all clustering approaches, the necessary generation of whole brain tractograms in test subjects and the additional construction of example or reference tractography data present barriers to application, as well as, in some cases, long processing times and high memory requirements\autocite{Wasserthal2018}.

The whole brain approach is computationally extremely wasteful, as the vast number of streamlines generated will not even represent \textit{any} anatomically valid pathway through the brain, let alone one belonging to the tract of interest.
Furthermore, if streamlines are randomly seeded throughout the brain, then longer tracts covering a larger volume are more likely to be sampled, the tendency to continue straight along a ``path of least resistance" at diverging or crossing fibres results in inordinate overrepresentation of certain pathways\autocite{Smith2013}.
All this means that, after perhaps hours of tracking and billions of streamlines created, only a handful may be included in a final bundle reconstruction.

\begin{SCfigure}
  \includegraphics[width=0.5\textwidth]{chapter_2/tg_rois_glass.png}
  \caption{\gls{wm} tracts are virtually dissected with streamline tractography and anatomically informed \glspl{roi}. In this toy example, streamlines for the \gls{cst} are seeded in the cerebral peduncles (blue ring), selected with an inclusion \gls{roi} in the posterior limb of the internal capsule (green ring). Streamlines following the paths of the \gls{cc} or cerebellar peduncles are excluded (red rings).}
  \label{fig:tg_rois}
\end{SCfigure}

The second approach may be called ``targeted tractography", an involves only seeding streamlines in a tract-specific \gls{roi} and retaining those that fulfil tract-specific selection criteria, provided as additional inclusion and exclusion regions (Fig. \ref{fig:tg_rois}), until a target number of streamlines have been selected.
A seed region can but does not necessarily have to be placed at one of the actual anatomical ends of the tract, and in some cases it makes more sense to seed from the middle of the tract and propagate bidirectionally, placing additional include regions at the ends to ensure complete coverage.
This approach does not mean that no streamlines are discarded (seeded fibres may be terminated before fulfilling all inclusion criteria, or stray into exclusion regions) but targeted seeding and selection certainly leads to a higher number of admissible streamlines being generated in far less computational time than in the whole brain approach, while discarding unwanted streamlines on the fly reduces storage requirements.
Targeted tractography is the more common approach particularly in applications where only a few or even just a single tract are relevant, such as in neurosurgery\autocite{Yang2021}. \note{the pnt approach is closer to this than clustering}

Manual placement of \glspl{roi} in both whole-brain and targeted pipelines represents a significant intellectual burden, relies on expert anatomical knowledge and can be extremely time consuming, so it is often automated by registering structural atlases and defining tracts in terms of logical relations to atlas structures, as in TractQuerier\autocite{Wassermann2016} or a similar proposed method using fuzzy logic\autocite{Delmonte2019}, and Tracula \autocite{Yendiki2011}, or pre-defined \glspl{roi}, as in XTRACT \autocite{Warrington2020}.
The former two are examples of methods that rely on comprehensive cortical parcellations, typically obtained with a software tool such as FreeSurfer\autocite{Desikan2006,FischlSalat2002} which can take many hours to run.
In many scenarios, manual \gls{roi} placement remains the default method, particularly in clinical contexts where automatic \gls{roi} registration or segmentation may fail due to pathology.
Here, not only is a good understanding of the anatomy of a tract is vital to produce high quality reconstructions, but the user will also need to understand the biases and pitfalls of their chosen \gls{fod} model and tractography algorithm to ensure proper interpretation and qualification of the results\autocite{Rheault2020,Rheault2022}.
Even to an experienced user, producing quality bundles is often time-consuming and tedious.
While modern research applications and increasingly more clinical applications almost exclusively favour probabilistic and multi-fibre \gls{odf} algorithms thanks to higher sensitivity to complex fibre configurations\autocite{Yang2021}, an inevitable trade-off is a high prevalence of false positive streamlines, representing either irrelevant or unphysical connections.
Filtering out these unwanted streamlines remains a considerable challenge\autocite{Jorgens2021}.
Attempts to reduce their creation in the first instance include injecting more anatomical priors into the tracking process, such as by modifying \glspl{fod} to favour the directions associated with the target tract\autocite{Rheault2019}, using directional \glspl{roi} for particularly tricky geometries \autocite{Chamberland2017}, or designing alternatives to the piece-wise linear tracking paradigm that aim to generate streamlines with more anatomical plausibility \autocite{Schomburg2017,Aydogan2021}.
Finally, due to a combination of the different computational methods available, and a general lack of consensus on the precise anatomical extents of many commonly reconstructed pathways, tractography suffers from notoriously low reproducibility\autocite{Schilling2021a}.

In view of these limitations, some in the field are continuing efforts to improve streamline tractography with novel tracking algorithms, finding new ways to incorporate anatomical priors and developing more powerful streamline filtering, clustering and selection strategies.
Others are looking towards white matter segmentation solutions that do not rely on tractography at the point of application, but instead produce voxel-wise tract segmentations directly from \gls{dmri} or \gls{fod} data.

\subsection{White matter imaging part 2: Direct methods}

There have been numerous works addressing the \gls{wm} tract identification task as a classic voxel-wise segmentation problem, utilising techniques including multi-label supervised clustering \autocite{Ratnarajah2014}, level-set and front propagation\textcite{Nazem-Zadeh2011, Hao2014}, and deep learning for direct segmentation from fibre orientation representations \autocite{Wasserthal2018,Li2020}.
Typically, direct methods require some number of samples with which to train a classifier, atlas, Bayesian model or neural network.

In \textcite{Hagler2009}, a fibre location and orientation atlas is created by averaging the \gls{dt} and tractography-derived information from multiple subjects and subsequently used to estimate the voxel-wise \textit{a posteriori} tract probability in a test subject.
As orientation information was encoded by averaging \gls{dt} principal eigenvectors across subjects this approach is not optimised for crossing fibre configurations.
The spatial probability was given by the averaged, normalised track density values from individual deterministic streamline tractography, although tracking biases discussed above mean that equating streamline density with likelihood of tract location is problematic\autocite{Rheault2019,Smith2013}.
\textcite{Bazin2011} also proposed a direct approach based on diffusion tensor-derived priors (``Diffusion-Oriented Tract Segmentation", or DOTS) also based on \gls{dt} modelling.
Here the atlas orientation prior consisted of a single principal direction per voxel, and comparisons with the test subject data are made using Markov random field models and neighbouring tensor connectivity.

More recent developments have made use of advances in data science techniques including deep learning segmentation models, of which TractSeg\autocite{Wasserthal2018} and Neuro4Neuro\autocite{Li2020} are notable examples, using \gls{fod} peaks and diffusion tensors as inputs, respectively.
Deep learning-based approaches have the advantage of producing highly reproducible results in short processing time, without the need for template or atlas registration.
However, drawbacks of direct, deep learning-based methods which produce binary segmentations include a lack of explainability, and a dependence on large volumes of annotated training data which are labour-intensive to produce.
This limits their flexibility: if a user requires a tract segmentation which is either anatomically different or not covered by an existing pre-trained model, then the necessary production of new training data and subsequent model training represents a high logistical and computational barrier.

Inference models trained on large volumes of healthy data may not be entirely robust to pathologies, particularly those causing significant topological changes.
In addition to healthy data, Neuro4Neuro\autocite{Li2020} was validated only in a dementia dataset, and TractSeg\autocite{Wasserthal2018} was qualitatively validated in schizophrenia and autism datasets in the original work.
TractSeg has also been qualitatively validated in a tumour dataset with mostly successful results, with more complete segmentations in cases with minimally deforming tumours.\autocite{Richards2021}
In \textcite{Moshe2022}, the authors trained their own TractSeg model, on approximately 500 datasets, to segment the \gls{cst} in brain tumour patients.
The results were more reproducible than for the compared manual method, and obtained an average dice similarity score of 0.64, almost 25\% worse than the performance in healthy data reported in the original TractSeg study (for the same tract).
The authors cite a lack of reliable and sufficient labelled training data as a reason for limiting their study to a single tract, despite the importance of other tracts in preoperative fibre mapping.


\chapter{Neurosurgery}\label{chap:neurosurgery}
%========================

% \note{This is for all the reasons for surgery, including tumour, epilepsy DBS.
% Also more detail into the types of tumours, and locations in the brain}

Many types of interventions fall under the remit of cranial and spinal neurosurgery, including inserting electrodes for \gls{dbs}, diagnostic biopsies, vascular procedures, and insertion of \gls{csf} shunts.
In all cases, precision is paramount and tools for accurate navigation form vital components of the surgical workflow.
The following review will focus on some of the most complex and invasive procedures, involving the removal of tumours and epileptogenic brain tissue.
In England between 2013--2018, oncological procedures were the third most common of all neurosurgery subspecialties comprising approximately 9\% of total, while functional neurosurgeries (including for epilepsy and deep brain stimulation) made up 8\%\autocite{Wahba2022}.
At \gls{gosh} in London, a leading paediatric centre, 10\% of neurosurgical procedures between 2018--2022 were tumour-related, while 12\% were for epilepsy\autocite{gosh2023}.

Invasive brain tumour operations are both highly complex and variable in their neurophysiology, microbiology, treatment plans, and prognosis.
Such diversity presents a significant barrier to the development of image processing methods intended for generalised use in tumour patients\autocite{Bauer2013}.
Neoplasms occur throughout the brain, with the location having unique impact on surrounding structures and associated function.
A tumour's natural history and histopathology also play a large role in determining its effects on its environment.
Malignant gliomas, a category of tumours arising from glial cells, often have complex structures, with infiltrating components and peritumoural oedema blurring the distinction between tumour and non-tumour tissue\autocite{Weller2021}.
On the other hand, many non-malignant tumours including most meningiomas and low-grade astrocytomas, are encapsulated, with clear demarcation from neighbouring brain tissues, which are displaced rather than infiltrated\autocite{Lu2004,Gerard2017}.

This project is concerned with the visualisation of cerebral white matter tracts, and therefore this review will focus on those indications and interventions in which damage to and navigation around such structures is of particular concern.
This is typically not the case for posterior fossa and suprasellar lesions, although there is growing interest in the role of the cerebellum in wider cognition and the brain functional network, and the surgical community is paying increasing attention on the effects of posterior fossa surgery on cerebellar tracts\autocite{Toescu2021,Skye2023}.
Brainstem tumours are often not candidates for surgical removal due to their eloquent location, with limited access and excessive risk to vital brainstem function.
The discussions in this section therefore apply primarily to supratentorial lesions in the cerebral hemispheres and thalamus, candidates for biopsy or resection via craniotomy.
The most common intracranial tumours are meningiomas, arising from the protective membranes surrounding the \gls{cns}, of which the majority are benign with good overall survival rates and relatively low-risk surgical treatment \autocite{Rogers2015,Spena2022}.
Far more complex and controversial are decisions surrounding the surgical treatment of gliomas, those tumours originating in glial cells including astrocytes and oligodendrocytes which are the most common malignant primary \gls{cns} neoplasms in both adults\autocite{Ostrom2015,Wanis2021} and children\autocite{Ostrom2015,Bauchet2009}.
While challenges regarding functional preservation and optimal surgical strategy are relevant to all intracranial surgeries, they are acutely highlighted within the context of glioma literature.
These tumours can embed themselves insidiously within the brain's functional architecture with devastating prognosis, facing challenging oncologists and surgeons with stark dilemmas in their bid to maximise both patient survival and quality of life.
Gliomas are classified into four \gls{who} grades, commonly split into \gls{lgg} (\gls{who} grades 1--2) and \gls{hgg} (\gls{who} grades 3--4) to reflect differences is malignancy and prognosis.
There are many subtypes based on histological and genetic characteristics which are periodically updated\autocite{Louis2021}, but this overview will focus on the broad categories of \gls{hgg} and \gls{lgg}.

\section{Extent of resection}\label{sec:eor}

Complete removal of all pathological tissue, perhaps counterintuitively, is not always the surgical objective.
Though it may in many cases be the ideal outcome from an oncological perspective, this scenario would frequently be in conflict with other equally important outcome indicators, such as the preservation of surrounding brain structures and the patient's neurological wellbeing.
Successfully balancing these consequences is a central dilemma in neurosurgical practice, with the key measure being \gls{eor}, the amount of tumour removed.
In theory, \gls{eor} is a straightforward concept, but in practice it is ill-defined and inconsistently reported, while remaining central to studies of surgical efficacy and outcomes.

Easily defined in oncology%
\footnote[2]{\gls{eor} is relevant to epilepsy surgery, although the terminology and calculations here are different. Epileptogenic centres cannot always be distinguished and measured on imaging, and functionally eloquent tissue may be the clear source of epileptic activity, and thus subject to removal.}
as either the absolute volume or relative percentage of tumour tissue removed, accurately and consistently determining \gls{eor} is very difficult.
It is often reported in terms of broad categories, the most common being biopsy, \gls{str} or partial resection (PR), near total resection, \gls{gtr} and supratotal resection\autocite{Wykes2021,Karschnia2021}.
There is also no general consensus on how these categories are defined, making comparison between studies even more difficult\autocite{Karschnia2021}.
Many studies simply give very rough percentage values as estimated visually by the operating surgeon or a radiologist based on whether or not tumour residue is visible in the resection cavity or on a postoperative scan, with limited accuracy\autocite{Sanai2008,Martino2013,Lau2018,Sezer2020}.
Over time, the definitions for \gls{eor} have evolved with the availability of techniques for measuring it, and the current accepted standard for quantifying \gls{eor} is with volumetric measurement on pre- and postoperative imaging\autocite{Rincon-Torroella2019}, but here too practices are inconsistent\autocite{Wykes2021}.
Full volumetric analysis requires accurately segmenting the entire lesion, though sometimes \gls{eor} is calculated by simply taking the diameter of the lesion on a single or multiple slices, or with approximate ellipsoid segmentation\autocite{Sanai2008,Albuquerque2021}.
% This also can't account for resected portions of the tumour being filled with fluid, or differing amounts of tissue compression caused by mass effect and postoperative brain shift.

Even manual delineation can be unreliable and inconsistent, especially for tumours with poorly defined borders and on postoperative imaging \autocite{Ertl-Wagner2009,Bo2017,Visser2019}.
Semi- or fully automatic segmentation improves reproducibility\autocite{Ertl-Wagner2009,Sezer2020} and modern algorithms are proving ever more accurate, although there are still challenges regarding computational performance and robust clinical translation\autocite{Angulakshmi2017,Wadhwa2019,Fawzi2021}.
Estimates of \gls{eor} can also be compromised by post-operative brain tissue shifting and obscuring the actual volume of resected tumour \autocite{Schucht2014a}, while microscopic tumour cell invasion means that complete resection, as viewed either on imaging or by intraoperative visual assessment, does not necessarily mean no tumour residue remains\autocite{Yordanova2017}.
Finally, while there is the most focus on reporting relative reductions in tumour volume as a percentage of original size, more recent studies have argued that absolute residual tumour volume is as, if not more relevant for determining postoperative outcomes\autocite{Ius2012,Rincon-Torroella2019,Smith2008,Karschnia2021}.

\section{Oncological and neurological outcomes: Necessarily in opposition?}

Inconsistent reporting of \gls{eor} is one factor complicating the study of its effects on clinical outcomes, even as there is widespread agreement on the importance of studying those effects\autocite{Rincon-Torroella2019,Wykes2021,Weller2021}.
% \note{defs: lgg = grades 1--2, hgg = grades 3--4, glioblastoma = grade 4 glioma}
% For many tumour types, particularly aggressive tumours such as \glspl{hgg}, subtotal resection is \note{never} curative even with adjuvant therapy.
Broadly speaking, \gls{gtr} has been shown to increase overall and progression free survival over \gls{str} across age groups in both high \autocite{Hatoum2022, Han2020, Adams2016, McCrea2015, Bloch2012, McGirt2009, Kramm2006} and low-grade \autocite{Keles2001, Pollack1995, Sanai2008} gliomas.
For \gls{lgg}, and especially in paediatric patients, \gls{gtr} has become the recommended standard of care, as complete resection leads to a lower rate of recurrence\autocite{Berger1994,Claus2005}.
In particular, maximal resection of \glspl{lgg} drastically reduces the risk of residual tumour evolving into \gls{hgg} (known as malignant transformation \autocite{Duffau2013,Hervey-Jumper2016,Rincon-Torroella2019}, though this is only a concern in adult patients, as malignant transformation in paediatric \glspl{lgg} is exceedingly rare\autocite{Collins2020}.
More recent voices have even argued for supratotal resection, beyond the margins of any abnormally enhancing areas on $T_1$-weighted and FLAIR $T_2$-weighted \gls{mri} scans, as reviewed in \textcite{deLeeuw2019}.
There is limited evidence, though controversial, to suggest that supratotal resection of \gls{who} grade 2 gliomas in adults is followed by fewer cases of malignant transformation and improved progression-free survival \autocite{Yordanova2011}.

However, due to a general lack of prospective randomisation and robust comparison with appropriately matched controls, drawing definitive conclusions from studies investigating the effects of \gls{eor} (or other surgical variables) on post-operative outcomes is contentious\autocite{deLeeuw2019,Keles2001}.
Results may be confounded by selection biases, for example, different tumour histological subtypes may lend themselves more or less easily to greater \gls{eor}, or arise more frequently in eloquent areas of the brain (which include cortex and subcortical \gls{wm} subserving language, motor, and sensory functions, as well as the thalamus, midline structures involved in memory processing, and the brain stem), where an aggressive surgical strategy is likely to be discounted\autocite{deLeeuw2019}.
Adult \glspl{lgg} tend to occur more frequently than \glspl{hgg} in highly eloquent cortical regions\autocite{Duffau2004}, indeed the control group for the supratotal \gls{lgg} study\autocite{Yordanova2011} mentioned above consisted of patients whose gliomas were located in eloquent brain areas, and who therefore underwent only \gls{gtr}.
One might therefore expect supratotal resection to be associated with worse postoperative neurological outcomes, and indeed \textcite{Rossi2019a} found higher probabilities of immediate postoperative deficits in supratotal versus total resection of \glspl{lgg}.
These were however significantly reversed at three month and one year follow-ups, and initial overall evidence suggests that neuropsychological outcomes are comparable between total and supratotal groups\autocite{Tabor2021}.

In adults with glioma, maximal safe resection, combined with adjuvant radio- and chemotherapy, has been the established standard of care for some time.
It has been less clear, however, whether the same should apply to children.
The most prevalent anatomical locations in which gliomas arise may differ between adults and children\autocite{Duffau2004}.
Thalamic gliomas, for example, are more frequent in children than in adults\autocite{Cinalli2018,Palmisciano2021,GomezVecchio2021}:
Adult gliomas are located most frequently in the hemispheres, mostly the frontal lobe, with only approximately 4--7\%\autocite{GomezVecchio2021,Larjavaara2007} situated in the thalamus, while as many as 19\% of paediatric \glspl{hgg} are thalamic\autocite{McCrea2015}.
There is also concern that oncological differences between adult and paediatric type gliomas preclude safe extrapolation of treatment plans from one patient group to the other\autocite{Jones2012,Greuter2021}.
% Adult LGG 31\% eloquent Jakola2012
% Adult mixed grade 2--3 65\% ''presumed eloquence'' GomezVecchio2021 ; 80\% eloquent adult LGG Greuter2021
In addition to paediatric tumours frequently arising in high-risk areas such as the thalamus and brain stem \autocite{Ostrom2015},  neurocognitive and functional preservation is an especially critical concern in children.
A meta-analysis published in 2022 analysed 37 articles to assess the association between \gls{eor} and survival in paediatric patients with \gls{hgg}\autocite{Hatoum2022}.
Notwithstanding the difficulties in consistently defining and reporting \gls{eor} as discussed above, the study found strong evidence for improved overall survival in \gls{gtr} over \gls{str} of gliomas located in the cerebral hemispheres, but no association between \gls{eor} and survival was observed in midline cases.
The authors emphasise that midline (thalamic and brain stem) gliomas are not often indicated for aggressive resection due to the elevated risk to critical neurological function, and the lack of observed association may stem from measurement biases, including lower sample size and the pooling of histologically distinct tumour types which may respond differently to treatment.
Moreover, no comparison was made for postoperative functional neurological outcomes, thus failing to capture the full picture of factors contributing to a decision to pursue radical surgery.

New postoperative neurological deficits occur in over one third of glioma surgeries \autocite{Zetterling2020a}, although most patients improve significantly over longer-term followup.
Unsurprisingly, higher chances of postoperative deficits were associated with higher \gls{eor} and with tumours situated in eloquent areas\autocite{Zetterling2020a}.
In \textcite{Gil-Robles2010}, authors argue for a more conservative resection margin in \gls{who} grade 2 gliomas (low-grade) to protect functional structures, although current consensus recommends total resection in \gls{lgg} wherever possible\autocite{Rincon-Torroella2019,Albuquerque2021}. %but this is not the dominant opinion
For the most malignant tumour types, even maximal resection combined with adjuvant therapy is rarely curative, and may only lead to a survival advantage of just a few months\autocite{Rincon-Torroella2019,Karschnia2023}.
Given the overall poor survival outcomes associated with aggressive gliomas, oftentimes the risks to quality of life and postoperative neurological function associated with pursuing \gls{gtr} outweigh any potential oncological benefits\autocite{Rahman2016,Tabor2021}.
In \glspl{hgg}, the justification for \gls{gtr} or even supratotal resection is weaker than in \gls{lgg}, given that it cannot secure long-term survival for affected patients.
Where radical resection carries no likely oncological benefit and is contraindicated by a high functional risk to eloquent areas, the goal of surgery may be conservative debulking of the lesion to relieve pressure on the brain and reduce current neurological symptoms.
% In certain types of tumour, particularly when a combined treatment approach of surgery and radiotherapy is taken, little evidence has been found that \gls{gtr} offers greater tumour-related (that is to say, neurological condition affected by the presence of the tumour) outcomes than subtotal resection, while increasing risk to surrounding brain tissue.
% On the other hand, certain types of tumour, in particular lower grade and less aggressive types, show a lower risk of recurrence and malignant transformation when radically resected, offering a particularly good long-term outlook.
With the widespread evidence of an oncological advantage associated with more extensive resection, physicians have increasingly advocated for \gls{gtr} as the standard treatment for \gls{lgg} and maximal safe resection for \gls{hgg} \autocite{Rincon-Torroella2019}.
But this comes with the caveat that tumours of lower malignancy are also often those found to be more operable, muddying the causal link between overall survival and extent of resection\autocite{Weller2021}.
Furthermore, post-operative neurological deficits, due to cortical and subcortical injury, themselves have a negative impact on overall survival, independently of differences in pre-operative symptoms \autocite{Trinh2013,Rahman2016,Rincon-Torroella2019}.
Hence even if only overall survival is considered as the measure of treatment success, the evidence that both neurological injury and un-resected tumour negatively impact survival highlights the dilemma of surgically treating tumours in eloquent brain regions\autocite{Rincon-Torroella2019,Duffau2004,Rahman2016}.
The European Association of Neuro-Onocology's recommendation, as of 2021, is that prevention of new neurological deficits should be prioritised over maximal extent of resection in the surgical treatment of gliomas\autocite{Weller2021}.

A further consideration on the feasibility of \gls{gtr} or supratotal resection is neuroplasticity\autocite{Duffau2005}.
Slowly growing, low-grade, or recurring tumours may lead to functional reorganisation of surrounding brain tissue\autocite{Takahashi2012,Southwell2016,Das2019} or compensatory recruitment of equivalent contralateral regions\autocite{Mitolo2022}, enabling the safe removal of a greater margin of tissue than would otherwise be accepted for eloquent areas\autocite{Rossi2019a}.
Current understanding of neuroplasticity and brain tumours is limited to a small number of case studies, and more systematic research into the mechanisms and robust detection of functional reorganisation are required before these findings can be put into routine clinical practice\autocite{Duffau2005,Abel2015,Satoer2017}.
Taken together with the emergence of a hodological framework for neurosurgery discussed in Section \ref{sec:hodology}\autocite{Sala2019}, improved study of neuroplasticity could gradually lead to wider applicability of total or supratotal resection without compromising on neurological function and postoperative quality of life.

Early neurological concepts of rigid functional localisation formed the basis for the concepts of eloquence and operability of tumours guiding neurosurgeons throughout much of recent decades.
The recent move towards a more individualised view has only been made possible through developments in imaging and functional monitoring tools, allowing clinical teams to adapt the surgical strategy to each unique brain-tumour system, rather than relying on received assumptions about functional organisation\autocite{Boerger2023}.
The next section will explore some of those advanced technologies instrumental in the planning and execution of state-of-the-art neurosurgical practice.

% Overall view: moving goal posts, as changes in classifications of glioma and new understanding of mutations and associated risk, changing priorities from topological to hodological mapping and considering neuroplasticity, selection bias, ...

\section{Surgical planning and preoperative imaging}

Tumours can interact with their surroundings in a number of ways, depending on their nature and location.
Some tumours, including some \glspl{lgg} and meningiomas, are fully encapsulated and displace surrounding brain as they grow.
This strong demarcation between tumour and healthy tissue can facilitate surgical treatment and total removal of the tumour without undue risk to functioning neural tissue, but such lesions can cause neurological impairments as surrounding structures are stretched or compressed, leading to a recommendation for surgical removal of a tumour which poses an otherwise lower oncological threat.
Others cause almost no spatial displacement of brain tissue, with cancer cells instead invading the parenchyma and blurring the boundaries between disease and healthy brain.
Infiltrating tumours pose a particular surgical challenge due to the risk of surgical injury to eloquent tissue, and may only be conservatively debulked to relieve intracranial pressure and improve the effectiveness of adjuvant radiation or chemotherapy treatment.
In order to meet the goal of safely balancing maximal resection and functional preservation, the full tumour-brain interaction must be comprehensively mapped to determine the optimal resection margin.

Preoperatively, structural and functional non-invasive imaging are used for diagnosis and, if surgery is indicated, surgical planning.
At this stage the goal is to assess the spatial and functional relationships between involved and healthy tissues, map out a safe operative corridor to access the lesion, and determine the appropriate \gls{eor} under all considerations explored in the previous section.
Structural imaging with \gls{ct} and \gls{mri} provide critical anatomical information in high spatial detail.
Multi-contrast \gls{mri} examinations, including FLAIR, $T_1$-weighted and $T_2$-weighted imaging sequences, each provide unique contrasts for visualising different aspects of a tumour, such as necrotic and infiltrating regions, which can aid in determining tumour type, what \gls{eor} to aim for, or which region of the tumour to target for biopsy.
Angiography, detailed mapping of blood vessels with \gls{ct} and specialised \gls{mri} sequences, can also be employed for determining a tumour's vasculature and identifying major vessels involved\autocite{Kashimura2008,Kim2019}.
\Gls{fmri} and navigated transcranial magnetic stimulation \autocite{WeissLucas2020} map out eloquent cortex lying in proximity to the lesion, including the motor, language, and sensory cortices, supplementing the purely structural data obtained from conventional \gls{mri} or \gls{ct}.
In epilepsy patients, \gls{eeg} may be employed to monitor epileptogenic regions\autocite{Sarco2006}.

\Gls{dmri} is playing an ever-increasingly important role for neurosurgical planning and navigation\autocite{Manan2022}, especially as focus moves from functional localisation towards viewing the brain as an interconnected network.
Tumour and \gls{wm} interactions are varied and can be difficult to distinguish on conventional contrast \gls{mri} alone.
Depending on the infiltrative nature of a lesion, \gls{wm} tracts may be displaced due to mass effect, invaded but remain functionally intact, disrupted or destroyed, or experience a combination of effects\autocite{Essayed2017,DSouza2019,Manan2023}.
\Gls{dmri} can be instrumental in differentiating these circumstances and assessing tract integrity\autocite{Field2004,Manan2023}, but care must be taken to recognise how tumour effects may disturb diffusion patterns and impact the results.
Peritumoural oedema and invading cells can lead to drastically altered diffusion signal measurements and reduction of anisotropy, complicating their interpretation and inhibiting accurate streamline tracking\autocite{Bulakbas2009,Nimsky2010,Kuhnt2013}.
Nevertheless, \gls{dti} and streamline tractography have brought dramatic improvements to neurosurgical planning, unlocking detailed visualisations of \gls{wm} tracts and their spatial relationships to the surgical target and even acting as a predictor of postoperative deficits with implications for preoperative patient counselling\autocite{Manan2022}.
This potential was recognised almost immediately, with \gls{dti} and early tractography quickly making their way into clinical practice \autocite{Lee2001,Mori2002a,Nimsky2005}.
% \paragraph*{White matter mapping}
%%%% --! The below may be picked up as plagiarism; heavily copied from HBM manuscript; check academic regulations !--
In the intervening years, research imaging has largely transitioned to the multi-fibre models and probabilistic algorithms described in Chapter \ref{chap:neuroimaging}, but in clinical practice tractography is frequently still based on \gls{dt} fibre orientation models \autocite{Toescu2020, Yang2021} and deterministic tracking algorithms.
As a notable example, popular neuronavigation platform provider Brainlab's iPlan\textregistered{} (single tensor)\autocite{Brainlab2012} and Elements (dual tensor)\autocite{Sollmann2020a} Fibre Tracking applications (Brainlab AG, Munich, Germany) use a version of the probabilistic \gls{fact} algorithm \autocite{Mori1999}.

Regardless of the particular combination of fibre model, algorithm and tracking criteria, streamline tractography is compromised by weaknesses that can lead to flawed results or interpretations if not accounted for\autocite{Rheault2020, Schilling2022, Schilling2019}.
The same techniques and associated limitations for reconstructing individual \gls{wm} bundles already described apply here too, and can even be exasperated by additional tumour-related effects.
\Gls{dt}-based tractography, already afflicted by low sensitivity in healthy applications, often encounters particular difficulties tracking through oedema and areas of infiltration even where intact and functioning fibres may persist\autocite{Leclercq2010}, leading to missed connections and dangerous blind spots in the very regions at risk during surgery, where accurate navigation is most critical \autocite{Kuhnt2013,Ashmore2020}.
Meanwhile, the high propensity for false positive streamlines typical of probabilistic algorithms can be even more difficult to manage when tumour deformations disturb normal fibre orientations and inhibit accurate placement of \glspl{roi}\autocite{Yang2021}.

Perhaps clinical translation of multi-fibre probabilistic tractography has also been muted on account of its lower ease of use and practicality.
\Gls{dt} acquisitions can have as few as six diffusion-weighted directions, resulting in much shorter scan times compared to full \gls{hardi} scans.
Deterministic tracking itself is rapid, and the placement of \glspl{roi} need not be as strict as with probabilistic tractography owing to a lower sensitivity to false positives \autocite{ODonnell2017}.
A general lack of availability of the necessary expertise and time limits neurosurgical centres' access to state-of-the-art tractography \autocite{Toescu2020}.
Until recently, commercially available neurosurgical navigation platforms have exclusively supported \gls{dt} modelling and deterministic tractography (a recent exception is the Medtronic Stealth\texttrademark{} S8 Tractography application (Medtronic, USA), which implements \gls{csd}-based tractography\autocite{Pozzilli2023} as well as \gls{dt}).
This lack of readily available alternatives in the neurosurgeon's workflow and certification for safe clinical use is undoubtedly a major factor in the persisting preference for deterministic methods in clinical practice.
Nonetheless, there is growing consensus that (pending appropriate regulatory approval) the clinical community ought to adopt probabilistic, non-\gls{dt} tractography\autocite{Yang2021, Beare2022, Petersen2017}.
There is evidence that this shift is gradually underway, at least in the context of presurgical planning\autocite{Toescu2020}, driven probably by a combination and feedback loop of growing demand and better availability and integration of advanced techniques into the clinical workflow.

\begin{SCfigure}[][htb!]
  \includegraphics[width=0.5\textwidth]{chapter_2/neuronav.png}
  \caption[Streamline tractography for neurosurgical planning and navigation]{Idealised demonstration of tractography for neurosurgical planning and navigation in a paediatric patient with a left ependymoma (orange, outlined) involved with the \gls{or} (green) and \gls{cst} (blue). (Illustrative only, not a depiction of real clinical tractography.)}
  \label{fig:nav}
\end{SCfigure}

\section{Neuronavigation and brain shift}

During the surgical procedure itself, multimodal information streams continue to guide the safest possible resection.
Functional monitoring with \gls{des} is a crucial component of neurosurgical workflows and widely considered the gold standard for localising neural function after craniotomy.
Electrical current is applied to the cortical surface at increasing strengths\autocite{Saito2015}, and where stimulation elicits a functional response or disruption, the corresponding region is deemed eloquent.
Additionally, stimulation of subcortical white matter behind the resection cavity wall can be used to indicate when surgery should be halted as underlying eloquent structures are approached\autocite{Sala2019}.
\Gls{des} can be utilised in awake or asleep paradigms.
In the former, patients are awakened after craniotomy, and perform structured cognitive tasks involving those cortical hubs that may be at risk while undergoing electrical stimulation.
It is commonly considered for \gls{lgg} treatment within the UK, and to a lesser extent for \gls{hgg}\autocite{WykesV.2017}.
Language function is perhaps the most common target for awake stimulation as well as high-level motor tasks (such as playing an instrument) and vision\autocite{Mazerand2017}.
Awake surgery is technically complex, psycho-cognitively and emotionally demanding of the patient, and not universally tolerated\autocite{Nossek2013a,Wang2019}.
In very young children, awake surgery is rarely possible except in the most cooperative and resilient patients, and with appropriate preparation\autocite{Zolotova2022}.
\Gls{des} may also be performed in asleep patients, albeit limited to assessing motor and somatosensory function, where stimulation elicited somatosensory or motor evoked potentials (SSEPs, MEPs) can be measured in the patient's sensory cortex or muscles\autocite{Stone2019} (although the effects of anaesthesia can limit the sensitivity and accuracy of this approach\autocite{Stone2019,WeissLucas2020}).
There is also a considerable risk of intraoperative seizures, particularly in younger patients, which can lead to failure of awake functional mapping and potential increased risk of postoperative deficits\autocite{Nossek2013,Wang2019,Rigolo2020a}.
The neurological and oncological benefits associated with greater \gls{eor} discussed in Section \ref{sec:eor} have been achieved in large part with the help of awake functional mapping, primarily in adult patients.
It is therefore vital to develop and improve alternatives to cortical mapping to achieve maximal safe resection in all populations and especially in those patients who cannot undergo awake surgery, including some children.

Imaging and functional data acquired in preparation for surgery is not only instrumental to surgical planning, it also serves to guide the surgeon throughout the procedure, providing real-time multidimensional navigational information to supplement their live view through the surgical microscope.
Image guidance can involve simply displaying preoperative imaging and mapping in the theatre, while more advanced systems can also indicate the positions of surgical tools, or overlay imaging information on the microscope view.
This is achieved through stereotactic image guided surgery, which arrived with the introduction of frame-based systems in the later half of the 20th century, later largely giving way to frameless setups for craniotomies\autocite{Sandeman1995}, which are considered more time and cost effective\autocite{Sattur2019}.
In the former case, the patient's head is fixed within a stereotactic frame which guides the positioning of surgical tools, while in modern frameless systems the tools are tracked remotely, most commonly by an infrared camera system\autocite{Sattur2019}.
Fiducial markers, affixed either to the frame or patient, are detected on imaging and registered to the operating room coordinate system, allowing the tools' and patient's positions to be mapped and displayed on imaging in real time.
Intraoperative navigation with preoperative \gls{fmri} and \gls{dti} or tractography can improve \gls{eor} and preservation of critical cortical and subcortical function\autocite{Wu2007,Bello2008,Bello2010d}, particularly when combined with \gls{des} or awake surgery\autocite{Aibar-Duran2020}.
Where awake surgery is contraindicated or abandoned, preoperative functional mapping remains the only guidance available for higher cognitive functions, playing a crucial role in improving surgical care for patients who would not qualify for awake surgery.
\textcite{Rigolo2020a} found that preoperative \gls{fmri} guidance enabled safe resection of tumours or epileptic foci to continue after failed or incomplete \gls{des} function mapping, with no significant difference in postoperative morbidity.

Maximising \gls{eor} has further been significantly improved with the introduction of 5-aminolevulinic acid (5-ALA) guidance.
This compound is administered orally prior to surgery and is converted within cells to protoporphyrin IX (PPIX), which fluoresces when excited by short wavelength light.
Uptake of 5-ALA is highest in tumour cells due to \gls{bbb} disruption, and the metabolic pathways producing PPIX are more active in tumour cells, allowing the surgeon to distinguish them from healthy tissue under the surgical microscope.
5-ALA guided surgery results in improved \gls{eor} and higher progression free survival, while maintaining preservation of functional tissue\autocite{Coburger2019,Golub2020}, has seen widespread adoption in the surgical treatment of gliomas\autocite{Stummer2006}, and is included in UK national care guidelines\autocite{NICE2021}.

\begin{SCfigure}[][h!]
  \includesvg[height=\textheight,pretex=\small\sffamily]{chapter_2/brain_shift.svg}
  \caption[Intraoperative brain shift]{Illustration of intraoperative brain shift and its effect on neuronavigation. \textbf{a.} Preoperative imaging of a right \gls{who} grade 1 epidermoid lesion patient. Image is a $T_1$ weighted structural scan overlaid with \gls{csd}-derived \gls{dec} map. White arrowhead indicates medial displacement of the \gls{cst} (coloured blue/purple). \textbf{b.} Intraoperative imaging with partially resected lesion. Brain shift has caused the \gls{cst} to relax laterally towards the craniotomy (white arrowhead). \textbf{c.} Streamline tractography reconstructions of the \gls{cst} from preoperative (red) and intraoperative (green) \gls{dmri}, with areas of overlap in yellow. Note how the red streamlines give the impression of a tract further from the resection margin.}
  \label{fig:shift}
\end{SCfigure}

The dynamic conditions of brain surgery result in the unpredictable and often substantial movement, compression and deformation of tissue referred to as brain shift.
A range of factors contribute to this phenomenon, including \gls{csf} drainage, sagging due to gravity, decompression of tissue surrounding the resection cavity, swelling, craniotomy herniation and the effects of surgical instruments\autocite{Gerard2017}.
These factors may act in competing directions and combine in complex ways, for example swelling and tumour debulking can cause brain shift towards the craniotomy, while gravity and \gls{csf} drainage may have the opposite effect\autocite{Roberts1998}.
With the magnitude and direction of brain shift being so unpredictable, ranging from 1~mm to as much as 50~mm\autocite{Gerard2017}, accounting for it with predictive modelling is very difficult\autocite{Bayer2017b}.

Brain shift can affect the neurosurgeon's perception of the shape and location of the target lesion and invalidate preoperative imaging used for navigation\autocite{Nimsky2000}.
Many neurosurgeons rely on intuition to update their mental map of the surgical site throughout the procedure.
On more advanced neuronavigational platforms which integrate preoperative imaging and intraoperative data such as \gls{des} stimulation sites, brain shift can lead to misleading and inaccurate depictions of the spatial relationships between tumour and surrounding structures.
In particular, accurate localisation on image-guided navigation systems of deep tumour margins and the functionally eloquent structures beyond is significantly compromised by brain deformation and cannot be as easily mentally compensated for by the neurosurgeon as visible surface movements\autocite{Nimsky2000}.
Numerous techniques have been proposed to address the problem of brain shift\autocite{Bayer2017b} by dynamically adjusting preoperative imaging with patient specific deformation modelling.
Some rely entirely on preoperative imaging in combination with predictive modelling to simulate deformations, others include sparse or alternative modality intraoperative data, including sparse tracking of cortical surface features\autocite{Luo2019}, optical imaging of the cortical surface\autocite{Skrinjar2002,Audette2005,Fan2017} and intraoperative \gls{us} \autocite{Letteboer2005,Reinertsen2007,Bucki2012,Machado2019}, to estimate brain shift and update preoperative imaging accordingly.
Yet \textcite{Yang2017a} found that \gls{wm} tract shift direction was independent of cortical surface shift.
Ultimately, the most accurate 3D patient anatomy information can only be obtained with full 3D structural imaging after brain shift has occurred.

% Options without intraop imaging rely heavily on computationally modelling, with either limited accuracy or high computational time

\section{Intraoperative imaging}

To mitigate the effects of brain shift on neuronavigation accuracy, new structural and functional guidance information can be acquired intraoperatively.
Once more, different modalities offer different strengths and weaknesses.
Ultrasound imaging can probe into tissue beyond the surface, is safe, and can be operated at the surgical table without needing to move the patient\autocite{Elmesallamy2019,Eljamel2016}.
Doppler ultrasound is particularly useful for detecting intra- and peritumoural vasculature\autocite{Steno2016}, although image quality is limited and can be difficult to compare with other imaging modalities such as \gls{mri}\autocite{Eljamel2016}.
Specialised \gls{ct} systems can also be utilised intraoperatively \autocite{Bayer2018}, but they cause additional patient exposure to ionising radiation which is to be avoided wherever possible.

\Gls{imri} is becoming an increasingly common and valued addition to neurosurgical set-ups.
This includes low field ($<1$T) open bore systems which can be installed in the operating room, and allow for easy transfer of the patient into the scanner\autocite{Steinmeier1998,Senft2010}, as well as full high-field ($1.5-3$T) systems which acquire far higher quality images with potentially more clinical utility\autocite{Makary2011} at the expense of practicality, as interrupting surgery for an extended scan session and safely transferring a patient from the operating table to inside the scanner bore in an adjoining room is a substantial logistical and medical undertaking\autocite{Senft2010,Giordano2016a,Sattur2019}.
Technical challenges notwithstanding, \gls{imri} is incredibly valuable for determining surgical margins and providing guidance after the effects of brain shift deformations have invalidated preoperative imaging.

Numerous works demonstrate the benefits of \gls{imri} for improving postsurgical outcomes in both tumour and epilepsy surgery, including greater \gls{eor}, fewer new postoperative deficits, greater postoperative seizure freedom, and reduced length of hospital stay in both adults and children
\autocite{Shah2012,Zhang2015a,Sacino2016,Rao2017c,Giordano2017,Lu2018a,Garzon-Muvdi2019,Leroy2019,Karsy2019,Golub2020,Hlavac2020,Englman2021}.
%Giordano2017: safe in children; positioning
Even with advanced image and surgical guidance, postoperative \gls{mri} may show tumour nodules unintentionally left in the brain, concealed in corners of the resection cavity.
In some cases early repeat surgery is required to remove residual disease, resulting in significant additional clinical burden to the patient, longer hospital stays, heightened risk of complications including wound infection\autocite{Tenney1985,Chang2003}, delays in the planning of adjuvant treatment, and greater financial expense\autocite{Shah2012}.
Intraoperative imaging enables these remaining tumour margins to be detected and fully resected within the same surgery\autocite{Sattur2019,Hlavac2020}.
For low-grade lesions, the use of \gls{imri} could mean the difference between a potentially curative procedure and one leaving the patient at risk for recurrence and/or reoperation\autocite{Shah2012}.
Where \gls{imri} indicates no need for further resection, it can replace the need to subject patients to an additional scan immediately following surgery with all the discomfort and possible sedation, or in children, general anaesthesia, that it would entail.

While there is still some debate over the overall cost to benefit ratio of high-field \gls{imri} systems\autocite{Eljamel2016,Giordano2016a,Giussani2022}, and indeed high initial capital expense remains one of the primary barriers to their installation, cost-effectiveness analyses have indicated that upfront investments are recouped by lifetime savings associated with shorter hospital stays, and improved postoperative recovery and survival\autocite{Giordano2016a,Sacino2018}.
Cost-effectiveness can be further increased with a dual use set-up, in which an \gls{imri} system is used both for intraoperative and routine diagnostic scanning\autocite{Giordano2016a}, as is the case at \gls{gosh}.
Further challenges of implementing \gls{imri} include longer operating times, equipment compatibility, and patient positioning for both navigated surgery and scanning\autocite{Giordano2017}.
It is worth noting that the strength of evidence supporting all \gls{imri} use and cost-effectiveness is still under debate \autocite{Jenkinson2018,Garzon-Muvdi2019,Caras2020}, and interpretation of \gls{imri} studies is confounded by selection bias and a lack of randomised control trials\autocite{Kubben2011}.

% Next: skim open papers; discuss conventional contrast imaging, move to other acquisitions (dwi: infarct etc, then more advanced); compare with iop acquisition of dmri with coregistration (deformable) between preop advanced and intraop structural.
% end with technical considerations (in particular for children), costs etc.
% Note that more research needed to assess intraop diffusion mri.

The majority of \gls{imri} sequences are $T_1$-weighted (with or without injected contrast agent enhancement), $T_2$-weighted, and other conventional acquisitions with good tumour tissue contrast\autocite{Kubben2011,Coburger2019}.
In line with this finding, the majority of literature reviewed surrounding \gls{imri} focusses on evaluating \gls{eor}, and in this regard the technique has gradually established itself as a clearly beneficial and in some centres indispensable surgical aid\autocite{Garzon-Muvdi2019,Hlavac2020}.
What has been less extensively studied is the potential for \gls{imri} to provide updated advanced functional neuronavigation.
Broadly speaking, this can be approached in two ways.
The first is to use conventional \gls{imri} to dynamically adjust preoperative functional information (including \gls{fmri} and tractography) using deformable image registration and/or biomechanical brain shift modelling, the second is to directly acquire new advanced \gls{mri} sequences intraoperatively.
While the former may seem like the preferred choice, as one avoids having to acquire and process additional lengthy scans and further prolonging surgery interruption, achieving robust deformable registration in a reasonable timeframe is by no means trivial.
It is especially complicated given the significantly disturbed anatomy following craniotomy and complex brain shift, as well as the effects of air in the resection cavity not present on preoperative imaging.
Intraoperative \gls{fmri} has been reported in a very limited number of studies in \gls{dbs}\autocite{Hiss2015,Knight2015}, tumour resection\autocite{Roder2016a,Qiu2017a}, and neurovascular\autocite{Muscas2019} surgeries, but it's unlikely to find widespread use on account of practicality and an established preference for \gls{des} for cortical mapping.
By contrast, intraoperative \gls{dmri} has several potential advantages.

Use of tractography for neuronavigation is valued by many, but its accuracy diminishes with increasing brain shift, with precision most compromised during later stages of a resection, at the same time as it is potentially approaching critical subcortical \gls{wm}\autocite{Yang2019}.
Some account for this by registering preoperative tractography or \gls{dti} \gls{dec} maps to intraoperative structural \gls{mri}\autocite{Nimsky2006a,Tamura2022}, which depends on robust and accurate registration\autocite{Beare2016}.
Alternatively, intraoperative acquisition of \gls{dti} and even \gls{hardi} scans has been garnering interest.
Early work by \textcite{Nimsky2005}$^,$\autocite{Nimsky2005a} demonstrated the feasibility of intraoperative \gls{dmri} and tractography, illustrating \gls{wm} tract shifting after tumour resection.
Safe resection of tumours and epilepsy foci aided by intraoperative reconstruction of motor\autocite{Maesawa2010,Javadi2017}, language\autocite{DAndrea2016,Li2021} and visual\autocite{Daga2012,Cui2015} pathways has been confirmed in subsequent studies.
Overall, a systematic review by \citeauthor{Aylmore} (in production) found 26 articles reporting intraoperative \gls{dti} or tractography and associated outcome measures, with moderate suggestion of benefit to surgical success\autocite{Aylmore}.

There are considerable technical considerations associated with intraoperative \gls{dmri}, even more so than with routine imaging.
Duration is of course one, with minimum possible scan time a priority for interrupted surgeries.
Secondly, intraoperative \gls{dmri} can suffer from degraded image quality\autocite{Roder2019}, and distortion artefacts common in \gls{dmri} sequences utilising \gls{epi} are exasperated by the tissue--air interface at the craniotomy site\autocite{Elliott2020}, which can significantly diminish the accuracy of \gls{wm} tract localisation\autocite{Yang2022}.
Early studies suggested that intraoperative tract reconstructions may not be not reliable on their own, but useful in combination with adjuvant functional mapping such as \gls{des}\autocite{Ostry2013}.
Nearly all current implementations of intraoperative tractography are limited to commercially available \gls{dt}-based deterministic algorithms, which have known limitations particularly in application to pathology.
Nevertheless, concrete evidence of benefits to postoperative outcomes with intraoperative advanced \gls{dmri} is mounting\autocite{DAndrea2012,Cui2015,Maesawa2010}, and in combination with what we already know about conventional \gls{imri} and preoperative imaging for \gls{wm} neuronavigation, it stands to reason that intraoperative \gls{dmri} could bring significant improvements to brain surgery if some of the technical and image processing limitations can be overcome.

All neuronavigational tools are complementary, each providing distinct streams of information, and where possible, combination of techniques can increase the chances of achieving an optimal outcome.
For example, maximal safe resection with combined 5-ALA and \gls{imri} guidance is recommended by UK national clinical guidelines for surgical treatment of gliomas, which further advise considering the use of \gls{dti} and awake craniotomy to optimise safety and effectiveness\autocite{NICE2021}.
Reviews have found similar benefits to \gls{eor} from both \gls{imri} and 5-ALA guidance separately\autocite{Golub2020} and even better outcomes when combined\autocite{Nickel2018,Coburger2019}.
5-ALA provides the surgeon with a direct visualisation of tumour cells in the surgical view, but residual nodules may remain out of sight behind cavity convexities and only show up on \gls{imri}\autocite{SueroMolina2019}.
Combining awake surgery with \gls{imri} is also viable and beneficial in some patients \autocite{Motomura2017,Tuleasca2021}.
Advanced functional mapping, presurgical planning, and intraoperative monitoring, where the appropriate resources and expertise are available\autocite{GeorgeZakiGhali2020}, all increase the operability of highly eloquent gliomas
\autocite{Bello2008,Krieg2013,DellaPuppa2013b,Magill2018}.

\chapter{Summary and objectives}
\label{chap:problem}

We have seen how, along the long journey towards understanding and mapping the brain to advance neuroscience and neurology, \gls{mri} has played a pivotal role.
In particular, advanced \gls{dmri} \gls{wm} imaging with fibre orientation modelling and streamline tractography has brought us closer to comprehensively exploring the brain's structural connectivity pathways, at the same time as those pathways have achieved higher recognition as structures of interest to neurosurgical planning and navigation.
The unique challenges of the neurosurgical imaging environment, including limited resources and pathology-affected data, have so far held back effective clinical translation of the most advanced capabilities of \gls{dmri}.
These challenges are especially acute in intraoperative imaging, which has the potential to improve surgical outcomes through overcoming the destabilising effects of brain shift on image-guided surgery accuracy.
Considering the existing evidence for improved \gls{eor} with conventional contrast \gls{imri}, in combination with the proven beneficial contribution of preoperative \gls{dmri}, it is reasonable to hypothesise that the addition of intraoperative \gls{dmri} \gls{wm} imaging to the surgical workflow could significantly improve outcomes still further\autocite{Manan2022}.

Streamline tractography, as the dominant \gls{wm} imaging tool, has found widespread adoption for surgical planning while remaining error prone and easily affected by disease-specific image effects, including disturbed diffusion around tumour oedema and infiltration.
Direct voxel-based segmentation does not have the same sensitivity to error propagation and magnification inherent in the point-wise tracking process, but like tractography it still requires careful consideration of the anatomical criteria defining a tract, only these are required during a preparatory training phase, rather than at the point of application.
Data-hungry solutions including deep-learning models demonstrate impressive accuracy and speed, but often have limited generalisability without extensive retraining.
Notably, inference can fail in the presence of large tumour distortions, limiting applicability in neurosurgical practice.
Finally, a methodology dependent on large volumes of training data is arguably less than ideal for an application where the accuracy of that data may be subjective and change over time, as is the case with white matter tract anatomical definitions, which are evolving and lacking consensus.

This thesis describes the conceptualisation and implementation of a \gls{wm} tract imaging method tailored to neurosurgical application in light of all considerations laid out above.
The preceding sections reviewing the current state of the art motivate the following guiding principles and objectives for the proposed method:
Of course, applicability and robustness to clinical data and specifically brain tumour patient data featuring large spatial distortions is paramount.
Secondly, streamline tracking at the point of application is to be avoided, owing to its demands on resources and expertise and vulnerability to pathology.
Instead the rich orientation information available in \gls{dmri} acquisitions should be directly compared with \textit{a priori} tract feature knowledge to infer the tract's likely location in the target subject.
More generally we require a pipeline compatible with the specific constraints of the intraoperative environment, including an acquisition--to--result timescale of under ten minutes, and minimal to no reliance on user input.
Finally, the training data requirements should be kept to a minimum, to allow flexibility and adaptability to new tracts or modified tract definitions in an environment where obtaining and curating high quality reference data is restricted.

The remainder of this thesis is divided into two parts.
A proposed methodology is presented next, consisting of a tract specific statistical atlas expressing priors on the spatial and orientational distribution to guide inference in the target image, and a tumour deformation model for adaptability to patient-specific tumour distortions.
Subsequently, feasibility, applicability, and technical considerations are explored through a series of quantitative benchmark evaluations and case studies.
Along the way the above objectives will be continually invoked to frame and evaluate the proposed approaches.


\epart[\epigraph{I've spent my life trying to make things simpler. \\ Because I find ultimately that complicated doesn't reach the heart.}{Hans Zimmer}]{Tractfinder}
\chapter{Tract orientation atlases and mapping}
\label{chap:atlas}

The chosen approach to fulfil the aims laid out in Chapter \ref{chap:problem} combines aspects of both traditional atlas-based and direct data-driven approaches to segmentation, and is named \textit{tractfinder}.
It is clear that the task of \gls{wm} tract identification cannot be accomplished without incorporating anatomical priors, and that those priors cannot account for individual differences.
The incorporation of anatomical priors can be automated for streamline tractography, as is done for several prior works including the \gls{fsl} tool XTRACT\autocite{Warrington2020} and the white matter query language framework\autocite{Wassermann2016}.
These approaches still rely on tractography as the means for parsing the local orientation information in the test subject, and consequently inherit many associated drawbacks such as the computational time and power required for whole brain tractography, or the difficulties tracking through oedema or around tumours.
As an alternative to tractography, in which the attribution of a voxel to a tract of interest is contingent on it being visited by a streamline which, based on its entire length, has been attributed to the tract, we will instead develop a voxel-wise approach, in which a voxel can be associated with a tract based on its local diffusion properties and global coordinates.

We'll thus consider a voxel's membership with a target tract to be determined by its location in the brain and local directional diffusion characteristics, and how well they align with prior expectations for the tract.
This chapter is concerned with the encoding of those expectations in the form of a tract-specific location and orientation atlas, and the voxel-wise comparison of those expectations with observed directional diffusion information.
The necessary intermediate step, of establishing spatial correspondence between the atlas and observed data, is the subject of Chapter \ref{chap:reg}.

\section{Theoretical motivation}

The tract orientation atlas aims to capture the typical spatial and orientational distribution of a bundle across a sample of healthy subjects.
In order to motivate its format and derivation, it is worth first considering the information channels aimed to be contained within the atlas and their interpretation.

First is the spatial distribution, i.e. the likelihood of finding the tract in a given voxel.
How should we quantify this likelihood?
At the individual level, this is sometimes conflated with or loosely equated to streamline density derived from tractography, when in reality the relative densities between different portions of a streamline bundle have far more to do with the biases of tracking algorithms than the underlying anatomy.
Instead, let's imagine we pick a single \gls{voxel} in a brain scan and ask, ``are fibres of the \gls{cst} present in this voxel?".
In attempting to answer this question, we encounter several sources of uncertainty.

\textit{a) Definitional uncertainty.}
While we can more or less agree that the \gls{cst} forms the pyramids of the medulla at one end, and terminates in the primary motor cortex in the precentral gyrus at the other, its precise boundaries in the subcortical \gls{wm} are less obvious.
Within the corona radiata, for instance, where is the transition, if a clear one exists, between motor fibres and ascending sensory pathways?
This lack of highly detailed neurophysiological knowledge is our first source of uncertainty.

\textit{b) Inter-subject variability.}
Even if we could nail down the exact shape of a pathway in a hypothetical ideal brain, we would be left to contend with natural differences in brain shape and organisation.
How can we know if, in this particular individual, the transition from afferent to efferent fibres occurs slightly more posteriorly than on average?

\textit{c) Measurement uncertainty.}
Suppose we have overcome our doubts about the expected shape of the \gls{cst} in our specific individual.
Our voxel is in or near the tract's domain, but it also sits adjacent to grey matter.
The diffusion profile measured in this voxel is consistent with the presence of some fibres travelling in the expected direction, but only according to one estimate of the \gls{fod}, using \gls{ssst} \gls{csd}.
Using \gls{msmt} \gls{csd}, we obtain a different \gls{fod}, one which doesn't align with the expected \gls{cst} fibres, suggesting the data only weakly supports the presence of such fibres, and possibly only thanks to an artefact or noise.
% This measurement and modelling uncertainty, incidentally, is what is captured in the probabilistic tractography algorithm probtrackx\autocite{Behrens2007}.

Faced with this tangle of unknowns, a simplistic route was chosen.
At the individual level, we will decide upon a theoretical bundle structure that is best supported by known neuroanatomy (effectively eliminating the definitional uncertainty), and attempt to faithfully reconstruct that structure using probabilistic streamline tractography and careful manual filtering of implausible streamlines.
From this reconstruction we will derive a binary spatial segmentation, effectively assuming no measurement uncertainty for the purpose of training an atlas. % \note{disregarding all those uncertainties?}
By repeating this process and considering the differences in segmentations across a population of reference subjects, we will in doing so capture an approximation of the inter-subject variability.
% its true we cant trust the mri signal so theres some uncertainty associated with that but given that we don't have a definitive definition of where the tract is even in a perfect situation with zero measurement uncertainty, trying to disentangle the ``we don't know if this voxel contains the tract necessarily because the signal might be noisy" and ``we don't know if the tract is here because we don't actually have a specific definition of the tract's anatomy" is sort of pointless. So we might as well decide on a definition and, wherever that definition is supported by the data we have via the presence of anatomically plausible streamlines, we'll define that voxel as having the tract.

The second channel of information to be conveyed by the atlas is the orientation distribution, or the spread of likely directions along which fibres of a bundle might be aligned at a given location.
This property, too, is subject to uncertainties as described above.
However, once a tract's location is known, or at least assumed, we can in most cases be fairly certain of its local orientations based on our knowledge of its global shape and trajectory.
In a multi-fibre voxel, the distinct population in the apparent \gls{fod} which corresponds to our tract of interest is assumed to be known.
In this case, tractography which probabilistically draws samples from that distribution will reconstruct some streamlines travelling through that voxel in a distribution of directions consistent with our expectations for that tract, alongside potential spurious streamlines traversing the voxel in the ``wrong" direction.
After carefully ensuring, as much as is feasible, that only those consistent streamlines remain, we can retrieve the orientation distribution of the tract of interest in isolation from the other crossing populations using \gls{tod} mapping.
% The objective is to create a map in a template space capturing, at each location, the range of possible orientations the tract can take on as a single spherical distribution.
After aggregating this information in template space we should be left with a single spherical distribution per voxel expressing the range of possible orientations typical of the tract.
A narrow distribution may be found where the tract's orientation is highly consistent across all subjects, whereas a more spread-out distribution would reflect a wider inter-subject variability, which may be seen in regions of fanning or sharp turning.

To obtain such a mapping, a combination of streamline tractography and \gls{tod} mapping \autocite{Dhollander2014} is used.
While tractography has significant limitations as discussed previously, it remains the standard way of segmenting white matter bundles from \textit{in vivo} \gls{dmri} data, and biases and errors can, with appropriate post-processing steps, be at least partially corrected for.
In addition, tractography uniquely enables the extraction of orientation information specific to the reconstructed bundle, which would not be possible from a binary voxel-wise segmentation.
Being derived from carefully curated streamline tractography reconstructions, we can conceptualise the atlas as a store of anatomical prior expectations which would otherwise be utilised to draw appropriate \glspl{roi} for tractography in a target image.

\section{Tract definitions}

It is important that the atlases closely reflect the anatomical definitions most prevalent in the literature, and most useful for clinical applications.
To this end, an extensive literature search was conducted for each tract before confirming the \gls{roi} strategy that would in turn determine the shape of its normative atlas.
A combination of neuroanatomical studies using blunt dissection, functional studies using either non-invasive methods including \gls{fmri} or intraoperative functional stimulation were sought out as well as tractography studies detailing recommended \gls{roi} strategies.

There is of yet no perfect method for determining the exact course of individual white matter fibres in the human brain.
A technique called tract-tracing, which involves the invasive injection of tracer compounds \textit{in vivo} and subsequent post-mortem histological analysis, is widely accepted as the gold standard, but only ethically feasible in monkeys, and there is disagreement and controversy surrounding the transferability of tract definitions between no-human primates and humans.\autocite{Becker2022,ThiebautdeSchotten2012}.
Blunt dissection in post-mortem specimens, greatly advanced through the fixation methods developed by Joseph Klingler (1888--1963) in the early 20th century\autocite{Agrawal2011}, is therefore the most direct technique available for studying global white matter anatomy in humans.
Though greatly advancing the study of white matter anatomy and widely used to evaluate the findings of tractography studies, post-mortem dissections, in which layers of white matter fibres are painstakingly separated and isolated, cannot flawlessly uncover the complete and intact extent of an entire bundle\autocite{Martino2010, Dick2012}, and its findings have also been subject of rigorous debate (see, for example, \textcite{Giampiccolo2022a},\textcite{Becker2022},\textcite{Giampiccolo2022b}).
Nevertheless, post-mortem dissection and \textit{in vivo} functional evidence from intraoperative \gls{des} are preferred over conclusions drawn from purely non-invasive tractography for informing our tract definitions.
By relying wherever possible on such ``primary source" evidence, it is hoped that the circularity of basing our understanding of a tract's anatomy on tractography studies alone can be avoided. %\note{explain this better}
For example, there are several studies describing the cortical terminations of streamlines in tract reconstructions and presenting them as evidence for connectivity\autocite{Conner2018,Hau2016}, when in reality inferring the existence of a physical connection based on tractography alone is highly inadvisable\autocite{Rheault2020}.

For projection tracts, such as the optic radiation, the function and anatomy almost define each-other:
The optic radiation transmits optical signals from the thalamus to the first point of contact in the cortex, the primary visual cortex.
With association tracts, one can observe a backwards and forwards between deducing possible cortical terminations based on observed function, e.g. particular functional deficits induced through stimulation of the core of the tract, and theorising on possible functions based on observed cortical terminations, e.g. through blunt postmortem dissections.

Once an anatomical definition was decided upon based on the balance of available evidence, tractography studies detailing recommended \gls{roi} strategies were compared.
Prior studies acted as guidance for the choice of \glspl{roi} used for each tract if a clear consensus consistent with the chosen definition emerged.
Due to the difficulties in constraining tractography streamlines as they approach the cortex, grey matter \glspl{roi} derived from cortical parcellations were used wherever practical.

As is invariably necessary for probabilistic tractography, additional exclusion \glspl{roi} are needed to filter false positive streamlines that are clearly not part of the tract, and would not feasibly be included in the tract in an alternative definition.
These include, for example, midline exclusion \glspl{roi} to remove \gls{cst} streamlines straying into the corpus callosum, or a posterior \gls{roi} to exclude \gls{af} streamlines from propagating into the occipital lobe.
These exclusion \glspl{roi} will not be described, as their motivation is trivial, unless they result in reconstructions that differ notably from other anatomical definitions.
For example, some reconstructions of the \gls{cst} include parts of the afferent dorsal-column medial lemniscus system within the brainstem, whereas in tractfinder these fibres are explicitly excluded.
Where relevant, such distinctions are described below.

\subsection{Optic radiation}

\begin{SCfigure}[][htb!]
  \includegraphics[width=0.5\textwidth]{chapter_3/45_or.png}
  \caption{Schematic reconstruction of the \gls{or}, shown from an inferior axial viewpoint. Fibres originating in the \gls{lgn} of the thalamus arch anteriorly forming Meyer's loop, before returning towards the visual centres in the occipital cortex.}
  \label{fig:or}
\end{SCfigure}

As a projection tract, the \gls{or} is among the more easily definable in terms of its end points.
Arising from the \gls{lgn} of the thalamus, its fibres swoop around the lateral ventricle, some directly and some in an elaborate arc reaching as far forward as the tip of the lateral ventricle's temporal pole before curving back towards the occipital lobe in a formation named Meyer's loop (Fig. \ref{fig:or})\autocite{Sarubbo2015}, the extent and shape of which vary significantly between individuals\autocite{Ebeling1988,Yogarajah2009}.
As it projects towards the most posterior reaches of the occipital lobe, the \gls{or} forms part of the \gls{ss}, a confluence of fibres superficial to the \gls{trigone}, comprising the antero-posteriorly oriented fibres of association and projection fibres of the occipital, temporal and frontal lobes\autocite{Maldonado2021}.
The fibres of the optic radiation, conferring low-level visual data on contrast, colour and shapes, then terminate in the primary visual areas of the occipital cortex surrounding the calcarine fissure, a sulcus on the medial surface of the occipital lobe.

When it comes to the course of the optic radiations through the \gls{ss} and posterior termination in the occipital lobes, there is no disagreement between targeted tractography and other segmentation methods.
However, there remain significant differences in the regions of the \gls{lgn} and \gls{ml}.
The \gls{lgn} is a small nucleus, its localisation on \gls{mri} images not straightforward, resulting in often very generous \glspl{roi} including most of or even the entire ipsilateral thalamus.
In addition, due to the complex arrangement of white matter structures in the upper midbrain and thalamus regions, it is easy for streamlines to extend into the entire posterior thalamus and fornix and even descend into the brainstem.
This contributes to often broad \gls{or} segmentations in the thalamic portion at the start of the tract.
Secondly, the full anterior extent of Meyer's loop is often not comprehensively reconstructed by tractography, due to the extreme and tight curvature\autocite{Lilja2015,Chamberland2018}.

The \gls{or} is unsurprisingly an integral structure for visual function and thus quality of life.
With the extension of \gls{ml} deep into the temporal pole, this structure is at high risk of injury in common surgeries for treatment of epilepsy, often resulting in postoperative visual field deficits\autocite{Lacerda2020}.
Evidence of \gls{or} infiltration by tumours as evidenced by preoperative tractography has been significantly correlated with worse postoperative visual outcomes\autocite{Soumpasis2023}.
Intraoperative mapping with subcortical \gls{des} can be used to assess for visual field deficits during surgery\autocite{Duffau2004a,Mazerand2017}, although \textcite{Shahar2018} have suggested that this may be of limited value if visual responses can only be elicited once the resection has already reached a critical proximity to the tract.

To achieve the best possible representation of this important pathway, the \gls{lgn} was segmented with precision in all training subjects using the incoming optic tracts and relative positioning with the cerebral peduncles for orientation.
Using the \gls{lgn} for seeding, an inclusion waypoint was drawn around the \gls{ss} \gls{roi} on the coronal plane.
Finally, to improve the coverage of Meyer's loop, an anterior waypoint was positioned just posterior to the anterior pole of the temporal horn.
Since not all of the \gls{or} takes the full scenic route, tractography was run with and without this additional anterior waypoint and the two sets of streamlines combined.

\subsection{Corticospinal tract}

\begin{SCfigure}[][htb!]
  \includegraphics[width=0.5\textwidth]{chapter_3/45_cst.png}
  \caption{Schematic reconstruction of the \glspl{cst}, viewed coronally, showing efferent fibres originating in the primary motor cortices converging into a compact bundle in the internal capsule and descending into the spinal cord.}
  \label{fig:cst}
\end{SCfigure}

Voluntary movement is the cornerstone of life and our means of interacting with the outside world, without which motion, speech, and vision would be impossible.
It stands to reason that an outsized portion of the central nervous system is dedicated to the precise innervation of muscles throughout the body, from the fine control of vocal chords to automated touch-typing, and that such a vast repertoire of motor function is mediated by an intricate interplay of different systems including the basal ganglia circuits, cerebellum and the sensori-motor cortices of the cerebral hemispheres\autocite{Kandel2021}.
Within the primary motor cortex, situated in the precentral gyri, are the bodies of neurons whose axons project all the way into the spinal cord, where they synapse with the motor neurons of the peripheral nervous system to control muscles.
These axons, together, form a critical pathway called the corticospinal tract (Fig. \ref{fig:cst}).
The motor cortex exhibits a somatotopic organisation, with correspondence between specific cortical areas and the muscles in different regions of the body, while recent evidence suggests that this somatotopy is lost in the spinal cord as descending fibres converge and overlap\autocite{Lemon2023}.
Standardised white matter atlases and tractography protocols varyingly describe the corticospinal and pyramidal tracts.
These two terms are often used interchangeably in tractography-oriented publications, while in anatomical terms they are distinct:
The pyramidal tracts are commonly defined as encompassing both the \gls{cst} and the corticobulbar (or corticonuclear) tract, the latter of which synapses with cranial nerve nuclei of the medulla to control motion of the head, neck and face\autocite{Chenot2019}.
This despite the fact that only the \gls{cst} fibres continue to form the pyramids of the medulla, after which pyramidal tract is named.
Since the two are not practically distinguishable on \gls{dmri} from the brainstem upwards, we will hereon refer only to the corticospinal tract.% \note{reference this section}
%
% Tractography studies and related white matter segmentation research sometimes conflate the major descending (motor) and ascending (sensory pathways), whether intentionally or not.
% This is evident in two main regions.
% Firstly, the inclusion of the medial lemniscus is frequently seen in \gls{cst} segmentations (usually as it is not explicitly excluded, rather than being actively included).
% This includes TractSeg (and associated reference streamline bundles), XTRACT to some extent, and TractoInferno.
% By contrast, the tractography protocol employed in this research includes an exclusion mask on the medial lemniscus.
% Secondly, while it has been suggested that the primary motor cortex can reside in the post-central gyrus\autocite{Kumar2009}, it is generally accepted that the somatosensory cortex is located in the latter, while the motor areas are in the precentral gyri.
% However, particularly with probabilistic tractography, it is near impossible to constrain streamlines exiting the internal capsule into the fanning corona radiata to one side of the central sulcus, without additional exclusion planes or the use of cortical target regions, which are especially time-consuming to produce, whether manually or through automatic parcellation.
% Thus streamline-based \gls{cst} segmentations often contain parts of the somatosensory cortex\autocite{Poulin2021} while others, such as those utilising cortical parcellation-derived target regions, will be restricted to the motor cortex.

Classically, the \gls{cst} was thought to arise solely from the primary motor cortex (M1), with supportive areas including supplementary motor and somatosensory areas mediating voluntary movement via interaction with M1.
More recent analysis has revealed that these secondary cortices also send projections directly via the \gls{cst}\autocite{Kandel2021}, with up to 35\% of neurons originating in non-motor (primary and supplementary) areas, although these figures are based on animal studies\autocite{Welniarz2017}.
Evidence in humans for non-primary motor cortex origins of \gls{cst} fibres is limited to tractography and lesion studies\autocite{Kumar2009,Jane1967}.
However, particularly with probabilistic tractography, it is near impossible to constrain streamlines exiting the internal capsule into the fanning corona radiata to only a narrow region of cortex, and of course tractography is unable to distinguish efferent and afferent pathways.
Thus streamline-based \gls{cst} segmentations often contain parts of the somatosensory cortex whether intentionally or not\autocite{Poulin2022a}.
In another conflation of descending and ascending pathways, the inclusion of the medial lemniscus is frequently seen in \gls{cst} segmentations (usually as it is not explicitly excluded, rather than being actively included).\autocite{Wasserthal2018,Warrington2020,Poulin2022a}

The tractfinder \gls{cst} atlas streamlines were obtained using FreeSurfer parcellations \autocite{Desikan2006,FischlSalat2002} of the primary motor cortex.
To boost the coverage of the lateral precentral gyrus\autocite{Ebeling1992}, which is typically neglected by tractography due to the difficulty in crossing the centrum semiovale and its preference for taking the straightest path, two sets of streamlines were generated.
First, streamlines were seeded from the entire precentral gyrus, with include regions on the cerebral peduncles and mid-pons level, retaining the coordinates of the successful (included) seeds.
Then, a dilated mask of those coordinates was subtracted from the motor cortex mask, and the tracking was repeated from this modified low-count seed mask.
An exclusion \gls{roi} on the medial lemniscus in the mid-pons excludes afferent fibres at the brainstem level.

\subsection{Arcuate fasciculus}

% \begin{quote}
%   In dieser Windung liegt der starke hakenförmige Markbündel, welcher beide Hirnlappen verbindet und gleichsam als der eigentliche Schlüssel zur Organisation der Sylvischen Grube angesehen werden muss.
%
%   [In this curve lies the strongly hook-shaped matter bundle, which connects both brain lobes and must be regarded as the essential key to the organisation of the Sylvian fissure.] (\citeauthor{Reil1809}, \citeyear{Reil1809}\autocite{Reil1809})
% \end{quote}
% % In 1809,  wrote:

\begin{SCfigure}[][htb!]
  \includegraphics[width=0.5\textwidth]{chapter_3/45_af.png}
  \caption{Schematic reconstruction of the \gls{af}, viewed sagittally, showing fibres arcing between the posterior temporal and inferior frontal lobes.}
  \label{fig:af}
\end{SCfigure}

The \gls{af} has long held the fascination of neuroscientists, anatomists and psychologists since its earliest descriptions in the 19th century\autocite{Burdach1822}.
Ever since, the nomenclature and anatomical description of the \gls{af} has been the subject of much disagreement and confusion, with researchers variably dividing it into a series of subcomponents, disputing its cortical terminals and declaring it synonymous with either part or all of the \gls{slf}.
A comprehensive review of this jumble and its history can be read in \textcite{PortodeOliveira2021}.
Intertwined with this discussion on the white matter connections between (and within) frontal and temporal lobes are the ever-evolving models of language reception, processing and production in the human brain\autocite{Becker2022a}.
In particular, conflicts over the distinction between the \gls{slf} and \gls{af} and their respective terminations in the superior, middle or inferior temporal gyri are directly motivated by a desire to disentangle the distribution of different language associated functions in the temporal cortex and their processing by means of the dorsal and ventral language streams.\autocite{Hickok2004,Friederici2013a,Kljajevic2014a,Giampiccolo2022a,Becker2022a}

Even the core precondition for the classical definition of the \gls{af}, namely the existence of a direct connection between the inferior frontal gyrus (loosely, Broca's area) and the posterior superior temporal gyrus (Wernicke's area), has been called into question\autocite{Dick2012,Giampiccolo2022a}.
While we cannot dispute the existence of a core bundle of fibres arching around the Sylvian fissure, no-one can seem to agree on where exactly those fibres are headed, in either direction.
Within the temporal lobe, attempts to definitively disentangle the \gls{af} from other temporal lobe tracts have fallen short\autocite{Becker2022}, and with regards to the frontal terminations, there is disagreement over the relationships between the various dorsal stream components and the premotor cortex and inferior frontal gyrus.\autocite{Kljajevic2014a,Giampiccolo2022a}.

Numerous lengthy articles and thesis chapters have been devoted to this discussion, and it is beyond the scope of this short review to attempt a definitive summary.
Instead, for the purposes of this report, we will consider only the narrowest and perhaps most canonical definition of the \gls{af}:
That of a bundle connecting the posterior-superior aspect of the temporal lobe with the inferior frontal gyrus (Fig. \ref{fig:af}), with a critical role in speech production\autocite{Baldo2015}.
By this definition, we will exclude longer extensions into the anterior temporal cortex\autocite{Giampiccolo2022a} and any frontal gyri other than the inferior frontal gyrus, and reject the tripartite paradigm which includes additional subdivisions connecting to Geschwind's territory of the parietal lobe\autocite{Catani2005,Martino2013a}.
Streamlines were seeded from an \gls{roi} covering the \gls{af} on a coronal slice at the level of the central sulcus, with an additional axial inclusion \gls{roi} placed on the descending fibres, posterior to the tip of the Sylvian fissure.


\subsection{Inferior fronto-occipital fasciculus}

% \paragraph*{Anatomy and function}

\begin{SCfigure}[][htb!]
  \includegraphics[width=0.5\textwidth]{chapter_3/45_ifof.png}
  \caption{Schematic reconstruction of the \gls{ifof}, viewed sagittally, showing association fibres passing through the narrow external capsule and temporal stem to connect the occipital and frontal lobes}
  \label{fig:ifof}
\end{SCfigure}

Even compared to the arcuate fasciculus, whose canonical description, however since disputed, was at least anchored in a strongly motivated connection between two well described cortical regions and an attached clear linguistic function, study of a direct white matter fibre connection between the occipital and frontal lobes has an unsettled history\autocite{Forkel2014a}.
Thanks to a confusing mix of attempting to directly compare simian and human anatomy\autocite{Schmahmann2007,ThiebautdeSchotten2012,Mandonnet2018,Sarubbo2019}, the historical study of and extrapolation from atypically developed or lesioned brains\autocite{Schmahmann2007,Forkel2014a}, and the difficulty by any means of following long-range connections in the brain,
\autocite{Martino2010} the \gls{ifof} is an exceedingly difficult tract to pin down from available literature\autocite{Sarubbo2019,Weiller2021}.

The most widely cited functional role of the \gls{ifof} is one of semantic processing and fluency as part of the wider language network, supported by lesion studies\autocite{Ille2018b,Almairac2015} and the eliciting of semantic paraphasia by \gls{des} along its entire course,\autocite{Duffau2013a,Herbet2017,Voets2017,Vigren2020a} while the connections to the occipital lobes point more specifically to lexical and visual-semantic involvement.
\autocite{Martino2010,Rollans2017,Rollans2018}
From a surgical perspective, damage to and infiltration of the \gls{ifof} has been associated with transient or permanent semantic deficits and is commonly dislocated or infiltrated by tumours. \autocite{Almairac2015,Voets2017,Altieri2019,Binding2023}
Situated within the temporal lobe alongside the \gls{or} and the dense language association network, the \gls{ifof} is at risk of injury in temporal lobectomy surgeries common in the treatment of epilepsy. \autocite{Baran2020,Shah2022,Binding2023}
Furthermore, \textcite{Bello2010d} found preoperative reconstruction of the \gls{ifof} from \gls{dmri} to be in strong agreement with intraoperative functional mapping with \gls{des}.

The \gls{ifof} is frequently described as the key pathway underpinning a proposed ventral stream of language processing\autocite{Duffau2013a,Rollans2018,Voets2017,DavidPoeppel2012} (inspired by the dual-stream ``what" and ``where" model of visual processing), even though early works on this dual-stream framework for language make practically no mention of the occipital cortex.\autocite{Hickok2004,DavidPoeppel2012,Kummerer2013}.
Others conflate it with the ``extreme capsule fibre system"\autocite{Friederici2013a,Zhang2018} (analogous to the structure described in non-human primates\autocite{Mandonnet2018,ThiebautdeSchotten2012}), said to connect the temporal and frontal cortices\autocite{Kummerer2013},
or otherwise declare the \gls{ifof}'s origins as being, at least partly, in the temporal lobe\autocite{Bajada2015a}.
This casual blurring of functional and anatomical postulation and evidence has led to a fair degree of uncertainty surrounding the cortical, in particular the posterior,\autocite{Martino2010,Forkel2014a,Weiller2021} terminations of the \gls{ifof} and how it ties in with the wider models of cerebral language networks.
\autocite{Duffau2013a, Mandonnet2018, Rollans2018, Friederici2013a}
The cognitive functions residing in the frontal cortex are often nebulously defined and difficult to study and pin down, making the anatomical definition of pathways associated with those functions correspondingly difficult.
Beyond the commonly cited semantic functional role, researchers have also speculated, often based on at best tenuous evidence, on roles for the \gls{ifof} that include reading and writing functions associated with parts of the parietal cortex\autocite{Motomura2014}, sensorimotor processing\autocite{Martino2010} and planning\autocite{Sarubbo2013}, and involvement in goal-oriented behaviour\autocite{Conner2018}.
% tenuous because e.g. Sarubbo2013 is referring to Rizzolatti and Matelli 2003 which actually is only talking about the dorsal visual stream in monkeys? and no mention of \gls{ifof}? and Conner because the goal oriented behaviour ref is a review of \gls{dti} studies.
\textcite{Sarubbo2013}, \textcite{Martino2010}, and \textcite{Rollans2018}, considering terminations in the parietal lobe in addition to occipital regions, have all proposed possible subdivisions of the \gls{ifof} into language and non-language related functions.
% \paragraph*{Posterior terminations}

We will look for an anatomical consensus among post-mortem dissection studies aimed at uncovering the anatomical path of the \gls{ifof}, considering first the posterior and subsequently the frontal terminations.
In their detailed dissection studies, \textcite{Martino2010}, and \textcite{Sarubbo2013} all confirmed the extrastriate occipital cortex, in particular the middle occipital gyrus, as primary posterior targets, while lateral occipital lobe terminations were also studied in \textcite{Palejwala2020}.
% \textcite{Hau2016} agree as-well, however that study was based on tractography alone and therefore difficult to trust when it comes to detailed distinction of cortical terminations.
In addition to the largely agreed upon occipital regions, additional terminations have been found in the parietal and temporal cortices, in particular the superior parietal lobe, fusiform and inferior temporal gyri in \textcite{Martino2010}, and the lingual gyrus in \textcite{Sarubbo2013}.
The lingual gyrus and cuneus are cited as \gls{ifof} origination regions in \textcite{Palejwala2021}
However, it must be pointed out that the \gls{ifof} fibres have not been unambiguously distinguished from other parietal-occipital bundles, including the \gls{or} and temporal association tracts.
For example, the lingual gyrus is cited in \textcite{Sarubbo2013} as one of three ``main apparent origins" of the \gls{ifof}, albeit to a limited extent, while \textcite{Martino2010} asserted that any fibres connected to the lingual gyrus belonged to the optic radiations.
% (\textcite{Hau2016} reported tractography projections to the lingual gyrus ``in almost all subjects".)

After coursing through the temporal lobe above the roof of the temporal horn\autocite{Martino2010,Kljajevic2014a}, the \gls{ifof} converges into a narrow bundle to squeeze, alongside the \gls{uf}, through the temporal stem and into the frontal lobe (Fig. \ref{fig:ifof})\autocite{Martino2010,Sarubbo2013}.
\textcite{Martino2010} maintained that ``at the level of the frontal operculum, the fibres of the \gls{ifof} strongly intersect with the terminal branches of other long association fasciculus", making dissection extremely difficult, and indeed not possible for that particular study.
Nevertheless, \textcite{Sarubbo2013} have conducted the only dissection study to my knowledge analysing in detail of the frontal connections specifically of the \gls{ifof}.
Other studies, e.g. \textcite{Burks2017}, focus on a particular cortical region and study all connections to the region.
The disadvantage with such studies is in the aforementioned difficulty in simultaneously keeping the fibres of multiple bundles intact and traceable at once.
Connections to the inferior frontal gyrus, the seat of Broca's area, were confirmed in \textcite{Sarubbo2013} as well as \textcite{Hau2016}, and are consistent with a linguistic function.
\textcite{Sarubbo2013} also found terminations in the middle frontal gyrus (and more specifically the dorsolateral prefrontal cortex) ``in all specimens".
Finally, the orbitofrontal cortex, which is considered to play an important role in decision making, and frontal pole terminations were found in \textcite{Sarubbo2013} as well as in \textcite{Burks2017}.

Taken together, the dissection studies confirm (i.e. reported in all or most specimens) primarily posterior terminations in the superior, middle, and inferior occipital gyri and anterior terminations in the middle and inferior frontal gyri, frontal pole, and orbitofrontal cortex.
The evidence is less clear for wider connections to the parietal, posterior temporal and superior frontal cortices, so these will be excluded from the tractfinder atlas.
Streamlines were seeded in the temporal stem (extreme capsule) and selected according to cortical targets derived from the FreeSurfer (v4.5) Destrieux atlas\autocite{Destrieux2010} (2009 version) parcellation.\footnote[2]{frontal labels: 1\{1,2\}1\{01,05,15,54,12,13,14,53,63,24,65\}; occipital labels: 1\{1,2\}1\{02,19,20,58,59\}}

\section{Streamline tractography and filtering}

A dataset of \gls{hardi} acquisitions from 16 healthy adult volunteers (``EEG, fMRI and NODDI dataset"\autocite{Clayden2020}, available online at osf.io/94c5t) was used in the creation of reference bundles for developing the atlas.
\Gls{dmri} scans were acquired at 2.5~mm isotropic voxel size with three shells at $b=2400$~s~mm$^{-2}$ (60 directions), $b=800$~s~mm$^{-2}$ (30 directions), $b=300$~s~mm$^{-2}$ (9 directions), and a single $b=0$ image with no diffusion weighting.
In addition, a high resolution anatomical $T_1$-weighted scan with 1~mm isotropic voxel size was acquired in each subject.
The \gls{dmri} was preprocessed to remove noise, bias field, susceptibility distortions and motion artefacts using tools provided by the MRtrix3 (v3.0\_RC3) and FMRIB (v6.0.5) software libraries.
\Gls{fod} images where produced using multi-shell multi-tissue \gls{csd}\autocite{Jeurissen2014,Tournier2019} and response functions estimated using the Dhollander algorithm\autocite{Dhollander2016}.
Then each bundle of interest was reconstructed in both hemispheres using probabilistic streamline tractography with iFOD2 \autocite{Tournier2010} and an \gls{roi} strategy based on anatomical landmarks as described above. (See \ref{app:rois} for full details.)

After streamline generation, each streamline bundle was transformed into MNI space using affine registration implemented in the \gls{fsl} Linear Image Registration Tool\autocite{Jenkinson2002} between the subject's $T_1$-weighted image and the MNI152 $T_1$ template\autocite{Fonov2011}.
Affine registration rather than non-linear, was used for this step to capture individual anatomical variation and minimise unrealistic warping of streamlines from local registration errors or overfitting.
With all subject streamlines aggregated in MNI space, manual filtering of streamlines was performed according to the definitions previously laid out for each tract (Fig. \ref{fig:filter}), to remove not only ``volumetric false positives", which depart from the accepted volume of the tract, but also ``orientational false positives" (OFPs), which remain entirely within the tract volume but are at least in part aligned with a different, intersecting bundle.
An example of such OFPs are depicted in Figure \ref{fig:ofps}. Such streamlines have little effect on any volumetric applications of the reconstruction, e.g. via a track density depiction.
However, their removal is vital for the construction of the orientation atlas, which summarises the orientational distribution of streamlines on a voxel-wise basis.
Filtering was performed in DSI studio (v2021\_04, \url{https://dsi-studio.labsolver.org/})\autocite{Yeh2021a}, which enables the filtering of streamlines based on angle of intersection with a cutting plane.

\begin{figure}[htb!]
  \centering
  \includesvg[width=\textwidth,pretex=\small\sffamily]{chapter_3/ofps.svg}
  \caption{Orientational false positives (OFPs) are streamlines that remain within the target bundle volume, but partially follow the path of a different tract.
  \textbf{a.} Example orientational false positive streamlines, isolated from a \gls{cst} bundle, viewed from a superior vantage. Ascending streamlines following the CST abruptly turn along an anterior-posterior direction to follow the course of the intersecting superior longitudinal fasciculus, before again turning back towards the motor cortex.
  \textbf{b.} Atlas \glspl{tod} before filtering out OFPs, with green lobes that do not correspond the expected \gls{cst} orientations.
  \textbf{c.} Atlas \glspl{tod} after filtering, showing only directions corresponding to the \gls{cst}. A = anterior, P = posterior.}
  \label{fig:ofps}
\end{figure}

The percentage of streamlines filtered for each tract and summarised reasons for removal are presented in Table \ref{tab:filt}.
What is immediately apparent is the large discrepancy in the number of streamlines filtered between tracts, an indication of how much easier it is to reconstruct some tracts over others.
For the \gls{cst}, thanks to the well targeted cortical seeding and structurally highly distinct pathway, it was only necessary to remove a small percentage of streamlines, mostly OFPs in the centrum semiovale.
By contrast, approximately half of all \gls{ifof} streamlines were discarded, due thanks to the tight association of multiple tracts in the \gls{ss} resulting in overwhelming contamination from abutting structures, in particular the \gls{vof}.
The absolute number of retained streamlines is of little consequence to subsequent steps in creating the atlas, after ensuring full coverage of the tract in every training subject.

%%%%%%%%%%%%%%%%%%%%%%%%%%%%%%%%%%%%%%%%%%%%%%%%%%%%%%%%%%%%%%%%%%%%%%%%%%%%%%%%
\begin{table*}[t]
  \caption{Streamline filtering statistics. Abbreviations: \acrolist{af,crp,cc,cst,ec,slf,sfof,vof}}
  \label{tab:filt}
  \small
  \begin{tabularx}{\textwidth}{llllll X}\toprule
   &  & Original & Filtered & Difference & Reduction & Reasons for discarding \\
   \midrule
  CST & left & 148833 & 145300 & 3533 & 2.37\% & \multirow{3}{=}{Contamination from: AF / SLF, SFOF, CC, CrP} \\
   & right & 144759 & 139019 & 5740 & 3.97\% &  \\
   & total & 293592 & 284319 & 9273 & 3.16\% &  \\ \addlinespace
  AF & left & 61922 & 49778 & 12144 & 19.61\% & \multirow{3}{=}{Contamination from:  EC, CST, CC Overextension into: Motor, anterior temporal, and superior frontal cortex} \\
   & right & 61834 & 43027 & 18807 & 30.42\% &  \\
   & total & 123756 & 92805 & 30951 & 25.01\% &  \\ \addlinespace
  OR & left & 123842 & 99984 & 23858 & 19.26\% & \multirow{3}{=}{Contamination from: Tapetum of CC, SLF} \\
   & right & 122534 & 109265 & 13269 & 10.83\% &  \\
   & total & 246376 & 209249 & 37127 & 15.07\% & \\ \addlinespace
  IFOF & left & 80000 & 44224 & 35776 & 44.72\% & \multirow{3}{=}{Contamination from: Tapetum of CC, VOF, superior frontal cortex} \\
   & right & 80000 & 31753 & 48247 & 60.31\% &  \\
   & total & 160000 & 75977 & 84023 & 52.51 \% & \\ \toprule
 \end{tabularx}
\end{table*}
%%%%%%%%%%%%%%%%%%%%%%%%%%%%%%%%%%%%%%%%%%%%%%%%%%%%%%%%%%%%%%%%%%%%%%%%%%%%%%%%

\begin{figure}
  \makebox[\linewidth][c]{%
    \includegraphics[width=0.28\textwidth]{chapter_3/cst_removed.png}%
    \includegraphics[width=0.28\textwidth]{chapter_3/or_removed.png}%
    \includegraphics[width=0.28\textwidth]{chapter_3/af_removed.png}%
    \includegraphics[width=0.28\textwidth]{chapter_3/ifo_removed.png}}
  \makebox[\linewidth][c]{%
    \includegraphics[width=0.28\textwidth]{chapter_3/cst_filtered.png}%
    \includegraphics[width=0.28\textwidth]{chapter_3/or_filtered.png}%
    \includegraphics[width=0.28\textwidth]{chapter_3/af_filtered.png}%
    \includegraphics[width=0.28\textwidth]{chapter_3/ifo_filtered.png}}
  \caption{False positive streamlines filtered from each tract (top row), aggregated from all subjects, and final set of selected streamlines (bottom row). Reproduced from \textcite{Young2024}}\label{fig:filter}
\end{figure}


\section{TOD mapping}

After aggregate filtering, the retained streamlines were re-separated into individual subject bundles and the \gls{tod} was computed from the individual bundles as described in \textcite{Dhollander2014} and implemented in MRtrix3. \autocite{Tournier2019}
\Gls{tod} mapping is the generalisation of track density imaging into the angular domain, creating a 5D spatio-angular representation of streamline tracks on a voxel-wise basis.
The \gls{tod} image is represented in modified \gls{sh} basis \autocite{Descoteaux2006} using only even orders up to a maximum order $l_{max}=8$, meaning each image consists of 45 coefficients, denoted $t_j$, per voxel.
The distribution is described by those coefficients and the modified \gls{sh} basis functions $Y_{l,m}$ defined in (\ref{eq:sh}) \autocite{Descoteaux2006} as

\begin{align}
  T(\theta, \phi) = \sum_{l=0}^{l_{max}} \sum_{m=-l}^l t_{l,m} Y_{l,m}(\theta, \phi) = \sum_j t_jY_j(\theta, \phi)
\end{align}

The individual \gls{tod} images at this stage still contain significant density bias, with exaggerated differences in magnitude between the core bundle portions and fanning extremities owing to tractography's tendency towards early termination outside of the densest collinear tract regions.\autocite{Smith2013,Calamante2015,Rheault2020}.
The purpose of the atlas is to capture only the likelihood of a tract's presence in any given voxel (spatial prior) and, in the case that it is present, its expected orientation (orientational prior).
If the spatial prior is to be determined by considering the spatial variation of the tract between subjects, then the only information needed for each individual subject is a binary visitation map for the bundle and orientational data.
Thus to remove the streamline density component, the \gls{tod} maps for each subject are normalised.
The spherical integral of each \gls{sh} basis function $Y_{l,m}$ is

\begin{align}
  \int_{\Omega} Y^m_l(\theta, \phi) = \begin{cases}
   \sqrt{4\pi} & \text{ if } l=m=0\\
   0 & \text{ otherwise. }
  \end{cases}
\end{align}

Using the sum and constant rules of integration, the spherical integral of $T(\theta,\phi)$ is
\begin{align}
  \int_{\Omega} T(\theta,\phi) = t_0 \sqrt{4\pi}
\end{align}

where $t_0$ is the first \gls{sh} coefficient for $l=m=0$. Thus to remove density information the \gls{tod} map is normalised to unit integral as

\begin{align}
  \widetilde{T}(\theta, \phi) = \frac{T(\theta,\phi)}{\sqrt{4\pi} t_0}
\end{align}

After each individual \gls{tod} map has been normalised in MNI space, what remains contains only information about the tract's streamline orientations, and none about the number of streamlines passing through a given voxel in the original reconstruction.

\begin{SCfigure}
  \includesvg[width=0.5\textwidth]{chapter_3/tod_mean.svg}
  \caption{Averaging the \gls{tod} contributions produces from all subjects (smaller cutouts) produces a smooth map of voxel-wise orientation distributions (large cutout), illustrated here in the anterior point of Meyer's loop, a part of the \gls{or} with significant inter-subject variation. Adapted from \textcite{Young2024}}
  \label{fig:todmean}
\end{SCfigure}

Finally, the mean of all individual normalised \gls{tod} maps is computed to produce the final population tract \gls{tod} atlas.
Averaging all maps results in distributions that reflect all possible ranges of tract orientations in each voxel (Fig. \ref{fig:todmean}), while the first \gls{sh} coefficient of the atlas will reflect the proportion of training subjects in which the tract was present in a given voxel.
Outlier voxels visited by streamlines in only a single subject's reconstruction will contribute little weight to the final atlas.
This atlas can then be registered to a target subject for further processing.
Atlases have been created for the most commonly indicated pathways in neurosurgical planning and guidance, namely the \gls{cst}, \gls{af}, \gls{or} and \gls{ifof} (Fig. \ref{fig:atlases}).

% \section{Discussion}
%
% \note{Consider different elements of the atlas construction, e.g. what is the effect (smoothing) of using only non-linear registration and possibly put the discussion on number of training subjects here too?}

%%%%%%%%%%%%%%%%%%%%%%%%%%%%%%%%%%%%%%%%%%%%%%%%%%%%%%%%%%%%%%%%%%%%%%%%%%%%%%%%
\begin{figure*}[htb!]
  \centering
    \includegraphics[width=0.45\textwidth,align=c]{chapter_3/cst_tod.png}%
    \includegraphics[width=0.55\textwidth,align=c]{chapter_3/or_tod.png}
    \includegraphics[width=0.55\textwidth,align=c]{chapter_3/af_tod.png}%
    \includegraphics[width=0.45\textwidth,align=c]{chapter_3/ifo_tod.png}
  \caption{Tract orientation atlases, composite images projected onto coronal (CST), axial (OR, IFOF) and sagittal (AF) views in MNI152 reference space. Reproduced from \textcite{Young2024}}\label{fig:atlases}
\end{figure*}
%%%%%%%%%%%%%%%%%%%%%%%%%%%%%%%%%%%%%%%%%%%%%%%%%%%%%%%%%%%%%%%%%%%%%%%%%%%%%%%%

\section{Tract mapping}
\label{chapterlabel4}

\note{This part is about the inner product, plus any other comparison between atlas and data, including e.g. KL divergence, fixel anaylsis...}

The methodological description has been, up to this point, concerned with capturing
prior information about tracts and bringing those priors into alignment with the target anatomy.
Finally, it remains to compare those priors with the information present in the target diffusion data itself.

\subsection{Inner Product}


The orientation atlas is registered from MNI to subject space using affine registration.
The tract atlas intentionally conveys a degree of spatial tolerance to account for individual variations in tract location, with the following step acting to refine the estimate according to observed local information in the target image.
The objective is to obtain a measure per voxel of how closely the predicted tract orientation distribution overlaps with the observed FOD, modelled from dMRI data using constrained spherical deconvolution (CSD). \autocite{Tournier2007}

This can be achieved by taking the inner product of the two functions, i.e. multiplying them and integrating the product over all spherical angles.
As with the TOD atlas, the FOD is represented in the modified spherical harmonic (SH) basis:

\begin{align}
  F(\theta, \phi) = \sum_{l=0}^{l_{max}} \sum_{m=-l}^l f_{l,m} Y_{l,m}(\theta, \phi) = \sum_j f_jY_j(\theta, \phi)
\end{align}


The spherical integral of the product of two spherical harmonic basis functions $Y_{l_1,m_1}$ and $Y_{l_2,m_2}$ is

\begin{align}
  \int_0^{\pi} \int_0^{2\pi} Y_{l_1,m_1}(\theta, \phi) Y_{l_2,m_2}(\theta, \phi) sin(\theta) d\theta d\phi \\
    = \delta_{m_1, m_2} \delta_{l_1, l_2}
\end{align}


Therefore, for two functions  $F(\theta, \phi)$ and $T(\theta, \phi)$ the integral of their product can be expressed as

\begin{align}
  \begin{split}
    & \int_0^{\pi} \int_0^{2\pi} F(\theta, \phi) T(\theta, \phi) sin(\theta) d\theta d\phi \\
    = & \int_0^{\pi} \int_0^{2\pi} (\sum_j f_jY_j(\theta, \phi)) (\sum_k t_kY_k(\theta, \phi)) sin(\theta) d\theta d\phi \\
    = & \sum_{k,j} f_j t_k \delta_{jk}
  \end{split}
\end{align}


Thus for two distributions represented by a vector containing their spherical harmonic coefficients, the integrated product can be obtained by taking the inner product of the two coefficient vectors.
The final result is this voxel-wise inner product of the registered atlas and subject FOD images. The resulting image is a pseudo-probability map of tract location, in arbitrary and dimensionless units. Typical values range from [0 - 0.5], with 0.05 empirically determined to be a suitable threshold for converting to binary segmentation.
We refer to this proposed segmentation approach, of registering a pre-constructed orientation atlas to a target image and computing the inner product as described as "tractfinder".


\subsubsection{Components of final value}

The tract atlases and FOD data contain both spatial and directional information, all of which contribute to the final inner product value.
It is worth investigating, however, to what extent those two factors influence the final result.
It is conceivable for example that the final map value is predominantly determined by the overall amplitude of one or both of the ODFs, with the non-zeroth SH coefficients having little additional effect.


\subsection{Fixel analysis}

The value of the inner product between an estimated orientation distribution representing a mixture of fibre populations and the expected orientation distribution for a specific isolated population will depend on a range of interacting components, making interpretation of the final value a complicated task.

\note{missing figure: fantasy voxels comparing different atlas/fod combo scenarios (same amp, matching, crossing etc.)}

For example, consider two voxels, both containing FOD lobes of equal amplitude perfectly aligned with the expected orientation represented by the atlas TOD.
\note{I actually can't describe these scenarios without running some simulations}

To disentangle the competing effects in crossing fibre voxels, we consider an extension of the inner product analysis using the concept of fixels.
A portmanteau of ``fibre" and ``voxel", a fixel describes a sub-voxel element representing a single fibre population.
A voxel can contain any number of fixels, and each can be analysed in isolation with fixel-level properties such as orientation, dispersion and fibre density.
By segmenting a full \gls{fod} into individual lobes and computing properties on those sub-distributions, we can separate the effects of different fibre populations within a single voxel.

The fixel framework has been adapted here to extend the inner product to consider individual \gls{fod} lobes in isolation.
\gls{fod} segmentation works by sampling the \gls{sh} distribution on a dense set of directions (>1000).
Each of these direction samples is binned into a fixel based on its neighbourhood, resulting in each fixel being represented by a collection of directional samples \note{and zero everywhere else. ? better explanation of FMLS algorithm}
These fixel samples are then converted to their \gls{sh} decomposition via a linear least-squares fit \note{to what? the basis functions or amplitudes?}
The result is a set of individual \gls{sh} distributions per voxel, each representing a different fixel.

The amplitude of the fixel \glspl{odf} will reflect that of the corresponding lobe in the original full multi-fibre distribution, and thus the integral of the fixel \gls{odf} will always be smaller or equal to that of the original \gls{fod}.
However, we are interested in whether the fibre population of interest is present in a voxel based on orientation alone, regardless of its relative weight to the other fibres occupying the same voxel.
The location probability, in other words, should be influenced only by the expected location probability \note{$t_0$} and the overall white matter signal integral \note{$FOD_0$}.
\note{need to figure out these notations!!}
To remove the relative weighting effect of crossing fibres, we renormalise each fixel \gls{odf} to the overall \gls{fod} amplitude:

\begin{align}
  \overline{I} = \frac{I f_0}{i_0}
\end{align}

where the fixel \gls{odf} is $I(\theta, \phi) = \sum_j i_j Y_j(\theta,\phi)$.
Then we compute the inner product of the atlas \gls{tod} and each fixel \gls{odf} $I_k$, taking the maximum value as the result:

\begin{align}
  p = \operatorname*{argmax}_k \int_\Omega I_k(\theta,\phi) T(\theta,\phi) \sin\theta d\theta d\phi
\end{align}

The result is improved tract sensitivity in areas of crossing fibres where there is otherwise sufficient evidence that the tract is present in the voxel.
During the \gls{fod} segmentation process, a minimum threshold can be placed on the lobe amplitude for a single fixel, reducing noise amplification.

\begin{figure}
  \centering
  \begin{subfigure}[b]{0.3\textwidth}
  \centering
  \includegraphics[width=\textwidth,draft=false]{chapter_2/FCF_ground_truth_fibres.png}
  \caption{Ground truth fibres}
  \label{}
  \end{subfigure}%
  \begin{subfigure}[b]{0.3\textwidth}
  \includegraphics[width=\textwidth,draft=false]{chapter_2/normal_ip_FCF.png}
  \caption{Simple inner product}
  \label{}
  \end{subfigure}%
  \begin{subfigure}[b]{0.3\textwidth}
  \includegraphics[width=\textwidth,draft=false]{chapter_2/fixel_ip_FCF.png}
  \caption{Fixel inner product}
  \label{}
  \end{subfigure}
  \caption{Crossing fibres results in low signal for the brown tract in upper intersection region. In fixel IP this signal drop is eliminated resulting in smooth tract map.}\label{fig:fixip}
\end{figure}

\section{Atlas data requirements}\label{sec:ntrain}

In segmentation methods which ``learn" patterns from seen data to apply to unseen data, the volume and range of training data influences the prediction accuracy and generalisability.
For complex deep learning models, which have many thousands of network parameters to learn, the amount of training data required to achieve accurate and stable performance can be immense, posing a particular barrier to the use of such models in applications where suitably annotated data is scarce.
In the case of the deep learning tool TractSeg\autocite{Wasserthal2018}, for example, 105 subjects in total were used for cross-validation training, with each fully trained model having seen randomly sampled data slices from 63 unique subjects.
%https://github.com/MIC-DKFZ/TractSeg/issues/240
In tractfinder, the influence of the number of subjects used to construct each atlas is on the amount of inter-subject anatomical variation reflected in the spatial and orientational components.
It is to be expected that, save for extreme outliers, the additional information gained from adding more training subjects would reach a point of saturation.

To investigate this, an experiment was conducted whereby the number of subjects included in atlas construction was varied, and the effect on segmentation accuracy compared.
For this purpose the TractSeg \gls{hcp} bundles were used in order to enable objective evaluation against reference segmentations defined in the same manner as the training data, and direct comparison with TractSeg itself.
Using the same train - test data split as described in \textcite{Wasserthal2018b}, subsets of 1, 3, 5, 10, 15 and 30, as well as the full 63 training subjects were randomly selected, from which separate \gls{tod} atlases where constructed.
Tractfinder maps were then generated in the 42 test subjects using each of the different subset atlases and compared with the reference segmentations using the \gls{dice} and density correlation metrics.

\begin{figure}[hb!]
    \centering
    \includegraphics[width=0.8\textwidth]{compare_ntrain_.png}
    \caption{Comparison of segmentation performance using different numbers of atlas training subjects. Results are grouped by tract, colour represents number of training subjects. The IFO and \gls{or} are in places indistinguishable. \acrolist{af,cst,or}}
    \label{fig:ntrain}
\end{figure}

When using only a single subject's normalised \gls{tod} map as an ``atlas", mean \gls{dice}s ranged from 0.65 to 0.71 for the IFOF and \gls{cst} respectively (Fig. \ref{fig:ntrain}).
% These figures are from using the script compare_atlas_size.py committed at sha 33777217
The maximum increase in mean \gls{dice} between the 15 and 63 subjects atlases was 0.00835, for the \gls{cst}, representing only a 1\% increase from the lower score of 0.759.
Across all tracts and both comparison metrics, differences in performance between the different atlases were consistently negligible.
These results indicate that additional atlas subjects beyond a minimum number of around 10 to 15 do little or nothing to improve tractfinder results.
This can be interpreted as the extra training subjects offering minimal additional information on inter-subject variability, as a lot of this variability is already smoothed out due to affine (instead of diffeomorphic) co-registration of training subjects into template space.

The effects of additional training data may present differently if the atlases are constructed with non-linear co-registration of training subjects. %\note{this might belong somewhere else.}
There are two sources of inter-subject variability wrapped up in the atlas: the first is the global anatomical variability including skull shape and differences in cortical surface geometry, the second is variability in position and shape of the tract itself.
Theoretically, diffeomorphic registration of training subjects would eliminate the first of these effects (global variability), leaving only the tract specific variation.
However, such an atlas would necessitate subsequent applications in new target subjects to also utilise diffeomorphic registration between subject and template space, as the atlas would contain no ``allowance" for global variability, expecting perfect alignment with a target image.
Requiring diffeomorphic registration at the point of application would greatly inhibit the robustness and speed of tractfinder, and is therefore not the preferred approach.

% Query link: https://www.webofscience.com/wos/woscc/summary/28ab4b31-92c1-4d9e-b117-d5cd310c9927-a4fe70c6/relevance/1
The issue of deep learning's data requirements has been attracting increased attention.
A search on the online publication database Web of Science for \spverb|(few OR single OR one) AND shot AND learning|  (excluding results about single-shot echo planar imaging) revealed a sharply increasing trend in publication volume within the field of medical imaging (Fig. \ref{fig:pubs}).
A recent example, by \textcite{Liu2023a} and building on previous work in \textcite{Lu2021}, looked at using single-shot learning to train a deep neural detwork, an extension of the TractSeg architecture, for white matter tract segmentation.
They trained a network on 60 of the 72 total TractSeg white matter tracts, and applied data-augmentation and transfer learning techniques to adapt the model to segment the remaining 12 tracts using only a single exemplar training dataset for those novel tracts, which included both the \gls{cst} and \gls{or}.

Considering only their best results (which differed depending on how many novel tracts the network was extended to) for these two tracts, they achieved a mean Dice score of 0.719 for the \gls{cst} and 0.624 for the \gls{or}, equally good or worse than for tractfinder ``trained" even on only a single dataset (Fig. \ref{fig:ntrain}).
While the computational time spent on network training is not disclosed in either \textcite{Liu2023a} or \textcite{Wasserthal2018}, \textcite{Berto2021} re-trained their own TractSeg model as a benchmark comparison for the streamline clustering method Classifyber and reported GPU-accelerated training times of 3-7 hours.
Even so, one-shot training of a deep neural network appears to offer no improvement over simply registering a segmentation from one subject to another.

\begin{figure}[h!]
  \centering
  \includegraphics[width=0.8\textwidth]{pubs.png}
  \caption{Publication records by year including the term ``single/one/few shot learning" (or similar) on the database Web of Science.}
  \label{fig:pubs}
\end{figure}


\chapter{Aligning atlas and data}
\label{chap:reg}

Each tract atlas is constructed in a standard template space, and before being used to estimate the location of a tract in a new patient it must first be aligned with that patient's anatomy.
In order to achieve this, some sort of registration strategy is needed, with the most appropriate registration approach depending on the type of application.
In healthy subjects, it is sufficient to use affine registration to transform the atlas from template into subject space, before the inner product (or whichever comparison is used) is calculated.
However, in some patient data, more advanced processing may be necessary to ensure satisfactory alignment with target anatomy.
The following sections will describe the registration methods used in tractfinder for different scenarios, and including the use of tumour deformation modelling to account for space-occupying lesions.

\section{Registration from standard space}
\label{sec:reg1}

There are two categories of image registration.
In global registration, a single transform applies to the entire image, preserving its topology.
Different degrees of freedom define some commonly used terms: rigid body registration comprises just translation and rotation, while affine registration further includes scaling and shearing (up to 12 degrees of freedom on $\mathbb{R}^3$).
Following purely algebraic definition, a linear transform does not include translation, however in practice (at least in the medical imaging sphere) ``linear" and ``affine" are often used interchangeably.
Non-linear, deformable, or diffeomorphic registrations compute local deformations on a voxel-by-voxel basis, and therefore have orders of magnitude more degrees of freedom.
They are useful for fine-grained alignment of local structures, but many algorithms are unstable and prone to converging on local minima, making them difficult to successfully integrate into automated pipelines.

As we saw in Chapter \ref{chap:atlas}, a certain degree of inter-subject variability, including both tract-specific and global anatomical differences, is a feature of the atlas due to the use of affine registration to combine the training data in template space.
We mirror this relationship by allowing the atlas to be registered to a new target dataset using affine registration alone, with the subsequent comparison with native \gls{dmri} data acting to correct slight misalignments due to anatomical variability.
Computing a non-linear registration between atlas and subject at the point of application is undesirable principally due the lack of robust, stable and generalisable algorithms that are open source.
Maintaining practicality is a key requisite of tractfinder, and various non-linear registration tools were found during prototyping to either be too unstable (requiring manual adjustment of parameters in difficult cases such as some clinical scans) or having too long a computation time.\autocite{Visser2020}

As we shall see in the range of evaluations presented in Part \ref{part3}, the use of affine registration does not result in significant segmentation inaccuracies or errors, thus there is no credible incentive to favour the use of non-linear registration at the cost of increased processing time and potential instabilities.
Indeed, \textcite{Visser2020} showed that subject-to-standard registration accuracy of the tumoural region in low and high grade gliomas is not significantly improved using non-linear registration across a range of different packages, concluding there is little to justify the additional time cost and lack of robust automation.
Nevertheless, the trade-offs associated with relying on affine registration must be acknowledged.
Particularly in brains with modest deformations or other lesions, the accuracy of anatomical alignment can be impacted, and there follows the risk of suboptimal spatial and / or orientational alignment, resulting in missed areas (false negative) or erroneously included regions (false positives).

The risk of misalignment is especially high for small structures, such as the fornix or anterior commissure, both narrow bundles which are reliably difficult to segment.
For these cases, the construction of the atlas with spatial inter-subject smoothing is advantageous, as even with slight misalignment there is still a good chance that enough of the atlas will overlap with the structure in the target image to achieve detection.
However, this same feature may also lead to false positives in some cases, such as when two distinct tracts run in parallel, oriented the same way in relative spatial proximity.
An example of this scenario can be seen at the temporoparietal fibre intersection area, where at least seven different identified bundles converge.\autocite{Martino2013b}
Here the vertical portion of the arcuate fasciculus lies lateral to the sagittal stratum containing antero-posteriorly oriented association fibres (including the \gls{or}), which in turn lies lateral to the tapetum of the corpus callosum.
The similar orientations of the the \gls{af} and tapetum in this region, together with their close proximity, could lead to fibres of one being wrongly attributed to the other if the atlas is too broad (Fig. \ref{fig:tpfia}).

\begin{SCfigure}[][h!]
  \captionsetup{format=plain}
  \includegraphics[width=0.6\textwidth]{chapter_3/tpfia.png}
  \caption{Example of potential for atlas misalignment. The \gls{af} (*) and tapetum (**) are proximal and parallel at the temporoparietal fibre intersection area. Linearly registered right \gls{af} atlas \glspl{tod} may overlap with tapetum (arrow).}
  \label{fig:tpfia}
\end{SCfigure}

Fortunately, such misalignment effects are small and unlikely to impact the overall segmentation quality, in part because only the margins of the atlas are likely to ``spill" into neighbouring tracts, and the low spatial probability will result in very low or sub-threshold mapped values.
To confirm this assessment, Figure \ref{fig:nrr} shows segmentation similarity scores compared with probabilistic targeted tractography when using either affine (FMRIB's Linear Image Registration Tool\autocite{Jenkinson2002}) or non-linear (ANTs registration package Symmetric Normalisation algorithm\autocite{Tustison2013,Avants2011}) atlas registration in 71 healthy subjects (\textit{TractoInferno} dataset, see Section \ref{sec:data} for more details).
Note that the atlases themselves are unchanged from those previously described and used in subsequent analyses, meaning the generation of the atlases still involves only affine registration between training subjects.
There is no discernible difference in the scores, indicating that non-linear registration does not improve the final output.
This comparison was included as supplementary material in a manuscript accepted for publication in \textit{Human Brain Mapping}\autocite{Young2024}.

\begin{figure}
  \centering
  \includegraphics[width=\textwidth]{chapter_3/compare_affine_nrr_2.png}
  \caption{Difference in tractfinder performance when using either affine (plain) or non-linear (hatched) atlas registration, compared with targeted ROI tractography, in the \textit{TractoInferno} dataset of healthy subjects. For binary measures, a threshold of 0.05 was applied.\note{replace with/add comparison with all tracts, ranked by size}}
  \label{fig:nrr}
\end{figure}

In conclusion, affine registration between template and subject space is preferred in most applications, including for healthy subjects and clinical subjects without large space-occupying lesions.
The next sections will address the scenarios where patient anatomical topology differs significantly from atlas space: in the presence of deforming tumours, or intraoperative brain shift.

\section{Tumour deformation modelling}
\label{chapterlabel3}

Tract orientation atlases represent the expected orientation and location of a tract in typical healthy subjects.
In the previous chapter we asserted that, for subjects with little structural divergence from the norm, linear registration is sufficient for aligning the atlas and target data.
However, in cases with large mass-effect, affine registration becomes clearly inadequate, as the distances between expected and actual location of brain structures is simply too great.
In order to correct for displacement of white matter tracts due to space-occupying lesions, the atlas will need to be deformed more dramatically before comparing with the native \gls{fod} map.

Anatomical non-correspondence between subject and template images caused by space-occupying lesions poses a substantial challenge to the use of atlas-based white matter segmentation methods in clinical subjects.
While nonlinear deformation tools can produce accurate registration in clinical images, they require usually manual adjustment of many input and regularisation parameters on a case by case basis, and a robust automation of this process is not available to best knowledge.
More significantly, in the case of registration between a normative template and a scan with a brain tumour, a fundamental assumption of many registration algorithms, that of topological equivalence between the two images, does not hold.\autocite{Zacharaki2009}
The core of the problem is that one image---the template---depicts the anatomy of an average, healthy brain, and the other that of a diseased brain harbouring a tumour, presenting two potential scenarios.
Either the lesion exists \textit{in addition} to all the structures and tissue expected of a healthy brain, which have been displaced or compressed to accommodate it, or it has \textit{replaced} brain landmarks or rendered them otherwise unrecognisable through the effects of oedema or differing MR properties between tumour and healthy brain tissue.
In either case, a non-linear registration algorithm is tasked with finding a mapping between two images with different sets of anatomical landmarks, with finding anatomical correspondence between tissue in one image which is nonexistent in the other.
The result of this violation of the topological equivalence assumption is often ridiculously contorted images, and particularly in the peritumoural zone, which is rather problematic for neurosurgical applications.

Deformable registration alone is thus largely insufficient for handling the anatomical mismatch problem. \autocite{Elazab2018, Visser2020}
It has long been proposed that an acceptable registration between atlas and tumour data can only be obtained with an additional step of artificially implanting or modelling a representation of the tumour in the atlas.\autocite{Cabezas2011,Mang2020}
This can be in the form of a seeded atlas deformation,\autocite{Dawant2002} in which a small seed is placed in the atlas before deformable registration, acting as a region of tissue which can be warped to match the full tumour in patient space.
Alternatively, the seed can be artificially ``grown" using a biophysical or mathematical model of tumour proliferation to simulate deformation, with optional further nonlinear registration of the deformed atlas to the patient image.\autocite{Cuadra2004, Zacharaki2009}
Many proposed tumour deformation models aim to achieve highly accurate modelling of tumour growth dynamics and the effects on surrounding tissues, by taking into account elastic tissue properties and microscopic tumour growth modelling (see \textcite{Elazab2018} for a comprehensive review).
Generally models will consider one or a coupling\autocite{Clatz2005,Hogea2007,Prastawa2009} of tumour cellular proliferation and infiltration into surrounding tissues using a reaction diffusion or similar framework,\autocite{Tunc2021,Scheufele2019b,Elaff2018}
or the biomechanical forces acting between tissues.\autocite{Mohamed2006,Hogea2007a,Zacharaki2009}
The resulting algorithms are often mathematically complex and implemented using finite element methods,\autocite{Elazab2018} require optimisation of tumour parameters through problem inversion or by other means \autocite{Mohamed2006, Zacharaki2009, Mang2020} and take anywhere between 1 and 36 hours to run, even on high performance computing setups.\autocite{Zacharaki2009,Bauer2012,Gooya2012,Bauer2013,Mang2012}
This is entirely reasonable for studies in which accuracy is a far greater priority than speed.
Typical applications of tumour deformation modelling include intra-patient longitudinal studies of tumour growth, and inter-patient registration and spatial normalisation for atlas-based segmentation or statistical analysis across patient populations (see \textcite{Bauer2013} and \textcite{Cabezas2011} for overviews).
Given the time constraints of intraoperative imaging and the practical constraints of the computing capacity which can reasonably be assumed to be available in a clinical setup, the aim for this project was to achieve an estimate of tract displacement with low computational complexity.

The tract orientation atlas described in the previous chapter provides a degree of spatial tolerance that alleviates the need for voxel-perfect registration and deformation, allowing the implementation of a minimal deformation algorithm.
The idea is to obtain a deformation model which is simple enough to compute using few temporal and computational resources, and use it in combination with affine registration, as before, to achieve spatial alignment between atlas and patient space.
To this end we will interest ourselves solely in the macroscopic spatial effects of tumour growth:
What effect does the presence of a tumour have on the physical position of and the fibre orientations within a given volume unit of tissue?
Of course, this question cannot be fully answered without considering the complicated factors described above, such as whether a tumour is encapsulated or infiltrating.
Nevertheless, as all models are wrong, and some are useful, the utility of our model will be measured by its ease of computation and accurate capturing of tract displacement.
Whether performance in the latter criteria is satisfactory will be measured through the resulting improvement in tract mapping as compared with patient-native methods such as streamline tractography (see Section \ref{sec:btcd}).
We will consider a radial model of tumour deformation, previously described in \textcite{Young2022}, assuming the tumour has expanded outwards from a central seed, and that surrounding tissue is displaced along the same radial directions.
\textcite{Cuadra2004} similarly used a radial expansion assumption, but for modelling the interior tumour region (rather than outside the tumour), while an optical flow algorithm was implemented for image matching outside the tumour.

\subsection{Development of a radial deformation model}

We begin with a radial deformation model described by \textcite{Nowinski2005}.
Their motivation was remarkably similar:
The rapid deformation of a morphological brain atlas to aid the interpretation of brain anatomies affected by tumour mass effect.
The required model inputs are the segmentations of the tumour and brain volumes.
We define the direction $\mathbf{\hat{e}}$, which is the unit vector along the line connecting a point $P(x,y,z)$ anywhere within the brain to the tumour centre of mass, $S$.
This is the direction along which we assume the tissue at that point to be shifted by the tumour:
Radially outward from the tumour centre.
Along $\mathbf{\hat{e}}$ we also define $D_p$ as the distance  $\|\overrightarrow{SP}\|$, $D_b$ as the distance from $S$ to the brain surface and $D_t$ as the distance from $S$ to the tumour surface (Fig. \ref{fig:virtue}).

\begin{figure}[htp]
  \centering
  \includesvg[inkscapelatex=true]{chapter_4/virtue_vars.svg}
  \caption{Graphical schema of the variables defined in the radial deformation model}
  \label{fig:virtue}
\end{figure}

Then for a point in the original image $P = (x,y,z)$ the transformed location in the deformed image $P' = (x',y',z')$ is

\begin{align}\label{eq:forwardP}
  P' = f(P) = P + \mathbf{\hat{e}}kD_ts.
\end{align}

The amount of displacement $\Delta P = kD_ts$ is thus determined by $D_t$, a scale factor $0<s \leq 1$ and a spatially varying displacement factor $k(P)$.
In \textcite{Nowinski2005}, $k$ is a linear function of $D_p$: $k = 1-\frac{D_p}{D_b}$. \footnote[2]{In the original \textcite{Nowinski2005} article, the deformation is described in reverse, as a shrinking model, and the variables there look a little different. They are consistent with the formulation used here, which has been chosen for ease of conceptualisation. Note that both forward and reverse models are required for different types of image transformation, and will be derived later.}
This can be conceptualised as a displacement force radiating from the centre of the tumour and decaying linearly with distance, reaching 0 only at the brain boundary.
However, initial experimentations with this model revealed that such a linearly decaying force doesn't do well at capturing the displacement fields observed in real tumour cases.
The elasticity and compressibility of brain tissue means that the radial force is absorbed by surrounding tissue more rapidly than a linear function accounts for.
Even with very large tumours, it is common for parts of the brain some distance from the tumour surface to experience no displacement at all, suggesting a  more rapidly decaying function would be a more appropriate choice for $k$.

An exponentially decaying function captures this well, while remaining easily computable, close-form and invertible.
We begin with a function in the form $k(P) \propto e^{-\lambda \frac{D_p}{D_t}}$.
There are two boundary conditions:
Points on the brain surface should not be displaced ($k(D_p = D_b) = 0$) and points at the centre of the tumour should be displaced by exactly $D_t$ ($k(D_p = 0) = 1$).
Note that the latter boundary condition is an assumption reflecting a fully encapsulated tumour, where no normal tissue remains inside the final tumour boundary after displacement.
Solving for these boundary conditions gives us a normalisation constant:

\begin{align}
  k &= a e^{-\lambda x} + c &\text{ where } x = D_p / D_b \nonumber \\
  k(x=0)=1 \longrightarrow 1 &= a e^{-0} + c = a + c \nonumber \\
  k(x=1)=0 \longrightarrow 0 &= a e^{-\lambda} + c \nonumber \\
  -c &= (1-c) e^{-\lambda} \nonumber \\
  c &= \frac{e^{-\lambda}}{e^{-\lambda} - 1} \label{eq:c}
\end{align}

giving

\begin{align}\label{eq:forwardk}
  k(P) = (1-c)e^{-\lambda \frac{D_p}{D_b}} +c.
\end{align}

Equations (\ref{eq:forwardP}) and (\ref{eq:forwardk}) describe the forward deformation transform $P'=f(P)$, which maps a point in the original image $P$ to a new position $P'$ in the deformed image.
Forward warping works well for continuous valued data such as streamlines:
As the transformed image is also defined in continuous coordinates, each vertex can be ``pushed" to its exact location in the warped output.
When transforming discreet image data defined on a pixel (or voxel) grid, however, the transformed coordinate $P'$ will not generally correspond to a grid point, resulting in voxels in the transformed image with no assigned value (``holes"), or voxels being assigned values from multiple overlapping mapped points.
To produce the transformed image, the value of $P'$ has to be distributed among the neighbouring voxel grid points (within a predefined kernel) using a process called ``splatting",\autocite{Niklaus2020} where each voxel is assigned the value weighted by its distance to $P'$ (Fig. \ref{fig:warp}).
As each voxel value will be determined by a weighted sum of all transformed values, this method requires an intermediate buffer to store all transformed points and distributed weights before the final voxel values can be determined, and filling techniques may be required to fill in any holes.

\begin{figure}[h!]
  \centering
  \includesvg[width=0.7\textwidth,inkscapelatex=true]{chapter_4/warping.svg}
  \caption{Forward image warping maps from original to transformed coordinates, where the transformed value is distributed among neighbourhood grid points in a process called ``splatting" (orange). Reverse image warping uses the inverse transform $f^{-1}(P')$ to determine the source position in the original image, with the appropriate value interpolated from the neighbourhood of $P$ (blue).}
  \label{fig:warp}
\end{figure}

It is therefore usually preferable, if the inverse transform $P = f^{-1}(P')$ is known, to use reverse warping, in which each grid point value in the transformed image is ``pulled" from the corresponding continuous-valued point in the source image.
As with forward warping, the reverse mapped point $f^{-1}(P')$ will not generally fall exactly on a grid point in the source image, so the appropriate value is interpolated from the neighbourhood (Fig. \ref{fig:warp}).
Reverse warp convention is preferred in medical image manipulation packages to smoothly deform (and resample) a gridded image, and so we need to obtain the inverse mapping $P = f^{-1}(P')$.
The inverse function for the linear model is (as formulated in \textcite{Nowinski2005}):

\begin{align}
  P = P' - \mathbf{\hat{e}}(1-\frac{D_{P'}-D_t}{D_b-D_t})D_ts
\end{align}

To obtain the inverse mapping for the exponential model, we solve equation (\ref{eq:forwardP}) for $P$, using the exponential $k(P)$ given by (\ref{eq:forwardk}):

\begin{align}
  P = P' - \mathbf{\hat{e}}(D_t s c - \frac{D_b}{\lambda}\mathcal{W}_0(\frac{-\lambda D_t s (1-c) e^{-\lambda(D_{p'}-D_tsc)/D_b}}{D_b}))
\end{align}

where $\mathcal{W}_0(y)$ is the principal branch of the lambert $\mathcal{W}$ function, defined as the inverse function of $ y(x) = xe^x $ for $x,y \in \mathbb{R}$.

The most appropriate value for the exponential decay parameter $\lambda$ will depend on characteristics of the specific lesion being modelled.
For example, smaller lesions (20-30mm diameter) typically displace tissue only in their immediate surroundings, with distant tissue remaining virtually unmoved.
In such cases, a higher value of $\lambda$ ($\geq 3$), indicating stronger fall-off in displacement force, would be appropriate.% (Fig. \note{fig:examples}).
In any case, in order to keep the transforms well-behaved, we need to enforce the boundary condition that every point $P$ in the source image that is within the tumour perimeter ends up strictly outside the tumour in the eventual deformed image.
In other words,

\begin{equation}\label{eq:lambdabound}
  k(P) \geq 1 - \frac{D_P}{D_t} = g(P)
\end{equation}

must hold for all $P$ (Fig. \ref{fig:k}).

Given that the gradient of $k$ is strictly decreasing and $g(P) = 1 - \frac{D_P}{D_t}$ is linear, it is sufficient to set
\begin{align*}
  \frac{d}{dP}\bigg\rvert_{D_P=0}k(P) &= \frac{d}{dP}\bigg\rvert_{D_P=0}g(P) &\text{ where } \frac{dk}{dP} &= -\frac{\lambda}{D_b}(1-c)e^{-D_p/D_b} &\text{ and } \frac{dg}{dP} &= -\frac{1}{D_t}
\end{align*}

We solve for $\lambda$:

\begin{align*}
  -\frac{\lambda_{max}}{D_b}(1-c) &= -\frac{1}{D_t} \\
  \lambda_{max} &= \frac{D_b}{D_t (1-c)}
\end{align*}

where $c$ is itself a function of $\lambda$ as defined in (\ref{eq:c}). This value can be determined iteratively, or with the expression

\begin{equation}
  \lambda_{max} = \mathfrak{Re} \left[ \mathcal{W}_0(-\frac{D_b}{D_t}e^{-D_b/D_t}) \right] +\frac{D_b}{D_t}
\end{equation}

Thus for strictly non-infiltrating lesions, we set $\lambda \leq \lambda_{max}$ to satisfy equation (\ref{eq:lambdabound}), where $\lambda_{max}$ is used as the default value if none is specified (referred to as ``adaptive $\lambda$").
Note that $\lambda_{max}$ varies throughout the brain, being a function of the relative distances to brain and tumour surfaces for each specific $P$.

\begin{SCfigure}[][h!]
  \includegraphics{chapter_4/k.pdf}
  \caption{Deformation factor $k$ as a function of $D_P$. $\lambda$ must be small enough such that $k_{\lambda}$ is strictly above the line $1-(\frac{D_P}{D_t})$ (dashed line). An exponential $k$ with $\lambda_{max}$ is plotted in solid black, compared with a linear $k$ as proposed in \textcite{Nowinski2005} (dotted line).}
  \label{fig:k}
\end{SCfigure}

The tumour deformation model is implemented in Python, and full execution takes on average 1 min for a 208 x 256 x 256 voxel image.
If lookup tables for $ D_t$ and $D_b$ are precomputed and saved, then subsequent executions of the model (e.g. with different values for $\lambda$ and $s$, as appropriate for a given tumour) take less than 10 seconds, as long as the tumour and brain segmentations remain unchanged.

\subsection{Limitations and future progress}

Due to its mathematical simplicity and crudeness in assumptions about tissue mechanics, the model described above naturally comes with several limitations, of which two in particular have a noticeable effect on real clinical applications and generalisability.
One is the inbuilt assumption of full tumour encapsulation, as expressed in the boundary condition that all existing tissue in the template brain remain outside the tumour in the deformed image.
Through this assumption we have explicitly excluded the modelling of infiltrative tumours, which may expand existing structures but not necessarily wholesale shunt them outside the tumour boundary.

The difficulty with modelling tumour infiltration under the current framework is in the appropriate demarcation of the lesion in the subject scan.
We can broadly categorise tumour regions into three components based on their interaction with the non-tumour environment:
A tumour core, non-infiltrating and possibly cystic or necrotic, an infiltrating component, and finally peritumoural oedema.
These distinctions are not enough to predict the mass effect of a tumour.
Some tumours that are entirely infiltrative exert no perceivable mass effect, while others do.
% If a tumour comprises both an infiltrating and non-infiltrating component, should both be included in the segmentation?
% Given that the aim here is to capture neighbourhood displacement, it seems reasonable to suggest that only a segmentation of the volume exerting mass effect should be considered.
In a lesion comprising both infiltrating and non-infiltrating parts, the latter may well be the only cause of mass effect, and readily demarcated, but not necessarily.
Hence even if it were practical to more finely segment a tumour, separately labelling oedema, infiltration and core tumour components, exploiting this information to more accurately model the full mass effect would be by no means straightforward, and perhaps not possible in a way that generalises to all tumours without resorting to the in-depth modelling described in the introduction to this chapter.

This is not to say that the tumour deformation model in its present form cannot handle any form of infiltrative tumour (indeed, we will see examples in Chapter \ref{chap:applications} of such cases), for example by adjusting the scale parameter $s$ if the segmentation partially includes infiltrative tumour.
However, tumours which are more or less wholly infiltrative, for example cases 4 and 5 in Figure \ref{fig:fa_hist}, cannot be modelled.
An alternative form of $k(P)$ could be considered, such as the radial polynomial models used in optics and digital imaging to describe and correct for lens distortions.\autocite{Zhang2000a}
The magnification effect near the centre of a radial distortion field is comparable with the distending of tissue seen in some infiltrative tumours, see for example case 5 in Figure \ref{fig:fa_hist}.
In depth exploration of such additional distortion models was not pursued for this project, principally because the relevant clinical cases are far less likely to be indicated for surgical resection owing to the grave risk to infiltrated structures.
Nevertheless, accurate white matter imaging could still be informative in assessing the lesion and informing treatment plans such as radiotherapy targeting,\autocite{Jena2005,Berberat2014} so the extension of this framework to include robust handling of infiltration could be an interesting avenue for future work.

\begin{figure}[htb!]
  \makebox[\linewidth][r]{%
  \includegraphics{chapter_4/tumour_fa.pdf}}
  \caption{A series of paediatric brain tumour cases. For each subject an axial slice through the tumour centre is displayed, with \gls{dti}-derived directionally encoded colour maps overlaid. Patients 4 and 5 have infiltrating tumours centred on the left and right basal ganglia respectively. Histograms depict the \gls{fa} values within the entire brain volume (including the tumour) and tumour volume only. The two infiltrating tumours, 4 and 5, have higher overall \gls{fa} values than the rest of the brain, indicating the presence of infiltrated anisotropic white matter within the tumour.}
  \label{fig:fa_hist}
\end{figure}

\begin{figure}[htb!]
  \includesvg[width=\textwidth,inkscapelatex=true,pretex=\small\sffamily]{chapter_4/ventricles.svg}
  \caption{Examples of ventricle deformation under tumour pressure, and tumour deformation modelling in template image. \textbf{a.} \gls{btc} sub-PAT03. \textbf{b.} \gls{btc} sub-PAT26}
  \label{fig:ventricles}
\end{figure}

A second key limitation of the present deformation model relates to the characteristics of the surrounding tissues rather than the tumour itself.
The displacement of a position in the brain $P$ depends only on the distances between it, the tumour, and the brain surface.
Otherwise, the brain is treated as a homogenous mass, when in reality it comprises regions with very different mechanical properties.
This becomes most apparent when we observe the deformations involving the ventricles.
The ventricles, forming a part of the wider \gls{csf} network, are far more compressible than \gls{wm} or \gls{gm}, allowing them to often absorb to a large extent the compressive forces exerted by the tumour and causing them to collapse more than the adjacent tissue.
These tissue-dependent effects are completely omitted from the presented model, leading to mis-estimations of the deformations surrounding ventricles in some cases (Fig. \ref{fig:ventricles}).
Controlling the magnitude of deformation adjacent to the tumour by adjusting the deformation decay constant $\lambda$ can mitigate this affect to an extent, however this increases the need for case-by-case manual parameter adjustment, which we wish to avoid.
Ultimately, accurate registration throughout the brain cannot be achieved with such a simple model, as indeed was never the aim.
Future work could investigate the feasibility of combining the radial deformation with adaptable tissue-dependent elasticity, modelled on previous similar works in the registration space.\autocite{Rohde2003,Duay2004}

\section{Intra-patient registration and brain shift}

Given the target application of intraoperative \gls{wm} imaging, it would be entirely remiss to not discuss the additional registration issues brought about by brain shift.
Predictably, the sheer variability in direction, magnitude and extent of brain shift means that a universally elegant and robust solution cannot be found.
Instead, we will discuss some common intraoperative scenarios and how tractfinder would be applied to them.

One major source of brain shift is tumour debulking.
As tumour is removed through the craniotomy, the surrounding tissue may collapse and the associated mass effect be correspondingly reduced.
For this scenario, tumour deformation can be utilised using a preoperative segmentation of the tumour and adjustment of the tumour scale parameter $s$, which if set to a value below 1 will virtually decrease the tumour radius, mimicking the debulking effect.
A case study demonstrating this is presented in Section \ref{sec:case}.
The necessary steps (that couldn't be precomputed preoperatively) in such a scenario would be affine registration between pre- and intraoperative subject space, re-computation of a tumour deformation field using preoperative tumour segmentation and $s<1$, atlas deformation, and finally tract mapping with an intraoperative \gls{fod} image.
The additional steps of pre- to intraoperative registration and tumour deformation would add less than five minutes to the total processing time.

Of course, tumour debulking does not always result in intraoperative brain shift that looks like a simple reduction in mass effect.
Often, a certain degree of tissue inertia means that surrounding structures don't instantly relax back into a more ``normal'' position, and the effects of gravity, \gls{csf} drainage, herniation, or intracranial pressure changes will then have a far more drastic influence on overall brain shift than the removal of tumour tissue.
In these cases, tumour deformation modelling cannot be leveraged to account for brain shift, and different registration strategies will be required.
Affine registration with anisotropic scaling may be sufficient for brain shift principally characterised by sagging or compression due to gravity.

Finally, there are those cases in which topological differences between pre- and intraoperative scans are so great that neither affine registration nor adjusted tumour deformation can sufficiently align the data.
In these subjects, advanced deformable registration is necessary.
The task of intraoperative registration to account for brain shift is a well studied one, although in most cases the approach is to deform preoperative data (such as streamlines) to continually provide accurate neuronavigation after brain shift\autocite{Clatz2005,Archip2007,Wittek2007,Archip2008}.
This requires highly accurate registration, as there is no intraoperative \gls{dmri} acquisition to inform tract identification.
For tractfinder, it would be preferable to impose stricter regularisation to increase registration stability and minimise the creation of unphysical distortions (e.g. surrounding the site of resection), since any consequent minor misalignments can be handled by the atlas smoothness and comparison with local diffusion data.
In individual case studies, non-linear registration has successfully been leveraged in this way (Fig. \ref{fig:nrrex}).
However, this is still only possible in an experimental setting and involves significant trial-and-error in determining the appropriate registration input parameters, and it remains the subject of future work to achieve robust generalisation and automation.

\begin{figure}[hb!]
  \centering
  \includesvg[width=\textwidth,pretex=\small\sffamily]{chapter_4/nrr_gosh.svg}
  \caption[Nonlinear registration for intraoperative MRI example]{Example of non-linear registration applied to a \glsentrylong{gosh} \glsentrylong{imri} case.
  \textbf{\sffamily a.} Intraoperative scan with colour \glsentrylong{fa} map overlaid to enhance \glsentrylong{wm} contrast, registered to preoperative scan using rigid registration. Outline of \glsentrylong{wm} segmentation of preoperative scan is overlaid in white, demonstrating substantial brain shift away from the craniotomy (*). Arrow highlights shifting of the ipsilateral external capsule. Dotted line indicates position of coronal plane in right image.
  \textbf{\sffamily b.} Tractfinder map of the \glsentrylongpl{cst} after non-linear registration between pre- and intraoperative scans, using the Fast Free-Form Deformation algorithm\autocite{Modat2010} from the NiftyReg package (\url{http://cmictig.cs.ucl.ac.uk/wiki/index.php/NiftyReg}).}
  \label{fig:nrrex}
\end{figure}

\section{Summary}

All atlas-based image segmentation or analysis techniques have to contend with the problem of aligning said atlas with the target image, and tractfinder is no exception.
In this chapter we have considered the available solutions to this problem, their respective advantages and drawbacks, and applicability to various practical scenarios.

By designing a fuzzy atlas which allows for inter-subject variability to provide an initial estimate for a tract's expected features in the target image, and relying on additional subject-specific information to refine that estimate, we can avoid the need for voxel-perfect atlas alignment, as long as the registered atlas fully covers the domain of the target tract.
In healthy and structurally normal data, affine registration with 12 degrees of freedom is fully sufficient to achieve this coverage.
We saw in Section \ref{sec:reg1} that non-linear registration is unlikely to provide any significant improvement in performance, and only decrease speed and practicality.

Where space occupying lesions are concerned, however, simple affine registration can leave the registered tract atlas entirely misaligned with the corresponding subject anatomy.
In such cases, more advanced atlas deformation is necessary to account for the effects of tumours, while still keeping the compute complexity to an acceptable minimum for clinical implementation.
A simple radial deformation algorithm is proposed to specifically account for mass effect after initial affine registration, which successfully models gross tract displacements.

Finally, we saw how the stringent assumptions of the tumour expansion model do not adequately support the modelling of infiltrating tumours, and how the complexities of brain shift may call for more involved registration tactics, including non-linear algorithms and others developed specifically for intraoperative registration.
In the next chapters we will put the methodological components together and assess the proposed techniques' performance in a series of quantitative evaluations and practical applications.


\epart[\epigraph{Unfortunately, nature seems unaware of our intellectual need for convenience and unity, and very often takes delight in complication and diversity.}{Santiago Ramón y Cajal, 1906 Nobel Lecture}]{Results and applications}
% \epigraph{What could be more natural than complexity?}{Hank Green}
% \epigraph{The most natural thing in the world is complexity}{Hank Green}
\chapter{Methodological evaluation}\label{chap:eval}

The following sections set out to provide an evaluation of the tractfinder method from a range of perspectives.
A common complaint in the white matter imaging field is over the lack of reliable ground truth information for fibre tract identification.
This goes deeper than the limitations of our scientific instruments:
Micrometer-resolution \gls{mri} or infallible tractography algorithms are not the answer.
As was explored in Chapter \ref{chap:atlas}, our understanding of white matter anatomy, connection, and function is neither fixed nor universal, meaning for many tracts a single definition that can be agreed upon by all does not even exist in a given moment, nor one to last over time.

We can nonetheless arrive at a reasonable assessment of tractfinder's accuracy, reliability and applicability by making justifiable assumptions and comparing it, quantitatively and visually, with benchmark methods in a variety of datasets and case studies.

The missing ground truth problem comes to the fore when comparing different methods, as we will see in Section \ref{sec:validation}, but the results are still instructive in assessing their respective characteristics.
To balance this issue, a further analysis directly comparing tractfinder against the deep learning method TractSeg by constructing new \gls{tod} atlases from the TractSeg model training data is presented in Section \ref{sec:tractseg}.
We'll further attempt to quantitatively evaluate the effect of tumour deformation modelling on improving segmentation performance in affected patients.

Subsequently, in Chapter \ref{chap:applications}, we'll move on from similarity scores and box plots to explore the practical applicability of tractfinder, considering the technical aspects of bringing it to the bedside, and illustrating the strengths and remaining roadblocks through a series of case studies.

\section{Datasets and image processing}
\label{sec:data}

A range of datasets are considered in which to compare the tractfinder approach against alternative tract segmentation methods, covering adult and paediatric, healthy and clinical populations (Tab. \ref{tab:datasets}).
Each dataset and any dataset-specific preprocessing is described in the following subsections.
In addition, the following default preprocessing steps were applied to all scans, unless otherwise specified:\footnote[2]{software versions: MRtrix3 v3.0.2--3.0.3 (\url{https://www.mrtrix.org/}), \gls{fsl} v.6.0}

\textit{Brain masking}\autocite{Tournier2019}
using a heuristic algorithm based on thresholding the mean of each diffusion-weighted shell described in \textcite{Dhollander2016}, implemented in MRtrix3\autocite{Tournier2019} as \verb|dwi2mask|.

\textit{Affine registration} between subject space and MNI152\autocite{Fonov2011} template space using the \gls{fsl} Linear Image Registration Tool (FLIRT)\autocite{Jenkinson2001,Jenkinson2002}.

\textit{\Gls{csd}}. Depending on the dataset and / or application, two different versions of \gls{csd} applied: \gls{ssst} \gls{csd}\autocite{Tournier2007,Tournier2019} (``original flavour") and \gls{msmt} \gls{csd}\autocite{Jeurissen2014} restricted to \gls{wm} and \gls{gm} tissue compartments for single-shelled acquisitions (those with a single nonzero $b$-value),
and \gls{msmt} \gls{csd} with three tissue compartments (\gls{wm}, \gls{gm}, \gls{csf}) for multi-shelled acquisitions (see Section \ref{sec:ismrmdiff} for further context on these choices).
In all cases, response functions were obtained using the Dhollander unsupervised 3-tissue response function estimation algorithm\autocite{Dhollander2016,Dhollander2019}.
All \gls{csd} was performed using the MRtrix3 image processing software package\autocite{Tournier2019}.

\subsection{HCP}

The first large healthy adult dataset (``HCP49") comprised 49 scans from the WU-Minn \gls{hcp} Young Adult S1200 data release (\url{https://www.humanconnectome.org/study/hcp-young-adult/document/1200-subjects-data-release}) \autocite{VanEssen2013}.
The original data comprises high resolution $T_1$-weighted and diffusion-weighted \gls{mri} acquired on a modified Siemens 3T Skyra scanner.
Raw diffusion data contains three diffusion-weighted shells at $b=$ 1000, 2000 and 3000~s~mm$^{-2}$ with 90 directions each and 18 $b=0$ volumes\autocite{Sotiropoulos2013}.
These images have been preprocessed as documented in \textcite{Glasser2013,Sotiropoulos2013},
and additionally for this analysis they were downsampled to 2.5~mm isotropic voxels (from the original resolution of 1.25~mm isotropic) and a subset of 60 optimally distributed directions at $b=1000$~s~mm$^{-2}$ were extracted, emulating a typical clinical acquisition in spatial and angular resolution.

A different set of 105 \gls{hcp} subjects (``HCP105") were used to train and test the deep learning segmentation method TractSeg\autocite{Wasserthal2018}.
The curated TractSeg reference streamline bundles for these subjects are publicly available\autocite{Wasserthal2018b} and were used in a separate analysis to directly compare tractfinder and TractSeg.
For these subjects, the same preprocessing steps were applied as in the set of 49 subjects, with the exception of selecting a subset of 30 instead of 60 $b=1000$~s~mm$^{-2}$ directions, to match the \textit{clinical quality} data described in \textcite{Wasserthal2018}.

\subsection{TractoInferno}

The recently released \textit{TractoInferno} database (v1.1.1, available at \url{https://openneuro.org/datasets/ds003900/versions/1.1.1})\autocite{Poulin2022a,Poulin2022}, created as an open dataset to support the training and comparison of machine learning tractography algorithms, contains diffusion and $T_1$-weighted \gls{mri} scans for 284 subjects pooled from several studies, accompanied by reference streamline tractography bundle reconstructions produced using four different tracking algorithms and the RecoBundlesX automatic streamline clustering method \autocite{Garyfallidis2018,Rheault2020a}, followed by semi-automatic quality control.
Of the 284 subjects included in the full \textit{TractoInferno} database, the 80 subjects with tractography of all of the \gls{cst}, \gls{or}, \gls{ifof} and \gls{af} were selected.
Nine subjects were excluded from the final analysis due to inadequate non-linear registration performance resulting in failed targeted \gls{roi} tractography (see Section \ref{sec:methods}), leaving a final 71 subjects.
Diffusion acquisition parameters and preprocessing steps are described in detail in \textcite{Poulin2022} and summarised in Table \ref{tab:datasets}, and all data was additionally resampled to 2.3~mm isotropic voxels, the lowest resolution present in the dataset and one in line with clinical acquisitions.

\subsection{Clinical (GOSH \& NHNN)}

To validate tractfinder in real clinical patient scans, a dataset of 15 individual scans from eight different subjects and two different institutions was collated.
They include four adult glioma subjects acquired in 2009 at the \gls{nhnn}, London (cases 4 and 5 from \textcite{Mancini2022}, others unpublished data),
three paediatric subjects from \gls{gosh} (each with one preoperative and one intraoperative scan),
a mock “intraoperative” scan on a healthy adult volunteer acquired with the \gls{gosh} intraoperative \gls{dti} protocol and using a simulated intraoperative setup (flex-coils wrapped around the head instead of a head coil, head significantly displaced from scanner isocenter etc.),
and a partridge in a pear tree.
The three \gls{gosh} paediatric patients all had low-grade astrocytoma (pilocytic astrocytoma, diffuse astrocytoma, and pleomorphic xanthoastrocytoma).
For fully acquisition details see Table \ref{tab:datasets}.
All clinical scans used in the systematic comparison with benchmark methods involved non-deforming tumours, in the sense that any lesions did not appreciably displace white matter structures from their typical positions.
Further \gls{gosh} and \gls{nhnn} scans with substantial tumours were analysed on a case-by case basis, as the low sample size and high tumour heterogeneity precluded a systematic quantitative evaluation.

Each \gls{dmri} scan was minimally preprocessed with \gls{mppca} denoising\autocite{Veraart2016, Cordero-Grande2019} Gibbs-ringing correction\autocite{Kellner2016} and bias field correction\autocite{Zhang2001, Smith2004}, as implemented in MRtrix3 \autocite{Tournier2019}.
Preoperative scans additionally had eddy current and motion distortion correction\autocite{Andersson2016a, Smith2004} (MRtrix3 v3.0.3 and FMRIB Software Library (\gls{fsl}, \url{https://fsl.fmrib.ox.ac.uk}) v6.0) applied, while this step was omitted for intraoperative scans to maintain a clinically realistic timeline.
No \gls{epi} distortion correction was performed, as it is frequently omitted from clinical pipelines due to lack of requisite reverse phase encoding or field map information and long processing times\autocite{Yang2022}.

\subsection{BTCD}
\label{sec:data_btcd}

The \gls{btc} dataset\autocite{Aerts2020a,Aerts2022a,Aerts2022} comprises a series of pre- and postoperative structural, functional, and diffusion-weighted \gls{mri} scans of glioma and meningioma patients.
Of the total 25 patients in the original dataset, ten with macroscopic, non-infiltrating lesions were selected for the validation of tractfinder in tumour patients.
In one of the ten subjects (sub-PAT22), only a preoperative session is available, giving a total of 19 unique scan sessions used for this dataset.
The data consists of high quality \gls{hardi} acquisitions, including reverse phase encoding for susceptibility distortion correction, and structural $T_1$-weighted images.
In addition, tumour segmentation masks for each preoperative scan are available in the original dataset, which can be used for tumour deformation modelling in those patients with strong tumour deformation.
The ten selected subjects include four with tumours large enough (thee meningioma and one anaplastic astrocytoma) to warrant the use of deformation modelling to improve atlas alignment.
Of these four subjects, one had a large midline frontal meningioma affecting both hemispheres, and for this subject all tracts were considered ``ipsilateral" tracts.
In all other subjects, the tracts were labelled ipsilateral or contralateral if they were in the same or opposite hemisphere as the tumour respectively.

\Gls{dmri} data was preprocessed with \gls{mppca} denoising, bias field correction and \gls{epi} distortion correction (eddy for CUDA 9.1, part of \gls{fsl} v.6.0.4).
\Gls{msmt}-\gls{csd}, tractfinder, TractSeg and manual tractography of the \gls{af}, \gls{cst}, \gls{ifof} and \gls{or} were performed as previously described.
Given the disruption caused by the space-occupying and in cases infiltrating lesions, manual tractography using multi-\gls{roi} targeting alone did not produce acceptable results.
Instead, the bundles were manually filtered to remove spurious and implausible streamlines using the same process as was used when curating the atlas training data.
These filtered bundles are used as reference segmentations for the 10 \gls{btc} subjects.

%%%%%%%%%%%%%%%%%%%%%%%%%%%%%%%%%%%%%%%%%%%%%%%%%%%%%%%%%%%%%%%%%%%%%%%%%%%%%%%%
\begin{landscape}
\begin{table}[t]
  \caption{Overview of acquisition parameters for the datasets included in quantitative benchmark evaluation. \dag Resampled from original, see text for details.}
  \label{tab:datasets}
  \footnotesize
  \begin{tabularx}{0.9\linewidth}{l c c c c c c c c} \toprule
             & \multicolumn{2}{c}{GOSH} & \multicolumn{2}{c}{NHNN} & \multicolumn{2}{c}{BTC\autocite{Aerts2018, Aerts2020a}} & \gls{hcp}\autocite{Sotiropoulos2013, Glasser2013} & \textit{TractoInferno}\autocite{Poulin2022} \\
             & pre-op.   & intra-op.      & pre-op. & intra-op.        & pre-op. & post-op.       & & \\
  \midrule%
  %           GOSH pre    GOSH intra          NHNN pre      NHNN intra                  BTC pre  BTC post                   \gls{hcp}           TractoInferno
  $n$ subjects & 3          & 4                & 4      & 4                               & 10    & 9                         & 49         & 71     \\[1em]
  Age range (y)  & \multicolumn{2}{c}{paediatric (3--12, n=3)} & \multicolumn{2}{c}{adult (30--39)} & \multicolumn{2}{c}{adult (39--74)} & adult (22--35) & adult (18--75) \\
             &            & adult (n=1)      &        &                                 &       &                           &            &         \\[1em]
  Diagnosis  & \multicolumn{2}{c}{low-grade astrocytoma (n=3)}  & \multicolumn{2}{c}{oligodendroglioma (n=2)}& \multicolumn{2}{c}{meningioma (n=4)} & healthy & healthy \\
              &           & healthy (n=1)  & \multicolumn{2}{c}{other tumour (n=2)}     & \multicolumn{2}{c}{glioma (n=5)}  &            & \\[1em]
  $b$-values  & 800 (n=1)   & 1000           & 1000     & 1000                       & \multicolumn{2}{c}{0, 700, 1200, 2800} & 1000       & 1000 (n=68) \\
  (s~mm$^2$) & 1000, 2200 (n=2) &          &          &                            &            &                           &            & 700 (n=3) \\[1em]
  $n$ directions & 15 (n=1)     & 30             & 64 (n=3) & 30 (n=3)                   & \multicolumn{2}{c}{8, 16, 30, 50}      & 60\dag     & 21--128 \\
           & 60, 60 (n=2) &                & 61 (n=1) & 3\x{}12 (n=1)               &            &                           &            & \\[2em]
  Voxel size & 1.75\x{}1.75\x{}2.5 (n=1) & 2.5 (n=1) & 2.5 & 2.5\x{}2.5\x{}2.7 & \multicolumn{2}{c}{2.5} & 2.5\dag    & 2.3\dag \\
  (mm)       & 2\x{}2\x{}2.2 (n=2)       & 2.3 (n=3) & & & & & & \\[1em]
  Scanner & Philips Ingenia 1.5T (n=1)  & Siemens  & Siemens & Siemens & \multicolumn{2}{c}{Siemens} & Siemens 3T & variable\\
          &  Siemens Prisma 3T (n=2)    &  Vida 3T  & Trio 3T  & Espree 1.5T                & \multicolumn{2}{c}{Trio 3T}           & ``Connectome Skyra” & \\
          &                          &                &          &                            &                   &                   &   & \\ \bottomrule
  \end{tabularx}
\end{table}
\end{landscape}
%%%%%%%%%%%%%%%%%%%%%%%%%%%%%%%%%%%%%%%%%%%%%%%%%%%%%%%%%%%%%%%%%%%%%%%%%%%%%%%%


\section{Validation: methods}

While a ground truth for white matter tract segmentation is not obtainable \textit{in vivo}, it is nevertheless instructive to compare the tractfinder technique with two other widely adopted methods for a quantitative estimation of reliability and accuracy.
We'll quantitatively evaluate tractfinder in three different datasets: Two large healthy datasets (HCP and TractoInferno, see \ref{sec:data}) and one smaller dataset of clinical neurosurgical acquisitions (NHNN and GOSH combined), together covering a range of acquisition protocols and scanner specifications.

In segmentation tasks, it is common to present a single numeric score of similarity with a ground truth by way of establishing accuracy.
However, in the absence of a ground truth for this particular task, we'll instead aim to present as rounded a picture as possible of the differences and characteristic features of each method through a range of different volumetric distance-based similarity metrics.
The purpose of this validation is therefore not to determine which method is best, as indeed cannot be determined without a reliable reference point, but to highlight the ways in which they are similar, and their characteristic tendencies.

\subsection{Benchmark methods}

We'll consider three alternative segmentation approaches and compare each with the proposed method: Probabilistic streamline tractography, representing the current standard, the deep learning direct segmentation technique TractSeg, and a ``naive" atlas registration.

\paragraph*{Streamline tractography}

Targeted probabilistic streamline tractography (iFOD2 algorithm\autocite{Tournier2010}, from MRtrix3\autocite{Tournier2019} v3.0.3)  was run in each scan using a multi-ROI approach (see \ref{sec:rois} for ROI details), with tractography input \glspl{fod} derived from \gls{msmt} \gls{csd} \autocite{Jeurissen2014} with white matter and grey matter tissue compartments.
In the clinical dataset, ROIs were placed manually for each subject.
For 193 HCP and TractoInferno subjects, manual ROI placement was infeasible.
Instead the same ROIs were drawn in MNI152 template space aided by the FSL HCP-1065 \gls{dti} template\autocite{FSLATLAS} and transformed to subject space using non-linear registration
(HCP data includes MNI transformation warps, while warps were created for the TractoInferno data using the ANTs registration package v2.4.2 (http://stnava.github.io/ANTs/).\autocite{Tustison2013,Avants2011}).
This in-house tractography is subsequently abbreviated to ``TG", while the reference TractoInferno bundles are referred to as ``TGR".
\note{same for btcd?}

\paragraph*{TractSeg}

TractSeg \autocite{Wasserthal2018} is a deep learning tract segmentation model which produces volumetric segmentations for 72 tracts directly from fibre orientation distribution peak directions (TractSeg v2.3-2.6, available at \url{https://github.com/MIC-DKFZ/TractSeg}).
There are two models available: one (``DKFZ") trained on modified streamline reconstructions using TractQuerier \autocite{Wassermann2016} as described in \textcite{Wasserthal2018}, and a second (``XTRACT") trained on streamline density maps output by FSL's XTRACT application. \autocite{Warrington2020}
Here both versions are compared, as they feature significant differences in anatomical tract definition.
Input peaks were derived from the SSST \gls{csd} \glspl{fod}.

\paragraph*{Atlas registration}

As well as the full tractfinder method (atlas alignment and inner product), we compared our results with a ``naive" tract atlas approach, taking only the density component (first SH coefficient) of the linearly registered tract atlases without any comparison with the native \glspl{fod}.
This amounts to a segmentation based on prior spatial expectation only, without taking into account the diffusion data.

\subsection{Comparison metrics}

Each technique is compared against the others, rather than designating any single technique as ``ground truth", with the exception of the TractoInferno dataset, where the published streamlines are regarded as a sort of independent reference \note{still problematic}.
Several comparison metrics are computed, to capture different kinds of agreement between segmentations.
The Dice-Soerensen similarity coefficient (\gls{dice}) \autocite{Dice1945} is a popular, symmetric measure of segmentation similarity given by

\begin{align}
  DSC &= \frac{2 |A \cap B|}{|A| + |B|} \\
\end{align}

for two binary voxel sets $A$ and $B$.
Since \gls{dice} is a measure for binary segmentations, it requires the thresholding of continuous-valued maps such as track density maps and the pseudo-probability maps produced by tractfinder, which is unideal for this task \note{?}.
Firstly, the conversion from continuous-valued to binary representation introduces a high degree of ambiguity over the appropriate choice of threshold value.
While the simplest and least ambigiuous approach may be to include all voxels with value $>0$ in the segmentation, this makes little sense in practice.
In the case of tractography, a small number of rogue false positive streamlines can massively increase the extent of the binary segmentation, and in the case of TractSeg, very few voxels actually are assigned a probability of 0.
The following thresholds were used throughout, wherever binary segmentations are concerned, are given in Table \ref{tab:thresh}.

\captionof{table}{\label{tab:thresh}}
\begin{tabularx}{\textwidth}{X c X}
  Method    & Threshold value & Reasoning \\
  \hline
  Tractfinder   & 0.05 & empirically determined \\
  Tractography (streamline density) & 10 & enough to exclude false positives \\
  Reference tractography (TractoInferno only) & 0 & Assume no false positives \\
  Atlas         & 0.1 & values are roughly double those of tractfinder \\
  TractSeg      & 0.5 & consistent with default TractSeg behaviour \\
  \hline
  \vspace{\baselineskip}
\end{tabularx}

Secondly, the binary nature of a \gls{dice} discounts the additional confidence information present in density and probability maps.
In addition to the binary \gls{dice} measure, therefore, we consider two candidates for a measure of agreement between two continuous valued segmentations.
The first is a generalisation of the \gls{dice} (\gls{gdice}):

\begin{align}
  gDSC &= \frac{2 \sum_i \sqrt{a_ib_i} }{\sum_ia_i + \sum_ib_i}
  =  \frac{2 \sum_i \sqrt{a_ib_i} }{||\mathbf{a}||_1 + ||\mathbf{b}||_1}
\end{align}

The density correlation metric provides an alternative option:
it is simply the Pearson correlation coefficient between the two sets of voxel values.
After considering computing and considering both the \gls{gdice} and density correlation for this analysis, it was determined that one did not provide any further information over the other.
Since the density correlation has been used in other studies for comparing streamline density \note{cite}, we will not consider the \gls{gdice} further in this evaluation.

In addition to the volumetric overlap and density metrics, the volumetric bundle adjacency as defined in \textcite{Schilling2021a} is also measured.
However, to avoid confusion with the streamline-based bundle adjacency\autocite{Radwan2022, Garyfallidis2012, Rheault2022} metric previously defined in \textcite{Garyfallidis2012},
and to give more intuitive meaning to the obtained values, we will refer to it as bundle distance $BD$.
It is computed by taking the mean of minimum distances from every non-overlapping voxel, in each segmentation, to the closest voxel in the other segmentation (Fig. \ref{fig:BD}).
Finally, to give a sense of whether the boundary of one segmentation is within or outside that of a second segmentation, we'll also consider the \textit{signed} bundle distance $BD_s$.
This metric is asymmetric, with $BD_s (A,B) = -BD_s(B,A)$.
Thus $BD$ and $BD_s$ are defined as

\begin{align}
  BD(A,B) &= \frac{\sum_{i \in A\setminus B} d_i(B) + \sum_{i \in B\setminus A} d_i(A)}{|A\Delta B|} \label{eq:bd} \\
  BD_s(A,B) &= \frac{\sum_{i \in A\setminus B} - d_i(B) + \sum_{i \in B\setminus A} d_i(A)}{|A\Delta B|} \label{eq:bds}
\end{align}

where $| \cdot |$ denotes set cardinality and $d_i(X)$ is the Euclidean distance transform (defined relative to the foreground of segmentation $X$, i.e. $d_i(X) = 0$ when $i \in X $ and $d_i(X) = |\overrightarrow{ij}|$ when $i \not\in X$ and where $j \in X$ is the voxel in $X$ closest to voxel $i$)  of segmentation $X$ at voxel $i$.

\begin{figure}[htbp!]
  \centering
  \includegraphics[width=0.3\textwidth]{segmentations_distance_euc.pdf}
  \caption{Illustration of regions involved in calculating bundle distance metric. Light grey is $A\setminus B$, dark grey area is $B\setminus A$. To compute bundle distance $BD(A,B)$ (Eq. \ref{eq:bd}), the mean minimum absolute distance to the intersection (solid black) is taken across all voxels in the two grey areas $BD(A,B) = (14+4\sqrt{2}+3\sqrt{5})/17 = 1.55$. To compute the signed bundle distance $BD_s(A,B)$ (Eq. \ref{eq:bds}), distance values in A are negated. $BD_s(A,B) = (2-2\sqrt{2}-\sqrt{5})/17 = -0.18$. The Dice score for these two segmentations would be $DSC = 2*4/(13+12) = 0.32$}
  \label{fig:BD}
\end{figure}

\section{Validation: results}

\subsection{Processing times}

Atlas transformation and inner product computation time per subject for all three \note{four!} tracts and both hemispheres was $18\pm5 s$, plus 1-2 minutes for MSMT-\gls{csd} and 20 seconds for MNI registration.
For TractSeg (DKFZ or XTRACT), mean processing time (for all tracts, 72 for DKFZ and 23 for XTRACT, both hemispheres) was 4:00$\pm$1:00 $min$, plus $15-20 s$ for SSST-\gls{csd}.
For a full processing time breakdown see Table \ref{tab:time}.

For manual streamline tractography, processing time was not explicitly measured, due to the high variability that comes with manual ROI drawing (between 10--25 minutes for all tracts in a single subject, although this varies significantly between operators).
\Gls{hcp} and TractoInferno tractography was run on a high performance computing cluster, taking approximately 10s per tract (single hemisphere), using 36 CPU cores, and additionally up to 2 minutes for non-linear ROI registration (Table \ref{tab:time}).
However, since the time taken depends greatly on several factors, including number of streamlines to select and streamline acceptance rate (often low in pathological brains due to oedema, deformation etc.), a detailed time analysis for manual tractography is not provided here.

%%%%%%%%%%%%%%%%%%%%%%%%%%%%%%%%%%%%%%%%%%%%%%%%%%%%%%%%%%%%%%%%%%%%%%%%%%%%%%%%
\begin{table*}[htp]
  \caption{Measured processing times mean and standard deviation for TractoInferno dataset. Individual steps shown and total average for the four different pipelines. Note that the tractography pipeline was partially run on a high performance computing cluster, so the reported total time is not representative of a typical setup. Further note that for the present study, tractography ROIs were drawn once for the whole dataset, whereas for clinical datasets manual ROI delineation will have to be repeated for each subject. \dag Desktop Mac with 4 GHz Quad-Core Intel Core i7 \ddag High performance computing cluster, 1 node per subject, 36 Intel(R) Xeon(R) Gold 6240 CPU @ 2.60GHz cores per node.}
  \label{tab:time}
  \small
  \begin{tabularx}{\textwidth}{+>{\raggedright}X ^c ^>{\sffamily}c ^>{\sffamily}c ^>{\sffamily}c ^>{\sffamily}c}
    \rowstyle{\rmfamily}
    Step & Processing time (per subject) & tractfinder & TractSeg & Atlas & tractography \\
    \hline
    \dag Brain masking & 3 $\pm$ 2 s & x & x & x & x\\
    \dag Affine MNI registration & 20 $\pm$ 4s & x &  & x &  \\
    \dag Response function & 5 $\pm$ 3 s & x & x & x & x\\
    \dag MSMT CSD & 01:50 min $\pm$ 55 s & x &  & x & x\\
    \dag SSST CSD + peaks estimation & 18 $\pm$ 8 s &  & x &  &  \\
    \dag Atlas transformation + inner product (3 tracts) & 18 $\pm$ 5 s & x &  & (x) &  \\
    \dag TractSeg (72 / 23 tracts) & 04:00 $\pm$ 01:00 min &  & x & & \\
    \dag Manual ROI delineation (once for whole dataset) & 20:00 min & & & & x \\
    \ddag Non-linear ROI registration + tractography (3 tracts, 2 hemispheres) & 4:05 $\pm$ 2:08 min & & & & x \\
    \rowstyle{\bfseries\rmfamily}
    Total &  & 2:36 min & 4:25 min & \textless2:36min & $\gtrsim$26:03 min
  \end{tabularx}
\end{table*}
%%%%%%%%%%%%%%%%%%%%%%%%%%%%%%%%%%%%%%%%%%%%%%%%%%%%%%%%%%%%%%%%%%%%%%%%%%%%%%%%


% Fullpage figures
\begin{figure}[htb!]
  \begin{subfigure}{\textwidth}
    \includegraphics{chapter_4/tractfinder_cst.png}
    \caption{}
    \label{}
  \end{subfigure}
  \begin{subfigure}{\textwidth}
    \includegraphics{chapter_4/tractfinder_or.png}
    \caption{}
    \label{}
  \end{subfigure}
  \caption{Lightboxes for the projection bundles CST and OR}
  \label{fig:lbcstor}
\end{figure}
\begin{figure}[htb!]
  \begin{subfigure}{\textwidth}
    \includegraphics{chapter_4/tractfinder_af.png}
    \caption{}
    \label{}
  \end{subfigure}
  \begin{subfigure}{\textwidth}
    \includegraphics{chapter_4/tractfinder_ifo.png}
    \caption{}
    \label{}
  \end{subfigure}
  \caption{Lightboxes for association bundles AF and IFO}
  \label{fig:lbcstor}
\end{figure}

\clearpage
\subsection{Qualitative evaluation}

Qualitative results can be seen in Figures \note{fig:SURFACE}, \note{fig:lb.cst}-\note{fig:lb.clin}.
The raw tract maps typically have values ranging from 0 to 0.5 (in arbitrary units, derived from the magnitudes of \gls{fod} and atlas distribution functions).
Due to the combined effects of \gls{odf} amplitude and orientation information, a low tract map value can have several causes: a) the \gls{fod} amplitude is low, indicating low evidence for white matter tissue in the voxel in question; b) the atlas amplitude is low, indicating low prior likelihood of the tract being present in that location; c) the peak orientations between the \gls{fod} and atlas are poorly aligned.

Thus combining information from the atlas and data-derived \gls{fod}s improves the tract map estimation over the ``raw" registered atlas in both the spatial and orientational domain. For example, the \gls{tod} atlases have poor definition of gyri and sulci, due to the effect of averaging over many subjects and linear registration. The reduced overall \gls{fod} amplitude in grey matter corrects this non-specificity. And in regions where different white matter structures lie in close proximity, where the atlas can erroneously predict the likely presence of the tract, and \gls{fod} amplitude is high, the lack of orientational agreement discounts the presence of the tract of interest in that location.

Qualitative results for a representative subject (identified as the only subject within the top 30 smallest deviations from the mean scores for all three of bundle distance, binary \gls{dice} and density correlation) are shown in Figures \note{fig:lb.afr}-\note{fig:lb.or}. \note{tractoinferno data}

\subsection{Quantitative results}

\begin{figure}[h!]
  \centering
  \includegraphics[width=\textwidth]{chapter_4/all_metrics_by_tract_tractoinferno+hcp.png}
  \caption{All methods compared against multi-ROI targeted tracotgraphy for the hcp and tractoinferno datasets combined (except TGR, which is tractoinferno data only)}
  \label{fig:combobox}
\end{figure}

\paragraph*{Performance}

In a realistic clinical context, our target segmentation is represented not by an independently determined and verifiable ground truth, but by the results of whatever approach would normally be taken to produce tract reconstructions for surgical guidance in the absence of any suitable alternative.
That benchmark is targeted multi-ROI streamline tractography, against which we need to evaluate tractfinder.
At the same time, quantitative evaluation using tractography as a reference \note{is difficult} as with typical use tractography will produce many false positives that are easily mentally discounted by an experienced viewer, but which will confound quantitative accuracy metrics.

This is one reason why the density correlation metric is particularly useful for comparing methods in this task: \note{place after metrics discussion?}
False positive streamlines are more likely to be apparent in areas of low streamline density, and if the proposed segmentation correspondingly predicts a low probability in the same areas, then this will be consistent with a high correlation value.
The density correlation thus helps illustrate the cases where the choice of threshold may have a disproportionate influence on subsequent binary comparisons.
For example, in the HCP dataset and for the corticospinal tract, mean binary \gls{dice} was $0.69$ between tractfinder and tractography and $0.51$ between TractSeg (XTRACT) and tractography (a difference of $0.18$).
For the same two comparisons, the density correlations differed only by $0.04$ ($0.63$ and $0.59$) respectively, indicating strong agreement around areas of high confidence and divergence only in peripheral areas of lower density values.

The signed bundle distance gives an indication of the nature of disagreement between two techniques where other metrics show little difference.
For example, in the HCP dataset and for the arcuate fasciculus, mean bundle distance between the naive atlas and tractography was $5.45 mm$ and mean bundle distance between TractSeg (DKFZ) and tractography was very similar at $5.41 mm$ (Tab. \note{tab:DATAHCP}).
However, the signed bundle distances for those same two comparisons were $+2.57 mm$ and $-2.68 mm$ respectively.
This indicates that, while if only considering the bundle distance metric, both TractSeg and the atlas appear to agree to a similar degree with tractography, TractSeg actually systematically over-segments the \gls{af} (relative to tractography), while the naive atlas segmentation tends towards under-segmentation.

With all this in mind, we find that tractfinder does indeed reliably perform well against tractography across all metrics, with the exception of the \gls{af} when measured on \gls{dice} or distance metrics.
There are a couple possible factors which may account for the worse perfomance for the \gls{af}.
One is the difficulty in consistently reconstructing this tract using only white matter based ROIs (as opposed to cortical ROIs which necessitate more extensive image preprocessing), which are unable to adequately constrain the streamlines as they fan out into numerous cortical regions.
Supporting this theory, tractfinder returned the highest standard deviation values for the arcuate fasciculus out of all \note{three} tracts across all metrics except the signed bundle distance.
It should be noted that, compared to tractography, TractSeg also returns higher standard deviations for the \gls{af}, pointing to an inter-subject variability in tractography segmentations, rather than an high degree of inconsistency in tractfinder performance.
We can also see that the signed bundle distance is significantly higher on average for the \gls{af}, meaning tractfinder typically undersegments the tract relative to tractography, which is to be expected with large volumes streamlines fanning to adjacent cortical termini not included in the curated tract atlas.
Another presumed contributing factor is the unique shape of the arcuate fasciculus, which includes a tight bend around the top of the sylvian fissure.
Alignment of an atlas with this shape is significantly affected by individual anatomy \note{??} which may also partly explain the lower average accuracy for this tract.

Notwithstanding the slightly worse accuracy for the \gls{af}, on balance we can see that tractfinder returns consistently strong agreement with tractography across all metrics.
The fact that comparing the expectations with the native diffusion data, rather than simply registering an atlas, refines and improves the segmentation is also borne out. \note{bit weak}
We also see a high degree of consistency in the levels of agreement between tractfinder and the benchmark methods (\note{figure out how to convey this graphically, alternative to those dice matrices}).
There is little variation in the comparison metrics across subjects, and the overall patterns also remain consistent between the different datasets, both healthy and clinical, which feature a range of aquisitions and ages.

\paragraph*{Tract variability}

Visual assessment reveals that \note{persistent} differences in the shapes of the segmented tracts accounts for a large part of the discrepancy between methods.
Again, this is most apparent in the arcuate fasciculus, where anatomical definitions differ widely (Fig. \note{fig:lb.afr}, \note{fig:lb.afl}).
For example, TractSeg (DKFZ) includes extensive coverage of the frontal and temporal lobe in its \gls{af} segmentations, including parts of the primary motor cortex.
Dice scores between different methods are low across the board for the \gls{af} owing to these anatomical disagreements.
Conversely in the corticospinal tract, which has a relatively well agreed-upon domain, segmentation results have much higher volumetric agreement between methods. \note{need graphic to show this}
Agreement in the optic radiations is somewhere in between, with slightly lower \glspl{dice} compared to the two TractSeg methods, which tend to include more thalamus and a lesser extent of Meyer's loop.
These differences highlight the difficulty in assessing the ``accuracy" of white matter segmentation methods given limited consensus on the precise anatomical definitions of different pathways.

% Figure \ref{fig:combobox} compares each studied method against the reference streamline bundles in the TractoInferno dataset.
% Noticeably, the differences in scores within a single method, between different tracts, are in places greater than the differences between methods within a tract.
% For example, the binary \gls{dice} scores for the \gls{cst} are similar for tractfinder and TractSeg (DKFZ) ($0.48$ and $0.45$ on average respectively), however the binary \gls{dice}s of TractSeg (DKFZ) are markedly different between the \gls{cst} and \gls{or} ($0.45$ and $0.59$ on average respectively). % This is all based on the TGR comparison, which have decided isn't very helpful anyway. Against the TG comparison, in HCP and TI data, this observation goes away

\paragraph*{Reproducibility}

As a final word on repoducibility, we note that the results in Figure \note{fig:METRICSBOXPLOTS} are consistent with the comparisons between TractSeg and RecoBundles published in \textcite{Wasserthal2018}.
There, a mean \gls{dice} of between 0.58 and 0.67 across all tracts was reported.
Our measured \gls{dice}s between TractSeg (DKFZ) and reference tractography (which is based on RecoBundlesX\autocite{Garyfallidis2018}) range between 0.45 and 0.59 across the three tracts studied (Tab. \note{tab:DATATI}).

\paragraph*{TractoInferno}

% \textit{\gls{dice}, \gls{gdice} and density correlation values for tractfinder were on par with TractSeg (XTRACT) in all three tracts, with the exception of density correlation in \gls{af}, while \gls{gdice} and density correlation were higher than TractSeg (DKFZ) in all tracts.
% Binary \gls{dice} scores were highest for TractSeg (DKFZ) in the \gls{cst} and \gls{af}, and equal between tractography, tractfinder and TractSeg (DKFZ)  for the optic radiation.}

While initially we intended to make use of the published TractoInferno streamlines as an unbiased reference for comparison, it became apparent that they were unsuited for this application.
Regardless of tract, metric, or compared methods, the tractofinderno reference streamlines yielded extremely variable results with large numbers of outliers.
Further investigation into these outliers revealed numerous subjects with incomplete or highly asymmetric bundles.
For example, in several cases, optic radiation streamlines only reach the superior portion of the occipital lobe (Fig. \note{fig:DUDSOR}).
In others, the right arcuate fasciculus is significantly smaller than the left (Fig. \note{fig:DUDSAF}).

Given these inconsistencies it was decided to leave the tractoinferno reference streamlines out of any critical analysis on the relative performances of different segmentation approaches.
The tractoinferno dataset features 284 subjects in total (135 were used for the present analysis), and was published with the intention of providing a large and high quality dataset of reference streamlines explicitly for the purpose of training deep learning models for improved tractography and other data-intensitve applications.
The comparatively \note{low quality to harsh} of the bundles is evidence of the difficulty in producing and quality controlling \note{?} streamline tractography bundles with a high degree of anatomical fidelity and inter-subject consistency.
The reliance on such large datasets thus presents a significant challenge to methods such as deep learning and demonstrates the advantage of an approach like tractfinder, as was seen in (\ref{sec:ntrain}).

\paragraph*{Clinical (GOSH \& NHNN)}

For the present analysis we included clinical scans with non-deforming lesions, meaning the orientation atlas could be registered to the target image using only affine registration without the need for tumour deformation modelling.
For qualitative results in clinical scans featuring deforming lesions, see \note{sec:applications?} %\textcite{Young2022}

Two example clinical subjects, one adult and one paediatric, are displayed in Figure \note{fig:lb.clin}.
In Figure \note{fig:lb.nh}, a sagittal view displays the interaction between the surgical resection cavity and the \gls{cst}.
Here our proposed method maps the \gls{cst} in relatively close proximity to the resection site, where the TractSeg segmentations are far more conservative, potentially missing \gls{cst} locations influenced by oedema or other tumour effects.
In Figure \note{fig:lb.gosh}, the extent of Meyer's loop depicted by tractography is similarly included in the proposed segmentation, but absent from the TractSeg results.

When the results for the clinical dataset were split on hospital / age group (paediatric or adult), no appreciable difference in results was observed (data not shown).
Equally, no systematic difference was observed between intraoperative and preoperative datasets.
The mean score results for all tracts and comparisons are given in Supplementary Table \note{tab:DATACL}.

\subsection{Two methods, same training data}

In order to directly compare tractfinder with the popular deep learning method TractSeg, a new set of tract atlases were created from the TractSeg reference bundles, using the same split into training and testing subjects as used in the deep learning model (published version).
TractSet was trained on 63 subjects and tested on the remaining 42, and the same was done for tractfinder (where ``training" means computing the atlas from individual bundles), even though tractfinder atlases can be constructed from just 10-15 subjects (see \ref{sec:ntrain}).

Tractfinder scores equally well or better on all metrics with the exception of the Dice coefficient, where in particular the scores for the IFOF and OR are lower.
\note{could redo for every tract}

\begin{figure}[htb!]
  \includegraphics[width=\textwidth]{chapter_4/tractfinder_trained_on_tractseg.png}
  \caption{Scores for tractfinder (using atlases constructed from the TractSeg training data) and TractSeg validated on the TractSeg reference bundles.}
  \label{fig:ts_atlas}
\end{figure}

\subsection{The effect of deformation modelling}

The BTC dataset, with its range of tumour types, high quality \gls{dmri} scans and tumour masks, provides an ideal \note{opportunity} to evaluate the effect of tumour deformation modelling on segmentation accuracy.
Four out of the ten selected BTCD subjects had tumours substantial enough to warrant deformation modelling, which was done using \note{the same parameters which ones?? automatic lambda} for all subjects.

\begin{figure}[htb!]
  \includegraphics{chapter_4/btcd_deformation.png}
  \caption{Effect of deformation modelling on segmentation accuracy, compared to tractfinder without deformation modelling, and TractSeg. Each datapoint represents the average across tracts for a single subject. There are only three contralateral datapoints because one of the four subjects had a midline tumour (all tracts considered ipsilateral).}
  \label{fig:btcd_def}
\end{figure}

Figure \ref{fig:btcd_def} shows how tractfinder with deformation modelling improves on average segmentation technique.
Notably in these subjects, where large mass effect was present, TractSeg performed significantly worse than tractfinder (although it should be noted that the reference tractography more closely follows the tract definitions used to train tractfinder).
There are two subjects worth considering in detail, where there is no clear improvement with the addition of deformation modelling.
For subject 11 (highest scores in Fig. \ref{fig:btcd_def}), the superior frontal location of the tumour barely affected any of the studied tracts, hence the overall high score and imperceptible improvement with the addition of deformation modelling.
In subject 26, which in Fig. \ref{fig:btcd_def} is the only subject showing an average decrease in DSC with the addition of deformation modelling, the \gls{cst} and \gls{af} actually saw an increase in scores, while the \gls{ifof} and \gls{or} saw a decrease.
In this case, the deformation modelled in the direct vicinity of the temporal lobe tumour was too strong, leading to slightly worse detection of the \gls{ifof} and \gls{or}.
At a greater distance from the tumour, however, the deformation modelled accurately matched the patient anatomy, leading to improved detection of more distant tracts, the \gls{af} and \gls{cst}.

\section{Data requirements}

\note{Basically the findings from ISMRM diffusion workshop with different data inputs.}

\section{Clinical case studies}

Having discussed some of the technological challenges of intraoperative image acquisition and processing, we will now turn our minds to the other side of the camera and explore the complex reality of tumour radiology and image guided neurosurgery.
The use of tractfinder in the presence of both significant tumour deformation and brain shift is illustrated in an \gls{imri} case from the \gls{nhnn}, first described in \textcite{Young2022}.
A further example, also from the \gls{nhnn}, demonstrates the effect of peritumoural oedema on tract imaging.
Finally, Section \ref{sec:case} will look at two examples of some of the most challenging neurosurgical procedures, involving paediatric tumours of the thalamus, in \gls{imri} cases taking place at \gls{gosh}.

\subsection{Intraoperative brain shift}
\label{sec:imri}

When considering the application of tractfinder to intraoperative imaging, we need to take brain shift into account, which is unpredictable:
Differing effects stem from drainage of fluid, pressure changes, tumour debulking and gravity.
Nevertheless, the aim is to achieve accurate intraoperative tract segmentation while avoiding the need to perform additional tumour or resection segmentation intraoperatively.

As the atlas is designed to be fairly inclusive, with the inner product acting to correct small spatial inaccuracies, it is possible in some cases where brain shift is minimal to reuse the preoperative tumour deformation field.
In cases of significant tumour debulking, the deformation field can be recomputed from the preoperative tumour segmentation by adjusting the value of $s$ in the deformation expression (\ref{eq:forwardP}) to simulate a reduction in tumour volume.
This scenario is demonstrated in Figure \ref{fig:shrink}, showing the resection of a large temporal epidermoid cyst in an adult patient, the same case depicted in an earlier demonstration of brain shift and tractography in Figure \ref{fig:shift}.
Intraoperative imaging shows a significant reduction in cyst volume and the adjacent \gls{cst} has shifted accordingly, resulting in significant discrepancy between preoperative tract reconstruction and intraoperative tract position.
Even so, the pathway is still shifted medially compared to its normal position (which can be clearly seen in comparison with the contralateral tract), such that tumour deformation modelling is still required for accurate segmentation.
By reusing the preoperative lesion segmentation and setting $s=0.8$ (effectively modelling a 20\% smaller tumour), the resulting deformation field is able to capture the approximate location of the shifted tract, without the need for additional tumour segmentation on the intraoperative structural imaging.
Adjusting the value of $s$ and reusing preoperatively computed values of $D_t$ and $D_b$ avoids time and resource-intensive intraoperative lesion segmentation, brain shift modelling, or non-linear registration.

\begin{figure}[h!]
  \centering
  \includesvg[width=\textwidth,pretex=\sffamily]{chapter_6/shrink.svg}
  \caption{Demonstration of successful pre- and intraoperative tract imaging using only preoperative tumour segmentation (blue outline). Setting $s=0.8$ (green outline) simulates intraoperative decompression and models accurate atlas deformation to align with intraoperative anatomy after brain shift. The resulting \gls{cst} tractfinder segmentations are in strong agreement with streamline tractography. An earlier version of this figure was first published in \textcite{Young2022}.}
  \label{fig:shrink}
\end{figure}

\subsection{Tumour deformation and oedema}

In this example case of a \gls{nhnn} patient with a tumour embedded within the right precentral gyrus (tumour histology information is unavailable for this subject) we will consider the combined effects of tumour mass effect and peritumoural oedema on tract reconstruction.
(This subject was not included in the non-deformation clinical dataset described in Section \ref{sec:data} and the quantitative evaluation in Chapter \ref{chap:eval}.)
Not only is there substantial distortion to the \gls{cst} and cortical topology, the tumour is surrounded with oedema, which is affecting diffusion measurements.
Due to the large free water compartment within the \gls{wm}, the usual \gls{msmt} \gls{csd} approach with \gls{wm} and \gls{gm} compartments is unsuitable in this situation.
Instead, \gls{wm} and \gls{csf} compartments were modelled, allowing modelling of \gls{wm} \glspl{fod} within oedema while reducing noise slightly compared to \gls{ssst} \gls{csd}.
After \gls{fod} modelling, the \gls{cst} was reconstructed using probabilistic tractography, tractfinder with tumour deformation modelling and TractSeg (DKFZ).

In Figure \ref{fig:oedema}, panel \textbf{a}, fibre orientations clearly indicate the presence of \gls{csd} fibres circumventing the tumour, yet the presence of oedema has inhibited their detection in both the tractography (panel \textbf{d}) and TractSeg (panel \textbf{e}) reconstructions.
The deformed tract atlas, by contrast, aligns well with the \gls{fod} orientations (panel \textbf{b}) resulting in a successful tractfinder segmentation  (panel \textbf{c}).
This example highlights the particular advantages of tractfinder:
All three methods utilised the same \glspl{fod} as the basis for identifying the tract, which clearly depict the presence of \gls{wm} fibres within the oedematous zone, yet the combination of crossing \gls{af} fibres, altered diffusion, and tract displacement presented a significant barrier to streamlines, with few penetrating to the \gls{cst} beyond.
Whether TractSeg neglected the same region because of deformation, oedema, or a combination of both is difficult to determine.
Meanwhile, the deformed atlas predicted the presence of fibres within the affected region and their expected orientation, both of which are confirmed by the \gls{fod} estimates, resulting in appropriate detection of the tract of interest.
Note how the atlas probability is also high within the tumour itself:
Its prediction alone is not sufficient for accurate segmentation, and a comparison with the local diffusion data is required to correct for imperfect atlas alignment.

\begin{figure}
  \includesvg[width=\textwidth,pretex=\sffamily]{chapter_6/oedema.svg}
  \caption{Adult patient with tumour in the right precentral gyrus. \textbf{a.} \glspl{fod}, \textbf{b.} Deformed \gls{cst} \gls{tod} atlas, \textbf{c.} Tractfinder map of the \gls{cst}, thresholded at 0.05, \textbf{d.} Probabilistic tractography streamlines, \textbf{e.} TractSeg segmentation probability, thresholded at 0.5, \textbf{f.} Coronal and sagittal slices indicating the location of panels \textbf{a.-e.}}
  \label{fig:oedema}
\end{figure}

\clearpage
\subsection{A tale of two paediatric thalamic gliomas}\label{sec:case}

From mid 2021 to late 2023, nine tumour resections were carried out at \gls{gosh} with \gls{imri} guidance that also included multi-directional diffusion-weighted imaging (that is, not trace-weighted or \gls{adc} imaging).
Throughout this period, neurosurgical and radiological staff were continually learning how best to integrate the \gls{imri} facility into their practice.

The cases included two posterior fossa tumours, two midline (thalamic) gliomas and five supratentorial hemispheric tumours.
Of these, \gls{imri} guidance with diffusion tractography is potentially the most useful for thalamic tumours, given their complex functional environment and difficult surgical access.
Both low-grade \autocite{Wong2016} and high-grade\autocite{Dorfer2021} thalamic tumours may be candidates for maximal safe resection, although infiltrating lesions cannot be safely resected completely without serious risk of neurological detriment to the patient.
With advances in surgical navigation, maximal safe resection has become an important consideration in thalamic tumours once considered inoperable\autocite{Souweidane1996,Puget2007,Steinbok2016,Grewal2019,Sunderland2021}.
Children in particular\autocite{Ferroli2023} experience better overall survival rates when thalamic tumours are more radically resected compared to subtotal resection or biopsy\autocite{Cinalli2018}.
The Alder Hey Children's Hospital in Liverpool, UK experienced an increase in substantial resection ($<1.5$cm$^3$ residual) rates from 37\% to 94\% with the introduction of \gls{imri} navigation, without any associated increase in postoperative morbidity\autocite{Sunderland2021}.
Similarly, a review of 38 thalamic tumour patients (paediatric and adult) treated at the Chinese PLA General Hospital in Beijing, China found that the use of \gls{imri} increased \gls{gtr} rates from 42\% to 68\%\autocite{Zheng2016}.
Preservation of the posterior limb of the internal capsule, containing the corticospinal tract, during resection is critical, with motor deficits being the most common functional symptoms of thalamic tumours\autocite{Puget2007, Zheng2016, Palmisciano2021}.
As the specific interactions between thalamic tumours and surrounding \gls{wm} are complex and unpredictable, as we will see in the following case studies, \gls{dmri} \gls{wm} imaging is especially valuable for safely planning and conducting resections\autocite{Celtikci2017}.

The following will look at two of the \gls{gosh} diffusion \gls{imri} cases in detail, beginning with the most recent, considering the imaging and tumour features, peri-operative clinical presentation, and neurology.
Hopefully this will provide a balanced perspective into the challenges and potential of intraoperative \gls{dmri} for such cases.

\paragraph*{Patient 8: High grade glioma}

The first of the two patients presented at seven years of age with a diffuse midline astrocytoma (\gls{who} grade 4) also in the left thalamus.
A strong degree of mass effect called for tumour deformation modelling, so the lesion was segmented on the preoperative structural scan (Fig. \ref{fig:8p}).

\begin{figure}[htb!]
  \centering
  \includegraphics[width=\textwidth]{chapter_6/case_studies/gosh_8_preop_t1.png}
  \includegraphics[width=\textwidth]{chapter_6/case_studies/gosh_8_preop_def.png}
  \caption{Preoperative $T_1$-weighted imaging (top) of patient 8 with high-grade left thalamic glioma. Bottom: Tumour deformation modelling in a co-registered MNI template image. No \gls{dmri} acquired preoperatively was available.}
  \label{fig:8p}
\end{figure}

The diffuse and infiltrative nature of this tumour made identifying the posterior capsule, thalamus and tumour margins exceedingly difficult.
Intraoperative \gls{dti} was requested by the radiologist after extensive discussion of the already acquired conventional contrast scans, from which the extent of internal capsule infiltration was indiscernible (Fig. \ref{fig:8i}).
\Gls{dti} \gls{dec} maps showed a largely intact and slightly displaced internal capsule with extensive tumour infiltration, so the decision was made not to attempt further resection of that part of the tumour.
Roughly half of the lesion was resected, predominantly in the ventral and medial parts.
Postoperatively the patient's right side weakness was slightly improved and the patient received adjuvant radiotherapy after confirmation of high-grade histology with H3K27M mutation.
Over the following weeks the patient's hemiparesis improved slowly, consistent with the decompression of the \gls{csd} and the tumour's infiltrative nature.

\begin{figure}
  \centering
  \includegraphics[width=\textwidth]{chapter_6/case_studies/gosh_8_iop_t1.png}
  \includesvg[width=\textwidth]{chapter_6/case_studies/gosh_8_iop_fa.svg}
  \caption{Intraoperative imaging showing transcallosal approach to patient 8's thalamic glioma. Top: $T_1$-weighted imaging. Bottom: \gls{dti} \gls{dec} \gls{fa} map. Arrowheads indicate partially infiltrated and displaced internal capsule.}
  \label{fig:8i}
\end{figure}

Identification of the internal capsule and assessment of its condition on intraoperative imaging may have been significantly simplified with advanced tract-specific \gls{dmri} analysis.
Interpreting colours on the directional \gls{fa} map is complicated by crossing fibres and disturbed diffusion caused by infiltration and oedema.
What is unclear is the state of the neuronal fibres in the infiltrated portion: Are they intact, distorted, or destroyed?
Visualising the fibre \glspl{odf} using \gls{csd} gives us further insight (Fig. \ref{fig:8i_fod}).

\begin{figure}
  \centering
  \includegraphics[width=0.33\textwidth]{chapter_6/case_studies/gosh_8_close_tensor.png}\,%
  \includegraphics[width=0.33\textwidth]{chapter_6/case_studies/gosh_8_close_csd.png}\,%
  \includegraphics[width=0.33\textwidth]{chapter_6/case_studies/gosh_8_close_atlas.png}
  \caption{Magnified coronal view of partially resected tumour and internal capsule and modelled fibre orientations. \textbf{a.} \gls{dt}. \textbf{b.} \gls{csd} \glspl{fod} (\gls{msmt} \gls{csd} with \gls{wm} and \gls{csf} compartments). \textbf{c.} Deformed \gls{tod} \gls{cst} atlas. While the main body of the tract is well defined in all three images, the boundaries between intact fibres, infiltrated \gls{wm}, and tumour are indiscernible.}
  \label{fig:8i_fod}
\end{figure}

Using a tumour deformation field modelled from the preoperative scan, tractfinder reconstructed the corticospinal tract in very close agreement with probabilistic streamline tractography.
Both indicate a largely intact bundle slightly laterally displaced, although neither can necessarily rule out the presence of infiltrated fibres closer to the tumour, as the expected resulting decrease in anisotropy would produce fewer streamlines through that region and reduced tractfinder probability.
Surprisingly, TractSeg was unable to successfully detect the \gls{cst} in the affected hemisphere, despite the relatively manageable amount of distortion.
It is unclear what led to failure in this particular subject, as TractSeg has shown some success in cases with similar tumour deformations\autocite{Moshe2022}.

\begin{figure}
  % 5~mm slice increments
  \includegraphics[width=\textwidth]{chapter_6/case_studies/gosh_8_iop_tg.png}
  \includegraphics[width=\textwidth]{chapter_6/case_studies/gosh_8_iop_tf.png}
  \caption{Coronal slices in 5~mm increments with \gls{cst} reconstructions on intraoperative imaging for patient 8. Top: Probabilistic streamline tractography, Bottom: Tractfinder \gls{cst} with tumour deformation modelling based on preoperative tumour segmentation.}
  \label{fig:8i_tf}
\end{figure}

\paragraph*{Patient 5: Low grade pilocytic astrocytoma}

A second patient presented with a low-grade pilocytic astrocytoma in the left thalamus at 23 months of age with progressive right-sided weakness.
On preoperative \gls{mri} examination, a sliver of tissue within the tumour, presumed to be a part of the \gls{cst}, was evident, while the rest of the tract was displaced (Fig. \ref{fig:5p}).
This being a low-grade and well-encapsulated tumour,

\begin{figure}[htb!]
  \centering
  \includesvg[width=\textwidth]{chapter_6/case_studies/gosh_5_preop_t1.svg}
  \includesvg[width=\textwidth]{chapter_6/case_studies/gosh_5_preop_fa.svg}
  \caption{Preoperative radiological presentation of \gls{gosh} patient 5, with a low-grade astrocytoma of the left thalamus. Top: $T_1$-weighted scan. Bottom: \gls{dti} \gls{dec} \gls{fa} map. Arrowheads indicate the \gls{cst}, partially located within the tumour.}
  \label{fig:5p}
\end{figure}

Confirming the location of the \gls{cst}, divided into an intratumoural and a displaced portion, was particularly difficult in this case.
A \gls{dmri} sequence with \gls{dec} visualisation was able to confirm that the strip of tissue inside the tumour indeed constituted part of the \gls{cst}.
Streamline tractography also depicted the \gls{cst} as both within and displaced posteriorly around the tumour.
The part of the pathway involved was at the cerebral peduncle level, which even in controls appears as a relatively narrow bundle.
Consequently, the tractfinder atlas was unable to account for both the tumour-engulfed section, which remained spatially relatively unmoved from a normal position, and the posteriorly displaced portion simultaneously.
Basic tract mapping reconstructed the first part, while additional tumour deformation modelling enabled detection of the second part (Fig. \ref{fig:5p_cst}).
Of course, such an \textit{ad hoc} solution would be entirely impractical and difficult to interpret, compared to tractography which is better at exploring the available pathways regardless of prior anatomical expectations.
As such, this is a case in which tractfinder was unsuited to the complex anatomy and tumour deformation effects at hand.

\begin{figure}[htb!]
  \centering
  \includegraphics[width=\textwidth]{chapter_6/case_studies/gosh_5_preop_tg_5mm.png}
  \includegraphics[width=\textwidth]{chapter_6/case_studies/gosh_5_preop_tf_5mm.png}
  \caption{Reconstruction of the left \gls{cst} on preoperative imaging. Top: Streamline tractography \gls{tdi} map, thresholded at ten streamlines. Bottom: Tractfinder with and without deformation modelling merged into a single segmentation.}
  \label{fig:5p_cst}
\end{figure}

Due to this complicated involvement of the internal capsule a debulking resection under \gls{imri} guidance was indicated to relieve pressure on the \gls{cst}.
After a large part of the tumour was removed, the patient was brought through to \gls{imri} (Fig \ref{fig:5i}) to confirm maximal safe resection with some residual tumour remaining.
There was concern of an area of infarction involving the \gls{cst}, however the patient's hemiparesis, after experiencing worsened motor deficit immediately postoperatively, improved over the following days after decompression of the motor fibres.
The patient went on to receive adjuvant chemotherapy and showed further improvement over longer term follow-up, their hemiparesis returning to preoperative levels.

\begin{figure}[htb!]
  \centering
  \includegraphics[width=\textwidth]{chapter_6/case_studies/gosh_5_iop_t1.png}
  \includegraphics[width=\textwidth]{chapter_6/case_studies/gosh_5_iop_fa.png}
  \caption{Intraoperative $T_1$-weighted (top) and \gls{dec} \gls{fa} (bottom) for patient 5, showing left craniotomy and surgical corridor through the temporal lobe}
  \label{fig:5i}
\end{figure}

\begin{figure}[htb!]
  \centering
  \includegraphics[width=\textwidth]{chapter_6/case_studies/gosh_5_iop_tg_5mm.png}
  \includegraphics[width=\textwidth]{chapter_6/case_studies/gosh_5_iop_tf_5mm.png}
  \caption{Reconstruction of the \gls{cst} on intraoperative imaging using tractography (top) and tractfinder (bottom) with standard affine registration.}
  \label{fig:5i_cst}
\end{figure}

Reconstruction of the \gls{cst} on intraoperative imaging again proved difficult.
Here, significant brain shift away from the craniotomy site prevented accurate atlas registration, resulting in a mismatch in anatomical alignment between atlas and target image (Fig \ref{fig:5i_cst}).
Non-linear registration, with harsh penalties on strong local deformation to reduce overfitting around the resection surface, could mitigate this, but would be impractical for routine intraoperative use.


\epipage{It is good to have an end to journey towards, but it is the journey that matters, in the end.}{Ursula K. Le Guin, \textit{The Left Hand of Darkness}}
\chapter{Conclusions}

The growing availability of advanced \gls{mri} capabilities in health centres is bringing attention to the shortcomings of current image processing techniques in fully exploiting potential benefits for patients.
In a small internal survey of five neuroradiologists and neurosurgeons at \gls{gosh}, all respondents expressed that they would find a tool for intraoperative imaging of \gls{wm} bundles after brain shift to be either slightly (2/5) or very (3/5) useful, while confirming reservations about the reliability or accuracy of tractography.
Against this background, this thesis has set out to explore the current capabilities of \gls{dmri} to map brain \gls{wm} for surgical planning and neuronavigation, and propose a novel technique to fulfil the requirements for rapid and robust \gls{wm} tract detection.
Guided by the objectives set out in Chapter \ref{chap:problem}, the proposed pipeline, called tractfinder, involves constructing tract-specific orientation atlases which are compared with a subject's \gls{fod} image, to achieve direct voxel-wise segmentation with incorporated \textit{a priori} anatomical knowledge.
Each atlas is constructed from meticulously filtered training streamlines based on available consensus neuroanatomical definitions which are then mapped to voxel-wise orientation distributions using \gls{tod} imaging and averaged over the training population (Chapter \ref{chap:atlas}).
Upon registration with the target image, the tract's location is estimated by comparing the voxel-wise tract orientation and spatial priors with the \gls{fod} modelled from \gls{dmri} data, which can be achieved by taking the inner product of the two spherical distributions.
If large tumour deformation effects are present in the image, then the atlas is adjusted accordingly with an exponential radial deformation model (Chapter \ref{chap:reg}) prior to computing the tract map.
Through detailed evaluation against benchmark methods presented in Chapter \ref{chap:eval}, tractfinder has been shown to produce consistent and accurate segmentations at a standard comparable with streamline tractography and deep learning, issues with variable tract definitions and reference data quality notwithstanding.

Several design choices in the tractfinder pipeline are supported by the objective to keep point-of-application processing time and user-interaction to a minimum.
The anatomical priors, which would for tractography either be drawn by hand or automatically via cortical parcellation or deformable registration, are provided by the tract atlases which also account for a degree of inter-subject variability.
As we saw in Chapter \ref{chap:reg}, only linear registration is required to align atlas and subject data, which is faster and more robust than the non-linear or deformable registration that would be necessary for accurate segmentation using registration alone without comparison with subject diffusion data.
It is also applicable to challenging clinical data without need for manually adjusting registration parameters, where non-linear algorithms can fail to reach stable convergence.
Using the \gls{sh} basis for representing orientation distribution data allows for efficient inner product computation, and could flexibly support alternative comparison metrics.
Overall, the full pipeline can be run in under five minutes for a single tract including minimal preprocessing, with the only potential need for user input being the choice of tumour deformation model parameter $\lambda$, if the default adaptive value produces suboptimal results.

Given that tumour infiltration and oedema affect diffusion measurements and thus all downstream processing and modelling, we cannot claim that these effects do not impact tract mapping using tractfinder.
However, a voxel-wise approach is not susceptible to a compounding of errors in the same way streamline tractography is, where oedema in one part of the tract can derail tracking for the entire bundle, including parts not directly affected by infiltration.
Where disturbed diffusion does result in lower tractfinder values, they can be interpreted in the context of other imaging and even provide clinically useful information about \gls{wm} integrity.
By contrast, we have seen examples of the deep learning method TractSeg failing to fully recognise tracts disturbed by tumours (Sections \ref{sec:quant}, \ref{sec:case}), displaying a lack of explainability that is unacceptable for effective clinical translation.
In this way, the requirement of a clinically applicable and robust method is achieved, although this has yet to be rigorously verified in a prospective study (see Section \ref{sec:future}).

A final stipulation was to keep reference data requirements for atlas creation to a minimum, to accommodate evolving neuroanatomical definitions and the difficulty in producing high quality reconstructions.
With only 16 training subjects used for the original atlases, and subsequent analysis presented in Section \ref{sec:ntrain} indicating that as few as ten may be sufficient, tractfinder has a significant advantage over more data-intensive statistical and deep learning alternatives.

The work described in this thesis came about during a time when increasingly sizeable contingents of the research and medical communities are focussing minds on the potential of machine learning techniques to disrupt hitherto intractable problems.
In the \gls{wm} imaging space, machine learning is gaining traction both for direct segmentation methods and as a means for finally overcoming the fundamental roadblocks in streamline tractography which have entangled researchers for years.
Against this backdrop of excitement for new algorithmic and big-data possibilities, proposals for a new atlas-based technique have been met with some skepticism.
Nevertheless, just as streamline tractography has long delivered astonishing benefits while simultaneously remaining unable to expel its sensitivity-specificity trade-off gremlins, so too are the difficulties of bringing deep learning solutions to real clinical translation beginning to show.
Fulfilling the need for large volumes of accurately annotated data may be easy where the resources for producing said data are freely available, or at least justified if the application is a well-defined and static problem.
Tract segmentation is neither of those things:
Producing the ground truth reference annotated data is burdensome, and the likelihood that efforts may need duplicating as our understanding of \gls{wm} anatomy evolves is high.
We have seen this in Section \ref{sec:quant} with the \textit{TractoInferno} dataset, which represents a substantial contribution to the machine learning tractography research community, but which nonetheless contains a number of poor-quality samples undetected by rigorous quality control.
Work on single-shot and transfer learning as discussed in Section \ref{sec:ntrain} has underscored the need for methods which can easily be retrained or extended to support new tracts as and when they become relevant.
In addition, a key concern in the translation of complex ``black box'' models is that clinical decision making must remain traceable and transparent, including when aided by computers.
In this light, a flexible atlas, which can be re-trained if needed using only a handful of exemplar datasets and which leads to intuitive and interpretable results can play a unique role alongside machine learning.

\section{Limitations and future directions}\label{sec:future}

The reality of achieving reliable \gls{wm} neuronavigation with intraoperative \gls{dmri} is not simply a matter of replacing tractography with a ``better'' automated tract segmentation pipeline.
As we have seen, the challenges of brain shift and tumour mass effect are substantial and will unlikely be solved with a single one-size-fits-all approach, given the degree of patient heterogeneity.

Exponential radial tumour deformation modelling can go a long way towards extending the applicability of atlas-based tract mapping to cases involving mass effect, however, the results must be inspected carefully as the effects of many tumours are not well captured by such a simplistic model.
It works particularly well for encapsulated tumours and for tumours with a relatively simple shape and location:
Lesions growing in the cortex or subcortical \gls{wm} tend to displace tissue around them in a fairly predictable way which is captured in a radial model.
By contrast, tumours of the diencephalon, midbrain, or hindbrain can produce deformations in non-radial directions, owing to the complex arrangement of surrounding structures and their biomechanical relationships to one another.
In addition, a notable shortcoming of the radial deformation model is the lack of awareness of different brain tissues' elasticities, especially in the ventricles.
The ability of the fluid filled ventricles to absorb a sizeable amount of displacement force and prevent the mass effect from propagating further through the brain than it otherwise might in a medium of uniform elasticity, is what frequently causes atlas registration inaccuracies in tumour cases.
This shortcoming should be the first to be addressed in any future modification and improvements to the tumour deformation model.
A further significant simplification is the strict non-infiltration assumption, in which all healthy tissue is fully displaced to beyond the tumour boundary.
Indeed this assumption is necessary to ensure the formula for the displacement factor $k$ remains well-behaved and invertible.
It may be argued that explicitly modelling tumour infiltration is of lower priority, as tumours infiltrating eloquent tracts are not generally candidates for total resection, as we saw in the case of patient 8 in Section \ref{sec:case}, where resection of the infiltrating tumour portion was abandoned to protect the \gls{cst}.
We also saw that tumour deformation modelling can still be effective in such cases, if only the solid component is segmented, or a scale factor $s<1$ is employed to reduce the effective tumour radius.
In this way, \gls{dmri}-based \gls{wm} mapping could play a role together with direct stimulation functional monitoring in subtotal resections and biopsies in high risk locations close to critical \gls{wm}.

\Gls{imri} is a relatively new technique, and as additional time spent scanning under general anaesthetic and with an open craniotomy carries potential risks for the patient, acquiring supplementary sequences with no clear or confirmed clinical benefit may be unethical.
Particularly with regards to intraoperative \gls{dmri}, a standard of care or guidelines for its use based on large-cohort trials has yet to be established.
As a consequence, while the new \gls{imri} system at \gls{gosh} has been used extensively since installation, the acquisition of diffusion sequences has remained infrequent, leading to a lack of available modern data with which to rigorously validate tractfinder.
Two of the available \gls{gosh} datasets were discussed in detail in Section \ref{sec:case}, while two more where included in the quantitative analysis of Chapter \ref{chap:eval}, providing illustrative insights into the use of tractfinder in real-world scenarios.
There are also outstanding challenges to obtaining consistently high quality diffusion images intraoperatively within a short-enough scan time, as discussed in reference to \gls{epi} artefacts and accelerated imaging in Section \ref{sec:technical}.
A further aspect of clinical uptake which was not addressed in this thesis is acceptance of a volumetric, voxel-based intensity map where radiologists and neurosurgeons may prefer visualising tracts in three dimensions as streamline bundles.
Addressing this potential barrier to translation, for example through a combination of the two techniques or improved volumetric visualisation strategies, should be considered as part of future investigations.

These limitations are to be addressed in an upcoming follow-on prospective study, funded by a grant from Children with Cancer UK (Ref: CwC2022\textbackslash 100006), which will evaluate tractfinder against conventional tractography in a series of children undergoing brain tumour surgery with \gls{imri} at \gls{gosh} and assess its clinical applicability in a range of tumour histological types and locations.
It is hoped that the methodologies and technical considerations presented in this work can contribute to the wider exploitation and adoption of advanced \gls{dmri}-based \gls{wm} imaging in neurosurgical practice, bringing to bear the full potential of modern technological and neuroscientific developments to the benefit of patients.

\chapter*{List of publications and outputs}\addcontentsline{toc}{chapter}{List of publications and outputs}

\subparagraph*{Peer-reviewed journal articles}
\begin{itemize}
  \item[] \fullcite{Young2022}
  \item[] \fullcite{Young2024}
\end{itemize}
\subparagraph*{Articles in production}
\begin{itemize}
  \item[] \fullcite{Aylmore}
\end{itemize}
\subparagraph*{International conferences, accepted abstracts}
\begin{itemize}
  \item[] \fullcite{Young2022b}
  \item[] \fullcite{Young2022a}
  \item[] \fullcite{Young2023}
\end{itemize}

\subparagraph*{Software and data}
\begin{itemize}
  \item[] Tractfinder external MRtrix3 module, available at: \url{github.com/fionaEyoung/tractfinder}
  \item[] Tract orientation atlases and training streamlines, available at: \url{https://doi.org/10.5281/zenodo.10149873}
\end{itemize}

\noindent A follow-on prospective two-year study applying the techniques presented in this thesis has been granted funding by Children with Cancer UK (Grant Ref: CwC2022\textbackslash100006, Title:  A tailored image guidance approach for children undergoing surgery for brain tumours).


\phantomsection
% The \appendix command resets the chapter counter, and changes the chapter numbering scheme to capital letters.
%\chapter{Appendices}
\appendix
\addtocontents{toc}{\protect\setcounter{tocdepth}{0}}

\chapter{Tractography parameters and ROI protocols}
\label{app:rois}

Default parameters as documented for the \verb|tckgen| command of MRtrix3 (release version 3.0.3, available at \url{https://mrtrix.readthedocs.io/en/3.0.3/reference/commands/tckgen.html}) (including \verb|-select 5000 -algorithm iFOD2|) were used for all tractography:

%%%%%%%%%%%%%%%%%%%%%%%%%%%%%%%%%%%%%%%%%%%%%%%%%%%%%%%%%%%%%%%%%%%%%%%%%%%%%%%%
\begin{center}
\begin{tabular}{ l l }\toprule
  Parameter & Value \\
 \midrule
 Algorithm      &   iFOD2\autocite{Tournier2010} \\
 Number of streamlines selected &   5000 \\
 Maximum angle  &   45\degree  \\
 Step size & $0.5 \, \mathsf{x}$ voxel size \\
 \glsentryshort{fod} amplitude threshold & 0.1 \\ \bottomrule
\end{tabular}
\end{center}
%%%%%%%%%%%%%%%%%%%%%%%%%%%%%%%%%%%%%%%%%%%%%%%%%%%%%%%%%%%%%%%%%%%%%%%%%%%%%%%%

In addition, the parameter \verb|-seed_unidirectional| was included for \gls{or} reconstructions, to ensure streamlines are propagated from a single direction out of the \gls{lgn}.


\section{ROI definitions}
\label{sec:rois}

The following ROI strategies were used for atlas constructions and subsequent validation tractography (differences between the two specified where applicable).
Visualisations of each ROI are shown on MNI152 template in Figures \ref{fig:rois.af}, \ref{fig:rois.cst} and \ref{fig:rois.or}.

\subsection{Arcuate fasciculus}

\begin{description}
  \item[Seed] White matter medial of angular gyrus, visible on coronal views of colour fractional anisotropy maps as a ``green triangle", drawn on the coronal plane.
  Level of coronal plane selected from sagittal view by locating the central sulcus (Fig. \ref{fig:rois.af}, arrow).
  \item[Include] Descending section of the arcuate fasciculus, drawn on the axial plane
  \item[Exclude] Exclusion ROIs targeting: midline, superior fronto-occipital fasciculus, ipsilateral cerebral penduncles, sagittal stratum, corona radiata and external capsules.
\end{description}

The following publications were reviewed to inform the above ROI strategy: \textcite{Brown2014a},\textcite{Catani2002},\textcite{Catani2005},\textcite{Chen2015c},
\textcite{Eluvathingal2007},\textcite{Kamali2014},\textcite{Martino2013a},\textcite{Nucifora2005},
\textcite{Parker2005},\textcite{Bain2019},\textcite{Talozzi2018}

\begin{figure*}[h]
  \centering
    \includegraphics[width=\textwidth]{appendix/AF_include.png}
    \includegraphics[width=\textwidth]{appendix/AF_exclude.png}
  \caption{Seed (yellow), inclusion (green) and exclusion (red) regions of interest for the arcuate fasciculus. Arrow indicates central sulcus, landmark for seed ROI.}
  \label{fig:rois.af}
\end{figure*}

\subsection{Corticospinal tract}

Corticospinal tract tracography strategy differed between the atlas creation and general tractography applied to new subjects.

\begin{description}
  \item[Seed (atlas)] For the orientation atlas, Freesurfer cortical parcellations were used to obtain more complete coverage of the motor cortex via the following process:
  \begin{enumerate}
    \item Seed in precentral gyrus and output successful seed location
    \item Generate binary mask from successful seed locations, subtract from precentral gyrus mask to create seed mask
    \item Re-run tractography with second seed mask to cover rest of precentral gyrus
  \end{enumerate}
  \item[Seed (general)] Posterior limb of internal capsule, drawn on three consecutive axial slices
  \item[Include] Posterior limb on internal capsule (if not used for seed), cerebral penduncles, CST in mid-pons
  \item[Exclude] Cerebellar peduncles (drawn on coronal slice), medial lemniscus (drawn on axial slice), midline, superior fronto-occipital fasciculus,
\end{description}

The following publications were reviewed to inform the above ROI strategy: \textcite{Ciccarelli2006},\textcite{Han2010},\textcite{Hattingen2009a},
\textcite{Niu2016},\textcite{Radmanesh2015},\textcite{Reich2006},
\textcite{Rosenstock2017},\textcite{Szmuda2021},\textcite{Vargas2013}

\begin{figure*}[h]
  \centering
    \includegraphics[width=\textwidth]{appendix/CST_include.png}
    \includegraphics[width=\textwidth]{appendix/CST_exclude.png}
  \caption{Seed (yellow), inclusion (green) and exclusion (red) regions of interest for the corticospinal tract}
  \label{fig:rois.cst}
\end{figure*}

\subsection{Inferior fronto-occipital fasciculus}

\begin{description}
  \item[Seed] Temporal stem, between anterior tip if Meyer's loop and descending portion of the uncinate fasciculus
  \item[Include (atlas)] Posterior: inferior, middle, and superior occipital gyri and middle and superior occipital sulci (Freesurfer (v4.5) Destrieux atlas\autocite{Destrieux2010} (2009 version) parcellation labels 1\{1,2\}1\{02,19,20,58,59\}).
  Anterior: frontal pole, middle and inferior frontal gyri and sulci, orbital gyrus and sulci (Freesurfer labels 1\{1,2\}1\{01,05,15,54,12,13,14,53,63,24,65\})
  \item[Include (general)] Frontal lobe coronal slice, anterior to genu of the corpus callosum
  \item[Exclude] Coronal slice on frontal lobe at the level of the central sulcus, coronal slice on tip of anterior temporal lobe
\end{description}

The following publications reviewed to inform the above ROI strategy: \textcite{Martino2010},\textcite{Sarubbo2013},\textcite{Hau2016},
\textcite{Catani2008},\textcite{Wakana2007},\textcite{Wu2016}

\begin{figure*}[h]
  \centering
    \includegraphics[width=\textwidth]{appendix/IFO_include.png}
    \includegraphics[width=\textwidth]{appendix/IFO_exclude.png}
  \caption{Seed (yellow), inclusion (green) and exclusion (red) regions of interest for the inferior fronto-occipital fasciculus}
  \label{fig:rois.ifo}
\end{figure*}

\subsection{Optic radiation}

\begin{description}
  \item[Seed] Lateral geniculate nucleus (LGN; drawn on axial planes)
  \item[Include] Sagittal stratum (drawn on coronal plane)
  \item[Exclude] Coronal slice anterior of and axial slice inferior of most anterior point of lateral ventricles, axial slice at level of superior reach lateral ventricles, splenium of corpus callosum, fornix
\end{description}

The following publications reviewed to inform the above ROI strategy:
\textcite{Yogarajah2009},\textcite{Hofer2010},\textcite{Dayan2015}

\begin{figure*}[h]
  \centering
    \includegraphics[width=\textwidth]{appendix/OR_include.png}
    \includegraphics[width=\textwidth]{appendix/OR_exclude.png}
  \caption{Seed (yellow), inclusion (green) and exclusion (red) regions of interest for the optic radiation}
  \label{fig:rois.or}
\end{figure*}


\addcontentsline{toc}{chapter}{Bibliography}
{\setstretch{1.0}
\printbibliography
}
\chapter*{Colophon}

This document was typeset in Libertinus Serif (a fork of Linux Libertine), {\sffamily Source Sans Pro} and {\ttfamily Courier} typefaces, with \LaTeX\ and Bib\TeX, and using the UCL \LaTeX\ \href{https://github.com/UCL/ucl-latex-thesis-templates}{thesis template} created by Ian Kirker.
Original vector graphics were produced in \href{https://inkscape.org/}{Inkscape}, Keynote, in Python using the \href{https://matplotlib.org/}{Matplotlib} package, and in MATLAB, and typeset using the \verb|svg| package.
Original raster graphics were produced with Inkscape and \verb|mrview| from the \href{https://www.mrtrix.org/}{MRtrix3} software package, and edited in \href{https://www.gimp.org/}{GIMP} 2.1 and with the \href{https://imagemagick.org/index.php}{ImageMagick} software suite.

% All done. \o/
\end{document}
