% UCL Thesis LaTeX Template
%  (c) Ian Kirker, 2014
% 
% This is a template/skeleton for PhD/MPhil/MRes theses.
%
% It uses a rather split-up file structure because this tends to
%  work well for large, complex documents.
% We suggest using one file per chapter, but you may wish to use more
%  or fewer separate files than that.
% We've also separated out various bits of configuration into their
%  own files, to keep everything neat.
% Note that the \input command just streams in whatever file you give
%  it, while the \include command adds a page break, and does some
%  extra organisation to make compilation faster. Note that you can't
%  use \include inside an \include-d file.
% We suggest using \input for settings and configuration files that
%  you always want to use, and \include for each section of content.
% If you do that, it also means you can use the \includeonly statement
%  to only compile up the section you're currently interested in.
% You might also want to put figures into their own files to be \input.

% For more information on \input and \include, see:
%  http://tex.stackexchange.com/questions/246/when-should-i-use-input-vs-include


% Formatting and binding rules for theses are here: 
%  https://www.ucl.ac.uk/students/exams-and-assessments/research-assessments/format-bind-and-submit-your-thesis-general-guidance

% This package goes first and foremost, because it checks all 
%  your syntax for mistakes and some old-fashioned LaTeX commands.
% Note that normally you should load your documentclass before 
%  packages, because some packages change behaviour based on
%  your document settings.
% Also, for those confused by the RequirePackage here vs usepackage
%  elsewhere, usepackage cannot be used before the documentclass
%  command, while RequirePackage can. That's the only functional
%  difference as far as I'm aware.
\RequirePackage[l2tabu, orthodox]{nag}


% ------ Main document class specification ------
% The draft option here prevents images being inserted,
%  and adds chunky black bars to boxes that are exceeding 
%  the page width (to show that they are).
% The oneside option can optionally be replaced by twoside if
%  you intend to print double-sided. Note that this is
%  *specifically permitted* by the UCL thesis formatting
%  guidelines.
%
% Valid options in terms of type are:
%  phd
%  mres
%  mphil
%\documentclass[12pt,phd,draft,a4paper,oneside]{ucl_thesis}
\documentclass[12pt,phd,a4paper,oneside]{ucl_thesis}


% Package configuration:
%  LaTeX uses "packages" to add extra commands and features.
%  There are quite a few useful ones, so we've put them in a 
%   separate file.
% -------- Packages --------

% This package means empty pages (pages with no text) won't get stuff
%  like chapter names at the top of the page. It's mostly cosmetic.
\usepackage{emptypage}

% The graphicx package adds the \includegraphics command,
%  which is your basic command for adding a picture.
\usepackage{graphicx}

% The float package improves LaTeX's handling of floats,
%  and also adds the option to *force* LaTeX to put the float
%  HERE, with the [H] option to the float environment.
\usepackage{float}

% The amsmath package enhances the various ways of including
%  maths, including adding the align environment for aligned
%  equations.
\usepackage{amsmath}
\usepackage{amssymb}


% Use these two packages together -- they define symbols
%  for e.g. units that you can use in both text and math mode.
\usepackage{gensymb}
\usepackage{textcomp}
% You may also want the units package for making little
%  fractions for unit specifications.
%\usepackage{units}


% The setspace package lets you use 1.5-sized or double line spacing.
\usepackage{setspace}
\setstretch{1.35}

% That just does body text -- if you want to expand *everything*,
%  including footnotes and tables, use this instead:
%\renewcommand{\baselinestretch}{1.5}


% PGFPlots is either a really clunky or really good way to add graphs
%  into your document, depending on your point of view.
% There's waaaaay too much information on using this to cover here,
%  so, you might want to start here:
%   http://pgfplots.sourceforge.net/
%  or here:
%   http://pgfplots.sourceforge.net/pgfplots.pdf
%\usepackage{pgfplots}
%\pgfplotsset{compat=1.3} % <- this fixed axis labels in the version I was using

% PGFPlotsTable can help you make tables a little more easily than
%  usual in LaTeX.
% If you're going to have to paste data in a lot, I'd suggest using it.
%  You might want to start with the manual, here:
%  http://pgfplots.sourceforge.net/pgfplotstable.pdf
%\usepackage{pgfplotstable}

% These settings are also recommended for using with pgfplotstable.
%\pgfplotstableset{
%	% these columns/<colname>/.style={<options>} things define a style
%	% which applies to <colname> only.
%	empty cells with={--}, % replace empty cells with '--'
%	every head row/.style={before row=\toprule,after row=\midrule},
%	every last row/.style={after row=\bottomrule}
%}


% Alternatively, you can use the ifdraft package to let you add
%  commands that will only be used in draft versions
\usepackage{ifdraft}
\ifdraft{
  % Draft mode geometry
  \usepackage[margin=1in]{geometry}
  \setstretch{1}
  \setlength \topmargin{10mm}
  \setlength \oddsidemargin {20mm} % Allow a mm for the bleed.
  \setlength \evensidemargin {20mm}
  % Line numbers
  \usepackage[modulo]{lineno}
  \linenumbers
}


% The multirow package adds the option to make cells span
%  rows in tables.
\usepackage{multirow}


% Subfig allows you to create figures within figures, to, for example,
%  make a single figure with 4 individually labeled and referenceable
%  sub-figures.
% It's quite fiddly to use, so check the documentation.
%\usepackage{subfig}

% The natbib package allows book-type citations commonly used in
%  longer works, and less commonly in science articles (IME).
% e.g. (Saucer et al., 1993) rather than [1]
% More details are here: http://merkel.zoneo.net/Latex/natbib.php
%\usepackage{natbib}

% The bibentry package (along with the \nobibliography* command)
%  allows putting full reference lines inline.
%  See:
%   http://tex.stackexchange.com/questions/2905/how-can-i-list-references-from-bibtex-file-in-line-with-commentary
\usepackage{bibentry}

% The isorot package allows you to put things sideways
%  (or indeed, at any angle) on a page.
% This can be useful for wide graphs or other figures.
%\usepackage{isorot}

% The caption package adds more options for caption formatting.
% This set-up makes hanging labels, makes the caption text smaller
%  than the body text, and makes the label bold.
% Highly recommended.
\usepackage[format=hang,font=small,labelfont=bf]{caption}
\usepackage{subcaption}

% If you're getting into defining your own commands, you might want
%  to check out the etoolbox package -- it defines a few commands
%  that can make it easier to make commands robust.
\usepackage{etoolbox}

% The microtype package adds `micro-typographic extensions' which
% generally makes text more readable and hyphenation less likely.
\usepackage{microtype}

% ---- Other Packages (PERSONALISED)

\usepackage{pdflscape}
\usepackage{rotating}


% For multicolumns
%\usepackage{multicol}
%\setlength{\columnsep}{.7cm}

% separate paragraphs with empty lines
\usepackage[parfill]{parskip}

% Create paragraph title format
\usepackage{titlesec}
\titleformat{\paragraph}[hang]{\normalfont\it\raggedright}{}{0pt}{\qquad}[]
\titlespacing*{\paragraph}{0pt}{0pt}{0pt}

% Section header spacing
\titlespacing*{\section}{0pt}{3\baselineskip}{\baselineskip}
\titlespacing*{\subsection}{0pt}{2\baselineskip}{\baselineskip}
\titlespacing*{\subsubsection}{0pt}{\baselineskip}{\baselineskip}

% Code listings
\usepackage{verbatim}
\usepackage{spverbatim} % not sure what this is for but oh well

% For formatting matlab code specifically, **MUST COPY mcode.sty FILE INTO FOLDER**
%\usepackage[framed,numbered,autolinebreaks,useliterate]{mcode}
\usepackage{listings} % Use like this in place: \lstinputlisting{filname.ext}

% Enumeration, lists
\usepackage{enumerate, enumitem}
\usepackage{framed, color}


% Sets up links within your document, for e.g. contents page entries
%  and references, and also PDF metadata.
% You should edit this!
%%
%% This file uses the hyperref package to make your thesis have metadata embedded in the PDF, 
%%  and also adds links to be able to click on references and contents page entries to go to 
%%  the pages.
%%

% Some hacks are necessary to make bibentry and hyperref play nicely.
% See: http://tex.stackexchange.com/questions/65348/clash-between-bibentry-and-hyperref-with-bibstyle-elsart-harv
\usepackage{bibentry}
\makeatletter\let\saved@bibitem\@bibitem\makeatother
\usepackage[pdftex,hidelinks]{hyperref}
\makeatletter\let\@bibitem\saved@bibitem\makeatother
\makeatletter
\AtBeginDocument{
    \hypersetup{
        pdfsubject={Thesis Subject},
        pdfkeywords={Thesis Keywords},
        pdfauthor={Author},
        pdftitle={Title},
    }
}
\makeatother
    


% And then some settings in separate files.
% These settings are partly from:
%  http://mintaka.sdsu.edu/GF/bibliog/latex/floats.html

% They give LaTeX more options on where to put your figures, and may
%  mean that fewer of your figures end up at the tops of pages far
%  away from the thing they're related to.

% Alters some LaTeX defaults for better treatment of figures:
% See p.105 of "TeX Unbound" for suggested values.
% See pp. 199-200 of Lamport's "LaTeX" book for details.

%   General parameters, for ALL pages:
\renewcommand{\topfraction}{0.9}	% max fraction of floats at top
\renewcommand{\bottomfraction}{0.8}	% max fraction of floats at bottom

%   Parameters for TEXT pages (not float pages):
\setcounter{topnumber}{2}
\setcounter{bottomnumber}{2}
\setcounter{totalnumber}{4}     % 2 may work better
\setcounter{dbltopnumber}{2}    % for 2-column pages
\renewcommand{\dbltopfraction}{0.9}	% fit big float above 2-col. text
\renewcommand{\textfraction}{0.2}	% page must be at least 20% text,
%                                  less than that and we get a floatpage

%   Parameters for FLOAT pages (not text pages):
\renewcommand{\floatpagefraction}{0.7}	% require fuller float pages
% N.B.: floatpagefraction MUST be less than topfraction !!
\renewcommand{\dblfloatpagefraction}{0.7}	% require fuller float pages

% remember to use [htp] or [htpb] for placement,
% e.g.
%  \begin{figure}[htp]
%   ...
%  \end{figure}

\usepackage[inkscapearea=page,
            inkscapeexe=/Applications/Inkscape.app/Contents/MacOS/inkscape,
            draft=false,
            inkscapelatex=false]{svg}

\ifdraft{%
  % Figure path
  \graphicspath{{draft_figs/}{figs/}}
  \svgpath{{draft_figs/}{figs/}}
  \setkeys{Gin}{draft=false}
}{%
  \graphicspath{{figs/}{draft_figs/}}
  \svgpath{{figs/}{draft_figs/}}
}

% "Times" symbol for resolutions etc.
\newcommand{\x}{\nobreak\hspace{.1em minus .045em}$\times$\nobreak\hspace{.1em minus .045em}}
 % For things like figures and tables
%\bibliographystyle{unsrt}

% Bibliography and referencing settings

\usepackage[doi=true,
            isbn=true,
            url=false,
            natbib=true,
            maxcitenames=1,
            maxbibnames=50,
            style=nature,
            backend=bibtex]{biblatex} % Add citation styles as needed (see docs for biblatex)
\IfFileExists{bibliographies/Neuroimaging.bib}
          {\addbibresource{bibliographies/Neuroimaging.bib}}
          {\addbibresource{bibs/Neuroimaging.bib}}
\IfFileExists{bibliographies/Neurosurgery.bib}
          {\addbibresource{bibliographies/Neurosurgery.bib}}
          {\addbibresource{bibs/Neurosurgery.bib}}
\IfFileExists{bibliographies/Figures.bib}
          {\addbibresource{bibliographies/Figures.bib}}
          {\addbibresource{bibs/Figures.bib}}
\IfFileExists{bibliographies/Oncology.bib}
          {\addbibresource{bibliographies/Oncology.bib}}
          {\addbibresource{bibs/Oncology.bib}}
\addbibresource{bibliographies/Maths.bib}
\addbibresource{bibliographies/Neuroscience.bib}
\addbibresource{bibliographies/Thesis.bib}


% Make DOIs look normal
\urlstyle{same}

 % Path to bib file (e.g. generated from Mendeley)
% \renewcommand*{\bibfont}{\small} % Make bibligraphy small

% Use the below in conjunction with nature style referencing to get proper behaviour of textcite command (numeric citation still superscript rather than bracket)
\usepackage{ifthen}
\renewcommand{\textcite}[2][]{
\ifthenelse { \equal {#1} {} }  %
    {\citeauthor{#2}\autocite{#2}}   % if #1 == blank
    {\citeauthor{#1}\autocite{#2}}}

\DeclareCiteCommand{\fullcite}
  {\usebibmacro{prenote}}
  {\usedriver
     {\defcounter{maxnames}{50}}
     {\thefield{entrytype}}.}
  {\multicitedelim}
  {\usebibmacro{postnote}}
   % For bibliographies

% These control how many number sections your subsections will take
%    e.g. Section 2.3.1.5.6.3
%  and how many of those will get put into the contents pages.
\setcounter{secnumdepth}{3}
\setcounter{tocdepth}{3}


\begin{document}

\nobibliography*
% ^-- This is a dumb trick that works with the bibentry package to let
%  you put bibliography entries whereever you like.
% I used this to put references to papers a chapter's work was 
%  published in at the end of that chapter.
% For more information, see: http://stefaanlippens.net/bibentry

% If you haven't finished making your full BibTex file yet, you
%  might find this useful -- it'll just replace all your
%  citations with little superscript notes.
% Uncomment to use.
%\renewcommand{\cite}[1]{\emph{\textsuperscript{[#1]}}}

% At last, content! Remember filenames are case-sensitive and 
%  *must not* include spaces.
% I may change the way this is done in a future version,
%  but given that some people needed it, if you need a different degree title
%  (e.g. Master of Science, Master in Science, Master of Arts, etc)
%  uncomment the following 3 lines and set as appropriate (this *has* to be before \maketitle)
% \makeatletter
% \renewcommand {\@degree@string} {Master of Things}
% \makeatother

\title{ Fibre tract imaging with intraoperative diffusion MRI for neurosurgical navigation }
\author{ Fiona Young }
\department{ Department of Medical Physics and Biomedical Engineering \\ UCL GOS Institute of Child Health }

\maketitle
\makedeclaration

\begin{abstract} % 300 word limit
My research is about stuff.

It begins with a study of some stuff, and then some other stuff and things.

There is a 300-word limit on your abstract.
\end{abstract}

\begin{impactstatement}

	UCL theses now have to include an impact statement. \textit{(I think for REF reasons?)} The following text is the description from the guide linked from the formatting and submission website of what that involves. (Link to the guide: {\scriptsize \url{http://www.grad.ucl.ac.uk/essinfo/docs/Impact-Statement-Guidance-Notes-for-Research-Students-and-Supervisors.pdf}})

\begin{quote}
The statement should describe, in no more than 500 words, how the expertise, knowledge, analysis,
discovery or insight presented in your thesis could be put to a beneficial use. Consider benefits both
inside and outside academia and the ways in which these benefits could be brought about.

The benefits inside academia could be to the discipline and future scholarship, research methods or
methodology, the curriculum; they might be within your research area and potentially within other
research areas.

The benefits outside academia could occur to commercial activity, social enterprise, professional
practice, clinical use, public health, public policy design, public service delivery, laws, public
discourse, culture, the quality of the environment or quality of life.

The impact could occur locally, regionally, nationally or internationally, to individuals, communities or
organisations and could be immediate or occur incrementally, in the context of a broader field of
research, over many years, decades or longer.

Impact could be brought about through disseminating outputs (either in scholarly journals or
elsewhere such as specialist or mainstream media), education, public engagement, translational
research, commercial and social enterprise activity, engaging with public policy makers and public
service delivery practitioners, influencing ministers, collaborating with academics and non-academics
etc.

Further information including a searchable list of hundreds of examples of UCL impact outside of
academia please see \url{https://www.ucl.ac.uk/impact/}. For thousands more examples, please see
\url{http://results.ref.ac.uk/Results/SelectUoa}.
\end{quote}
\end{impactstatement}

\begin{acknowledgements}
THANKS!
\end{acknowledgements}

%	Self-plagiarism declaration form template for these typeset in LaTeX
%	Prepared by David Sheard 2022 and made available free of copyright

%	If you use the results of your own published, accepted or submitted data (text or figures) in your final
%	doctoral thesis, you have to give a clear indication of the previous work, stating the exact source of the
%	previous material, irrespective of whether copyright is owned by you or by a publisher. This indication
%	should take the form of
%		a) an appropriate citation of the original source in the relevant Chapter; and
%		b) completion of the UCL Research Paper Declaration form---this should be embedded after the
%		Acknowledgements page in the thesis.

%	For more information consult the following links:
%	\url{https://www.grad.ucl.ac.uk/essinfo/guidance-on-selfplagiarism/?utm_source=Students\%27+Union+UCL\&utm_campaign=2ed9e73ab7-\&utm_medium=email\&utm_term=0_fe8c0cbcf2-2ed9e73ab7-209240456\&mc_cid=2ed9e73ab7\&mc_eid=0496c22bfc}
%	\url{https://www.grad.ucl.ac.uk/essinfo/guidance-on-selfplagiarism/Declaration-form_published-work-in-thesis.docx}

%	I recommend using this template by simply copying everything between \begin{document} and \end{document}
%	into your thesis after the acknowledgements section. By changing the preamble appropriately this template
%	may also be altered to work using \input{...} or the subfiles package.

% \documentclass[12pt, twoside]{article}
%
% \usepackage[a4paper,inner=40mm,outer=20mm, top=30mm, bottom=30mm]{geometry}
% \usepackage{amssymb}
% \usepackage{array}
% \usepackage{setspace}
% 	\setstretch{1.5}
%
% \begin{document}
{\sffamily
\section*{UCL Research Paper Declaration Form: referencing the doctoral candidate’s own published work(s)}
% Uncomment the following line if you would like to add this declaration to your table of contents
% \addcontentsline{toc}{section}{UCL Research Paper Declaration Form}
%
% Please use this form to declare if parts of your thesis are already available in another format,
% e.g. if data, text, or figures:
% •	have been uploaded to a preprint server
% •	are in submission to a peer-reviewed publication
% •	have been published in a peer-reviewed publication, e.g. journal, textbook.
%
% This form should be completed as many times as necessary. For instance, if a student had seven
% thesis chapters, two of which having material which had been published, they would complete this form twice.

\begin{enumerate}[leftmargin=*,label={\bfseries\arabic*.}]\itemsep0em
	\item \textbf{For a research manuscript that has already been published} (if not yet published, please skip to section 2)\textbf{:}
	%
	\begin{enumerate}[label={\alph*)}]\itemsep0em
	%
	\item \textbf{What is the title of the manuscript?}

	\citetitle{Young2022}

	\item \textbf{Please include a link to or doi for the work:}

	\url{https://doi.org/10.1007/s11548-022-02617-z}

	\item \textbf{Where was the work published?}

	\citefield{Young2022}{journaltitle}% Answer here: e.g. journal name

	\item \textbf{Who published the work?}

	\citelist{Young2022}{publisher}% Answer here: e.g. Elsevier/Oxford University Press

	\item \textbf{When was the work published?}

	\citefield{Young2022}{year}% Answer here

	\item \textbf{List the manuscript's authors in the order they appear on the publication:}
	\citeauthor*{Young2022}% Answer here

	\item \textbf{Was the work peer reviewd?}

	Yes

	\item \textbf{Have you retained the copyright?}

	Yes, from the Licence to Publish agreement:
	\begin{quote}
		Ownership of copyright in the Article will be vested in the name of the Author.
	\end{quote}% Answer here

	\item \textbf{Was an earlier form of the manuscript uploaded to a preprint server (e.g. medRxiv)? If `Yes’, please give a link or doi}

	No% Answer here:
	\\
	If ‘No’, please seek permission from the relevant publisher and check the box next to the below statement:
	%
		\begin{itemize}\itemsep0em
		% To check this box, replace \Box with \boxtimes
		\item[$\boxtimes$] {\itshape I acknowledge permission of the publisher named under 1d to include in this thesis portions of the publication named as included in 1c.}
		\end{itemize}
	%
	\end{enumerate}
%
\item \textbf{For a research manuscript prepared for publication but that has not yet been published} (if already published, please skip to section 3)\textbf{:}
%
\begin{enumerate}[label={\alph*)}]\itemsep0em
	%
	\item \textbf{What is the current title of the manuscript?}
	% Answer here:
	\item \textbf{Has the manuscript been uploaded to a preprint server `e.g. medRxiv'?
	\\
	If `Yes', please please give a link or doi:}
	% Answer here:
	\item \textbf{Where is the work intended to be published?}
	% Answer here: e.g. journal name
	\item \textbf{List the manuscript's authors in the intended authorship order:}
	% Answer here
	\item \textbf{Stage of publication:}
	% answer here: e.g. in submission
	%
\end{enumerate}

\item \textbf{For multi-authored work, please give a statement of contribution covering all authors} (if single-author, please skip to section 4)\textbf{:}
\begin{description}[font=\sffamily]
	\item[Fiona Young] Methodology (conceptualisation and implementation), analysis, manuscript original draft and graphics
	\item[Kristian Aquilina] Supervision, manuscript review and editing
	\item[Chris A. Clark] Supervision, manuscript review and editing
	\item[Jonathan D. Clayden] Conceptualisation, supervision, manuscript review and editing
\end{description}
% Answer here
\item \textbf{In which chapter(s) of your thesis can this material be found?}
% Answer here

Chapters \ref{chap:reg}, \ref{chap:applications}

\end{enumerate}

\textbf{e-Signatures confirming that the information above is accurate}
(this form should be co-signed by the supervisor/ senior author unless this is not appropriate, e.g. if the paper was a single-author work)\textbf{:}\\
\textbf{}\\
\textbf{Candidate:}\\
\textbf{Date:}\\
% this form should be co-signed by the supervisor/ senior author unless this is not appropriate, e.g. if the paper was a single-author work):
\textbf{}\\
\textbf{Supervisor/Senior Author signature} (where appropriate)\textbf{:}\\
\textbf{Date:}
%

%%%%%%%%%%%%%%%%%%%%%%%%%%%%% --- HBM --- %%%%%%%%%%%%%%%%%%%%%%%%%%%%%%%%%%%%%%
\newpage
\begin{enumerate}[leftmargin=*,label={\bfseries\arabic*.}]\itemsep0em
%
\item \textbf{For a research manuscript that has already been published} (if not yet published, please skip to section 2)\textbf{:}
%
\begin{enumerate}[label={\alph*)}]\itemsep0em
	%
	\item \textbf{What is the title of the manuscript?}

	\citetitle{Young2024}

	\item \textbf{Please include a link to or doi for the work:}

	\citefield{Young2024}{doi}

	\item \textbf{Where was the work published?}

	\citefield{Young2024}{journaltitle}% Answer here: e.g. journal name

	\item \textbf{Who published the work?}

	\citelist{Young2024}{publisher}% Answer here: e.g. Elsevier/Oxford University Press

	\item \textbf{When was the work published?}

	\citefield{Young2024}{year}% Answer here

	\item \textbf{List the manuscript's authors in the order they appear on the publication:}

	\citeauthor*{Young2024}% Answer here

	\item \textbf{Was the work peer reviewd?}

	Yes

	\item \textbf{Have you retained the copyright?}

	Yes, from license agreement:
	\begin{quote}
		The Author and each Co-author or, if applicable, the Author’s or Co-author’s employer, retains all proprietary rights, such as copyright
	\end{quote}

	\item \textbf{Was an earlier form of the manuscript uploaded to a preprint server (e.g. medRxiv)? If `Yes’, please give a link or doi}

	No% Answer here:
	\\
	If ‘No’, please seek permission from the relevant publisher and check the box next to the below statement:
	%
	\begin{itemize}\itemsep0em
	% To check this box, replace \Box with \boxtimes
	\item[$\boxtimes$] {\itshape I acknowledge permission of the publisher named under 1d to include in this thesis portions of the publication named as included in 1c.}
	\end{itemize}
	%
\end{enumerate}
%
\item \textbf{For a research manuscript prepared for publication but that has not yet been published} (if already published, please skip to section 3)\textbf{:}
%
\begin{enumerate}[label={\alph*)}]\itemsep0em
	%
	\item \textbf{What is the current title of the manuscript?}
	% Answer here:
	\item \textbf{Has the manuscript been uploaded to a preprint server `e.g. medRxiv'?
	\\
	If `Yes', please please give a link or doi:}
	% Answer here:
	\item \textbf{Where is the work intended to be published?}
	% Answer here: e.g. journal name
	\item \textbf{List the manuscript's authors in the intended authorship order:}
	% Answer here
	\item \textbf{Stage of publication:}
	% answer here: e.g. in submission
	%
\end{enumerate}

\item \textbf{For multi-authored work, please give a statement of contribution covering all authors} (if single-author, please skip to section 4)\textbf{:}
\begin{description}[font=\sffamily]
	\item[Fiona Young] Methodology (conceptualisation and implementation), analysis, manuscript original draft and graphics
	\item[Kristian Aquilina] Supervision, conceptualisation, manuscript review and editing
	\item[Laura Mancini] Data contribution, manuscript review and editing
	\item[Kiran K. Seunarine] Data contribution and curation
	\item[Chris A. Clark] Supervision, manuscript review and editing
	\item[Jonathan D. Clayden] Conceptualisation, supervision, manuscript review and editing
\end{description}
% Answer here
\item \textbf{In which chapter(s) of your thesis can this material be found?}
% Answer here

Chapters \ref{chap:review}, \ref{chap:atlas}, \ref{chap:reg} (Section \ref{sec:reg1}), \ref{chap:eval}, Appendix
\end{enumerate}

\textbf{e-Signatures confirming that the information above is accurate}
(this form should be co-signed by the supervisor/ senior author unless this is not appropriate, e.g. if the paper was a single-author work)\textbf{:}\\
\textbf{}\\
\textbf{Candidate:}\\
\textbf{Date:}\\
% this form should be co-signed by the supervisor/ senior author unless this is not appropriate, e.g. if the paper was a single-author work):
\textbf{}\\
\textbf{Supervisor/Senior Author signature} (where appropriate)\textbf{:}\\
\textbf{Date:}
%

%%%%%%%%%%%%%%%%%%%%%%%%%%%%% --- IEEE --- %%%%%%%%%%%%%%%%%%%%%%%%%%%%%%%%%%%%%
\newpage
\begin{enumerate}[leftmargin=*,label={\bfseries\arabic*.}]\itemsep0em
	%
	\item \textbf{For a research manuscript that has already been published} (if not yet published, please skip to section 2)\textbf{:}
	%
	\begin{enumerate}[label={\alph*)}]\itemsep0em
	%
	\item \textbf{What is the title of the manuscript?}

	\item \textbf{Please include a link to or doi for the work:}

	\item \textbf{Where was the work published?}

	\item \textbf{Who published the work?}

	\item \textbf{When was the work published?}

	\item \textbf{List the manuscript's authors in the order they appear on the publication:}

	\item \textbf{Was the work peer reviewd?}

	\item \textbf{Have you retained the copyright?}

	\item \textbf{Was an earlier form of the manuscript uploaded to a preprint server (e.g. medRxiv)? If ‘Yes’, please give a link or doi}
	\\
	If ‘No’, please seek permission from the relevant publisher and check the box next to the below statement:
	%
\begin{itemize}\itemsep0em
% To check this box, replace \Box with \boxtimes
\item[$\Box$] {\itshape I acknowledge permission of the publisher named under 1d to include in this thesis portions of the publication named as included in 1c.}
\end{itemize}
%
\end{enumerate}
%
\item \textbf{For a research manuscript prepared for publication but that has not yet been published} (if already published, please skip to section 3)\textbf{:}
%
\begin{enumerate}[label={\alph*)}]\itemsep0em
	%
	\item \textbf{What is the current title of the manuscript?}

	\citetitle{Young2023}
	\item \textbf{Has the manuscript been uploaded to a preprint server `e.g. medRxiv'?
	\\
	If `Yes', please please give a link or doi:}

	Yes, \url{https://www.researchgate.net/publication/375601134_Training_Data_Requirements_for_Atlas-Based_Brain_Fibre_Tract_Identification}
	% Answer here:
	\item \textbf{Where is the work intended to be published?}
	% Answer here: e.g. journal name

	Proceedings of the \citefield{Young2023}{eventtitle}
	\item \textbf{List the manuscript's authors in the intended authorship order:}
	% Answer here

	\citeauthor*{Young2023}
	\item \textbf{Stage of publication:}
	% answer here: e.g. in submission

	Work presented at conference, copyright transferred to IEEE, yet now indication of if/when proceedings may be published (copyright on submitted version retained).
\end{enumerate}

\item \textbf{For multi-authored work, please give a statement of contribution covering all authors} (if single-author, please skip to section 4)\textbf{:}
\begin{description}[font=\sffamily]
	\item[Fiona Young] Methodology (conceptualisation and implementation), analysis, manuscript original draft and graphics
	\item[Kristian Aquilina] Supervision
	\item[Chris A. Clark] Supervision
	\item[Jonathan D. Clayden] Conceptualisation, supervision, manuscript review and editing
\end{description}
% Answer here
\item \textbf{In which chapter(s) of your thesis can this material be found?}
% Answer here

Chapter \ref{chap:atlas}
\end{enumerate}

\textbf{e-Signatures confirming that the information above is accurate}
(this form should be co-signed by the supervisor/ senior author unless this is not appropriate, e.g. if the paper was a single-author work)\textbf{:}\\
\textbf{}\\
\textbf{Candidate:}\\
\textbf{Date:}\\
% this form should be co-signed by the supervisor/ senior author unless this is not appropriate, e.g. if the paper was a single-author work):
\textbf{}\\
\textbf{Supervisor/Senior Author signature} (where appropriate)\textbf{:}\\
\textbf{Date:}
%

%%%%%%%%%%%%%%%%%%%%%%%%% --- Diffusion ISMRM --- %%%%%%%%%%%%%%%%%%%%%%%%%%%%%
\newpage
\begin{enumerate}[leftmargin=*,label={\bfseries\arabic*.}]\itemsep0em
	%
	\item \textbf{For a research manuscript that has already been published} (if not yet published, please skip to section 2)\textbf{:}
	%
	\begin{enumerate}[label={\alph*)}]\itemsep0em
	%
	\item \textbf{What is the title of the manuscript?}

	\item \textbf{Please include a link to or doi for the work:}

	\item \textbf{Where was the work published?}

	\item \textbf{Who published the work?}

	\item \textbf{When was the work published?}

	\item \textbf{List the manuscript's authors in the order they appear on the publication:}

	\item \textbf{Was the work peer reviewd?}

	\item \textbf{Have you retained the copyright?}

	\item \textbf{Was an earlier form of the manuscript uploaded to a preprint server (e.g. medRxiv)? If ‘Yes’, please give a link or doi}
	\\
	If ‘No’, please seek permission from the relevant publisher and check the box next to the below statement:
	%
\begin{itemize}\itemsep0em
% To check this box, replace \Box with \boxtimes
\item[$\Box$] {\itshape I acknowledge permission of the publisher named under 1d to include in this thesis portions of the publication named as included in 1c.}
\end{itemize}
%
\end{enumerate}
%
\item \textbf{For a research manuscript prepared for publication but that has not yet been published} (if already published, please skip to section 3)\textbf{:}
%
\begin{enumerate}[label={\alph*)}]\itemsep0em
	%
	\item \textbf{What is the current title of the manuscript?}
	% Answer here:

	\citetitle{Young2022a}
	\item \textbf{Has the manuscript been uploaded to a preprint server `e.g. medRxiv'?
	\\
	If `Yes', please please give a link or doi:}

	Yes, \url{https://www.researchgate.net/publication/367116849_Stability_of_white_matter_tract_segmentation_methods_with_decreasing_data_quality}
	% Answer here:
	\item \textbf{Where is the work intended to be published?}
	% Answer here: e.g. journal name

	N/A
	\item \textbf{List the manuscript's authors in the intended authorship order:}
	% Answer here

	\citeauthor*{Young2022a}
	\item \textbf{Stage of publication:}
	% answer here: e.g. in submission

	Work presented at conference, no proceedings published.
\end{enumerate}

\item \textbf{For multi-authored work, please give a statement of contribution covering all authors} (if single-author, please skip to section 4)\textbf{:}
\begin{description}[font=\sffamily]
	\item[Fiona Young] Methodology (conceptualisation and implementation), analysis, manuscript original draft and graphics
	\item[Jonathan D. Clayden] Conceptualisation, supervision, manuscript review and editing
\end{description}
% Answer here
\item \textbf{In which chapter(s) of your thesis can this material be found?}
% Answer here

Chapter \ref{chap:applications}
\end{enumerate}

\textbf{e-Signatures confirming that the information above is accurate}
(this form should be co-signed by the supervisor/ senior author unless this is not appropriate, e.g. if the paper was a single-author work)\textbf{:}\\
\textbf{}\\
\textbf{Candidate:}\\
\textbf{Date:}\\
% this form should be co-signed by the supervisor/ senior author unless this is not appropriate, e.g. if the paper was a single-author work):
\textbf{}\\
\textbf{Supervisor/Senior Author signature} (where appropriate)\textbf{:}\\
\textbf{Date:}
%
}% End font switch


\setcounter{tocdepth}{2}
% Setting this higher means you get contents entries for
%  more minor section headers.

\tableofcontents
% \listoffigures
% \listoftables

\chapter{Introductory Material}
\label{chapterlabel1}

% -- DELETE AND REWRITE THIS; DEMO ONLY
Intraoperative dMRI has the potential to supplement existing imaging practices by offering a means of imaging fibre tracts after brain shift has invalidated preoperative imaging.\autocite{Nimsky2001}
Thus informed, surgeons would be better equipped to resect as much tumour as possible while leaving eloquent brain tissue intact.

\include{Chapter2}
\include{Chapter3}
\chapter{Conclusions}

The growing availability of advanced \gls{mri} capabilities in health centres is bringing attention to the shortcomings of current image processing techniques in fully exploiting potential benefits for patients.
In a small internal survey of five neuroradiologists and neurosurgeons at \gls{gosh}, all respondents expressed that they would find a tool for intraoperative imaging of \gls{wm} bundles after brain shift to be either slightly (2/5) or very (3/5) useful, while confirming reservations about the reliability or accuracy of tractography.
Against this background, this thesis has set out to explore the current capabilities of \gls{dmri} to map brain \gls{wm} for surgical planning and neuronavigation, and propose a novel technique to fulfil the requirements for rapid and robust \gls{wm} tract detection.
Guided by the objectives set out in Chapter \ref{chap:problem}, the proposed pipeline, called tractfinder, involves constructing tract-specific orientation atlases which are compared with a subject's \gls{fod} image, to achieve direct voxel-wise segmentation with incorporated \textit{a priori} anatomical knowledge.
Each atlas is constructed from meticulously filtered training streamlines based on available consensus neuroanatomical definitions which are then mapped to voxel-wise orientation distributions using \gls{tod} imaging and averaged over the training population (Chapter \ref{chap:atlas}).
Upon registration with the target image, the tract's location is estimated by comparing the voxel-wise tract orientation and spatial priors with the \gls{fod} modelled from \gls{dmri} data, which can be achieved by taking the inner product of the two spherical distributions.
If large tumour deformation effects are present in the image, then the atlas is adjusted accordingly with an exponential radial deformation model (Chapter \ref{chap:reg}) prior to computing the tract map.
Through detailed evaluation against benchmark methods presented in Chapter \ref{chap:eval}, tractfinder has been shown to produce consistent and accurate segmentations at a standard comparable with streamline tractography and deep learning, issues with variable tract definitions and reference data quality notwithstanding.

Several design choices in the tractfinder pipeline are supported by the objective to keep point-of-application processing time and user-interaction to a minimum.
The anatomical priors, which would for tractography either be drawn by hand or automatically via cortical parcellation or deformable registration, are provided by the tract atlases which also account for a degree of inter-subject variability.
As we saw in Chapter \ref{chap:reg}, only linear registration is required to align atlas and subject data, which is faster and more robust than the non-linear or deformable registration that would be necessary for accurate segmentation using registration alone without comparison with subject diffusion data.
It is also applicable to challenging clinical data without need for manually adjusting registration parameters, where non-linear algorithms can fail to reach stable convergence.
Using the \gls{sh} basis for representing orientation distribution data allows for efficient inner product computation, and could flexibly support alternative comparison metrics.
Overall, the full pipeline can be run in under five minutes for a single tract including minimal preprocessing, with the only potential need for user input being the choice of tumour deformation model parameter $\lambda$, if the default adaptive value produces suboptimal results.

Given that tumour infiltration and oedema affect diffusion measurements and thus all downstream processing and modelling, we cannot claim that these effects do not impact tract mapping using tractfinder.
However, a voxel-wise approach is not susceptible to a compounding of errors in the same way streamline tractography is, where oedema in one part of the tract can derail tracking for the entire bundle, including parts not directly affected by infiltration.
Where disturbed diffusion does result in lower tractfinder values, they can be interpreted in the context of other imaging and even provide clinically useful information about \gls{wm} integrity.
By contrast, we have seen examples of the deep learning method TractSeg failing to fully recognise tracts disturbed by tumours (Sections \ref{sec:quant}, \ref{sec:case}), displaying a lack of explainability that is unacceptable for effective clinical translation.
In this way, the requirement of a clinically applicable and robust method is achieved, although this has yet to be rigorously verified in a prospective study (see Section \ref{sec:future}).

A final stipulation was to keep reference data requirements for atlas creation to a minimum, to accommodate evolving neuroanatomical definitions and the difficulty in producing high quality reconstructions.
With only 16 training subjects used for the original atlases, and subsequent analysis presented in Section \ref{sec:ntrain} indicating that as few as ten may be sufficient, tractfinder has a significant advantage over more data-intensive statistical and deep learning alternatives.

The work described in this thesis came about during a time when increasingly sizeable contingents of the research and medical communities are focussing minds on the potential of machine learning techniques to disrupt hitherto intractable problems.
In the \gls{wm} imaging space, machine learning is gaining traction both for direct segmentation methods and as a means for finally overcoming the fundamental roadblocks in streamline tractography which have entangled researchers for years.
Against this backdrop of excitement for new algorithmic and big-data possibilities, proposals for a new atlas-based technique have been met with some skepticism.
Nevertheless, just as streamline tractography has long delivered astonishing benefits while simultaneously remaining unable to expel its sensitivity-specificity trade-off gremlins, so too are the difficulties of bringing deep learning solutions to real clinical translation beginning to show.
Fulfilling the need for large volumes of accurately annotated data may be easy where the resources for producing said data are freely available, or at least justified if the application is a well-defined and static problem.
Tract segmentation is neither of those things:
Producing the ground truth reference annotated data is burdensome, and the likelihood that efforts may need duplicating as our understanding of \gls{wm} anatomy evolves is high.
We have seen this in Section \ref{sec:quant} with the \textit{TractoInferno} dataset, which represents a substantial contribution to the machine learning tractography research community, but which nonetheless contains a number of poor-quality samples undetected by rigorous quality control.
Work on single-shot and transfer learning as discussed in Section \ref{sec:ntrain} has underscored the need for methods which can easily be retrained or extended to support new tracts as and when they become relevant.
In addition, a key concern in the translation of complex ``black box'' models is that clinical decision making must remain traceable and transparent, including when aided by computers.
In this light, a flexible atlas, which can be re-trained if needed using only a handful of exemplar datasets and which leads to intuitive and interpretable results can play a unique role alongside machine learning.

\section{Limitations and future directions}\label{sec:future}

The reality of achieving reliable \gls{wm} neuronavigation with intraoperative \gls{dmri} is not simply a matter of replacing tractography with a ``better'' automated tract segmentation pipeline.
As we have seen, the challenges of brain shift and tumour mass effect are substantial and will unlikely be solved with a single one-size-fits-all approach, given the degree of patient heterogeneity.

Exponential radial tumour deformation modelling can go a long way towards extending the applicability of atlas-based tract mapping to cases involving mass effect, however, the results must be inspected carefully as the effects of many tumours are not well captured by such a simplistic model.
It works particularly well for encapsulated tumours and for tumours with a relatively simple shape and location:
Lesions growing in the cortex or subcortical \gls{wm} tend to displace tissue around them in a fairly predictable way which is captured in a radial model.
By contrast, tumours of the diencephalon, midbrain, or hindbrain can produce deformations in non-radial directions, owing to the complex arrangement of surrounding structures and their biomechanical relationships to one another.
In addition, a notable shortcoming of the radial deformation model is the lack of awareness of different brain tissues' elasticities, especially in the ventricles.
The ability of the fluid filled ventricles to absorb a sizeable amount of displacement force and prevent the mass effect from propagating further through the brain than it otherwise might in a medium of uniform elasticity, is what frequently causes atlas registration inaccuracies in tumour cases.
This shortcoming should be the first to be addressed in any future modification and improvements to the tumour deformation model.
A further significant simplification is the strict non-infiltration assumption, in which all healthy tissue is fully displaced to beyond the tumour boundary.
Indeed this assumption is necessary to ensure the formula for the displacement factor $k$ remains well-behaved and invertible.
It may be argued that explicitly modelling tumour infiltration is of lower priority, as tumours infiltrating eloquent tracts are not generally candidates for total resection, as we saw in the case of patient 8 in Section \ref{sec:case}, where resection of the infiltrating tumour portion was abandoned to protect the \gls{cst}.
We also saw that tumour deformation modelling can still be effective in such cases, if only the solid component is segmented, or a scale factor $s<1$ is employed to reduce the effective tumour radius.
In this way, \gls{dmri}-based \gls{wm} mapping could play a role together with direct stimulation functional monitoring in subtotal resections and biopsies in high risk locations close to critical \gls{wm}.

\Gls{imri} is a relatively new technique, and as additional time spent scanning under general anaesthetic and with an open craniotomy carries potential risks for the patient, acquiring supplementary sequences with no clear or confirmed clinical benefit may be unethical.
Particularly with regards to intraoperative \gls{dmri}, a standard of care or guidelines for its use based on large-cohort trials has yet to be established.
As a consequence, while the new \gls{imri} system at \gls{gosh} has been used extensively since installation, the acquisition of diffusion sequences has remained infrequent, leading to a lack of available modern data with which to rigorously validate tractfinder.
Two of the available \gls{gosh} datasets were discussed in detail in Section \ref{sec:case}, while two more where included in the quantitative analysis of Chapter \ref{chap:eval}, providing illustrative insights into the use of tractfinder in real-world scenarios.
There are also outstanding challenges to obtaining consistently high quality diffusion images intraoperatively within a short-enough scan time, as discussed in reference to \gls{epi} artefacts and accelerated imaging in Section \ref{sec:technical}.
A further aspect of clinical uptake which was not addressed in this thesis is acceptance of a volumetric, voxel-based intensity map where radiologists and neurosurgeons may prefer visualising tracts in three dimensions as streamline bundles.
Addressing this potential barrier to translation, for example through a combination of the two techniques or improved volumetric visualisation strategies, should be considered as part of future investigations.

These limitations are to be addressed in an upcoming follow-on prospective study, funded by a grant from Children with Cancer UK (Ref: CwC2022\textbackslash 100006), which will evaluate tractfinder against conventional tractography in a series of children undergoing brain tumour surgery with \gls{imri} at \gls{gosh} and assess its clinical applicability in a range of tumour histological types and locations.
It is hoped that the methodologies and technical considerations presented in this work can contribute to the wider exploitation and adoption of advanced \gls{dmri}-based \gls{wm} imaging in neurosurgical practice, bringing to bear the full potential of modern technological and neuroscientific developments to the benefit of patients.

\chapter*{List of publications and outputs}\addcontentsline{toc}{chapter}{List of publications and outputs}

\subparagraph*{Peer-reviewed journal articles}
\begin{itemize}
  \item[] \fullcite{Young2022}
  \item[] \fullcite{Young2024}
\end{itemize}
\subparagraph*{Articles in production}
\begin{itemize}
  \item[] \fullcite{Aylmore}
\end{itemize}
\subparagraph*{International conferences, accepted abstracts}
\begin{itemize}
  \item[] \fullcite{Young2022b}
  \item[] \fullcite{Young2022a}
  \item[] \fullcite{Young2023}
\end{itemize}

\subparagraph*{Software and data}
\begin{itemize}
  \item[] Tractfinder external MRtrix3 module, available at: \url{github.com/fionaEyoung/tractfinder}
  \item[] Tract orientation atlases and training streamlines, available at: \url{https://doi.org/10.5281/zenodo.10149873}
\end{itemize}

\noindent A follow-on prospective two-year study applying the techniques presented in this thesis has been granted funding by Children with Cancer UK (Grant Ref: CwC2022\textbackslash100006, Title:  A tailored image guidance approach for children undergoing surgery for brain tumours).

\phantomsection
% The \appendix command resets the chapter counter, and changes the chapter numbering scheme to capital letters.
%\chapter{Appendices}
\appendix
\addtocontents{toc}{\protect\setcounter{tocdepth}{0}}

\chapter{Tractography parameters and ROI protocols}
\label{app:rois}

Default parameters as documented for the \verb|tckgen| command of MRtrix3 (release version 3.0.3, available at \url{https://mrtrix.readthedocs.io/en/3.0.3/reference/commands/tckgen.html}) (including \verb|-select 5000 -algorithm iFOD2|) were used for all tractography:

%%%%%%%%%%%%%%%%%%%%%%%%%%%%%%%%%%%%%%%%%%%%%%%%%%%%%%%%%%%%%%%%%%%%%%%%%%%%%%%%
\begin{center}
\begin{tabular}{ l l }\toprule
  Parameter & Value \\
 \midrule
 Algorithm      &   iFOD2\autocite{Tournier2010} \\
 Number of streamlines selected &   5000 \\
 Maximum angle  &   45\degree  \\
 Step size & $0.5 \, \mathsf{x}$ voxel size \\
 \glsentryshort{fod} amplitude threshold & 0.1 \\ \bottomrule
\end{tabular}
\end{center}
%%%%%%%%%%%%%%%%%%%%%%%%%%%%%%%%%%%%%%%%%%%%%%%%%%%%%%%%%%%%%%%%%%%%%%%%%%%%%%%%

In addition, the parameter \verb|-seed_unidirectional| was included for \gls{or} reconstructions, to ensure streamlines are propagated from a single direction out of the \gls{lgn}.


\section{ROI definitions}
\label{sec:rois}

The following ROI strategies were used for atlas constructions and subsequent validation tractography (differences between the two specified where applicable).
Visualisations of each ROI are shown on MNI152 template in Figures \ref{fig:rois.af}, \ref{fig:rois.cst} and \ref{fig:rois.or}.

\subsection{Arcuate fasciculus}

\begin{description}
  \item[Seed] White matter medial of angular gyrus, visible on coronal views of colour fractional anisotropy maps as a ``green triangle", drawn on the coronal plane.
  Level of coronal plane selected from sagittal view by locating the central sulcus (Fig. \ref{fig:rois.af}, arrow).
  \item[Include] Descending section of the arcuate fasciculus, drawn on the axial plane
  \item[Exclude] Exclusion ROIs targeting: midline, superior fronto-occipital fasciculus, ipsilateral cerebral penduncles, sagittal stratum, corona radiata and external capsules.
\end{description}

The following publications were reviewed to inform the above ROI strategy: \textcite{Brown2014a},\textcite{Catani2002},\textcite{Catani2005},\textcite{Chen2015c},
\textcite{Eluvathingal2007},\textcite{Kamali2014},\textcite{Martino2013a},\textcite{Nucifora2005},
\textcite{Parker2005},\textcite{Bain2019},\textcite{Talozzi2018}

\begin{figure*}[h]
  \centering
    \includegraphics[width=\textwidth]{appendix/AF_include.png}
    \includegraphics[width=\textwidth]{appendix/AF_exclude.png}
  \caption{Seed (yellow), inclusion (green) and exclusion (red) regions of interest for the arcuate fasciculus. Arrow indicates central sulcus, landmark for seed ROI.}
  \label{fig:rois.af}
\end{figure*}

\subsection{Corticospinal tract}

Corticospinal tract tracography strategy differed between the atlas creation and general tractography applied to new subjects.

\begin{description}
  \item[Seed (atlas)] For the orientation atlas, Freesurfer cortical parcellations were used to obtain more complete coverage of the motor cortex via the following process:
  \begin{enumerate}
    \item Seed in precentral gyrus and output successful seed location
    \item Generate binary mask from successful seed locations, subtract from precentral gyrus mask to create seed mask
    \item Re-run tractography with second seed mask to cover rest of precentral gyrus
  \end{enumerate}
  \item[Seed (general)] Posterior limb of internal capsule, drawn on three consecutive axial slices
  \item[Include] Posterior limb on internal capsule (if not used for seed), cerebral penduncles, CST in mid-pons
  \item[Exclude] Cerebellar peduncles (drawn on coronal slice), medial lemniscus (drawn on axial slice), midline, superior fronto-occipital fasciculus,
\end{description}

The following publications were reviewed to inform the above ROI strategy: \textcite{Ciccarelli2006},\textcite{Han2010},\textcite{Hattingen2009a},
\textcite{Niu2016},\textcite{Radmanesh2015},\textcite{Reich2006},
\textcite{Rosenstock2017},\textcite{Szmuda2021},\textcite{Vargas2013}

\begin{figure*}[h]
  \centering
    \includegraphics[width=\textwidth]{appendix/CST_include.png}
    \includegraphics[width=\textwidth]{appendix/CST_exclude.png}
  \caption{Seed (yellow), inclusion (green) and exclusion (red) regions of interest for the corticospinal tract}
  \label{fig:rois.cst}
\end{figure*}

\subsection{Inferior fronto-occipital fasciculus}

\begin{description}
  \item[Seed] Temporal stem, between anterior tip if Meyer's loop and descending portion of the uncinate fasciculus
  \item[Include (atlas)] Posterior: inferior, middle, and superior occipital gyri and middle and superior occipital sulci (Freesurfer (v4.5) Destrieux atlas\autocite{Destrieux2010} (2009 version) parcellation labels 1\{1,2\}1\{02,19,20,58,59\}).
  Anterior: frontal pole, middle and inferior frontal gyri and sulci, orbital gyrus and sulci (Freesurfer labels 1\{1,2\}1\{01,05,15,54,12,13,14,53,63,24,65\})
  \item[Include (general)] Frontal lobe coronal slice, anterior to genu of the corpus callosum
  \item[Exclude] Coronal slice on frontal lobe at the level of the central sulcus, coronal slice on tip of anterior temporal lobe
\end{description}

The following publications reviewed to inform the above ROI strategy: \textcite{Martino2010},\textcite{Sarubbo2013},\textcite{Hau2016},
\textcite{Catani2008},\textcite{Wakana2007},\textcite{Wu2016}

\begin{figure*}[h]
  \centering
    \includegraphics[width=\textwidth]{appendix/IFO_include.png}
    \includegraphics[width=\textwidth]{appendix/IFO_exclude.png}
  \caption{Seed (yellow), inclusion (green) and exclusion (red) regions of interest for the inferior fronto-occipital fasciculus}
  \label{fig:rois.ifo}
\end{figure*}

\subsection{Optic radiation}

\begin{description}
  \item[Seed] Lateral geniculate nucleus (LGN; drawn on axial planes)
  \item[Include] Sagittal stratum (drawn on coronal plane)
  \item[Exclude] Coronal slice anterior of and axial slice inferior of most anterior point of lateral ventricles, axial slice at level of superior reach lateral ventricles, splenium of corpus callosum, fornix
\end{description}

The following publications reviewed to inform the above ROI strategy:
\textcite{Yogarajah2009},\textcite{Hofer2010},\textcite{Dayan2015}

\begin{figure*}[h]
  \centering
    \includegraphics[width=\textwidth]{appendix/OR_include.png}
    \includegraphics[width=\textwidth]{appendix/OR_exclude.png}
  \caption{Seed (yellow), inclusion (green) and exclusion (red) regions of interest for the optic radiation}
  \label{fig:rois.or}
\end{figure*}
 
% You could separate these out into different files if you have
%  particularly large appendices.

% Actually generates your bibliography. The fact that \include is 
% the last thing before this ensures that it is on a clear page.
\bibliography{example}

% All done. \o/
\end{document}
