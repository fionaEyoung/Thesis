% -------- Packages --------

% This package means empty pages (pages with no text) won't get stuff
%  like chapter names at the top of the page. It's mostly cosmetic.
\usepackage{emptypage}

% The graphicx package adds the \includegraphics command,
%  which is your basic command for adding a picture.
\usepackage{graphicx}

% The float package improves LaTeX's handling of floats,
%  and also adds the option to *force* LaTeX to put the float
%  HERE, with the [H] option to the float environment.
\usepackage{float}

% The amsmath package enhances the various ways of including
%  maths, including adding the align environment for aligned
%  equations.
\usepackage{amsmath}
\usepackage{amssymb}


% Use these two packages together -- they define symbols
%  for e.g. units that you can use in both text and math mode.
\usepackage{gensymb}
\usepackage{textcomp}
% You may also want the units package for making little
%  fractions for unit specifications.
%\usepackage{units}


% The setspace package lets you use 1.5-sized or double line spacing.
\usepackage{setspace}
\setstretch{1.35}

% That just does body text -- if you want to expand *everything*,
%  including footnotes and tables, use this instead:
%\renewcommand{\baselinestretch}{1.5}


% PGFPlots is either a really clunky or really good way to add graphs
%  into your document, depending on your point of view.
% There's waaaaay too much information on using this to cover here,
%  so, you might want to start here:
%   http://pgfplots.sourceforge.net/
%  or here:
%   http://pgfplots.sourceforge.net/pgfplots.pdf
%\usepackage{pgfplots}
%\pgfplotsset{compat=1.3} % <- this fixed axis labels in the version I was using

% PGFPlotsTable can help you make tables a little more easily than
%  usual in LaTeX.
% If you're going to have to paste data in a lot, I'd suggest using it.
%  You might want to start with the manual, here:
%  http://pgfplots.sourceforge.net/pgfplotstable.pdf
%\usepackage{pgfplotstable}

% These settings are also recommended for using with pgfplotstable.
%\pgfplotstableset{
%	% these columns/<colname>/.style={<options>} things define a style
%	% which applies to <colname> only.
%	empty cells with={--}, % replace empty cells with '--'
%	every head row/.style={before row=\toprule,after row=\midrule},
%	every last row/.style={after row=\bottomrule}
%}


% Alternatively, you can use the ifdraft package to let you add
%  commands that will only be used in draft versions
\usepackage{ifdraft}
\ifdraft{
  % Draft mode geometry
  \usepackage[margin=1in]{geometry}
  \setstretch{1}
  \setlength \topmargin{10mm}
  \setlength \oddsidemargin {20mm} % Allow a mm for the bleed.
  \setlength \evensidemargin {20mm}
  % Line numbers
  \usepackage[modulo]{lineno}
  \linenumbers
}


% The multirow package adds the option to make cells span
%  rows in tables.
\usepackage{multirow}


% Subfig allows you to create figures within figures, to, for example,
%  make a single figure with 4 individually labeled and referenceable
%  sub-figures.
% It's quite fiddly to use, so check the documentation.
%\usepackage{subfig}

% The natbib package allows book-type citations commonly used in
%  longer works, and less commonly in science articles (IME).
% e.g. (Saucer et al., 1993) rather than [1]
% More details are here: http://merkel.zoneo.net/Latex/natbib.php
%\usepackage{natbib}

% The bibentry package (along with the \nobibliography* command)
%  allows putting full reference lines inline.
%  See:
%   http://tex.stackexchange.com/questions/2905/how-can-i-list-references-from-bibtex-file-in-line-with-commentary
\usepackage{bibentry}

% The isorot package allows you to put things sideways
%  (or indeed, at any angle) on a page.
% This can be useful for wide graphs or other figures.
%\usepackage{isorot}

% The caption package adds more options for caption formatting.
% This set-up makes hanging labels, makes the caption text smaller
%  than the body text, and makes the label bold.
% Highly recommended.
\usepackage[format=hang,font=small,labelfont=bf]{caption}
\usepackage{subcaption}

% If you're getting into defining your own commands, you might want
%  to check out the etoolbox package -- it defines a few commands
%  that can make it easier to make commands robust.
\usepackage{etoolbox}

% The microtype package adds `micro-typographic extensions' which
% generally makes text more readable and hyphenation less likely.
\usepackage{microtype}

% ---- Other Packages (PERSONALISED)

\usepackage{pdflscape}
\usepackage{rotating}


% For multicolumns
%\usepackage{multicol}
%\setlength{\columnsep}{.7cm}

% separate paragraphs with empty lines
\usepackage[parfill]{parskip}

% Create paragraph title format
\usepackage{titlesec}
\titleformat{\paragraph}[hang]{\normalfont\itshape\raggedright}{}{0pt}{\qquad}[]
\titlespacing*{\paragraph}{0pt}{\baselineskip}{0pt}

% Section header spacing
\titlespacing*{\section}{0pt}{3\baselineskip}{\baselineskip}
\titlespacing*{\subsection}{0pt}{2\baselineskip}{\baselineskip}
\titlespacing*{\subsubsection}{0pt}{\baselineskip}{\baselineskip}

% Code listings
\usepackage{verbatim}
\usepackage{spverbatim} % not sure what this is for but oh well

% For formatting matlab code specifically, **MUST COPY mcode.sty FILE INTO FOLDER**
%\usepackage[framed,numbered,autolinebreaks,useliterate]{mcode}
\usepackage{listings} % Use like this in place: \lstinputlisting{filname.ext}

% Enumeration, lists
\usepackage{enumerate, enumitem}
\usepackage{framed, color}

% Better tables
\usepackage{tabularx, bigstrut, multirow, booktabs, array}
\newcolumntype{+}{>{\global\let\currentrowstyle\relax}}
\newcolumntype{^}{>{\currentrowstyle}}
\newcommand{\rowstyle}[1]{\gdef\currentrowstyle{#1}%
    #1\ignorespaces
}% allows formatting of a whole row
\usepackage[table]{xcolor}

% Footnotes are symbols rather than numeric
\renewcommand{\thefootnote}{\fnsymbol{footnote}}
