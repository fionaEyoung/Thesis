%%
%% Set up abbreviations and definitions
\usepackage[acronym,
            style=super,
            nogroupskip=true,
            nonumberlist,
            seeautonumberlist,
            nopostdot]
            {glossaries}
\makeglossaries

%% MRI
\newacronym{mri}{MRI}{magnetic resonance imaging}
% Recursive acronym \protect solution from https://tex.stackexchange.com/a/502313
\newacronym{dmri}{dMRI}{diffusion \protect\ifglsused{mri}{MRI}{magnetic resonance imaging}}
\newacronym{imri}{iMRI}{intraoperative \protect\ifglsused{mri}{MRI}{magnetic resonance imaging}}
\newacronym{hardi}{HARDI}{high angular resolution diffusion imaging}
\newacronym{epi}{EPI}{echo planar imaging}
\newacronym{rsepi}{RS-EPI}{readout-segmented \protect\ifglsused{epi}{EPI}{echo planar imaging}}
\newacronym{rf}{RF}{radio frequency}
\newacronym{snr}{SNR}{signal to noise ratio}
\newacronym{fid}{FID}{free induction decay}
\newacronym{2dft}{2DFT}{2D Fourier transform}
\newacronym{ft}{FT}{Fourier transform}
% Image computing
\newacronym{csd}{CSD}{constrained spherical deconvolution}
\newacronym{odf}{ODF}{orientation distribution function}
\newacronym{fodf}{fODF}{fibre \protect\ifglused{odf}{ODF}{orientation distribution function}}
\newacronym{fod}{FOD}{fibre orientation distribution}
\newacronym{tod}{TOD}{track orientation distribution}
\newacronym{dti}{DTI}{diffusion tensor imaging}
\newacronym{dt}{DT}{diffusion tensor}
\newacronym{fa}{FA}{fractional anisotropy}
\newacronym{ssst}{SSST}{single-shell, single-tissue}
\newacronym{msmt}{MSMT}{multi-shell, multi-tissue}
%% Neuroanatomy
\newacronym{cns}{CNS}{central nervous system}
\newacronym{pns}{PNS}{peripheral nervous system}
\newacronym{cst}{CST}{corticospinal tract}
\newacronym{or}{OR}{optic radiation}
\newacronym{af}{AF}{arcuate fasciculus}
\newacronym{ifof}{IFOF}{inferior fronto-occipital fasciculus}
\newacronym{uf}{UF}{uncinate fasciculus}
\newacronym{wm}{WM}{white matter}
\newacronym{gm}{GM}{grey matter}
\newacronym{csf}{CSF}{cerebrospinal fluid}
\newacronym{crp}{CrP}{cerebellar peduncles}
\newacronym{cp}{CP}{cerebral peduncles}
\newacronym{cc}{CC}{corpus callosum}
\newacronym{ec}{EC}{external capsule}
\newacronym{slf}{SLF}{superior longitudinal fasciculus}
\newacronym{sfof}{SFOF}{superior fronto-occipital fasciculus}
%% Neurosurgery & Neurology
\newacronym{dbs}{DBS}{deep brain stimulation}
\newacronym{eor}{EOR}{extent of resection}
\newacronym{gtr}{GTR}{gross total resection}
\newacronym{hgg}{HGG}{high grade glioma}
\newacronym{lgg}{LGG}{low grade glioma}
\newacronym{eeg}{EEG}{electroencephalography}
\newacronym{ct}{CT}{computed tomography}
\newacronym{us}{US}{ultrasound}
%% Other
\newacronym{hcp}{HCP}{human connectome project}
\newacronym{dice}{DSC}{Dice similarity coefficient}
\newacronym{gdice}{gDSC}{generalised Dice similarity coefficient}
\newacronym{sh}{SH}{spherical harmonic}

%% Definitions
\newglossaryentry{tract}{
name=tract,
description={an anatomical structure of a group of neuron fibres with a common functional or anatomical organisation}
}
\newglossaryentry{fibrebundle}{
name=fibre bundle,
description={\emph{see} \gls{tract}}
}
\newglossaryentry{streamline}{
name=streamline,
description={a collection of points, in the context of tractography forming an abstraction of a single fibre}
}
\newglossaryentry{track}{
name=track,
description={\emph{see} \gls{streamline}}
}
\newglossaryentry{bundle}{
name=bundle,
description={\emph{see} \gls{streamlinebundle}}
}
\newglossaryentry{streamlinebundle}{
name=streamline bundle,
description={a collection of tractography streamlines, usually serving as a representation of, but not synonymous with, a corresponding anatomical tract}
}
\newglossaryentry{voxel}{
name=voxel,
description={``volume pixel"; the smallest spatial unit of a 3D digital image, akin to an image pixel in three dimensions }
}

%% Here we set up a new command to generate a list of abbreviations, e.g. for
%% float captions, formatted in a consistent way

% Define the separtors between abbreviation and definition, and between list items
% including any spaces!
\def\acrodefsep{ = }
\def\acrolistdelim{; }

% Process list of arbitratry length
\NewDocumentCommand{\acrolist}{>{\SplitList{,}}m}
    {\ProcessList{#1}{\acroformat}\firstitemtrue}
% Format the list using each item's glossary entry and the defined separators
\newif\iffirstitem
\firstitemtrue
\newcommand\acroformat[1]{%
  \iffirstitem
    \firstitemfalse
  \else
    \acrolistdelim %
  \fi
  \glsname{#1}\acrodefsep \glsdesc{#1}}
% ^ proper hangling of delimiter solution from https://tex.stackexchange.com/a/110906

%% Alternative solution from https://tex.stackexchange.com/a/110909 if the above causes issues
% \NewDocumentCommand{\acrolist}{>{\SplitList{,}}m}
%     {%
%     \def\acrodelim{\def\acrodelim{, }}%
%     \ProcessList{#1}{\acroformat} }
% \newcommand{\acroformat}[1]{\acrodelim\glsentryshort{#1} = \glsentrylong{#1}}
