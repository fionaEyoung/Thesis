\phantomsection
% The \appendix command resets the chapter counter, and changes the chapter numbering scheme to capital letters.
%\chapter{Appendices}
\appendix
\addtocontents{toc}{\protect\setcounter{tocdepth}{0}}

\chapter{Tractography parameters and ROI protocols}
\label{app:rois}

Default parameters as documented for the \verb|tckgen| command of MRtrix3 (release version 3.0.3, available at \url{https://mrtrix.readthedocs.io/en/3.0.3/reference/commands/tckgen.html}) (including \verb|-select 5000 -algorithm iFOD2|) were used for all tractography:

%%%%%%%%%%%%%%%%%%%%%%%%%%%%%%%%%%%%%%%%%%%%%%%%%%%%%%%%%%%%%%%%%%%%%%%%%%%%%%%%
\begin{center}
\begin{tabular}{ l l }\toprule
  Parameter & Value \\
 \midrule
 Algorithm      &   iFOD2\autocite{Tournier2010} \\
 Number of streamlines selected &   5000 \\
 Maximum angle  &   45\degree  \\
 Step size & $0.5 \, \mathsf{x}$ voxel size \\
 \glsentryshort{fod} amplitude threshold & 0.1 \\ \bottomrule
\end{tabular}
\end{center}
%%%%%%%%%%%%%%%%%%%%%%%%%%%%%%%%%%%%%%%%%%%%%%%%%%%%%%%%%%%%%%%%%%%%%%%%%%%%%%%%

In addition, the parameter \verb|-seed_unidirectional| was included for \gls{or} reconstructions, to ensure streamlines are propagated from a single direction out of the \gls{lgn}.


\section{ROI definitions}
\label{sec:rois}

The following ROI strategies were used for atlas constructions and subsequent validation tractography (differences between the two specified where applicable).
Visualisations of each ROI are shown on MNI152 template in Figures \ref{fig:rois.af}, \ref{fig:rois.cst} and \ref{fig:rois.or}.

\subsection{Arcuate fasciculus}

\begin{description}
  \item[Seed] White matter medial of angular gyrus, visible on coronal views of colour fractional anisotropy maps as a ``green triangle'', drawn on the coronal plane.
  Level of coronal plane selected from sagittal view by locating the central sulcus (Fig. \ref{fig:rois.af}, arrow).
  \item[Include] Descending section of the arcuate fasciculus, drawn on the axial plane
  \item[Exclude] Exclusion ROIs targeting: midline, superior fronto-occipital fasciculus, ipsilateral cerebral penduncles, sagittal stratum, corona radiata and external capsules.
\end{description}

The following publications were reviewed to inform the above ROI strategy: \textcite{Brown2014a},\textcite{Catani2002},\textcite{Catani2005},\textcite{Chen2015c},
\textcite{Eluvathingal2007},\textcite{Kamali2014},\textcite{Martino2013a},\textcite{Nucifora2005},
\textcite{Parker2005},\textcite{Bain2019},\textcite{Talozzi2018}

\begin{figure*}[h]
  \centering
    \includegraphics[width=\textwidth]{appendix/AF_include.png}
    \includegraphics[width=\textwidth]{appendix/AF_exclude.png}
  \caption{Seed (yellow), inclusion (green) and exclusion (red) regions of interest for the arcuate fasciculus. Arrow indicates central sulcus, landmark for seed ROI.}
  \label{fig:rois.af}
\end{figure*}

\subsection{Corticospinal tract}

Corticospinal tract tracography strategy differed between the atlas creation and general tractography applied to new subjects.

\begin{description}
  \item[Seed (atlas)] For the orientation atlas, Freesurfer cortical parcellations were used to obtain more complete coverage of the motor cortex via the following process:
  \begin{enumerate}
    \item Seed in precentral gyrus and output successful seed location
    \item Generate binary mask from successful seed locations, subtract from precentral gyrus mask to create seed mask
    \item Re-run tractography with second seed mask to cover rest of precentral gyrus
  \end{enumerate}
  \item[Seed (general)] Posterior limb of internal capsule, drawn on three consecutive axial slices
  \item[Include] Posterior limb on internal capsule (if not used for seed), cerebral penduncles, CST in mid-pons
  \item[Exclude] Cerebellar peduncles (drawn on coronal slice), medial lemniscus (drawn on axial slice), midline, superior fronto-occipital fasciculus,
\end{description}

The following publications were reviewed to inform the above ROI strategy: \textcite{Ciccarelli2006},\textcite{Han2010},\textcite{Hattingen2009a},
\textcite{Niu2016},\textcite{Radmanesh2015},\textcite{Reich2006},
\textcite{Rosenstock2017},\textcite{Szmuda2021},\textcite{Vargas2013}

\begin{figure*}[h]
  \centering
    \includegraphics[width=\textwidth]{appendix/CST_include.png}
    \includegraphics[width=\textwidth]{appendix/CST_exclude.png}
  \caption{Seed (yellow), inclusion (green) and exclusion (red) regions of interest for the corticospinal tract}
  \label{fig:rois.cst}
\end{figure*}

\subsection{Inferior fronto-occipital fasciculus}

\begin{description}
  \item[Seed] Temporal stem, between anterior tip if Meyer's loop and descending portion of the uncinate fasciculus
  \item[Include (atlas)] Posterior: inferior, middle, and superior occipital gyri and middle and superior occipital sulci (Freesurfer (v4.5) Destrieux atlas\autocite{Destrieux2010} (2009 version) parcellation labels 1\{1,2\}1\{02,19,20,58,59\}).
  Anterior: frontal pole, middle and inferior frontal gyri and sulci, orbital gyrus and sulci (Freesurfer labels 1\{1,2\}1\{01,05,15,54,12,13,14,53,63,24,65\})
  \item[Include (general)] Frontal lobe coronal slice, anterior to genu of the corpus callosum
  \item[Exclude] Coronal slice on frontal lobe at the level of the central sulcus, coronal slice on tip of anterior temporal lobe
\end{description}

The following publications reviewed to inform the above ROI strategy: \textcite{Martino2010},\textcite{Sarubbo2013},\textcite{Hau2016},
\textcite{Catani2008},\textcite{Wakana2007},\textcite{Wu2016}

\begin{figure*}[h]
  \centering
    \includegraphics[width=\textwidth]{appendix/IFO_include.png}
    \includegraphics[width=\textwidth]{appendix/IFO_exclude.png}
  \caption{Seed (yellow), inclusion (green) and exclusion (red) regions of interest for the inferior fronto-occipital fasciculus}
  \label{fig:rois.ifo}
\end{figure*}

\subsection{Optic radiation}

\begin{description}
  \item[Seed] Lateral geniculate nucleus (LGN; drawn on axial planes)
  \item[Include] Sagittal stratum (drawn on coronal plane)
  \item[Exclude] Coronal slice anterior of and axial slice inferior of most anterior point of lateral ventricles, axial slice at level of superior reach lateral ventricles, splenium of corpus callosum, fornix
\end{description}

The following publications reviewed to inform the above ROI strategy:
\textcite{Yogarajah2009},\textcite{Hofer2010},\textcite{Dayan2015}

\begin{figure*}[h]
  \centering
    \includegraphics[width=\textwidth]{appendix/OR_include.png}
    \includegraphics[width=\textwidth]{appendix/OR_exclude.png}
  \caption{Seed (yellow), inclusion (green) and exclusion (red) regions of interest for the optic radiation}
  \label{fig:rois.or}
\end{figure*}
