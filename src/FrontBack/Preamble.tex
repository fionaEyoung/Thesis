% I may change the way this is done in a future version,
%  but given that some people needed it, if you need a different degree title
%  (e.g. Master of Science, Master in Science, Master of Arts, etc)
%  uncomment the following 3 lines and set as appropriate (this *has* to be before \maketitle)
% \makeatletter
% \renewcommand {\@degree@string} {Master of Things}
% \makeatother

\title{ Fibre tract imaging with intraoperative diffusion MRI for neurosurgical navigation }
\author{ Fiona Young }
\department{ Department of Medical Physics and Biomedical Engineering \\ UCL GOS Institute of Child Health }

\maketitle
\makedeclaration

\begin{abstract} % 300 word limit
My research is about stuff.

It begins with a study of some stuff, and then some other stuff and things.

There is a 300-word limit on your abstract.
\end{abstract}

\begin{impactstatement}

	UCL theses now have to include an impact statement. \textit{(I think for REF reasons?)} The following text is the description from the guide linked from the formatting and submission website of what that involves. (Link to the guide: {\scriptsize \url{http://www.grad.ucl.ac.uk/essinfo/docs/Impact-Statement-Guidance-Notes-for-Research-Students-and-Supervisors.pdf}})

\begin{quote}
The statement should describe, in no more than 500 words, how the expertise, knowledge, analysis,
discovery or insight presented in your thesis could be put to a beneficial use. Consider benefits both
inside and outside academia and the ways in which these benefits could be brought about.

The benefits inside academia could be to the discipline and future scholarship, research methods or
methodology, the curriculum; they might be within your research area and potentially within other
research areas.

The benefits outside academia could occur to commercial activity, social enterprise, professional
practice, clinical use, public health, public policy design, public service delivery, laws, public
discourse, culture, the quality of the environment or quality of life.

The impact could occur locally, regionally, nationally or internationally, to individuals, communities or
organisations and could be immediate or occur incrementally, in the context of a broader field of
research, over many years, decades or longer.

Impact could be brought about through disseminating outputs (either in scholarly journals or
elsewhere such as specialist or mainstream media), education, public engagement, translational
research, commercial and social enterprise activity, engaging with public policy makers and public
service delivery practitioners, influencing ministers, collaborating with academics and non-academics
etc.

Further information including a searchable list of hundreds of examples of UCL impact outside of
academia please see \url{https://www.ucl.ac.uk/impact/}. For thousands more examples, please see
\url{http://results.ref.ac.uk/Results/SelectUoa}.
\end{quote}
\end{impactstatement}

\begin{acknowledgements}
THANKS!
\end{acknowledgements}

%	Self-plagiarism declaration form template for these typeset in LaTeX
%	Prepared by David Sheard 2022 and made available free of copyright

%	If you use the results of your own published, accepted or submitted data (text or figures) in your final
%	doctoral thesis, you have to give a clear indication of the previous work, stating the exact source of the
%	previous material, irrespective of whether copyright is owned by you or by a publisher. This indication
%	should take the form of
%		a) an appropriate citation of the original source in the relevant Chapter; and
%		b) completion of the UCL Research Paper Declaration form---this should be embedded after the
%		Acknowledgements page in the thesis.

%	For more information consult the following links:
%	\url{https://www.grad.ucl.ac.uk/essinfo/guidance-on-selfplagiarism/?utm_source=Students\%27+Union+UCL\&utm_campaign=2ed9e73ab7-\&utm_medium=email\&utm_term=0_fe8c0cbcf2-2ed9e73ab7-209240456\&mc_cid=2ed9e73ab7\&mc_eid=0496c22bfc}
%	\url{https://www.grad.ucl.ac.uk/essinfo/guidance-on-selfplagiarism/Declaration-form_published-work-in-thesis.docx}

%	I recommend using this template by simply copying everything between \begin{document} and \end{document}
%	into your thesis after the acknowledgements section. By changing the preamble appropriately this template
%	may also be altered to work using \input{...} or the subfiles package.

% \documentclass[12pt, twoside]{article}
%
% \usepackage[a4paper,inner=40mm,outer=20mm, top=30mm, bottom=30mm]{geometry}
% \usepackage{amssymb}
% \usepackage{array}
% \usepackage{setspace}
% 	\setstretch{1.5}
%
% \begin{document}
{\sffamily
\section*{UCL Research Paper Declaration Form: referencing the doctoral candidate’s own published work(s)}
% Uncomment the following line if you would like to add this declaration to your table of contents
% \addcontentsline{toc}{section}{UCL Research Paper Declaration Form}
%
% Please use this form to declare if parts of your thesis are already available in another format,
% e.g. if data, text, or figures:
% •	have been uploaded to a preprint server
% •	are in submission to a peer-reviewed publication
% •	have been published in a peer-reviewed publication, e.g. journal, textbook.
%
% This form should be completed as many times as necessary. For instance, if a student had seven
% thesis chapters, two of which having material which had been published, they would complete this form twice.

\begin{enumerate}[leftmargin=*,label={\bfseries\arabic*.}]\itemsep0em
	\item \textbf{For a research manuscript that has already been published} (if not yet published, please skip to section 2)\textbf{:}
	%
	\begin{enumerate}[label={\alph*)}]\itemsep0em
	%
	\item \textbf{What is the title of the manuscript?}

	\citetitle{Young2022}

	\item \textbf{Please include a link to or doi for the work:}

	\url{https://doi.org/10.1007/s11548-022-02617-z}

	\item \textbf{Where was the work published?}

	\citefield{Young2022}{journaltitle}% Answer here: e.g. journal name

	\item \textbf{Who published the work?}

	\citelist{Young2022}{publisher}% Answer here: e.g. Elsevier/Oxford University Press

	\item \textbf{When was the work published?}

	\citefield{Young2022}{year}% Answer here

	\item \textbf{List the manuscript's authors in the order they appear on the publication:}
	\citeauthor*{Young2022}% Answer here

	\item \textbf{Was the work peer reviewd?}

	Yes

	\item \textbf{Have you retained the copyright?}

	Yes, from the Licence to Publish agreement:
	\begin{quote}
		Ownership of copyright in the Article will be vested in the name of the Author.
	\end{quote}% Answer here

	\item \textbf{Was an earlier form of the manuscript uploaded to a preprint server (e.g. medRxiv)? If `Yes’, please give a link or doi}

	No% Answer here:
	\\
	If ‘No’, please seek permission from the relevant publisher and check the box next to the below statement:
	%
		\begin{itemize}\itemsep0em
		% To check this box, replace \Box with \boxtimes
		\item[$\boxtimes$] {\itshape I acknowledge permission of the publisher named under 1d to include in this thesis portions of the publication named as included in 1c.}
		\end{itemize}
	%
	\end{enumerate}
%
\item \textbf{For a research manuscript prepared for publication but that has not yet been published} (if already published, please skip to section 3)\textbf{:}
%
\begin{enumerate}[label={\alph*)}]\itemsep0em
	%
	\item \textbf{What is the current title of the manuscript?}
	% Answer here:
	\item \textbf{Has the manuscript been uploaded to a preprint server `e.g. medRxiv'?
	\\
	If `Yes', please please give a link or doi:}
	% Answer here:
	\item \textbf{Where is the work intended to be published?}
	% Answer here: e.g. journal name
	\item \textbf{List the manuscript's authors in the intended authorship order:}
	% Answer here
	\item \textbf{Stage of publication:}
	% answer here: e.g. in submission
	%
\end{enumerate}

\item \textbf{For multi-authored work, please give a statement of contribution covering all authors} (if single-author, please skip to section 4)\textbf{:}
\begin{description}[font=\sffamily]
	\item[Fiona Young] Methodology (conceptualisation and implementation), analysis, manuscript original draft and graphics
	\item[Kristian Aquilina] Supervision, manuscript review and editing
	\item[Chris A. Clark] Supervision, manuscript review and editing
	\item[Jonathan D. Clayden] Conceptualisation, supervision, manuscript review and editing
\end{description}
% Answer here
\item \textbf{In which chapter(s) of your thesis can this material be found?}
% Answer here

Chapters \ref{chap:reg}, \ref{chap:applications}

\end{enumerate}

\textbf{e-Signatures confirming that the information above is accurate}
(this form should be co-signed by the supervisor/ senior author unless this is not appropriate, e.g. if the paper was a single-author work)\textbf{:}\\
\textbf{}\\
\textbf{Candidate:}\\
\textbf{Date:}\\
% this form should be co-signed by the supervisor/ senior author unless this is not appropriate, e.g. if the paper was a single-author work):
\textbf{}\\
\textbf{Supervisor/Senior Author signature} (where appropriate)\textbf{:}\\
\textbf{Date:}
%

%%%%%%%%%%%%%%%%%%%%%%%%%%%%% --- HBM --- %%%%%%%%%%%%%%%%%%%%%%%%%%%%%%%%%%%%%%
\newpage
\begin{enumerate}[leftmargin=*,label={\bfseries\arabic*.}]\itemsep0em
%
\item \textbf{For a research manuscript that has already been published} (if not yet published, please skip to section 2)\textbf{:}
%
\begin{enumerate}[label={\alph*)}]\itemsep0em
	%
	\item \textbf{What is the title of the manuscript?}

	\citetitle{Young2024}

	\item \textbf{Please include a link to or doi for the work:}

	\citefield{Young2024}{doi}

	\item \textbf{Where was the work published?}

	\citefield{Young2024}{journaltitle}% Answer here: e.g. journal name

	\item \textbf{Who published the work?}

	\citelist{Young2024}{publisher}% Answer here: e.g. Elsevier/Oxford University Press

	\item \textbf{When was the work published?}

	\citefield{Young2024}{year}% Answer here

	\item \textbf{List the manuscript's authors in the order they appear on the publication:}

	\citeauthor*{Young2024}% Answer here

	\item \textbf{Was the work peer reviewd?}

	Yes

	\item \textbf{Have you retained the copyright?}

	Yes, from license agreement:
	\begin{quote}
		The Author and each Co-author or, if applicable, the Author’s or Co-author’s employer, retains all proprietary rights, such as copyright
	\end{quote}

	\item \textbf{Was an earlier form of the manuscript uploaded to a preprint server (e.g. medRxiv)? If `Yes’, please give a link or doi}

	No% Answer here:
	\\
	If ‘No’, please seek permission from the relevant publisher and check the box next to the below statement:
	%
	\begin{itemize}\itemsep0em
	% To check this box, replace \Box with \boxtimes
	\item[$\boxtimes$] {\itshape I acknowledge permission of the publisher named under 1d to include in this thesis portions of the publication named as included in 1c.}
	\end{itemize}
	%
\end{enumerate}
%
\item \textbf{For a research manuscript prepared for publication but that has not yet been published} (if already published, please skip to section 3)\textbf{:}
%
\begin{enumerate}[label={\alph*)}]\itemsep0em
	%
	\item \textbf{What is the current title of the manuscript?}
	% Answer here:
	\item \textbf{Has the manuscript been uploaded to a preprint server `e.g. medRxiv'?
	\\
	If `Yes', please please give a link or doi:}
	% Answer here:
	\item \textbf{Where is the work intended to be published?}
	% Answer here: e.g. journal name
	\item \textbf{List the manuscript's authors in the intended authorship order:}
	% Answer here
	\item \textbf{Stage of publication:}
	% answer here: e.g. in submission
	%
\end{enumerate}

\item \textbf{For multi-authored work, please give a statement of contribution covering all authors} (if single-author, please skip to section 4)\textbf{:}
\begin{description}[font=\sffamily]
	\item[Fiona Young] Methodology (conceptualisation and implementation), analysis, manuscript original draft and graphics
	\item[Kristian Aquilina] Supervision, conceptualisation, manuscript review and editing
	\item[Laura Mancini] Data contribution, manuscript review and editing
	\item[Kiran K. Seunarine] Data contribution and curation
	\item[Chris A. Clark] Supervision, manuscript review and editing
	\item[Jonathan D. Clayden] Conceptualisation, supervision, manuscript review and editing
\end{description}
% Answer here
\item \textbf{In which chapter(s) of your thesis can this material be found?}
% Answer here

Chapters \ref{chap:review}, \ref{chap:atlas}, \ref{chap:reg} (Section \ref{sec:reg1}), \ref{chap:eval}, Appendix
\end{enumerate}

\textbf{e-Signatures confirming that the information above is accurate}
(this form should be co-signed by the supervisor/ senior author unless this is not appropriate, e.g. if the paper was a single-author work)\textbf{:}\\
\textbf{}\\
\textbf{Candidate:}\\
\textbf{Date:}\\
% this form should be co-signed by the supervisor/ senior author unless this is not appropriate, e.g. if the paper was a single-author work):
\textbf{}\\
\textbf{Supervisor/Senior Author signature} (where appropriate)\textbf{:}\\
\textbf{Date:}
%

%%%%%%%%%%%%%%%%%%%%%%%%%%%%% --- IEEE --- %%%%%%%%%%%%%%%%%%%%%%%%%%%%%%%%%%%%%
\newpage
\begin{enumerate}[leftmargin=*,label={\bfseries\arabic*.}]\itemsep0em
	%
	\item \textbf{For a research manuscript that has already been published} (if not yet published, please skip to section 2)\textbf{:}
	%
	\begin{enumerate}[label={\alph*)}]\itemsep0em
	%
	\item \textbf{What is the title of the manuscript?}

	\item \textbf{Please include a link to or doi for the work:}

	\item \textbf{Where was the work published?}

	\item \textbf{Who published the work?}

	\item \textbf{When was the work published?}

	\item \textbf{List the manuscript's authors in the order they appear on the publication:}

	\item \textbf{Was the work peer reviewd?}

	\item \textbf{Have you retained the copyright?}

	\item \textbf{Was an earlier form of the manuscript uploaded to a preprint server (e.g. medRxiv)? If ‘Yes’, please give a link or doi}
	\\
	If ‘No’, please seek permission from the relevant publisher and check the box next to the below statement:
	%
\begin{itemize}\itemsep0em
% To check this box, replace \Box with \boxtimes
\item[$\Box$] {\itshape I acknowledge permission of the publisher named under 1d to include in this thesis portions of the publication named as included in 1c.}
\end{itemize}
%
\end{enumerate}
%
\item \textbf{For a research manuscript prepared for publication but that has not yet been published} (if already published, please skip to section 3)\textbf{:}
%
\begin{enumerate}[label={\alph*)}]\itemsep0em
	%
	\item \textbf{What is the current title of the manuscript?}

	\citetitle{Young2023}
	\item \textbf{Has the manuscript been uploaded to a preprint server `e.g. medRxiv'?
	\\
	If `Yes', please please give a link or doi:}

	Yes, \url{https://www.researchgate.net/publication/375601134_Training_Data_Requirements_for_Atlas-Based_Brain_Fibre_Tract_Identification}
	% Answer here:
	\item \textbf{Where is the work intended to be published?}
	% Answer here: e.g. journal name

	Proceedings of the \citefield{Young2023}{eventtitle}
	\item \textbf{List the manuscript's authors in the intended authorship order:}
	% Answer here

	\citeauthor*{Young2023}
	\item \textbf{Stage of publication:}
	% answer here: e.g. in submission

	Work presented at conference, copyright transferred to IEEE, yet now indication of if/when proceedings may be published (copyright on submitted version retained).
\end{enumerate}

\item \textbf{For multi-authored work, please give a statement of contribution covering all authors} (if single-author, please skip to section 4)\textbf{:}
\begin{description}[font=\sffamily]
	\item[Fiona Young] Methodology (conceptualisation and implementation), analysis, manuscript original draft and graphics
	\item[Kristian Aquilina] Supervision
	\item[Chris A. Clark] Supervision
	\item[Jonathan D. Clayden] Conceptualisation, supervision, manuscript review and editing
\end{description}
% Answer here
\item \textbf{In which chapter(s) of your thesis can this material be found?}
% Answer here

Chapter \ref{chap:atlas}
\end{enumerate}

\textbf{e-Signatures confirming that the information above is accurate}
(this form should be co-signed by the supervisor/ senior author unless this is not appropriate, e.g. if the paper was a single-author work)\textbf{:}\\
\textbf{}\\
\textbf{Candidate:}\\
\textbf{Date:}\\
% this form should be co-signed by the supervisor/ senior author unless this is not appropriate, e.g. if the paper was a single-author work):
\textbf{}\\
\textbf{Supervisor/Senior Author signature} (where appropriate)\textbf{:}\\
\textbf{Date:}
%

%%%%%%%%%%%%%%%%%%%%%%%%% --- Diffusion ISMRM --- %%%%%%%%%%%%%%%%%%%%%%%%%%%%%
\newpage
\begin{enumerate}[leftmargin=*,label={\bfseries\arabic*.}]\itemsep0em
	%
	\item \textbf{For a research manuscript that has already been published} (if not yet published, please skip to section 2)\textbf{:}
	%
	\begin{enumerate}[label={\alph*)}]\itemsep0em
	%
	\item \textbf{What is the title of the manuscript?}

	\item \textbf{Please include a link to or doi for the work:}

	\item \textbf{Where was the work published?}

	\item \textbf{Who published the work?}

	\item \textbf{When was the work published?}

	\item \textbf{List the manuscript's authors in the order they appear on the publication:}

	\item \textbf{Was the work peer reviewd?}

	\item \textbf{Have you retained the copyright?}

	\item \textbf{Was an earlier form of the manuscript uploaded to a preprint server (e.g. medRxiv)? If ‘Yes’, please give a link or doi}
	\\
	If ‘No’, please seek permission from the relevant publisher and check the box next to the below statement:
	%
\begin{itemize}\itemsep0em
% To check this box, replace \Box with \boxtimes
\item[$\Box$] {\itshape I acknowledge permission of the publisher named under 1d to include in this thesis portions of the publication named as included in 1c.}
\end{itemize}
%
\end{enumerate}
%
\item \textbf{For a research manuscript prepared for publication but that has not yet been published} (if already published, please skip to section 3)\textbf{:}
%
\begin{enumerate}[label={\alph*)}]\itemsep0em
	%
	\item \textbf{What is the current title of the manuscript?}
	% Answer here:

	\citetitle{Young2022a}
	\item \textbf{Has the manuscript been uploaded to a preprint server `e.g. medRxiv'?
	\\
	If `Yes', please please give a link or doi:}

	Yes, \url{https://www.researchgate.net/publication/367116849_Stability_of_white_matter_tract_segmentation_methods_with_decreasing_data_quality}
	% Answer here:
	\item \textbf{Where is the work intended to be published?}
	% Answer here: e.g. journal name

	N/A
	\item \textbf{List the manuscript's authors in the intended authorship order:}
	% Answer here

	\citeauthor*{Young2022a}
	\item \textbf{Stage of publication:}
	% answer here: e.g. in submission

	Work presented at conference, no proceedings published.
\end{enumerate}

\item \textbf{For multi-authored work, please give a statement of contribution covering all authors} (if single-author, please skip to section 4)\textbf{:}
\begin{description}[font=\sffamily]
	\item[Fiona Young] Methodology (conceptualisation and implementation), analysis, manuscript original draft and graphics
	\item[Jonathan D. Clayden] Conceptualisation, supervision, manuscript review and editing
\end{description}
% Answer here
\item \textbf{In which chapter(s) of your thesis can this material be found?}
% Answer here

Chapter \ref{chap:applications}
\end{enumerate}

\textbf{e-Signatures confirming that the information above is accurate}
(this form should be co-signed by the supervisor/ senior author unless this is not appropriate, e.g. if the paper was a single-author work)\textbf{:}\\
\textbf{}\\
\textbf{Candidate:}\\
\textbf{Date:}\\
% this form should be co-signed by the supervisor/ senior author unless this is not appropriate, e.g. if the paper was a single-author work):
\textbf{}\\
\textbf{Supervisor/Senior Author signature} (where appropriate)\textbf{:}\\
\textbf{Date:}
%
}% End font switch


\setcounter{tocdepth}{2}
% Setting this higher means you get contents entries for
%  more minor section headers.

\tableofcontents
% \listoffigures
% \listoftables
