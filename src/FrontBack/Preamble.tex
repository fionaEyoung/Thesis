% I may change the way this is done in a future version,
%  but given that some people needed it, if you need a different degree title
%  (e.g. Master of Science, Master in Science, Master of Arts, etc)
%  uncomment the following 3 lines and set as appropriate (this *has* to be before \maketitle)
% \makeatletter
% \renewcommand {\@degree@string} {Master of Things}
% \makeatother

\title{ Fibre tract imaging with intraoperative diffusion MRI for neurosurgical navigation }
\author{ Fiona Young }
\department{ Department of Medical Physics and Biomedical Engineering \\ and \\ UCL Great Ormond Street Institute of Child Health }

\maketitle
\makedeclaration

\begin{abstract} % 300 word limit
Mapping and understanding the brain's structure and function is never more critical than when it suffers injury or illness.
Lifesaving neurosurgical procedures may put essential neural communication pathways called \glsentrylong{wm} tracts at risk, with grave consequences for the patient, so accurately depicting their location using \gls{dmri} is becoming a key component of modern neurosurgical practice.
More recently, obtaining new \gls{imri} partway through surgery has demonstrated potential to further improve outcomes by providing updated anatomical information after the dynamic effects of intraoperative brain shift have diminished the accuracy of preoperative imaging.
With the ability to sample directional water diffusivity in tissue, \gls{dmri} produces millimetre-scale maps of \glsentrylong{wm} fibre orientations which are key to reconstructing individual tracts.
However, established image computational methods suffer from limitations in accuracy and practicality which restrict the wider clinical uptake of \gls{dmri} \glsentrylong{wm} imaging generally, and particularly for \gls{imri}.
After an in depth review of the state of the art in \glsentrylong{wm} imaging and image-guided neurosurgery, this thesis explores the development of a novel \glsentrylong{wm} tract mapping tool, named \textit{tractfinder}, which applies \textit{a priori} anatomical knowledge encoded within a statistical tract orientation and location atlas to achieve rapid tract segmentation in a patient \gls{dmri} scan.
The proposed pipeline includes explicit patient-specific modelling of tumour deformation effects, an element missing from many research-oriented tract reconstruction approaches.
Tractfinder's effectiveness in a range of applications is detailed through thorough quantitative evaluation, while clinical case studies demonstrate its key advantages over existing approaches.
In addition, the technical and practical challenges of intraoperative imaging are explored together with their implications for effective clinical translation of advanced \gls{dmri}-based \glsentrylong{wm} imaging.
\end{abstract}

\begin{impactstatement}
This thesis explores the use of \gls{dmri}'s ability to map brain fibre pathways (``tracts'') in neurosurgical planning and guidance, and in particular the potential for more advanced \gls{dmri} analysis methods than those currently under routine clinical use to improve outcomes for neurology patients.

Though \gls{imri} has gradually been recognised as a beneficial technology for improving neurosurgery, the use of high angular resolution \gls{dmri} specifically has not been widely adopted, partly because the associated potential benefits have yet to be determined.
The methods described in this thesis may make \gls{dmri} analysis more accessible to \gls{imri} procedures, which would allow their advantages to be systematically studied.
Over the longer term, intraoperative \gls{dmri}, if found to be beneficial, may be incorporated into national healthcare guidelines, as preoperative \gls{dti} and general \gls{imri} currently are.
In addition, the proposal of an alternative to streamline tractography, the current standard for \gls{wm} tract imaging, which relies less on the availability of imaging experts, time, and computational resources, could widen the availability of tract imaging for preoperative planning to more health centres, who may have the requisite equipment (\gls{mri} capabilities) but lack expertise in and/or staff availability for applying modern research methods.

The tract orientation atlas framework put forth here could inform the study of \gls{wm} architecture and connectivity in individuals and across populations, an area in which we are still lacking a lot of understanding, and which will be critical in the future study and combatting of psychological and neurological disorders.
It also contributes to current discussions on how to adopt modern data science methods and innovation in a field, medical imaging, where appropriate data is often scarce and difficult to annotate.
Brain shift is an intraoperative phenomenon which presents a key challenge to image guided surgery techniques.
The tumour deformation modelling described in this thesis represents a novel solution to achieving acceptable \gls{wm} tract segmentations on both pre- and intraoperative imaging, in patients with deforming tumours, which may be applicable to other imaging modalities as well as to tumours outside of the brain.

Findings presented in this thesis have been shared with the wider research community at several international meetings, and have led to funding being granted for a two year follow-on prospective study to apply these methods in intraoperative cases at \gls{gosh}, London.
All image processing tools developed for this thesis, including data and code for tumour deformation modelling and tract segmentation, have been made available for researchers to apply in their own data.
In addition, a patent application covering much of the methodology has been submitted and is currently under examination.
If granted, the intellectual property could be licensed to national or international commercial producers of neuronavigational equipment.
\end{impactstatement}

\begin{acknowledgements}
  Instead of living the sad doctoral student cliché, I've had a thoroughly good time working on this project, and I owe much of that enjoyment to having incredibly supportive and insightful supervisors: Professor Jon Clayden, who has kept me on course with dedicated kindness, immeasurable patience, and good humour, Mr Kristian Aquilina, who has given me the most valuable feedback and encouragement, and Professor Chris Clark, who has provided a wealth of knowledge and experience.
  I also sincerely thank my examiners, Dr Eleftheria (Laura) Panagiotaki and Professor Stephen Price, for their time, constructive and engaging discussions, and for ensuring my doctoral experience wrapped up in the most fulfilling way possible.

  My thanks go to the present and former staff and researchers at Great Ormond Street Hospital and the Institute for Child Health who have provided invaluable support and assistance, shared their data, and allowed me in small ways to participate in the ongoing development of the iMRI clinical service at GOSH, including Dr Kiran Seunarine, Dr Patrick Hales, Dr Enrico de Vita, and Dr Jan Sedlacik. Thanks as well to Dr Laura Mancini of the National Hospital for Neurology and Neurosurgery, for sharing their data and experiences with intraoperative diffusion imaging, and for improving this project with feedback and insight throughout.

  I'm grateful to the endless community of open source developers, maintainers, and forum members who give their time and talents to create the tools that make conducting and documenting research manageable and fun. Also, writing isn't my favourite activity, and I could hardly have produced this manuscript without the companionable and beautiful works of Hans Zimmer and Nils Frahm.

  I have amazing friends and family who have lifted my spirits from near and far, over walks, shared TV nights, pub quizzes, and voice notes; I love you all. And thank you Barley, for never (as far as I know) asking me when my thesis is due, and only ever asking me for my yoghurt.
\end{acknowledgements}

%	Self-plagiarism declaration form template for these typeset in LaTeX
%	Prepared by David Sheard 2022 and made available free of copyright

%	If you use the results of your own published, accepted or submitted data (text or figures) in your final
%	doctoral thesis, you have to give a clear indication of the previous work, stating the exact source of the
%	previous material, irrespective of whether copyright is owned by you or by a publisher. This indication
%	should take the form of
%		a) an appropriate citation of the original source in the relevant Chapter; and
%		b) completion of the UCL Research Paper Declaration form---this should be embedded after the
%		Acknowledgements page in the thesis.

%	For more information consult the following links:
%	\url{https://www.grad.ucl.ac.uk/essinfo/guidance-on-selfplagiarism/?utm_source=Students\%27+Union+UCL\&utm_campaign=2ed9e73ab7-\&utm_medium=email\&utm_term=0_fe8c0cbcf2-2ed9e73ab7-209240456\&mc_cid=2ed9e73ab7\&mc_eid=0496c22bfc}
%	\url{https://www.grad.ucl.ac.uk/essinfo/guidance-on-selfplagiarism/Declaration-form_published-work-in-thesis.docx}

%	I recommend using this template by simply copying everything between \begin{document} and \end{document}
%	into your thesis after the acknowledgements section. By changing the preamble appropriately this template
%	may also be altered to work using \input{...} or the subfiles package.

% \documentclass[12pt, twoside]{article}
%
% \usepackage[a4paper,inner=40mm,outer=20mm, top=30mm, bottom=30mm]{geometry}
% \usepackage{amssymb}
% \usepackage{array}
% \usepackage{setspace}
% 	\setstretch{1.5}
%
% \begin{document}
{\sffamily
\section*{UCL Research Paper Declaration Form: referencing the doctoral candidate’s own published work(s)}
% Uncomment the following line if you would like to add this declaration to your table of contents
% \addcontentsline{toc}{section}{UCL Research Paper Declaration Form}
%
% Please use this form to declare if parts of your thesis are already available in another format,
% e.g. if data, text, or figures:
% •	have been uploaded to a preprint server
% •	are in submission to a peer-reviewed publication
% •	have been published in a peer-reviewed publication, e.g. journal, textbook.
%
% This form should be completed as many times as necessary. For instance, if a student had seven
% thesis chapters, two of which having material which had been published, they would complete this form twice.

\begin{enumerate}[leftmargin=*,label={\bfseries\arabic*.}]\itemsep0em
	\item \textbf{For a research manuscript that has already been published} (if not yet published, please skip to section 2)\textbf{:}
	%
	\begin{enumerate}[label={\alph*)}]\itemsep0em
	%
	\item \textbf{What is the title of the manuscript?}

	\citetitle{Young2022}

	\item \textbf{Please include a link to or doi for the work:}

	\url{https://doi.org/10.1007/s11548-022-02617-z}

	\item \textbf{Where was the work published?}

	\citefield{Young2022}{journaltitle}% Answer here: e.g. journal name

	\item \textbf{Who published the work?}

	\citelist{Young2022}{publisher}% Answer here: e.g. Elsevier/Oxford University Press

	\item \textbf{When was the work published?}

	\citefield{Young2022}{year}% Answer here

	\item \textbf{List the manuscript's authors in the order they appear on the publication:}
	\citeauthor*{Young2022}% Answer here

	\item \textbf{Was the work peer reviewd?}

	Yes

	\item \textbf{Have you retained the copyright?}

	Yes, from the Licence to Publish agreement:
	\begin{quote}
		Ownership of copyright in the Article will be vested in the name of the Author.
	\end{quote}% Answer here

	\item \textbf{Was an earlier form of the manuscript uploaded to a preprint server (e.g. medRxiv)? If `Yes’, please give a link or doi}

	No% Answer here:
	\\
	If ‘No’, please seek permission from the relevant publisher and check the box next to the below statement:
	%
		\begin{itemize}\itemsep0em
		% To check this box, replace \Box with \boxtimes
		\item[$\boxtimes$] {\itshape I acknowledge permission of the publisher named under 1d to include in this thesis portions of the publication named as included in 1c.}
		\end{itemize}
	%
	\end{enumerate}
%
\item \textbf{For a research manuscript prepared for publication but that has not yet been published} (if already published, please skip to section 3)\textbf{:}
%
\begin{enumerate}[label={\alph*)}]\itemsep0em
	%
	\item \textbf{What is the current title of the manuscript?}
	% Answer here:
	\item \textbf{Has the manuscript been uploaded to a preprint server `e.g. medRxiv'?
	\\
	If `Yes', please please give a link or doi:}
	% Answer here:
	\item \textbf{Where is the work intended to be published?}
	% Answer here: e.g. journal name
	\item \textbf{List the manuscript's authors in the intended authorship order:}
	% Answer here
	\item \textbf{Stage of publication:}
	% answer here: e.g. in submission
	%
\end{enumerate}

\item \textbf{For multi-authored work, please give a statement of contribution covering all authors} (if single-author, please skip to section 4)\textbf{:}
\begin{description}[font=\sffamily]
	\item[Fiona Young] Methodology (conceptualisation and implementation), analysis, manuscript original draft and graphics
	\item[Kristian Aquilina] Supervision, manuscript review and editing
	\item[Chris A. Clark] Supervision, manuscript review and editing
	\item[Jonathan D. Clayden] Conceptualisation, supervision, manuscript review and editing
\end{description}
% Answer here
\item \textbf{In which chapter(s) of your thesis can this material be found?}
% Answer here

Chapters \ref{chap:reg}, \ref{chap:applications}

\end{enumerate}

\textbf{e-Signatures confirming that the information above is accurate}
(this form should be co-signed by the supervisor/ senior author unless this is not appropriate, e.g. if the paper was a single-author work)\textbf{:}\\
\textbf{}\\
\textbf{Candidate:}\\
\textbf{Date:}\\
% this form should be co-signed by the supervisor/ senior author unless this is not appropriate, e.g. if the paper was a single-author work):
\textbf{}\\
\textbf{Supervisor/Senior Author signature} (where appropriate)\textbf{:}\\
\textbf{Date:}
%

%%%%%%%%%%%%%%%%%%%%%%%%%%%%% --- HBM --- %%%%%%%%%%%%%%%%%%%%%%%%%%%%%%%%%%%%%%
\newpage
\begin{enumerate}[leftmargin=*,label={\bfseries\arabic*.}]\itemsep0em
%
\item \textbf{For a research manuscript that has already been published} (if not yet published, please skip to section 2)\textbf{:}
%
\begin{enumerate}[label={\alph*)}]\itemsep0em
	%
	\item \textbf{What is the title of the manuscript?}

	\citetitle{Young2024}

	\item \textbf{Please include a link to or doi for the work:}

	\citefield{Young2024}{doi}

	\item \textbf{Where was the work published?}

	\citefield{Young2024}{journaltitle}% Answer here: e.g. journal name

	\item \textbf{Who published the work?}

	\citelist{Young2024}{publisher}% Answer here: e.g. Elsevier/Oxford University Press

	\item \textbf{When was the work published?}

	\citefield{Young2024}{year}% Answer here

	\item \textbf{List the manuscript's authors in the order they appear on the publication:}

	\citeauthor*{Young2024}% Answer here

	\item \textbf{Was the work peer reviewd?}

	Yes

	\item \textbf{Have you retained the copyright?}

	Yes, from license agreement:
	\begin{quote}
		The Author and each Co-author or, if applicable, the Author’s or Co-author’s employer, retains all proprietary rights, such as copyright
	\end{quote}

	\item \textbf{Was an earlier form of the manuscript uploaded to a preprint server (e.g. medRxiv)? If `Yes’, please give a link or doi}

	No% Answer here:
	\\
	If ‘No’, please seek permission from the relevant publisher and check the box next to the below statement:
	%
	\begin{itemize}\itemsep0em
	% To check this box, replace \Box with \boxtimes
	\item[$\boxtimes$] {\itshape I acknowledge permission of the publisher named under 1d to include in this thesis portions of the publication named as included in 1c.}
	\end{itemize}
	%
\end{enumerate}
%
\item \textbf{For a research manuscript prepared for publication but that has not yet been published} (if already published, please skip to section 3)\textbf{:}
%
\begin{enumerate}[label={\alph*)}]\itemsep0em
	%
	\item \textbf{What is the current title of the manuscript?}
	% Answer here:
	\item \textbf{Has the manuscript been uploaded to a preprint server `e.g. medRxiv'?
	\\
	If `Yes', please please give a link or doi:}
	% Answer here:
	\item \textbf{Where is the work intended to be published?}
	% Answer here: e.g. journal name
	\item \textbf{List the manuscript's authors in the intended authorship order:}
	% Answer here
	\item \textbf{Stage of publication:}
	% answer here: e.g. in submission
	%
\end{enumerate}

\item \textbf{For multi-authored work, please give a statement of contribution covering all authors} (if single-author, please skip to section 4)\textbf{:}
\begin{description}[font=\sffamily]
	\item[Fiona Young] Methodology (conceptualisation and implementation), analysis, manuscript original draft and graphics
	\item[Kristian Aquilina] Supervision, conceptualisation, manuscript review and editing
	\item[Laura Mancini] Data contribution, manuscript review and editing
	\item[Kiran K. Seunarine] Data contribution and curation
	\item[Chris A. Clark] Supervision, manuscript review and editing
	\item[Jonathan D. Clayden] Conceptualisation, supervision, manuscript review and editing
\end{description}
% Answer here
\item \textbf{In which chapter(s) of your thesis can this material be found?}
% Answer here

Chapters \ref{chap:review}, \ref{chap:atlas}, \ref{chap:reg} (Section \ref{sec:reg1}), \ref{chap:eval}, Appendix
\end{enumerate}

\textbf{e-Signatures confirming that the information above is accurate}
(this form should be co-signed by the supervisor/ senior author unless this is not appropriate, e.g. if the paper was a single-author work)\textbf{:}\\
\textbf{}\\
\textbf{Candidate:}\\
\textbf{Date:}\\
% this form should be co-signed by the supervisor/ senior author unless this is not appropriate, e.g. if the paper was a single-author work):
\textbf{}\\
\textbf{Supervisor/Senior Author signature} (where appropriate)\textbf{:}\\
\textbf{Date:}
%

%%%%%%%%%%%%%%%%%%%%%%%%%%%%% --- IEEE --- %%%%%%%%%%%%%%%%%%%%%%%%%%%%%%%%%%%%%
\newpage
\begin{enumerate}[leftmargin=*,label={\bfseries\arabic*.}]\itemsep0em
	%
	\item \textbf{For a research manuscript that has already been published} (if not yet published, please skip to section 2)\textbf{:}
	%
	\begin{enumerate}[label={\alph*)}]\itemsep0em
	%
	\item \textbf{What is the title of the manuscript?}

	\item \textbf{Please include a link to or doi for the work:}

	\item \textbf{Where was the work published?}

	\item \textbf{Who published the work?}

	\item \textbf{When was the work published?}

	\item \textbf{List the manuscript's authors in the order they appear on the publication:}

	\item \textbf{Was the work peer reviewd?}

	\item \textbf{Have you retained the copyright?}

	\item \textbf{Was an earlier form of the manuscript uploaded to a preprint server (e.g. medRxiv)? If ‘Yes’, please give a link or doi}
	\\
	If ‘No’, please seek permission from the relevant publisher and check the box next to the below statement:
	%
\begin{itemize}\itemsep0em
% To check this box, replace \Box with \boxtimes
\item[$\Box$] {\itshape I acknowledge permission of the publisher named under 1d to include in this thesis portions of the publication named as included in 1c.}
\end{itemize}
%
\end{enumerate}
%
\item \textbf{For a research manuscript prepared for publication but that has not yet been published} (if already published, please skip to section 3)\textbf{:}
%
\begin{enumerate}[label={\alph*)}]\itemsep0em
	%
	\item \textbf{What is the current title of the manuscript?}

	\citetitle{Young2023}
	\item \textbf{Has the manuscript been uploaded to a preprint server `e.g. medRxiv'?
	\\
	If `Yes', please please give a link or doi:}

	Yes, \url{https://www.researchgate.net/publication/375601134_Training_Data_Requirements_for_Atlas-Based_Brain_Fibre_Tract_Identification}
	% Answer here:
	\item \textbf{Where is the work intended to be published?}
	% Answer here: e.g. journal name

	Proceedings of the \citefield{Young2023}{eventtitle}
	\item \textbf{List the manuscript's authors in the intended authorship order:}
	% Answer here

	\citeauthor*{Young2023}
	\item \textbf{Stage of publication:}
	% answer here: e.g. in submission

	Work presented at conference, copyright transferred to IEEE, yet now indication of if/when proceedings may be published (copyright on submitted version retained).
\end{enumerate}

\item \textbf{For multi-authored work, please give a statement of contribution covering all authors} (if single-author, please skip to section 4)\textbf{:}
\begin{description}[font=\sffamily]
	\item[Fiona Young] Methodology (conceptualisation and implementation), analysis, manuscript original draft and graphics
	\item[Kristian Aquilina] Supervision
	\item[Chris A. Clark] Supervision
	\item[Jonathan D. Clayden] Conceptualisation, supervision, manuscript review and editing
\end{description}
% Answer here
\item \textbf{In which chapter(s) of your thesis can this material be found?}
% Answer here

Chapter \ref{chap:atlas}
\end{enumerate}

\textbf{e-Signatures confirming that the information above is accurate}
(this form should be co-signed by the supervisor/ senior author unless this is not appropriate, e.g. if the paper was a single-author work)\textbf{:}\\
\textbf{}\\
\textbf{Candidate:}\\
\textbf{Date:}\\
% this form should be co-signed by the supervisor/ senior author unless this is not appropriate, e.g. if the paper was a single-author work):
\textbf{}\\
\textbf{Supervisor/Senior Author signature} (where appropriate)\textbf{:}\\
\textbf{Date:}
%

%%%%%%%%%%%%%%%%%%%%%%%%% --- Diffusion ISMRM --- %%%%%%%%%%%%%%%%%%%%%%%%%%%%%
\newpage
\begin{enumerate}[leftmargin=*,label={\bfseries\arabic*.}]\itemsep0em
	%
	\item \textbf{For a research manuscript that has already been published} (if not yet published, please skip to section 2)\textbf{:}
	%
	\begin{enumerate}[label={\alph*)}]\itemsep0em
	%
	\item \textbf{What is the title of the manuscript?}

	\item \textbf{Please include a link to or doi for the work:}

	\item \textbf{Where was the work published?}

	\item \textbf{Who published the work?}

	\item \textbf{When was the work published?}

	\item \textbf{List the manuscript's authors in the order they appear on the publication:}

	\item \textbf{Was the work peer reviewd?}

	\item \textbf{Have you retained the copyright?}

	\item \textbf{Was an earlier form of the manuscript uploaded to a preprint server (e.g. medRxiv)? If ‘Yes’, please give a link or doi}
	\\
	If ‘No’, please seek permission from the relevant publisher and check the box next to the below statement:
	%
\begin{itemize}\itemsep0em
% To check this box, replace \Box with \boxtimes
\item[$\Box$] {\itshape I acknowledge permission of the publisher named under 1d to include in this thesis portions of the publication named as included in 1c.}
\end{itemize}
%
\end{enumerate}
%
\item \textbf{For a research manuscript prepared for publication but that has not yet been published} (if already published, please skip to section 3)\textbf{:}
%
\begin{enumerate}[label={\alph*)}]\itemsep0em
	%
	\item \textbf{What is the current title of the manuscript?}
	% Answer here:

	\citetitle{Young2022a}
	\item \textbf{Has the manuscript been uploaded to a preprint server `e.g. medRxiv'?
	\\
	If `Yes', please please give a link or doi:}

	Yes, \url{https://www.researchgate.net/publication/367116849_Stability_of_white_matter_tract_segmentation_methods_with_decreasing_data_quality}
	% Answer here:
	\item \textbf{Where is the work intended to be published?}
	% Answer here: e.g. journal name

	N/A
	\item \textbf{List the manuscript's authors in the intended authorship order:}
	% Answer here

	\citeauthor*{Young2022a}
	\item \textbf{Stage of publication:}
	% answer here: e.g. in submission

	Work presented at conference, no proceedings published.
\end{enumerate}

\item \textbf{For multi-authored work, please give a statement of contribution covering all authors} (if single-author, please skip to section 4)\textbf{:}
\begin{description}[font=\sffamily]
	\item[Fiona Young] Methodology (conceptualisation and implementation), analysis, manuscript original draft and graphics
	\item[Jonathan D. Clayden] Conceptualisation, supervision, manuscript review and editing
\end{description}
% Answer here
\item \textbf{In which chapter(s) of your thesis can this material be found?}
% Answer here

Chapter \ref{chap:applications}
\end{enumerate}

\textbf{e-Signatures confirming that the information above is accurate}
(this form should be co-signed by the supervisor/ senior author unless this is not appropriate, e.g. if the paper was a single-author work)\textbf{:}\\
\textbf{}\\
\textbf{Candidate:}\\
\textbf{Date:}\\
% this form should be co-signed by the supervisor/ senior author unless this is not appropriate, e.g. if the paper was a single-author work):
\textbf{}\\
\textbf{Supervisor/Senior Author signature} (where appropriate)\textbf{:}\\
\textbf{Date:}
%
}% End font switch


\tableofcontents
\setcounter{tocdepth}{1}
\listoffigures
\listoftables
{\setstretch{1.0}
\setlength{\glsdescwidth}{0.8\linewidth}
\printglossary[type=\acronymtype,title={List of Abbreviations}]
}
